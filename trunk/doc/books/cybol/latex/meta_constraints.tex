%
% $RCSfile: meta_constraints.tex,v $
%
% Copyright (c) 2002-2007. Christian Heller. All rights reserved.
%
% Permission is granted to copy, distribute and/or modify this document
% under the terms of the GNU Free Documentation License, Version 1.1 or
% any later version published by the Free Software Foundation; with no
% Invariant Sections, with no Front-Cover Texts and with no Back-Cover
% Texts. A copy of the license is included in the section entitled
% "GNU Free Documentation License".
%
% http://www.cybop.net
% - Cybernetics Oriented Programming -
%
% Version: $Revision: 1.1 $ $Date: 2007-08-01 13:59:00 $ $Author: christian $
% Authors: Christian Heller <christian.heller@tuxtax.de>
%

\subsection{Meta Constraints}
\label{meta_constraints_heading}
\index{Meta Constraints Example}
\index{Debian GNU/Linux Package Definition}

The example of this section shows a possible \emph{Debian GNU/Linux}
\cite{debian} \emph{Package} definition, written in CYBOL:

\begin{scriptsize}
    \begin{verbatim}
<model>
    <part name="name" channel="inline" abstraction="string" model="resmedicinae"/>
    <part name="version" channel="inline" abstraction="string" model="0.1.0.0"/>
    <part name="section" channel="inline" abstraction="string" model="science"/>
    <part name="priority" channel="inline" abstraction="string" model="optional"/>
    <part name="architecture" channel="inline" abstraction="string" model="all"/>
    <part name="packages" channel="file" abstraction="cybol" model="resmedicinae-packages"/>
    <part name="files" channel="file" abstraction="cybol" model="resmedicinae-files"/>
    <part name="maintainer" channel="inline" abstraction="string" model="Happy Coder"/>
    <part name="description" channel="inline" abstraction="string" model="Medical System"/>
</model>
    \end{verbatim}
\end{scriptsize}

The part called \emph{packages} in the example above references an external
CYBOL knowledge template, which is displayed below. It represents a list of
packages having different versions and varying strengths of dependency. The
\emph{strength} property of the last of these packages has the model value
\emph{suggests} and, it contains meta information about that property, namely
a \emph{constraint}. Constraints can be, for example: minima, maxima or a
choice of possible values, as in this case.

\begin{scriptsize}
    \begin{verbatim}
<model>
    <part name="cyboi" channel="inline" abstraction="string" model="cyboi">
        <property name="strength" channel="inline" abstraction="string" model="depends"/>
        <property name="version" channel="inline" abstraction="string" model=">= 1.0.0.0"/>
        <property name="conflicts" channel="inline" abstraction="string" model="< 1.0.0.0"/>
    </part>
    <part name="cybol-healthcare" channel="inline" abstraction="string" model="cybol-healthcare">
        <property name="strength" channel="inline" abstraction="string" model="depends"/>
        <property name="version" channel="inline" abstraction="string" model=">= 0.1.0.0"/>
    </part>
    <part name="resadmin" channel="inline" abstraction="string" model="resadmin">
        <property name="strength" channel="inline" abstraction="string" model="recommends"/>
        <property name="version" channel="inline" abstraction="string" model=">= 0.8.0.0"/>
    </part>
    <part name="resmedicinae-doc" channel="inline" abstraction="string" model="resmedicinae-doc">
        <property name="strength" channel="inline" abstraction="string" model="suggests">
            <constraint name="choice" channel="inline" abstraction="set" model="suggests,recommends"/>
        </property>
        <property name="version" channel="inline" abstraction="string" model=">= 0.1.0.0"/>
    </part>
</model>
    \end{verbatim}
\end{scriptsize}
