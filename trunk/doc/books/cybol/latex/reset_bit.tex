%
% $RCSfile: reset_bit.tex,v $
%
% Copyright (c) 2002-2007. Christian Heller. All rights reserved.
%
% Permission is granted to copy, distribute and/or modify this document
% under the terms of the GNU Free Documentation License, Version 1.1 or
% any later version published by the Free Software Foundation; with no
% Invariant Sections, with no Front-Cover Texts and with no Back-Cover
% Texts. A copy of the license is included in the section entitled
% "GNU Free Documentation License".
%
% http://www.cybop.net
% - Cybernetics Oriented Programming -
%
% Version: $Revision: 1.1 $ $Date: 2007-07-17 20:02:36 $ $Author: christian $
% Authors: Christian Heller <christian.heller@tuxtax.de>
%

\subsection{Reset Bit}
\label{reset_bit_heading}
\index{Reset Bit}

This operation resets the bit at the given position within the given number.
RESET means setting the bit to FALSE.

\subsubsection{Number Property}

\emph{required}

name=\texttt{'number'}\\
abstraction=\texttt{'integer' \vline\ 'knowledge' \vline\ 'encapsulated'}\\
model=\texttt{the actual number (or knowledge model)}

This property specifies the number of which one bit is to be reset.

\subsubsection{Position Property}

\emph{required}

name=\texttt{'position'}\\
abstraction=\texttt{'integer' \vline\ 'knowledge' \vline\ 'encapsulated'}\\
model=\texttt{position of the bit}

The position of the bit whose value is to be reset (set to FALSE).

\subsubsection{Result Property}

\emph{required}

name=\texttt{'result'}\\
abstraction=\texttt{'knowledge' \vline\ 'encapsulated'}\\
model=\texttt{result knowledge model}

This is the result knowledge model in which to store the number of which one
bit was reset.
