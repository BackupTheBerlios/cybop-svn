%
% $RCSfile: analogy.tex,v $
%
% Copyright (c) 2002-2007. Christian Heller. All rights reserved.
%
% Permission is granted to copy, distribute and/or modify this document
% under the terms of the GNU Free Documentation License, Version 1.1 or
% any later version published by the Free Software Foundation; with no
% Invariant Sections, with no Front-Cover Texts and with no Back-Cover
% Texts. A copy of the license is included in the section entitled
% "GNU Free Documentation License".
%
% http://www.cybop.net
% - Cybernetics Oriented Programming -
%
% Version: $Revision: 1.1 $ $Date: 2007-08-01 13:59:00 $ $Author: christian $
% Authors: Christian Heller <christian.heller@tuxtax.de>
%

\section{Analogy}
\label{analogy_heading}
\index{Analogy between CYBOP and Java}
\index{Java-CYBOP Analogy}
\index{CYBOP-Java Analogy}
\index{CYBOI as Virtual Machine}

There are analogies to other systems run by language interpretation. Table
\ref{analogy_table} shows that between the \emph{Java-} and \emph{CYBOP} world.
Both are based on a programming theory, have a language and interpreter, the
latter sometimes being called a \emph{Virtual Machine} (VM).

\begin{table}[ht]
    \begin{center}
        \begin{footnotesize}
        \begin{tabular}{| p{35mm} | p{35mm} | p{35mm} |}
            \hline
            \textbf{Criterion} & \textbf{Java World} & \textbf{CYBOP World}\\
            \hline
            Theory & OOP in Java & CYBOP\\
            \hline
            Language & Java & CYBOL\\
            \hline
            Interpreter & Java VM & CYBOI\\
            \hline
        \end{tabular}
        \end{footnotesize}
        \caption{Analogy between the Java- and CYBOP World}
        \label{analogy_table}
    \end{center}
\end{table}

CYBOI provides low-level, platform-dependent system functionality, close to the
OS, together with a unified knowledge schema which allows CYBOL applications to
be truly portable, well extensible and easier to program, because developers
need to concentrate on domain knowledge only. Since CYBOI may interpret CYBOL
sources \emph{live} at system runtime, without the need for previous compilation
(as in Java), changes to CYBOL sources can get into effect right away, without
having to restart the system.
