%
% $RCSfile: neg.tex,v $
%
% Copyright (c) 2002-2007. Christian Heller. All rights reserved.
%
% Permission is granted to copy, distribute and/or modify this document
% under the terms of the GNU Free Documentation License, Version 1.1 or
% any later version published by the Free Software Foundation; with no
% Invariant Sections, with no Front-Cover Texts and with no Back-Cover
% Texts. A copy of the license is included in the section entitled
% "GNU Free Documentation License".
%
% http://www.cybop.net
% - Cybernetics Oriented Programming -
%
% Version: $Revision: 1.2 $ $Date: 2007-08-01 13:59:00 $ $Author: christian $
% Authors: Christian Heller <christian.heller@tuxtax.de>
%

\subsection{NEG}
\label{neg_heading}
\index{NEG}

This operation applies the logic NEG operator to the given boolean operand.

\subsubsection{Example}

\begin{scriptsize}
    \begin{verbatim}
<part name="apply_neg" channel="inline" abstraction="operation" model="neg">
    <property name="operand" channel="inline" abstraction="knowledge" model=".app.value"/>
    <property name="result" channel="inline" abstraction="knowledge" model=".app.result"/>
</part>
    \end{verbatim}
\end{scriptsize}

\subsubsection{Operand Property}

This is the operand of the boolean operation.

\emph{required}

name=\texttt{'operand'}\\
abstraction=\texttt{'boolean' \vline\ 'knowledge' \vline\ 'encapsulated'}\\
model=\texttt{boolean value or knowledge model}

\subsubsection{Result Property}

This is the result of the boolean operation. It may be either \emph{true} or
\emph{false}.

\emph{required}

name=\texttt{'result'}\\
abstraction=\texttt{'knowledge' \vline\ 'encapsulated'}\\
model=\texttt{knowledge model}
