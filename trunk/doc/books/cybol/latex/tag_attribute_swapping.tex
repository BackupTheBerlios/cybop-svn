%
% $RCSfile: tag_attribute_swapping.tex,v $
%
% Copyright (c) 2002-2007. Christian Heller. All rights reserved.
%
% Permission is granted to copy, distribute and/or modify this document
% under the terms of the GNU Free Documentation License, Version 1.1 or
% any later version published by the Free Software Foundation; with no
% Invariant Sections, with no Front-Cover Texts and with no Back-Cover
% Texts. A copy of the license is included in the section entitled
% "GNU Free Documentation License".
%
% http://www.cybop.net
% - Cybernetics Oriented Programming -
%
% http://www.resmedicinae.org
% - Information in Medicine -
%
% Version: $Revision: 1.1 $ $Date: 2007-08-01 13:59:00 $ $Author: christian $
% Authors: Christian Heller <christian.heller@tuxtax.de>
%

\subsection{Tag-Attribute Swapping}
\label{tag_attribute_swapping_heading}
\index{Tag-Attribute Swapping}

CYBOL swaps the meaning attributes and tags traditionally have in XML
documents, where tags represent elements that may be nested infinitely and
attributes hold additional (meta) data about a tag. Following an example of how
CYBOL might have looked that way:

\begin{scriptsize}
    \begin{verbatim}
<model>
    <part>
        <name="title"/>
        <channel="inline"/>
        <abstraction="character"/>
        <model="Res Medicinae"/>
    </part>
    <part layout="compass" position="north">
        <name="menu_bar"/>
        <channel="file"/>
        <abstraction="cybol"/>
        <model="gui/menu_bar.cybol"/>
    </part>
</model>
    \end{verbatim}
\end{scriptsize}

The current final specification of CYBOL, on the contrary, uses attributes to
define a nested element (part) and tags to give properties (meta information)
about such a nested element, in the following way:

\begin{scriptsize}
    \begin{verbatim}
<model>
    <part name="title" channel="inline" abstraction="character" model="Res Medicinae"/>
    <part name="menu_bar" channel="file" abstraction="cybol" model="gui/menu_bar.cybol">
        <property name="layout" channel="inline" abstraction="character" model="compass"/>
        <property name="position" channel="inline" abstraction="character" model="north"/>
    </part>
</model>
    \end{verbatim}
\end{scriptsize}

This is because:

\begin{enumerate}
    \item the number of attributes specifying a part in CYBOL is fixed, whereas
        the number of tags specifying a property of a part is not, and the
        number of XML tags is easier extensible than that of attributes;
    \item that way it is also possible to specify a part without any properties
        in just one CYBOL code line, while otherwise four tags would always
        have to be given;
    \item not only a part may be nested (consist of smaller parts), but also a
        property may be (for example a position consisting of three coordinates
        given as parts), which necessitates the four standard attributes to be
        given for properties and constraints as well.
\end{enumerate}
