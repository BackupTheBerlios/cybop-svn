%
% $RCSfile: loop.tex,v $
%
% Copyright (c) 2002-2007. Christian Heller. All rights reserved.
%
% Permission is granted to copy, distribute and/or modify this document
% under the terms of the GNU Free Documentation License, Version 1.1 or
% any later version published by the Free Software Foundation; with no
% Invariant Sections, with no Front-Cover Texts and with no Back-Cover
% Texts. A copy of the license is included in the section entitled
% "GNU Free Documentation License".
%
% http://www.cybop.net
% - Cybernetics Oriented Programming -
%
% Version: $Revision: 1.2 $ $Date: 2007-08-01 13:59:00 $ $Author: christian $
% Authors: Christian Heller <christian.heller@tuxtax.de>
%

\subsection{Loop}
\label{loop_heading}
\index{Loop}

This operation starts an endless loop that runs until the given break flag is
set.

\subsubsection{Example}

\begin{scriptsize}
    \begin{verbatim}
<part name="loop_addresses" channel="inline" abstraction="operation" model="loop">
    <property name="model" channel="inline" abstraction="knowledge" model=".app.process"/>
    <property name="break" channel="inline" abstraction="knowledge" model=".app.break_flag"/>
</part>
    \end{verbatim}
\end{scriptsize}

\subsubsection{Model Property}

This is the knowledge model to be executed repeatedly by the loop.

\emph{required}

name=\texttt{'model'}\\
abstraction=\texttt{'knowledge' \vline\ 'encapsulated'}\\
model=\texttt{knowledge model}

\subsubsection{Break Property}

This is the break flag. Once set, the loop will be left (exited). It may be
either \emph{true} or \emph{false}.

\emph{required}

name=\texttt{'break'}\\
abstraction=\texttt{'knowledge' \vline\ 'encapsulated'}\\
model=\texttt{knowledge model}
