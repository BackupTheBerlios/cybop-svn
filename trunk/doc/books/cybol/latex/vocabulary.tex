%
% $RCSfile: vocabulary.tex,v $
%
% Copyright (c) 2002-2007. Christian Heller. All rights reserved.
%
% Permission is granted to copy, distribute and/or modify this document
% under the terms of the GNU Free Documentation License, Version 1.1 or
% any later version published by the Free Software Foundation; with no
% Invariant Sections, with no Front-Cover Texts and with no Back-Cover
% Texts. A copy of the license is included in the section entitled
% "GNU Free Documentation License".
%
% http://www.cybop.net
% - Cybernetics Oriented Programming -
%
% Version: $Revision: 1.1 $ $Date: 2007-08-01 13:59:01 $ $Author: christian $
% Authors: Christian Heller <christian.heller@tuxtax.de>
%

\section{Vocabulary}
\label{vocabulary_heading}
\index{Vocabulary}
\index{Terms}
\index{Symbols}
\index{Document Type Definition}
\index{DTD}
\index{XML Schema Definition}
\index{XSD}
\index{Extended Backus Naur Form}
\index{EBNF}

The \emph{Vocabulary} is what fills a language with life. It delivers the
\emph{Terms} and \emph{Symbols} that are combined after the rules of a syntax.

XML allows to define and exchange the whole vocabulary of a language. It offers
two ways in which a list of legal elements can be defined: The traditional
\emph{Document Type Definition} (DTD) and the more modern
\emph{XML Schema Definition} (XSD). Besides the vocabulary, DTD and XSD define
the structure of an XML document and allow to typify, constrain and validate
items. In addition to DTD and XSD, the \emph{Extended Backus Naur Form} (EBNF)
of CYBOL is given following.

%
% $RCSfile: document_type_definition.tex,v $
%
% Copyright (c) 2002-2007. Christian Heller. All rights reserved.
%
% Permission is granted to copy, distribute and/or modify this document
% under the terms of the GNU Free Documentation License, Version 1.1 or
% any later version published by the Free Software Foundation; with no
% Invariant Sections, with no Front-Cover Texts and with no Back-Cover
% Texts. A copy of the license is included in the section entitled
% "GNU Free Documentation License".
%
% http://www.cybop.net
% - Cybernetics Oriented Programming -
%
% Version: $Revision: 1.2 $ $Date: 2007-08-01 13:59:00 $ $Author: christian $
% Authors: Christian Heller <christian.heller@tuxtax.de>
%

\subsection{Document Type Definition}
\label{document_type_definition_heading}
\index{Document Type Definition}
\index{DTD}
\index{Extensible Markup Language}
\index{XML}
\index{Markup Tag}
\index{model Tag}
\index{part Tag}
\index{property Tag}
\index{constraint Tag}
\index{name Attribute}
\index{channel Attribute}
\index{abstraction Attribute}
\index{model Attribute}

A DTD represents the type definition of an XML document. It consists of a set
of \emph{Markup Tags} and their \emph{Interpretation} \cite{foldoc}. DTDs can
be declared inline, within a document, or as an external reference
\cite{w3schools}. Figure \ref{dtd_figure} shows the DTD of the CYBOL language.

\begin{figure}[ht]
    \bigskip
    \begin{scriptsize}
        \begin{verbatim}
<!ELEMENT model (part*)>
<!ELEMENT part (property*)>
<!ELEMENT property (constraint*)>
<!ELEMENT constraint EMPTY>

<!ATTLIST part
    name CDATA #REQUIRED
    channel CDATA #REQUIRED
    abstraction CDATA #REQUIRED
    model CDATA #REQUIRED>
<!ATTLIST property
    name CDATA #REQUIRED
    channel CDATA #REQUIRED
    abstraction CDATA #REQUIRED
    model CDATA #REQUIRED>
<!ATTLIST constraint
    name CDATA #REQUIRED
    channel CDATA #REQUIRED
    abstraction CDATA #REQUIRED
    model CDATA #REQUIRED>
        \end{verbatim}
    \end{scriptsize}
    \caption{Recommended CYBOL DTD}
    \label{dtd_figure}
\end{figure}

Following the pure hierarchical structure of CYBOL, it would actually suffice
to use a DTD as simple as the one shown in figure \ref{simpledtd_figure}. Since
the three elements \emph{part}, \emph{property} and \emph{constraint} (compare
figure \ref{dtd_figure}) have the same list of required attributes, they could
be summarised under the name \emph{part}, for example. Because the structure of
a CYBOL model is non-ambiguous, the meaning of its elements can be guessed from
their position within the model.

\begin{figure}[ht]
    \bigskip
    \begin{scriptsize}
        \begin{verbatim}
<!ELEMENT part (part*)>

<!ATTLIST part
    name CDATA #REQUIRED
    channel CDATA #REQUIRED
    abstraction CDATA #REQUIRED
    model CDATA #REQUIRED>
        \end{verbatim}
    \end{scriptsize}
    \caption{Simplified CYBOL DTD}
    \label{simpledtd_figure}
\end{figure}

\clearpage

For the purpose of expressing knowledge in accordance with the schema suggested
by CYBOP \cite{cybop}, a CYBOL knowledge template (file) does not need to have
a root element. The file name clearly identifies it. For reasons of XML
conformity, however, an extra root element called \emph{model} was defined
(figure \ref{dtd_figure}). And for reasons of better readability and
programmability, the three kinds of embedded elements were given distinct names.

%
% $RCSfile: xml_schema_definition.tex,v $
%
% Copyright (c) 2002-2007. Christian Heller. All rights reserved.
%
% Permission is granted to copy, distribute and/or modify this document
% under the terms of the GNU Free Documentation License, Version 1.1 or
% any later version published by the Free Software Foundation; with no
% Invariant Sections, with no Front-Cover Texts and with no Back-Cover
% Texts. A copy of the license is included in the section entitled
% "GNU Free Documentation License".
%
% http://www.cybop.net
% - Cybernetics Oriented Programming -
%
% Version: $Revision: 1.1 $ $Date: 2007-07-17 20:02:36 $ $Author: christian $
% Authors: Christian Heller <christian.heller@tuxtax.de>
%

\section{XML Schema Definition}
\label{xml_schema_definition_heading}
\index{XML Schema Definition}
\index{XSD}
\index{CYBOL XSD}
\index{XML Schema}
\index{Extensible Markup Language}
\index{XML}

\emph{XML Schema} is an XML-based alternative to DTD \cite{w3schools}, and XSD
is its definition language. There is a lot of discussion going on about the
sense or \emph{Myth} of XML Schema \cite{browne}, that this document will not
take part in. Figure \ref{xsd_figure} shows the XSD of the CYBOL language.

\begin{figure}[ht]
    \bigskip
    \bigskip
    \begin{scriptsize}
        \begin{verbatim}
<?xml version="1.0"?>
<xs:schema xmlns:xs='http://www.w3.org/2001/XMLSchema' targetNamespace='http://www.cybop.net'
    xmlns='http://www.cybop.net' elementFormDefault='qualified'>
    <xs:element name='part'>
        <xs:complexType>
            <xs:sequence>
                <xs:element ref='part' minOccurs='0' maxOccurs='unbounded'/>
            </xs:sequence>
            <xs:attribute name='name' type='xs:string' use='required'/>
            <xs:attribute name='channel' type='xs:string' use='required'/>
            <xs:attribute name='abstraction' type='xs:string' use='required'/>
            <xs:attribute name='model' type='xs:string' use='required'/>
        </xs:complexType>
    </xs:element>
</xs:schema>
        \end{verbatim}
    \end{scriptsize}
    \caption{Simplified CYBOL XSD}
    \label{simplexsd_figure}
\end{figure}

\begin{figure}[ht]
    \bigskip
    \bigskip
    \begin{scriptsize}
        \begin{verbatim}
<?xml version="1.0"?>
<xs:schema xmlns:xs='http://www.w3.org/2001/XMLSchema' targetNamespace='http://www.cybop.net'
    xmlns='http://www.cybop.net' elementFormDefault='qualified'>
    <xs:element name='model'>
        <xs:complexType>
            <xs:sequence>
                <xs:element ref='part' minOccurs='0' maxOccurs='unbounded'/>
            </xs:sequence>
        </xs:complexType>
    </xs:element>
    <xs:element name='part'>
        <xs:complexType>
            <xs:sequence>
                <xs:element ref='property' minOccurs='0' maxOccurs='unbounded'/>
            </xs:sequence>
            <xs:attribute name='name' type='xs:string' use='required'/>
            <xs:attribute name='channel' type='xs:string' use='required'/>
            <xs:attribute name='abstraction' type='xs:string' use='required'/>
            <xs:attribute name='model' type='xs:string' use='required'/>
        </xs:complexType>
    </xs:element>
    <xs:element name='property'>
        <xs:complexType>
            <xs:sequence>
                <xs:element ref='constraint' minOccurs='0' maxOccurs='unbounded'/>
            </xs:sequence>
            <xs:attribute name='name' type='xs:string' use='required'/>
            <xs:attribute name='channel' type='xs:string' use='required'/>
            <xs:attribute name='abstraction' type='xs:string' use='required'/>
            <xs:attribute name='model' type='xs:string' use='required'/>
        </xs:complexType>
    </xs:element>
    <xs:element name='constraint'>
        <xs:complexType>
            <xs:attribute name='name' type='xs:string' use='required'/>
            <xs:attribute name='channel' type='xs:string' use='required'/>
            <xs:attribute name='abstraction' type='xs:string' use='required'/>
            <xs:attribute name='model' type='xs:string' use='required'/>
        </xs:complexType>
    </xs:element>
</xs:schema>
        \end{verbatim}
    \end{scriptsize}
    \caption{Recommended CYBOL XSD}
    \label{xsd_figure}
\end{figure}

Again, a simplified version of that XSD could be created (figure
\ref{simplexsd_figure}). But for reasons explained before, the recommended XSD
is the one shown in figure \ref{xsd_figure}.

%
% $RCSfile: extended_backus_naur_form.tex,v $
%
% Copyright (C) 2002-2008. Christian Heller.
%
% Permission is granted to copy, distribute and/or modify this document
% under the terms of the GNU Free Documentation License, Version 1.1 or
% any later version published by the Free Software Foundation; with no
% Invariant Sections, with no Front-Cover Texts and with no Back-Cover
% Texts. A copy of the license is included in the section entitled
% "GNU Free Documentation License".
%
% http://www.cybop.net
% - Cybernetics Oriented Programming -
%
% http://www.resmedicinae.org
% - Information in Medicine -
%
% Version: $Revision: 1.1 $ $Date: 2008-08-19 20:41:06 $ $Author: christian $
% Authors: Christian Heller <christian.heller@tuxtax.de>
%

\subsubsection{Extended Backus Naur Form}
\label{extended_backus_naur_form_heading}
\index{Extended Backus Naur Form}
\index{EBNF}
\index{Backus Naur Form}
\index{BNF}
\index{CYBOL EBNF}

The EBNF adds regular expression syntax to the \emph{Backus Naur Form} (BNF)
notatation \cite{naur}, in order to allow very compact specifications
\cite{kuhn}. Figure \ref{ebnf_figure} shows the EBNF of the CYBOL language.

\begin{figure}[ht]
    \bigskip
    \bigskip
    \begin{scriptsize}
        \begin{verbatim}
CYBOL       = '<model>'
                    {part}
                '</model>';

part        = '<part ' attributes '\>' |
                '<part ' attributes '>'
                    {property}
                '</part>';

property    = '<property ' attributes '\>' |
                '<property ' attributes '>'
                    {constraint}
                '</property>';

constraint  = '<constraint ' attributes '\>';

attributes  = name_attribute channel_attribute abstraction_attribute model_attribute

name_attribute          = 'name="' name '"';
channel_attribute       = 'channel="' channel '"';
abstraction_attribute   = 'abstraction="' abstraction '"';
model_attribute         = 'model="' model '"';

name        = description_sign;
channel     = description_sign;
abstraction = description_sign;
model       = value_sign;

description_sign    = { ( letter | number ) };
value_sign          = { ( letter | number | other_sign ) };

letter          = small_letter | big_letter;
small_letter    = 'a' | 'b' | 'c' | 'd' | 'e' | 'f' | 'g' |
                    'h' | 'i' | 'j' | 'k' | 'l' | 'm' | 'n' |
                    'o' | 'p' | 'q' | 'r' | 's' | 't' | 'u' |
                    'v' | 'w' | 'x' | 'y' | 'z';
big_letter      = 'A' | 'B' | 'C' | 'D' | 'E' | 'F' | 'G' |
                    'H' | 'I' | 'J' | 'K' | 'L' | 'M' | 'N' |
                    'O' | 'P' | 'Q' | 'R' | 'S' | 'T' | 'U' |
                    'V' | 'W' | 'X' | 'Y' | 'Z';
other_sign      = ',' | '.' | '/', '+', '-', '*';
number          = '0' | '1' | '2' | '3' | '4' |
                    '5' | '6' | '7' | '8' | '9';
        \end{verbatim}
    \end{scriptsize}
    \caption{CYBOL in EBNF}
    \label{ebnf_figure}
\end{figure}

\clearpage
