%
% $RCSfile: xor.tex,v $
%
% Copyright (c) 2002-2007. Christian Heller. All rights reserved.
%
% Permission is granted to copy, distribute and/or modify this document
% under the terms of the GNU Free Documentation License, Version 1.1 or
% any later version published by the Free Software Foundation; with no
% Invariant Sections, with no Front-Cover Texts and with no Back-Cover
% Texts. A copy of the license is included in the section entitled
% "GNU Free Documentation License".
%
% http://www.cybop.net
% - Cybernetics Oriented Programming -
%
% Version: $Revision: 1.1 $ $Date: 2007-07-17 20:02:36 $ $Author: christian $
% Authors: Christian Heller <christian.heller@tuxtax.de>
%

\subsection{XOR}
\label{xor_heading}
\index{XOR}

This operation applies the logic XOR operator to the given boolean operands.

\subsubsection{Operand 1 Property}

\emph{required}

name=\texttt{'operand\_1'}\\
abstraction=\texttt{'boolean' \vline\ 'knowledge' \vline\ 'encapsulated'}\\
model=\texttt{boolean value or knowledge model}

This is the first operand of the boolean operation.

\subsubsection{Operand 2 Property}

\emph{required}

name=\texttt{'operand\_2'}\\
abstraction=\texttt{'boolean' \vline\ 'knowledge' \vline\ 'encapsulated'}\\
model=\texttt{boolean value or knowledge model}

This is the second operand of the boolean operation.

\subsubsection{Result Property}

\emph{required}

name=\texttt{'result'}\\
abstraction=\texttt{'boolean' \vline\ 'knowledge' \vline\ 'encapsulated'}\\
model=\texttt{boolean value or knowledge model}

This is the result of the boolean operation.
