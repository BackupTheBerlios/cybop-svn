%
% $RCSfile: simple_assignment.tex,v $
%
% Copyright (c) 2002-2007. Christian Heller. All rights reserved.
%
% Permission is granted to copy, distribute and/or modify this document
% under the terms of the GNU Free Documentation License, Version 1.1 or
% any later version published by the Free Software Foundation; with no
% Invariant Sections, with no Front-Cover Texts and with no Back-Cover
% Texts. A copy of the license is included in the section entitled
% "GNU Free Documentation License".
%
% http://www.cybop.net
% - Cybernetics Oriented Programming -
%
% Version: $Revision: 1.1 $ $Date: 2007-08-01 13:59:00 $ $Author: christian $
% Authors: Christian Heller <christian.heller@tuxtax.de>
%

\subsection{Simple Assignment}
\label{simple_assignment_heading}
\index{Simple Assignment Example}

CYBOL does not know \emph{Variables} as used in classical languages. All states
a system may take on are represented by just one \emph{Knowledge Tree}, which
applications may access in a defined manner (dot-separated knowledge paths).
Consequently, \emph{Assignments} are done differently in CYBOL than in
classical programming languages. All kinds of state changes go back to a
manipulation of the one knowledge tree:

\begin{scriptsize}
    \begin{verbatim}
<model>
    <part name="copy_value" channel="inline" abstraction="operation" model="copy">
        <property name="source" channel="inline" abstraction="knowledge" model="domain.name"/>
        <property name="destination" channel="inline" abstraction="knowledge" model="gui.name"/>
    </part>
    <part name="move_branch" channel="inline" abstraction="operation" model="move">
        <property name="source" channel="inline" abstraction="knowledge" model="address_1.phone"/>
        <property name="destination" channel="inline" abstraction="knowledge" model="address_2"/>
    </part>
</model>
    \end{verbatim}
\end{scriptsize}

The first operation in the example above copies a value between two branches of
the tree. Only primitive values can be copied. The second operation removes a
whole tree branch (referenced by the \emph{source} property) from one parent
node, and adds it to another (referenced by the \emph{destination} property).
