%
% $RCSfile: receive.tex,v $
%
% Copyright (c) 2002-2007. Christian Heller. All rights reserved.
%
% Permission is granted to copy, distribute and/or modify this document
% under the terms of the GNU Free Documentation License, Version 1.1 or
% any later version published by the Free Software Foundation; with no
% Invariant Sections, with no Front-Cover Texts and with no Back-Cover
% Texts. A copy of the license is included in the section entitled
% "GNU Free Documentation License".
%
% http://www.cybop.net
% - Cybernetics Oriented Programming -
%
% Version: $Revision: 1.1 $ $Date: 2007-07-17 20:02:36 $ $Author: christian $
% Authors: Christian Heller <christian.heller@tuxtax.de>
%

\subsection{Receive}
\label{receive_heading}
\index{Receive}

This operation receives data from the given data source.

\subsubsection{Name Property}

\emph{required}

name=\texttt{'name'}\\
abstraction=\texttt{'character'}\\
model=\texttt{string of characters}

This is the name of the knowledge model to be created.

\subsubsection{Channel Property}

\emph{required}

name=\texttt{'channel'}\\
abstraction=\texttt{'character'}\\
model=\texttt{'inline' \vline\ 'file'}

This is the channel via which the knowledge model is received.

\subsubsection{Abstraction Property}

\emph{required}

name=\texttt{'abstraction'}\\
abstraction=\texttt{'character'}\\
model=\texttt{'boolean' \vline\ 'integer' \vline\ 'float' \vline\ 'character' \vline\ etc.}

This is the abstraction of the knowledge model to be created.

\subsubsection{Model Property}

\emph{required}

name=\texttt{'model'}\\
abstraction=\texttt{'character'}\\
model=\texttt{path to a file}

This is the data source (knowledge template) from where the actual knowledge
model is created.

\subsubsection{Details Property}

\emph{required}

name=\texttt{'details'}\\
abstraction=\texttt{'character'}\\
model=\texttt{path to a file}

This is the model details' data source (knowledge template) from where the
knowledge model's details are created.

\subsubsection{Element Property}

\emph{required}

name=\texttt{'element'}\\
abstraction=\texttt{'character'}\\
model=\texttt{'part' \vline\ 'meta'}

This property decides about the kind of model being created. A part model will
be added to the whole's model (part) hierarchy; a meta model will be added to the
whole's details (meta) hierarchy.

\subsubsection{Whole Property}

\emph{required}

name=\texttt{'whole'}\\
abstraction=\texttt{'knowledge' \vline\ 'encapsulated'}
model=\texttt{whole model knowledge path}

This property specifies the knowledge model that the new part/ meta model will
be added to.

\subsubsection{Root Property}

\emph{required}

name=\texttt{'root'}\\
abstraction=\texttt{'knowledge' \vline\ 'encapsulated'}
model=\texttt{root model knowledge path}

This property specifies the knowledge model that will serve as the root.

\subsubsection{Style Property}

\emph{required}

name=\texttt{'style'}\\
abstraction=\texttt{'knowledge' \vline\ 'encapsulated'}
model=\texttt{'stream' \vline\ 'datagram' \vline\ 'raw'}

This is the style of socket communication.

\subsubsection{Commands Property}

\emph{required}

name=\texttt{'commands'}\\
abstraction=\texttt{'knowledge' \vline\ 'encapsulated'}
model=\texttt{commands model knowledge path}

This property specifies the knowledge model containing the commands that the
user interface should react to.

\subsubsection{Blocking Property}

\emph{required}

name=\texttt{'blocking'}\\
abstraction=\texttt{'boolean'}
model=\texttt{'true' \vline\ 'false'}

This property specifies whether the receive process should be blocking. If it
is, then application signals will not be processed while the receive operation
is waiting for some message to arrive. Only if a message is actually received,
the application will process it in form of a signal and then continue to wait.
