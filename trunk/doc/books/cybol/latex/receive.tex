%
% $RCSfile: receive.tex,v $
%
% Copyright (c) 2002-2007. Christian Heller. All rights reserved.
%
% Permission is granted to copy, distribute and/or modify this document
% under the terms of the GNU Free Documentation License, Version 1.1 or
% any later version published by the Free Software Foundation; with no
% Invariant Sections, with no Front-Cover Texts and with no Back-Cover
% Texts. A copy of the license is included in the section entitled
% "GNU Free Documentation License".
%
% http://www.cybop.net
% - Cybernetics Oriented Programming -
%
% Version: $Revision: 1.2 $ $Date: 2007-08-01 13:59:00 $ $Author: christian $
% Authors: Christian Heller <christian.heller@tuxtax.de>
%

\subsection{Receive}
\label{receive_heading}
\index{Receive}

This operation receives data from the given data source.

\subsubsection{Example}

\begin{scriptsize}
    \begin{verbatim}
<part name="receive_patients" channel="inline" abstraction="operation" model="receive">
    <property name="channel" channel="inline" abstraction="character" model="file"/>
    <property name="language" channel="inline" abstraction="character" model="xdt"/>
    <property name="message" channel="inline" abstraction="character" model="import/1.bde"/>
    <property name="model" channel="inline" abstraction="knowledge" model=".app.xdt"/>
</part>
    \end{verbatim}
\end{scriptsize}

\subsubsection{Channel Property}

The channel via which to receive the message.

\emph{required}

name=\texttt{'channel'}\\
abstraction=\texttt{'character'}\\
model=\texttt{'inline' \vline\ 'file'}

\subsubsection{Language Property}

This is the language (abstraction, type, structure) of the data received.

\emph{required}

name=\texttt{'language'}\\
abstraction=\texttt{'character'}\\
model=\texttt{
    % Primitive.
    'boolean'
    \vline\ 'character'
    \vline\ 'wide\_character'
    \vline\ 'integer'
    \vline\ 'unsigned\_long'
    \vline\ 'double'
    \vline\ 'fraction'
    \vline\ 'complex'
    \vline\ 'date\_time'
    \vline\ 'yyyy-mm-dd\_date\_time'
    % Knowledge.
    \vline\ 'cybol'
    % User Interface.
    \vline\ 'tui'
    \vline\ 'gui'
    \vline\ 'wui'
    % Audio.
    \vline\ 'ogg'
    \vline\ 'mp3'
    % Image.
    \vline\ 'jpeg'
    \vline\ 'png'
    \vline\ 'gif'
    \vline\ 'bmp'
    % Text.
    \vline\ 'cybop\_model\_diagram'
    \vline\ 'xdt'
    \vline\ 'hxp'
    \vline\ 'latex'
    \vline\ 'rtf'
    \vline\ 'sgml'
    \vline\ 'tex'
    \vline\ 'xhtml'
    % Video.
    \vline\ 'mpeg'
    \vline\ 'avi'
    \vline\ 'qt'
    % Compression.
    \vline\ 'tar'
    \vline\ 'tgz'
    \vline\ 'zip'
    \vline\ 'rar'
    % Application.
    \vline\ 'kwd'
    \vline\ 'odt'
    \vline\ 'sxw'
    % Network.
    \vline\ 'http'
    \vline\ 'https'
    \vline\ 'ftp'
}

\subsubsection{Message Property}

This is the source (knowledge template) from where to receive data.

\emph{required}

name=\texttt{'message'}\\
abstraction=\texttt{'character'}\\
model=\texttt{path to a file}

\subsubsection{Model Property}

This is the compound model to be filled with the data received.

\emph{required}

name=\texttt{'model'}\\
abstraction=\texttt{'character'}\\
model=\texttt{knowledge model path}

\subsubsection{Details Property}

This is the compound details to be filled with the data received.

\emph{required}

name=\texttt{'details'}\\
abstraction=\texttt{'character'}\\
model=\texttt{knowledge model path}

\subsubsection{Root Property}

This property specifies the knowledge model that will serve as the root.

\emph{required}

name=\texttt{'root'}\\
abstraction=\texttt{'knowledge' \vline\ 'encapsulated'}
model=\texttt{root model knowledge path}

\subsubsection{Style Property}

This is the style of socket communication.

\emph{required}

name=\texttt{'style'}\\
abstraction=\texttt{'knowledge' \vline\ 'encapsulated'}
model=\texttt{'stream' \vline\ 'datagram' \vline\ 'raw'}

\subsubsection{Commands Property}

This property specifies the knowledge model containing the commands that the
user interface should react to.

\emph{optional}, only if a user interface thread is to react to certain commands

name=\texttt{'commands'}\\
abstraction=\texttt{'knowledge' \vline\ 'encapsulated'}
model=\texttt{commands model knowledge path}

\subsubsection{Blocking Property}

This property specifies whether the receive process should be blocking. If it
is, then application signals will not be processed while the receive operation
is waiting for some message to arrive. Only if a message is actually received,
the application will process it in form of a signal and then continue to wait.

\emph{optional}

name=\texttt{'blocking'}\\
abstraction=\texttt{'boolean'}
model=\texttt{'true' \vline\ 'false'}
