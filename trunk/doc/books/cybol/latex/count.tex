%
% $RCSfile: count.tex,v $
%
% Copyright (c) 2002-2007. Christian Heller. All rights reserved.
%
% Permission is granted to copy, distribute and/or modify this document
% under the terms of the GNU Free Documentation License, Version 1.1 or
% any later version published by the Free Software Foundation; with no
% Invariant Sections, with no Front-Cover Texts and with no Back-Cover
% Texts. A copy of the license is included in the section entitled
% "GNU Free Documentation License".
%
% http://www.cybop.net
% - Cybernetics Oriented Programming -
%
% Version: $Revision: 1.2 $ $Date: 2007-08-01 13:59:00 $ $Author: christian $
% Authors: Christian Heller <christian.heller@tuxtax.de>
%

\subsection{Count}
\label{count_heading}
\index{Count}

This operation counts those parts of the given whole (compound), that match the
given filter criteria.

\subsubsection{Example}

\begin{scriptsize}
    \begin{verbatim}
<part name="count_addresses" channel="inline" abstraction="operation" model="count">
    <property name="compound" channel="inline" abstraction="encapsulated" model=".app.name"/>
    <property name="selection" channel="inline" abstraction="character" model="prefix"/>
    <property name="filter" channel="inline" abstraction="character" model="address"/>
    <property name="result" channel="inline" abstraction="knowledge" model=".app.result"/>
</part>
    \end{verbatim}
\end{scriptsize}

\subsubsection{Compound Property}

The compound whose parts are to be counted.

\emph{required}

name=\texttt{'compound'}\\
abstraction=\texttt{'knowledge' \vline\ 'encapsulated'}\\
model=\texttt{compound knowledge model}

\subsubsection{Selection Property}

This property selects the kind of filter to be applied for counting the
compound's parts.

\emph{required}

name=\texttt{'selection'}\\
abstraction=\texttt{'character'}\\
model=\texttt{'full' \vline\ 'prefix' \vline\ 'suffix' \vline\ 'part'}

\subsubsection{Filter Property}

The filter to compare the compound parts' names with. Only those parts will be
counted whose name (full, prefix, suffix, part) matches the filter string.

\emph{required}

name=\texttt{'filter'}\\
abstraction=\texttt{'character'}\\
model=\texttt{filter string}

\subsubsection{Result Property}

The knowledge model in which to store the result.

\emph{required}

name=\texttt{'result'}\\
abstraction=\texttt{'knowledge' \vline\ 'encapsulated'}\\
model=\texttt{knowledge model}
