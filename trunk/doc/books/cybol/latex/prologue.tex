%
% $RCSfile: prologue.tex,v $
%
% Copyright (c) 2002-2007. Christian Heller. All rights reserved.
%
% Permission is granted to copy, distribute and/or modify this document
% under the terms of the GNU Free Documentation License, Version 1.1 or
% any later version published by the Free Software Foundation; with no
% Invariant Sections, with no Front-Cover Texts and with no Back-Cover
% Texts. A copy of the license is included in the section entitled
% "GNU Free Documentation License".
%
% http://www.cybop.net
% - Cybernetics Oriented Programming -
%
% Version: $Revision: 1.1 $ $Date: 2007-07-17 20:02:36 $ $Author: christian $
% Authors: Christian Heller <christian.heller@tuxtax.de>
%

\section*{Prologue}
\label{prologue_heading}
%\addcontentsline{toc}{section}{Prologue}

After having had completed and published my book on the theory of
\emph{Cybernetics Oriented Programming} (CYBOP), the next logical step was to
closer inspect and introduce the features of the
\emph{Cybernetics Oriented Language} (CYBOL). This book tries to achieve this
by:

\begin{enumerate}
    \item explaining the CYBOL syntax
    \item giving extensive CYBOL code examples
    \item developing a small CYBOL application from scratch
    \item listing all CYBOL keywords currently interpretable by the CYBOI interpreter
\end{enumerate}

A next book dealing with the internal architecture of the
\emph{Cybernetics Oriented Interpreter} (CYBOI) might follow later on.

This is a growing document undergoing steady development. It is not and doesn't
claim to be free of errors nor to contain the only possible way for application
system development. So, if you find errors of whatever kind or have any helpful
ideas or constructive critics, then please contribute them to
\(<\)christian.heller@tuxtax.de\(>\) or to the CYBOP developers mailing list
\(<\)cybop-developers@lists.berlios.de\(>\)!
