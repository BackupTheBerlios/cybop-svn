%
% $RCSfile: stylistic_means.tex,v $
%
% Copyright (c) 2002-2007. Christian Heller. All rights reserved.
%
% Permission is granted to copy, distribute and/or modify this document
% under the terms of the GNU Free Documentation License, Version 1.1 or
% any later version published by the Free Software Foundation; with no
% Invariant Sections, with no Front-Cover Texts and with no Back-Cover
% Texts. A copy of the license is included in the section entitled
% "GNU Free Documentation License".
%
% http://www.cybop.net
% - Cybernetics Oriented Programming -
%
% Version: $Revision: 1.1 $ $Date: 2007-07-17 20:02:36 $ $Author: christian $
% Authors: Christian Heller <christian.heller@tuxtax.de>
%

\section*{Stylistic Means and Notation}
\label{stylistic_means_heading}
%\addcontentsline{toc}{section}{Stylistic Means and Notation}

The language of choice in this document is \emph{British English}, more
precisely known as \emph{Commonwealth English}. Exceptions are citations or
proper names like \emph{Unified Modeling Language}, stemming from
\emph{American English} sources. (In Oxford English, \emph{Modelling} would be
written with double letter \emph{l}). I am thinking about writing a
\emph{German} version of this document, but am not sure if it will be worth the
effort. If you as reader are interested in a translation, send me a short note!
The more emails I receive, the more convinced I will be.

% Als Autor dieser Arbeit konnte ich mir die Freiheit nehmen, viele Textpassagen
% nicht wortgetreu, sondern vielmehr sinngem�� aus der englischen Originalfassung
% zu �bersetzen. Fachbegriffe aus der Computerwelt wurden in Englisch belassen,
% um keine Zweideutigkeiten und redundante Abk�rzungen entstehen zu lassen.

Correctly, masculine \emph{and} feminine forms are used in a work. When
describing a patient's record, for example, one would write: \textit{his or her
record}. In order to improve readability, and exclusively because of this
reason, only masculine forms are used in this work.

Pieces of software source code are displayed in \texttt{Typewriter Typeface}.
Emphasised words are \emph{italicised}.

Footnotes are not used on purpose. In my opinion, they only interrupt the
flow-of-reading. Remarks are placed in context instead, sometimes enclosed in
parentheses.
