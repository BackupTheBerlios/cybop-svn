%
% $RCSfile: create.tex,v $
%
% Copyright (c) 2002-2007. Christian Heller. All rights reserved.
%
% Permission is granted to copy, distribute and/or modify this document
% under the terms of the GNU Free Documentation License, Version 1.1 or
% any later version published by the Free Software Foundation; with no
% Invariant Sections, with no Front-Cover Texts and with no Back-Cover
% Texts. A copy of the license is included in the section entitled
% "GNU Free Documentation License".
%
% http://www.cybop.net
% - Cybernetics Oriented Programming -
%
% Version: $Revision: 1.1 $ $Date: 2007-07-17 20:02:36 $ $Author: christian $
% Authors: Christian Heller <christian.heller@tuxtax.de>
%

\subsection{Create}
\label{create_heading}
\index{Create}

This operation creates a new knowledge model.

\subsubsection{Name Property}

\emph{required}

name=\texttt{'name'}\\
abstraction=\texttt{'character'}\\
model=\texttt{knowledge model name}

This is the name of the knowledge model to be created.

\subsubsection{Channel Property}

\emph{required}

name=\texttt{'channel'}\\
abstraction=\texttt{'character'}\\
model=\texttt{'inline' \vline\ 'file'}

This is the channel via which to load the knowledge template for creating a
knowledge model.

\subsubsection{Abstraction Property}

\emph{required}

name=\texttt{'abstraction'}\\
abstraction=\texttt{'character'}\\
model=\texttt{'boolean' \vline\ 'integer' \vline\ 'float' \vline\ 'character' \vline\ 'compound'}

This is the abstraction of the knowledge model to be created.

\subsubsection{Model Property}

\emph{required}

name=\texttt{'model'}\\
abstraction=\texttt{'character'}\\
model=\texttt{knowledge template path}

This is the knowledge template from which to create a knowledge model.

\subsubsection{Element Property}

\emph{required}

name=\texttt{'element'}\\
abstraction=\texttt{'character'}\\
model=\texttt{'part' \vline\ 'meta'}

This property specifies the kind of element (knowledge model) to be created.

\subsubsection{Whole Property}

\emph{required}

name=\texttt{'whole'}\\
abstraction=\texttt{'knowledge' \vline\ 'encapsulated'}\\
model=\texttt{whole knowledge model}

This property determines the compound to which to add the new knowledge model.
