%
% $RCSfile: get_bit.tex,v $
%
% Copyright (c) 2002-2007. Christian Heller. All rights reserved.
%
% Permission is granted to copy, distribute and/or modify this document
% under the terms of the GNU Free Documentation License, Version 1.1 or
% any later version published by the Free Software Foundation; with no
% Invariant Sections, with no Front-Cover Texts and with no Back-Cover
% Texts. A copy of the license is included in the section entitled
% "GNU Free Documentation License".
%
% http://www.cybop.net
% - Cybernetics Oriented Programming -
%
% Version: $Revision: 1.2 $ $Date: 2007-08-01 13:59:00 $ $Author: christian $
% Authors: Christian Heller <christian.heller@tuxtax.de>
%

\subsection{Get Bit}
\label{get_bit_heading}
\index{Get Bit}

This operation retrieves the bit at the given position, within the given number.

\subsubsection{Example}

\begin{scriptsize}
    \begin{verbatim}
<part name="get_some_bit" channel="inline" abstraction="operation" model="get_bit">
    <property name="number" channel="inline" abstraction="integer" model="127"/>
    <property name="position" channel="inline" abstraction="integer" model="3"/>
    <property name="result" channel="inline" abstraction="knowledge" model=".app.result"/>
</part>
    \end{verbatim}
\end{scriptsize}

\subsubsection{Number Property}

This property specifies the number of which one bit is to be retrieved.

\emph{required}

name=\texttt{'number'}\\
abstraction=\texttt{'integer' \vline\ 'knowledge' \vline\ 'encapsulated'}\\
model=\texttt{the actual number (or knowledge model)}

\subsubsection{Position Property}

The position of the bit whose value is to be retrieved.

\emph{required}

name=\texttt{'position'}\\
abstraction=\texttt{'integer' \vline\ 'knowledge' \vline\ 'encapsulated'}\\
model=\texttt{position of the bit}

\subsubsection{Result Property}

This is the result knowledge model in which to store the value of the bit that
was retrieved from the given number. It may be either \emph{true} or \emph{false}.

\emph{required}

name=\texttt{'result'}\\
abstraction=\texttt{'knowledge' \vline\ 'encapsulated'}\\
model=\texttt{result knowledge model}
