%
% $RCSfile: rotate.tex,v $
%
% Copyright (c) 2002-2007. Christian Heller. All rights reserved.
%
% Permission is granted to copy, distribute and/or modify this document
% under the terms of the GNU Free Documentation License, Version 1.1 or
% any later version published by the Free Software Foundation; with no
% Invariant Sections, with no Front-Cover Texts and with no Back-Cover
% Texts. A copy of the license is included in the section entitled
% "GNU Free Documentation License".
%
% http://www.cybop.net
% - Cybernetics Oriented Programming -
%
% Version: $Revision: 1.2 $ $Date: 2007-08-01 13:59:00 $ $Author: christian $
% Authors: Christian Heller <christian.heller@tuxtax.de>
%

\subsection{Rotate}
\label{rotate_heading}
\index{Rotate}

This operation rotates the bits within the number by the given positions.

\subsubsection{Example}

\begin{scriptsize}
    \begin{verbatim}
<part name="rotate_bits" channel="inline" abstraction="operation" model="rotate">
    <property name="number" channel="inline" abstraction="knowledge" model=".app.number"/>
    <property name="direction" channel="inline" abstraction="character" model="left"/>
    <property name="position" channel="inline" abstraction="integer" model="1"/>
    <property name="result" channel="inline" abstraction="knowledge" model=".app.result"/>
</part>
    \end{verbatim}
\end{scriptsize}

\subsubsection{Number Property}

This property specifies the number whose bits are to be rotated.

\emph{required}

name=\texttt{'number'}\\
abstraction=\texttt{'integer' \vline\ 'knowledge' \vline\ 'encapsulated'}\\
model=\texttt{the actual number (or knowledge model)}

\subsubsection{Direction Property}

The direction in which to rotate.

\emph{required}

name=\texttt{'direction'}\\
abstraction=\texttt{'character'}\\
model=\texttt{'left' \vline\ 'right'}

\subsubsection{Position Property}

The number of positions by which the bits of the given number are to be rotated.

\emph{required}

name=\texttt{'position'}\\
abstraction=\texttt{'integer' \vline\ 'knowledge' \vline\ 'encapsulated'}\\
model=\texttt{number of positions to rotate by}

\subsubsection{Result Property}

This is the result knowledge model in which to store the number whose bits were
rotated.

\emph{required}

name=\texttt{'result'}\\
abstraction=\texttt{'knowledge' \vline\ 'encapsulated'}\\
model=\texttt{result knowledge model}
