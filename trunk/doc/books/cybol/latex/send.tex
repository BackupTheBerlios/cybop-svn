%
% $RCSfile: send.tex,v $
%
% Copyright (c) 2002-2007. Christian Heller. All rights reserved.
%
% Permission is granted to copy, distribute and/or modify this document
% under the terms of the GNU Free Documentation License, Version 1.1 or
% any later version published by the Free Software Foundation; with no
% Invariant Sections, with no Front-Cover Texts and with no Back-Cover
% Texts. A copy of the license is included in the section entitled
% "GNU Free Documentation License".
%
% http://www.cybop.net
% - Cybernetics Oriented Programming -
%
% Version: $Revision: 1.1 $ $Date: 2007-07-17 20:02:36 $ $Author: christian $
% Authors: Christian Heller <christian.heller@tuxtax.de>
%

\subsection{Send}
\label{send_heading}
\index{Send}

This operation is able to send a message via textual, graphical or web user
interface, or to the file system or also as shell output directly.

\subsubsection{Channel Property}

\emph{required}

name=\texttt{'channel'}\\
abstraction=\texttt{'character'}\\
model=\texttt{'inline' \vline\ 'file'}

The channel via which to send the message.

\subsubsection{Language Property}

\emph{required}

name=\texttt{'language'}\\
abstraction=\texttt{'character'}\\
model=\texttt{'tui' \vline\ 'gui' \vline\ 'wui' \vline\ 'file' \vline\ 'shell' \vline\ etc.}

The language into which to encode the message before sending it.

\subsubsection{Mode Property}

\emph{required}

name=\texttt{'mode'}\\
abstraction=\texttt{'character'}\\
model=\texttt{'client' \vline\ 'server'}

The mode of communication.

\subsubsection{Namespace Property}

\emph{required}

name=\texttt{'namespace'}\\
abstraction=\texttt{'character'}\\
model=\texttt{'local' \vline\ 'inet' \vline\ 'inet6' \vline\ 'ns' \vline\ 'iso' \vline\ 'ccitt' \vline\ 'implink' \vline\ 'route'}

The namespace of the socket.

\subsubsection{Style Property}

\emph{required}

name=\texttt{'style'}\\
abstraction=\texttt{'character'}\\
model=\texttt{'stream' \vline\ 'datagram' \vline\ 'raw'}

The style of communication.

\subsubsection{Sender Property}

\emph{required}

name=\texttt{'sender'}\\
abstraction=\texttt{'character'}\\
model=\texttt{name of sending system}

The name of the system sending the message.

\subsubsection{Receiver Property}

\emph{required}

name=\texttt{'receiver'}\\
abstraction=\texttt{'character'}\\
model=\texttt{name of receiving system}

The name of the system receiving the message.

\subsubsection{Message Property}

\emph{required}

name=\texttt{'message'}\\
abstraction=\texttt{'knowledge' \vline\ 'encapsulated'}\\
model=\texttt{message knowledge model path}

The actual message to be sent to another system.

\subsubsection{Area Property}

\emph{optional}

name=\texttt{'area'}\\
abstraction=\texttt{'knowledge' \vline\ 'encapsulated'}\\
model=\texttt{knowledge path to part model to be repainted}

The user interface area to be repainted. It is normally just a part of the
whole user interface model. This property helps to speed up repainting while
avoiding user interface flickering.

\subsubsection{Clean Property}

\emph{optional}

name=\texttt{'clean'}\\
abstraction=\texttt{'boolean'}\\
model=\texttt{'true' \vline\ 'false'}

This property indicates whether or not to clear the screen before painting a
user interface.

\subsubsection{New Line Property}

\emph{optional}

name=\texttt{'new\_line'}\\
abstraction=\texttt{'boolean'}\\
model=\texttt{'true' \vline\ 'false'}

This property indicates whether or not to add a new line after having printed
the message on screen.
