%
% $RCSfile: syntax.tex,v $
%
% Copyright (c) 2002-2007. Christian Heller. All rights reserved.
%
% Permission is granted to copy, distribute and/or modify this document
% under the terms of the GNU Free Documentation License, Version 1.1 or
% any later version published by the Free Software Foundation; with no
% Invariant Sections, with no Front-Cover Texts and with no Back-Cover
% Texts. A copy of the license is included in the section entitled
% "GNU Free Documentation License".
%
% http://www.cybop.net
% - Cybernetics Oriented Programming -
%
% Version: $Revision: 1.1 $ $Date: 2007-08-01 13:59:00 $ $Author: christian $
% Authors: Christian Heller <christian.heller@tuxtax.de>
%

\section{Syntax}
\label{syntax_heading}
\index{Syntax}
\index{Grammar}
\index{Extensible Markup Language}
\index{XML}
\index{XML Tag}
\index{XML Attribute}

CYBOL's syntax (grammar with rules for combining terms and symbols) is based on
the well-known \emph{Extensible Markup Language} (XML) \cite{xml}. It has a
clear text representation, is flexible, extensible and easy to use.

To mention just two of the syntactical elements of XML, \emph{Tag} and
\emph{Attribute} are considered here shortly. Tags are special, arbitrary
keywords that have to be defined by the system working with an XML document.
Attributes keep additional information about the contents enclosed by two tags.
Two examples:

\begin{scriptsize}
    \begin{verbatim}
    <tag attribute="value">
        contents
    </tag>
    \end{verbatim}
\end{scriptsize}

\begin{scriptsize}
    \begin{verbatim}
    <tag attribute1="value" attribute2="content"/>
    \end{verbatim}
\end{scriptsize}

A CYBOL knowledge template (XML document) carries a name and can thus represent
a \emph{Discrete Item}. Being a \emph{Compound}, the template consists of parts
-- and, it can link to other templates (files) treated as its parts. That way,
a whole hierarchy can be formed. Tag attributes can keep additional information
about the linked parts. Most importantly, the hierarchical structure is based
on tags, which may be used to store meta data about a part.

CYBOL, finally, is XML \emph{plus} a defined set of tags, attributes and values
used to structure and link documents meaningfully.
