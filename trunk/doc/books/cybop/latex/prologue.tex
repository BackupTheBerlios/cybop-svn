%
% $RCSfile: prologue.tex,v $
%
% Copyright (C) 2002-2008. Christian Heller.
%
% Permission is granted to copy, distribute and/or modify this document
% under the terms of the GNU Free Documentation License, Version 1.1 or
% any later version published by the Free Software Foundation; with no
% Invariant Sections, with no Front-Cover Texts and with no Back-Cover
% Texts. A copy of the license is included in the section entitled
% "GNU Free Documentation License".
%
% http://www.cybop.net
% - Cybernetics Oriented Programming -
%
% http://www.resmedicinae.org
% - Information in Medicine -
%
% Version: $Revision: 1.1 $ $Date: 2008-08-19 20:41:08 $ $Author: christian $
% Authors: Christian Heller <christian.heller@tuxtax.de>
%

\section*{Prologue}
\label{prologue_heading}
%\addcontentsline{toc}{section}{Prologue}

To me, basically, there are two ways to deal with a scientific subject:

\begin{enumerate}
    \item The deepened investigation on a special area aiming to find
        completely new phenomenons
    \item The systematic subsumption of multiple known aspects of one or many
        disciplines aiming to find new cross-correlations and ideas
\end{enumerate}

Both approaches may lead to new theories, methods and concepts. And both may use
laboratory trials to find and prove their theories. This work follows the second
approach. The idea behind is, simply spoken, to steal ideas from nature and
various fields of science, and to apply them to software design.

\emph{Laboratory Trials} are what \emph{Coding} is in informatics -- experiment
and proof of operability, at the same time. Some information scientists have
the opinion that coding weren't \emph{scientific} enough and not necessary to
create new theories or to achieve good results. I doubt this. In my opinion,
there are things that can only be found when actually implementing ideas in a
computer language. And in the end, a theory is worth much more when having been
proven in practice. This document contains proven ideas that were growing in my
mind over the last few years, while dealing with topics such as:

\begin{itemize}
    \item[-] Structured- and Procedural Programming
    \item[-] Object Oriented Programming
    \item[-] Design Patterns and Frameworks
    \item[-] Component Based Design and Agents
    \item[-] Ontology Structured Domain Knowledge
    \item[-] Document- and User Interface Markup
    \item[-] Persistence Mechanisms
    \item[-] System Communication
    \item[-] Operating System Concepts
\end{itemize}

The usage of typical buzzwords could not quite be avoided in this work, yet do
I hope that the ideas and results are nevertheless explained straightforward and
well enough to be really useful to some other developers out there.

This document claims to be an \emph{Academic Paper}. To all practitioners who do
not want to read it for that reason, I would like to point out that each and
every concept in it arose from practice, that is coding. Like most developers,
I started up with only a few lines of code in one Java class, later extended to
more classes, a whole framework and so on. Whenever I stumbled over difficulties,
I thought through and improved my current design by applying patterns recommended
by several software development Gurus. It was only when I realised that even
those concepts were not sufficient, that I made up my own. They are entitled
\emph{Cybernetics Oriented Programming} (CYBOP), because most ideas behind them
stem from nature.

Finally, this document has become my thesis, written to earn a doctorate
(Dr.-Ing./ PhD) in Informatics/ Software Engineering. You may wonder why I
release it under the \emph{Free Documentation License} (FDL). Well, I'm a full
supporter of the idea of \emph{Free Knowledge}, \emph{Free Software}, a
\emph{Society free of Patents} which are only hindering its development. There
are three reasons that have contributed to my decision:

\begin{enumerate}
    \item Hope to get helpful \emph{Feedback} from readers
    \item Trust in the scientific \emph{Fairness} of colleagues, worldwide, to
        properly reference this document even though it is licensed under the FDL
    \item Wish to contribute to the open source movement \emph{now} (and not in
        some years when the document might reach a more stable version), to
        speed up its successful development
\end{enumerate}

This is a growing document undergoing steady development. It is not and doesn't
claim to be free of errors nor to contain the only possible way for application
system development. So, if you find errors of whatever kind or have any helpful
ideas or constructive critics, then please contribute them to
\(<\)christian.heller@tuxtax.de\(>\) or to the CYBOP developers mailing list
\(<\)cybop-developers@lists.berlios.de\(>\)!
