%
% $RCSfile: functional_programming.tex,v $
%
% Copyright (C) 2002-2008. Christian Heller.
%
% Permission is granted to copy, distribute and/or modify this document
% under the terms of the GNU Free Documentation License, Version 1.1 or
% any later version published by the Free Software Foundation; with no
% Invariant Sections, with no Front-Cover Texts and with no Back-Cover
% Texts. A copy of the license is included in the section entitled
% "GNU Free Documentation License".
%
% http://www.cybop.net
% - Cybernetics Oriented Programming -
%
% http://www.resmedicinae.org
% - Information in Medicine -
%
% Version: $Revision: 1.1 $ $Date: 2008-08-19 20:41:06 $ $Author: christian $
% Authors: Christian Heller <christian.heller@tuxtax.de>
%

\subsection{Functional Programming}
\label{functional_programming_heading}
\index{Functional Programming}
\index{Lisp}
\index{Automatic Storage Management}
\index{Integrated Development Environment}
\index{Prolog}
\index{Pascal}
\index{C}
\index{Special Purpose Language}
\index{SPL}
\index{General Purpose Language}
\index{GPL}
\index{Imperative Language}
\index{Functional Language}
\index{Haskell}
\index{Side Effect}
\index{SML}
\index{Scheme}
\index{Association of Lisp Users}

Many languages such as \emph{Lisp} and its relatives cannot be characterised
cleanly as system programming language or scripting language; they are situated
somewhere between. Concepts like \emph{Interpretation} and \emph{Dynamic Typing},
now common in scripting languages, stem from Lisp \cite{ousterhout1998}. Others like
\emph{Automatic Storage Management} and \emph{Integrated Development Environments},
now used in both scripting- and system programming languages, were introduced by
Lisp as well. Peter Norvig writes in \cite{norvig}:

\begin{quote}
    There is a myth that Lisp (and Prolog) are \emph{Special Purpose Languages}
    (SPL), while languages such as Pascal and C are \emph{General Purpose} (GPL).
    Actually, just the reverse is true. Pascal and C are special-purpose languages
    for manipulating the registers and memory of a \emph{von Neumann}-style computer.
    The majority of their syntax is devoted to arithmetic and Boolean expressions,
    and while they provide some facilities for forming data structures, they have
    poor mechanisms for procedural abstraction or control abstraction. In addition,
    they are designed for the \emph{state-oriented} style of programming: computing
    a result by changing the value of variables through assignment statements.
\end{quote}

The \emph{Frequently Asked Questions} (FAQ) edited by Graham Hutton \cite{hutton}
distinguish between \emph{Imperative Languages} and \emph{Functional Languages}.
System programming languages as introduced in previous sections belong to the
first group. To calculate the sum of the integers from 1 to 10, for example,
they would probably use a simple loop and repeatedly update the values held in
an accumulator variable \emph{total} and a counter variable \emph{i}:

\begin{scriptsize}
    \begin{verbatim}
    int total = 0;
    for (int i = 1; i <= 10; ++i) {
        total += i;
    }
    \end{verbatim}
\end{scriptsize}

A functional language like \emph{Haskell} would express the same program
\emph{without} any variable updates, by evaluating an expression, as shown
below. Variable updates, that is \textit{computational effects caused by
expression evaluation that persist after the evaluation is completed}
\cite{hutton} are called \emph{Side Effects}.

\begin{scriptsize}
    \begin{verbatim}
    sum [1..10]
    \end{verbatim}
\end{scriptsize}

The following two examples \cite{hutton} show the same program in two other
functional languages, namely \emph{SML} and \emph{Scheme}:

\begin{scriptsize}
    \begin{verbatim}
    let fun sum i tot = if i = 0 then tot else sum (i - 1) (tot + i)
    in sum 10 0
    end
    \end{verbatim}
\end{scriptsize}

\begin{scriptsize}
    \begin{verbatim}
    (define sum
    (lambda (from total)
        (if (= 0 from)
            total
            (sum (- from 1) (+ total from)))))
    (sum 10 0)
    \end{verbatim}
\end{scriptsize}

The \emph{Association of Lisp Users} \cite{commonlisp} points out the absence
of side effects and explains \emph{Functional Programming} as follows:

\begin{quote}
    Functional programming describes all computer operations as mathematical
    functions on inputs. Typically, a function can be created and returned from
    other functions as first-class data. This function object may then be passed
    as input to other functions, perhaps be composed with other functions, and
    eventually, applied to inputs to produce a value. Objects can be defined in
    terms of functions that encapsulate certain data, and operations on objects
    can be defined by functions encapsulating the objects. Purely functional
    languages do not have assignment, as all side-effecting can be defined in
    terms of functions that encapsulate the changed data. Procedural languages
    essentially perform everything as side-effects to data structures. A purely
    procedural language would have no functions, but might have subroutines of
    no arguments that returned no values, and performed certain assignments and
    other operations based on the data it found stored in the system.
\end{quote}

The interpreter described in chapter \ref{cybernetics_oriented_interpreter_heading}
manipulates \emph{all} data (knowledge) as if it would be one huge side effect.
Data (knowledge models) are not bundled with-, but kept completely outside any
functions/ procedures. Only references to these data are handed over as
parameters.
