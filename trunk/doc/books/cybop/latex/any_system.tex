%
% $RCSfile: any_system.tex,v $
%
% Copyright (C) 2002-2008. Christian Heller.
%
% Permission is granted to copy, distribute and/or modify this document
% under the terms of the GNU Free Documentation License, Version 1.1 or
% any later version published by the Free Software Foundation; with no
% Invariant Sections, with no Front-Cover Texts and with no Back-Cover
% Texts. A copy of the license is included in the section entitled
% "GNU Free Documentation License".
%
% http://www.cybop.net
% - Cybernetics Oriented Programming -
%
% http://www.resmedicinae.org
% - Information in Medicine -
%
% Version: $Revision: 1.1 $ $Date: 2008-08-19 20:41:05 $ $Author: christian $
% Authors: Christian Heller <christian.heller@tuxtax.de>
%

\subsubsection{Any System?}
\label{any_system_heading}
\index{CYBOL for Any System}

While creating the CYBOP knowledge concepts and implementing them in the CYBOL
language, one main aim was to make that language as flexible as possible, in
order to be usable for the development of a variety of systems. It seems that
CYBOL is indeed applicable for developing standard business applications in
very different domains.

It might also be usable for creating desktop environments such as the
\emph{K Desktop Environment} (KDE) \cite{kde} -- and even configuration parts
of an \emph{Operating System} (OS) (the information stored in the files of the
\emph{/etc} directory and the \emph{/usr/src/linux/.config} file, when taking
the Linux kernel as example) could possibly be encoded in CYBOL. The same
counts for the configuration files of applications residing in the \emph{/etc}
directory of systems that follow the \emph{Filesystem Hierarchy Standard}
(FHS). That is, both applications and their configuration files may be written
in the same format: CYBOL.

Yet to limit the scope of this work, proof for these assumptions cannot be
given here. As well, the applicability of CYBOL for programming \emph{Real Time}
(RT) systems was not investigated yet. A slightly more extensive example,
however, is given in chapter \ref{res_medicinae_heading}, which describes the
\emph{Res Medicinae} prototype -- a yet very incomplete
\emph{Electronic Health Record} (EHR) application.
