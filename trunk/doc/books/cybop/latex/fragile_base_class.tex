%
% $RCSfile: fragile_base_class.tex,v $
%
% Copyright (C) 2002-2008. Christian Heller.
%
% Permission is granted to copy, distribute and/or modify this document
% under the terms of the GNU Free Documentation License, Version 1.1 or
% any later version published by the Free Software Foundation; with no
% Invariant Sections, with no Front-Cover Texts and with no Back-Cover
% Texts. A copy of the license is included in the section entitled
% "GNU Free Documentation License".
%
% http://www.cybop.net
% - Cybernetics Oriented Programming -
%
% http://www.resmedicinae.org
% - Information in Medicine -
%
% Version: $Revision: 1.1 $ $Date: 2008-08-19 20:41:06 $ $Author: christian $
% Authors: Christian Heller <christian.heller@tuxtax.de>
%

\subsubsection{Fragile Base Class}
\label{fragile_base_class_heading}
\index{Fragile Base Class (Problem)}
\index{Implementation Inheritance}
\index{Subclass}
\index{Superclass}
\index{Class Hierarchy}
\index{Cascade of Change}
\index{Reusability}
\index{Flexibility}
\index{Cyclic Method Dependencies}
\index{Self-calling Assumptions of a Class Method}
\index{Base Class Access}
\index{Modifier Invariant Function}

Despite the possible code reduction through class inheritance, there are some
negative effects that hinder just this code reduction and reuse. John K.
Ousterhout writes in his article \cite{ousterhout1998}:

\begin{quote}
    Implementation inheritance, where one class borrows code that was written
    for another class, is a bad idea that makes software harder to manage and
    reuse. It binds the implementations of classes together so that neither
    class can be understood without the other: a subclass cannot be understood
    without knowing how the inherited methods are implemented in its superclass,
    and a superclass cannot be understood without knowing how its methods are
    inherited in subclasses. In a complex class hierarchy, no individual class
    can be understood without understanding all the other classes in the
    hierarchy. Even worse, a class cannot be separated from its hierarchy for
    reuse. Multiple inheritance makes these problems even worse. Implementation
    inheritance causes the same intertwining and brittleness that have been
    observed when goto statements are overused. As a result, object-oriented
    systems often suffer from complexity and lack of reuse.
\end{quote}

Unwanted dependencies caused simply by the usage of inheritance are called
\emph{Fragile Base Class Problem} \cite[section \emph{Layers}; p. 48]{buschmann}.
The source code changes resulting from base class manipulation are also called
\emph{Cascade of Change} \cite[Vorwort]{gruhn}. They are just the opposite of
what inheritance was actually intended to be for: \emph{Reusability}. Leonid
Mikhajlov and Emil Sekerinski \cite{mikhajlov} write:

\begin{quote}
    This problem occurs in open object-oriented systems employing code
    inheritance as an implementation reuse mechanism. System developers unaware
    of extensions to the system developed by its users may produce a seemingly
    acceptable revision of a base class which may damage its extensions.
    The fragile base class problem becomes apparent during maintenance of open
    object-oriented systems, but requires consideration during design.
\end{quote}

They identify the following \emph{Restrictions} \cite{mikhajlov} disciplining the
code inheritance mechanism, thus avoiding the \emph{Fragile Base Class Problem},
but on the cost of general \emph{Flexibility}:

\begin{itemize}
    \item[-] \emph{No cycles:} A base class revision and a modifier should not
        jointly introduce new cyclic method dependencies.
    \item[-] \emph{No revision self-calling assumptions:} Revision class methods
        should not make any additional assumptions about the behaviour of the
        other methods of itself. Only the behaviour described in the base class
        may be taken into consideration.
    \item[-] \emph{No base class down-calling assumptions:} Methods of a modifier
        should disregard the fact that base class self-calls can get redirected
        to the modifier itself. In this case bodies of the corresponding methods
        in the base class should be considered instead, as if there were no
        dynamic binding.
    \item[-] \emph{No direct access to base class state:} An extension class
        may not access the state of its base class directly, but only through
        calling base class methods.
    \item[-] \emph{No modifier invariant function:} A modifier should not bind
        values of its instance variables with values of the intended base class
        instance variables to generate an invariant.
\end{itemize}

In order to remain highly flexible and to avoid the fragile base class problem,
the language described in chapter \ref{cybernetics_oriented_language_heading}
does not use inheritance, although it could be extended to do so. In this case,
of course, its interpreter (chapter \ref{cybernetics_oriented_interpreter_heading})
would have to be adapted as well.
