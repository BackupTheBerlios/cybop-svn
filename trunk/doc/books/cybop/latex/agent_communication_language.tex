%
% $RCSfile: agent_communication_language.tex,v $
%
% Copyright (C) 2002-2008. Christian Heller.
%
% Permission is granted to copy, distribute and/or modify this document
% under the terms of the GNU Free Documentation License, Version 1.1 or
% any later version published by the Free Software Foundation; with no
% Invariant Sections, with no Front-Cover Texts and with no Back-Cover
% Texts. A copy of the license is included in the section entitled
% "GNU Free Documentation License".
%
% http://www.cybop.net
% - Cybernetics Oriented Programming -
%
% http://www.resmedicinae.org
% - Information in Medicine -
%
% Version: $Revision: 1.1 $ $Date: 2008-08-19 20:41:05 $ $Author: christian $
% Authors: Christian Heller <christian.heller@tuxtax.de>
%

\subsection{Agent Communication Language}
\label{agent_communication_language_heading}
\index{Agent Communication Language}
\index{ACL}
\index{Artificial Intelligence}
\index{AI}
\index{Agent Oriented Programming}
\index{AGOP}

A whole palette of languages was suggested within the scientific field of
\emph{Artificial Intelligence} (AI). \emph{Agent Oriented Programming} (AGOP)
(section \ref{agent_oriented_programming_heading}), for example, uses
representation formats like the ones described following, for the knowledge
bases and communication of its agent systems. That is why such formats are
often labeled \emph{Agent Communication Language} (ACL).

%
% $RCSfile: knowledge_interchange_format.tex,v $
%
% Copyright (C) 2002-2008. Christian Heller.
%
% Permission is granted to copy, distribute and/or modify this document
% under the terms of the GNU Free Documentation License, Version 1.1 or
% any later version published by the Free Software Foundation; with no
% Invariant Sections, with no Front-Cover Texts and with no Back-Cover
% Texts. A copy of the license is included in the section entitled
% "GNU Free Documentation License".
%
% http://www.cybop.net
% - Cybernetics Oriented Programming -
%
% http://www.resmedicinae.org
% - Information in Medicine -
%
% Version: $Revision: 1.1 $ $Date: 2008-08-19 20:41:07 $ $Author: christian $
% Authors: Christian Heller <christian.heller@tuxtax.de>
%

\subsubsection{Knowledge Interchange Format}
\label{knowledge_interchange_format_heading}
\index{Knowledge Interchange Format}
\index{KIF}
\index{PostScript}
\index{PS}

The \emph{Knowledge Interchange Format} (KIF), as described in \cite{kif}, is:

\begin{itemize}
    \item[-] a language designed for use in the interchange of knowledge among
        disparate computer systems
    \item[-] not intended as a primary language for interaction with human users
    \item[-] not intended as an internal representation for knowledge within
        computer systems
    \item[-] in its purpose, roughly analogous to \emph{PostScript} (PS)
        (section \ref{page_description_language_heading})
    \item[-] not as efficient as a specialised representation for knowledge,
        but more general and programmer-readable
\end{itemize}

The idea behind KIF is that \cite{kif}: \textit{a computer system reads a
knowledge base in KIF, (and) converts the data into its own internal form
(pointer structures, arrays, etc.). All computation is done using these
internal forms. When the computer system needs to communicate with another
computer system, it maps its internal data structures into KIF.} KIF's design
is characterised by three features:

\begin{enumerate}
    \item \emph{Declarative Semantics:} independent from specific interpreters,
        as opposed to e.g. \emph{Prolog}
    \item \emph{Logically Comprehensive:} may express arbitrary logical
        sentences, as opposed to \emph{SQL} or \emph{Prolog}
    \item \emph{Meta Knowledge:} permits the introduction of new knowledge
        representation constructs, without changing the language
\end{enumerate}

The following syntax example \cite{kif} shows a logical term involving the
\emph{if} operator. If the object constant \emph{a} denotes a number, then the
term denotes the absolute value of that number:

\begin{scriptsize}
    \begin{verbatim}
    (if (> a 0) a (- a))
    \end{verbatim}
\end{scriptsize}

The language introduced in chapter \ref{cybernetics_oriented_language_heading}
may not only serve as interchange format between systems, but also for the
definition of user interfaces, workflows and domain models, altogether. It
treats state- and logic models as separate, composable concepts (chapter
\ref{state_and_logic_heading}), which KIF does not. Further, it provides the
means to express meta knowledge.

%
% $RCSfile: knowledge_query_and_manipulation_language.tex,v $
%
% Copyright (C) 2002-2008. Christian Heller.
%
% Permission is granted to copy, distribute and/or modify this document
% under the terms of the GNU Free Documentation License, Version 1.1 or
% any later version published by the Free Software Foundation; with no
% Invariant Sections, with no Front-Cover Texts and with no Back-Cover
% Texts. A copy of the license is included in the section entitled
% "GNU Free Documentation License".
%
% http://www.cybop.net
% - Cybernetics Oriented Programming -
%
% http://www.resmedicinae.org
% - Information in Medicine -
%
% Version: $Revision: 1.1 $ $Date: 2008-08-19 20:41:07 $ $Author: christian $
% Authors: Christian Heller <christian.heller@tuxtax.de>
%

\subsubsection{Knowledge Query and Manipulation Language}
\label{knowledge_query_and_manipulation_language_heading}
\index{Knowledge Query and Manipulation Language}
\index{KQML}
\index{Common Lisp}
\index{CL}

The \emph{Knowledge Query and Manipulation Language} (KQML) \cite{kqml} is a:
\textit{language and associated protocol by which intelligent software agents
can communicate to share information and knowledge}, as Tim Finin et al.
\cite{finin} write. Its syntax were based on a balanced parenthesis list,
because initial implementations had been done in Common Lisp (CL)
\cite{commonlisp}. After Finin et al., the initial element of the list were the
\emph{Performative} and the remaining elements were the performative's
\emph{Arguments} as keyword/ value pairs. The Free Wikipedia Encyclopedia
\cite{wikipedia} explains:

\begin{quote}
    The KQML message format and protocol can be used to interact with an
    intelligent system, either by an application program, or by another
    intelligent system. KQML's \emph{Performatives} are operations that agents
    perform on each other's \emph{Knowledge} and \emph{Goal} stores.
    Higher-level interactions such as \emph{Contract Nets} and
    \emph{Negotiation} are built using these. KQML's
    \emph{Communication Facilitators} coordinate the interactions of other
    agents to support \emph{Knowledge Sharing}.
\end{quote}

An example message representing a query about the price of a share of IBM stock
might be encoded as \cite{finin}:

\begin{scriptsize}
    \begin{verbatim}
    (ask-one
    :content (PRICE IBM ?price)
    :receiver stock-server
    :language LPROLOG
    :ontology NYSE-TICKS)
    \end{verbatim}
\end{scriptsize}

System communication and its elements like \emph{Sender}, \emph{Receiver},
\emph{Language} or \emph{Message Content} will be further investigated in
chapter \ref{state_and_logic_heading}. The new language introduced in chapter
\ref{cybernetics_oriented_language_heading} defines communication operations
(logic) accompanied by properties (meta information), much the same way
performatives have arguments. Also, that new language may not only be used to
encode knowledge for communication, but to represent knowledge of arbitrary
domains. By combining pre-defined, primitive operations, it may be used to
create more complex (higher-level) algorithms.

%
% $RCSfile: darpa_agent_markup_language.tex,v $
%
% Copyright (C) 2002-2008. Christian Heller.
%
% Permission is granted to copy, distribute and/or modify this document
% under the terms of the GNU Free Documentation License, Version 1.1 or
% any later version published by the Free Software Foundation; with no
% Invariant Sections, with no Front-Cover Texts and with no Back-Cover
% Texts. A copy of the license is included in the section entitled
% "GNU Free Documentation License".
%
% http://www.cybop.net
% - Cybernetics Oriented Programming -
%
% http://www.resmedicinae.org
% - Information in Medicine -
%
% Version: $Revision: 1.1 $ $Date: 2008-08-19 20:41:06 $ $Author: christian $
% Authors: Christian Heller <christian.heller@tuxtax.de>
%

\subsubsection{DARPA Agent Markup Language / Ontology Inference Layer}
\label{darpa_agent_markup_language_heading}
\index{DARPA Agent Markup Language}
\index{DAML}
\index{Ontology Inference Layer}
\index{OIL}
\index{DAML+OIL}
\index{Semantic Web}
\index{Extensible Markup Language}
\index{XML}
\index{Resource Description Framework}
\index{RDF}
\index{OWL}

The DAML+OIL language resulted from a combination of the DAML and OIL languages.
The \emph{DARPA Agent Markup Language} (DAML) \cite{damloil} was created in a
project run by the \emph{Defense Advanced Research Projects Agency} (DARPA) of the
\emph{United States of America} (USA); the \emph{Ontology Inference Layer} (OIL)
was created within the \emph{Information Science Technologies} (IST) program of
the \emph{European Union} (EU) \cite{rdfowlrelease}. Both projects aimed at
developing a language and tools to facilitate the concept of the
\emph{Semantic Web} (section \ref{semantic_web_heading}).

At the beginning of the project stood the realisation that: \textit{The use of
ontologies (section \ref{conceptual_network_heading}) provides a very powerful
way to describe objects and their relationships to other objects.} The DAML+OIL
language, being developed as an extension to the \emph{Extensible Markup Language}
(XML) (section \ref{extensible_markup_language_heading}) and the
\emph{Resource Description Framework} (RDF) (section \ref{semantic_web_heading}),
therefore provided a \cite{rdf} rich set of constructs with which to create
ontologies and to markup information so that it becomes machine-readable and
understandable. Much of the work in DAML and OIL has now been incorporated into
OWL (section \ref{web_ontology_language_heading}).

Chapter \ref{cybernetics_oriented_language_heading} will introduce a language
that is based on XML, too.

%\input{general_ontological_language}
%\emph{General Ontological Language} (GOL) \cite{degen}
%Dokument dazu liegt lokal in /tmp:
%explanation of KIF etc.; Sowa is referenced; see end of paper for relation to informatics
