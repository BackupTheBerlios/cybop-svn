%
% $RCSfile: lexical_scheme.tex,v $
%
% Copyright (C) 2002-2008. Christian Heller.
%
% Permission is granted to copy, distribute and/or modify this document
% under the terms of the GNU Free Documentation License, Version 1.1 or
% any later version published by the Free Software Foundation; with no
% Invariant Sections, with no Front-Cover Texts and with no Back-Cover
% Texts. A copy of the license is included in the section entitled
% "GNU Free Documentation License".
%
% http://www.cybop.net
% - Cybernetics Oriented Programming -
%
% http://www.resmedicinae.org
% - Information in Medicine -
%
% Version: $Revision: 1.1 $ $Date: 2008-08-19 20:41:07 $ $Author: christian $
% Authors: Christian Heller <christian.heller@tuxtax.de>
%

\subsubsection{Lexical Scheme}
\label{lexical_scheme_heading}
\index{Lexical Scheme}
\index{Lexical Tech}
\index{Unified Medical Language System}
\index{UMLS}

A \emph{Lexical Technique} is one that helps compare phrases based on what they
appear to say -- on which words appear, in which order, and in what grammatical
constructs -- rather than on what they might or might not actually mean. Such
techniques can provide a powerful (but not 100\% accurate) method for mapping
between phrases in existing schemes, or between such phrases and the text found
in papers, the \emph{World Wide Web} (WWW) or other electronic resources.

One example of a lexical scheme is the \emph{Unified Medical Language System}
(UMLS).

Where lexical techniques break is when the language gets more \emph{slippery},
that is ambiguities may occur. Humans might interpret such results correctly,
but automated decision support systems would fail. Rogers \cite{rogers}
concludes that: \textit{As an input for autonomous machine processing
applications such as decision support, the outputs of natural language
processing tools remain unsuitable.}
