%
% $RCSfile: terminology_systems.tex,v $
%
% Copyright (C) 2002-2008. Christian Heller.
%
% Permission is granted to copy, distribute and/or modify this document
% under the terms of the GNU Free Documentation License, Version 1.1 or
% any later version published by the Free Software Foundation; with no
% Invariant Sections, with no Front-Cover Texts and with no Back-Cover
% Texts. A copy of the license is included in the section entitled
% "GNU Free Documentation License".
%
% http://www.cybop.net
% - Cybernetics Oriented Programming -
%
% http://www.resmedicinae.org
% - Information in Medicine -
%
% Version: $Revision: 1.1 $ $Date: 2008-08-19 20:41:09 $ $Author: christian $
% Authors: Christian Heller <christian.heller@tuxtax.de>
%

\subsection{Terminology Systems}
\label{terminology_systems_heading}
\index{Medical Terminology Systems}
\index{Enumerative Scheme}
\index{Compositional Scheme}
\index{Lexical Scheme}

Besides defining the differences between a \emph{Lexicon} (list of pure words)
and \emph{Terminology} (also containing phrases), the latter sometimes called
\emph{Vocabulary}, section \ref{terminology_heading} introduced tree-like
\emph{Hierarchies} as one way to organise such sets of words or terms. Three
concrete schemes for organising terminologies were described in section
\ref{schemes_heading}: \emph{Enumerative}, \emph{Compositional} and
\emph{Lexical}. Controversial opinions about terminologies exist. Thomas Beale
wrote in \cite[December 2003]{openhealth}:

\begin{quote}
    \ldots\ trying to standardise the whole of medicine \ldots\ is a fruitless
    enterprise. Sam Heard has said this many times in presentations in
    Australia, and when he first started saying it, was amazed not to be
    stoned publicly; in fact many people have come to this conclusion through
    their own hard work, but aren't comfortable with saying it, since it goes
    against current orthodoxy (embodied in things like SNOMED CT).
\end{quote}

Nevertheless, terminologies \emph{are} a topic of research and sometimes used
in practice, as the example of ICD (see below) shows. This section therefore
briefly describes some medical terminologies and, by referring to Jeremy Rogers
\cite{rogers}, tries to assign them to one of the before-mentioned schemes.

%
% $RCSfile: icd.tex,v $
%
% Copyright (C) 2002-2008. Christian Heller.
%
% Permission is granted to copy, distribute and/or modify this document
% under the terms of the GNU Free Documentation License, Version 1.1 or
% any later version published by the Free Software Foundation; with no
% Invariant Sections, with no Front-Cover Texts and with no Back-Cover
% Texts. A copy of the license is included in the section entitled
% "GNU Free Documentation License".
%
% http://www.cybop.net
% - Cybernetics Oriented Programming -
%
% http://www.resmedicinae.org
% - Information in Medicine -
%
% Version: $Revision: 1.1 $ $Date: 2008-08-19 20:41:07 $ $Author: christian $
% Authors: Christian Heller <christian.heller@tuxtax.de>
%

\subsubsection{ICD}
\label{icd_heading}
\index{International Classification of Diseases}
\index{ICD}
\index{World Health Organisation}
\index{WHO}

The \emph{International Classification of Diseases} (ICD): \textit{has become
the international standard diagnostic classification for all general
epidemiological and many health management purposes \ldots\ It is used to
classify diseases and other health problems recorded on many types of health
and vital records including death certificates and hospital records.} \cite{icd}

Scheme: enumerative\\
Maintainer: World Health Organisation (WHO)

%
% $RCSfile: opcs.tex,v $
%
% Copyright (C) 2002-2008. Christian Heller.
%
% Permission is granted to copy, distribute and/or modify this document
% under the terms of the GNU Free Documentation License, Version 1.1 or
% any later version published by the Free Software Foundation; with no
% Invariant Sections, with no Front-Cover Texts and with no Back-Cover
% Texts. A copy of the license is included in the section entitled
% "GNU Free Documentation License".
%
% http://www.cybop.net
% - Cybernetics Oriented Programming -
%
% http://www.resmedicinae.org
% - Information in Medicine -
%
% Version: $Revision: 1.1 $ $Date: 2008-08-19 20:41:08 $ $Author: christian $
% Authors: Christian Heller <christian.heller@tuxtax.de>
%

\subsubsection{OPCS}
\label{opcs_heading}
\index{Office of Population Censuses and Surveys Classification of Surgical Operations and Procedures}
\index{OPCS}
\index{National Health Service Information Authority}
\index{NHSIA}

The \emph{Office of Population Censuses and Surveys Classification of Surgical
Operations and Procedures} (OPCS) is a: \textit{statistical classification of
diseases and surgical procedures, respectively.} It allows the: \textit{logical
translation of clinical statements into codes in a way that facilitates the
retrieval of data in a consistent manner and comparative analysis of aggregated
datasets compiled from multiple sources.} \cite{opcs}

Scheme: enumerative\\
Maintainer: \emph{National Health Service Information Authority} (NHSIA)

%
% $RCSfile: read.tex,v $
%
% Copyright (C) 2002-2008. Christian Heller.
%
% Permission is granted to copy, distribute and/or modify this document
% under the terms of the GNU Free Documentation License, Version 1.1 or
% any later version published by the Free Software Foundation; with no
% Invariant Sections, with no Front-Cover Texts and with no Back-Cover
% Texts. A copy of the license is included in the section entitled
% "GNU Free Documentation License".
%
% http://www.cybop.net
% - Cybernetics Oriented Programming -
%
% http://www.resmedicinae.org
% - Information in Medicine -
%
% Version: $Revision: 1.1 $ $Date: 2008-08-19 20:41:08 $ $Author: christian $
% Authors: Christian Heller <christian.heller@tuxtax.de>
%

\subsubsection{READ}
\label{read_heading}
\index{Read Codes}
\index{READ}
\index{Clinical Terms Version 3}
\index{CTV3}
\index{ICD-10}
\index{OPCS-4}
\index{National Health Service Information Authority}
\index{NHSIA}

The \emph{Read Codes} (READ), as their older name \emph{Clinical Terms Version 3}
(CTV3) says, are a: \textit{list of terms describing the care and treatment of
patients}. They: \textit{cover a wide range of topics in categories such as
signs and symptoms, treatments and therapies, investigations, occupations,
diagnoses and drugs and appliances.} Further, they: \textit{provide cross maps
to both ICD-10 and OPCS-4 classification codes.} \cite{read}

Scheme: enumerative\\
Maintainer: \emph{United Kingdom} (UK)
\emph{National Health Service Information Authority} (NHSIA)

%
% $RCSfile: loinc.tex,v $
%
% Copyright (C) 2002-2008. Christian Heller.
%
% Permission is granted to copy, distribute and/or modify this document
% under the terms of the GNU Free Documentation License, Version 1.1 or
% any later version published by the Free Software Foundation; with no
% Invariant Sections, with no Front-Cover Texts and with no Back-Cover
% Texts. A copy of the license is included in the section entitled
% "GNU Free Documentation License".
%
% http://www.cybop.net
% - Cybernetics Oriented Programming -
%
% http://www.resmedicinae.org
% - Information in Medicine -
%
% Version: $Revision: 1.1 $ $Date: 2008-08-19 20:41:07 $ $Author: christian $
% Authors: Christian Heller <christian.heller@tuxtax.de>
%

\subsubsection{LOINC}
\label{loinc_heading}
\index{Logical Observation Identifiers, Names and Codes}
\index{LOINC}
\index{SNOMED CT}
\index{Regenstrief Institute}

The \emph{Logical Observation Identifiers, Names and Codes} (LOINC) is a
database whose purpose is: \textit{to facilitate the exchange and pooling of
results \ldots\ for clinical care, outcomes management, and research.} Its
codes are: \textit{universal identifiers for laboratory- and other clinical
observations.} \cite{loinc} After \cite{rogers}, it were now closely allied to
SNOMED CT (see later section).

Scheme: hybrid enumerative-compositional\\
Maintainer: \emph{United States} (US) \emph{Regenstrief Institute}

%
% $RCSfile: icnp.tex,v $
%
% Copyright (C) 2002-2008. Christian Heller.
%
% Permission is granted to copy, distribute and/or modify this document
% under the terms of the GNU Free Documentation License, Version 1.1 or
% any later version published by the Free Software Foundation; with no
% Invariant Sections, with no Front-Cover Texts and with no Back-Cover
% Texts. A copy of the license is included in the section entitled
% "GNU Free Documentation License".
%
% http://www.cybop.net
% - Cybernetics Oriented Programming -
%
% http://www.resmedicinae.org
% - Information in Medicine -
%
% Version: $Revision: 1.1 $ $Date: 2008-08-19 20:41:07 $ $Author: christian $
% Authors: Christian Heller <christian.heller@tuxtax.de>
%

\subsubsection{ICNP}
\label{icnp_heading}
\index{International Classification for Nursing Practice}
\index{ICNP}
\index{International Council of Nurses}
\index{ICN}

The \emph{International Classification for Nursing Practice} (ICNP) is a:
\textit{combinatorial terminology for nursing practice that facilitates
crossmapping of local terms and existing vocabularies and classifications.} It
wants to: \textit{establish a common language for describing nursing practice
in order to improve communication among nurses, and between nurses and others.}
\cite{icnp}

Scheme: hybrid enumerative-compositional\\
Maintainer: \emph{International Council of Nurses} (ICN)

%
% $RCSfile: snomed_ct.tex,v $
%
% Copyright (C) 2002-2008. Christian Heller.
%
% Permission is granted to copy, distribute and/or modify this document
% under the terms of the GNU Free Documentation License, Version 1.1 or
% any later version published by the Free Software Foundation; with no
% Invariant Sections, with no Front-Cover Texts and with no Back-Cover
% Texts. A copy of the license is included in the section entitled
% "GNU Free Documentation License".
%
% http://www.cybop.net
% - Cybernetics Oriented Programming -
%
% http://www.resmedicinae.org
% - Information in Medicine -
%
% Version: $Revision: 1.1 $ $Date: 2008-08-19 20:41:08 $ $Author: christian $
% Authors: Christian Heller <christian.heller@tuxtax.de>
%

\subsubsection{SNOMED CT}
\label{snomed_ct_heading}
\index{Systematized Nomenclature of Medicine}
\index{SNOMED}
\index{SNOMED Clinical Terms}
\index{SNOMED CT}
\index{SNOMED Reference Terminology}
\index{SNOMED RT}
\index{Read Codes}
\index{READ}
\index{ICD-9-CM}
\index{ICD-10}
\index{ICD-03}
\index{OPCS-4}
\index{LOINC}
\index{SNOMED International}
\index{College of American Pathologists}
\index{CAP}

The \emph{Systematized Nomenclature of Medicine} (SNOMED) \emph{Clinical Terms}
(SNOMED CT) is: \textit{a dynamic, scientifically validated clinical health care
terminology and infrastructure that makes health care knowledge more usable and
accessible.} The SNOMED CT core terminology: \textit{contains over 364,400
health care concepts with unique meanings and formal logic-based definitions
organized into hierarchies. As of January 2005, the fully populated table with
unique descriptions for each concept contains more than 984,000 descriptions.
Approximately 1.45 million semantic relationships exist to enable reliability
and consistency of data retrieval.} \cite{snomed}

SNOMED CT was created by combining the content and structure of the SNOMED
\emph{Reference Terminology} (SNOMED RT) with the United Kingdom's (UK)
\emph{Read Codes} (READ) clinical terms. Meanwhile, mappings and integrations
for further standards exist, e.g. for several ICD versions (ICD-9-CM, ICD-10,
ICD-O3), OPCS-4 and LOINC.

Scheme: hybrid enumerative-compositional\\
Maintainer: \emph{SNOMED International} and \emph{College of American Pathologists} (CAP)

==

--- Klaus Veil <klaus@veil.net.au> wrote:

> Nandalal,
> �
> The concerns about closedness and excessive cost were indeed the main
> drivers that finally convinced CAP in the USA that an "open" international
> Standards Development Organisation was the only viable way forward. �CAP
> have committed to transfer the SNOMED IP to the new SDO.
> �
> Klaus
> 
> � _____ �
> 
> From: openhealth@yahoogroups.com [mailto:openhealth@yahoogroups.com] On
> Behalf Of Nandalal Gunaratne
> Sent: Monday, 5 February 2007 20:06
> To: openhealth@yahoogroups.com
> Subject: RE: [openhealth] OSHCA Conference Topics
> 
> 
> 
> Klaus,
> 
> Most of asia use ICD and other WHO standards. SNOMED
> is considered too expensive and too closed. I hope the
> new initiative would change that.
> 
> Nandalal

%
% $RCSfile: odyssee.tex,v $
%
% Copyright (C) 2002-2008. Christian Heller.
%
% Permission is granted to copy, distribute and/or modify this document
% under the terms of the GNU Free Documentation License, Version 1.1 or
% any later version published by the Free Software Foundation; with no
% Invariant Sections, with no Front-Cover Texts and with no Back-Cover
% Texts. A copy of the license is included in the section entitled
% "GNU Free Documentation License".
%
% http://www.cybop.net
% - Cybernetics Oriented Programming -
%
% http://www.resmedicinae.org
% - Information in Medicine -
%
% Version: $Revision: 1.1 $ $Date: 2008-08-19 20:41:07 $ $Author: christian $
% Authors: Christian Heller <christian.heller@tuxtax.de>
%

\subsubsection{Odyssee}
\label{odyssee_heading}
\index{Odyssee}
\index{Logiciel Nautilus}

The \emph{Odyssee} open source project \cite{nautilus} contains a terminology
(\emph{Lexique}) of more than 35,000 (French) terms, each with a code, at the
core of its system. Additionally, it contains a \emph{Semantic Network} of
links between terms of the Lexique, to give sense. Links can be \emph{is a},
\emph{belongs to} or \emph{has unit}. Philippe Ameline writes
\cite{openehrtechnical}:

\begin{quotation}
    In Odyssee, we describe all that we can with trees. If we compare the
    Lexique with medical vocabulary, trees are sentences made of its words.
    Each node of a tree is an object with fields like the Lexique's code,
    complement (to store numbers or external codes), degree of evidence (from
    0=no to 100=certain). Trees can also contain free text sentences \ldots

    In Odyssee, each and every structured document is a tree; you just have to
    look at the Lexique term at its root to know what it is. The whole patient
    record can even be seen as a huge tree with (the) term \emph{Patient} as
    root. Trees can be shown \emph{as is} or, for report generation, be
    translated to natural langage sentences.
\end{quotation}

Scheme: compositional\\
Maintainer: Odyssee \emph{Non-Profit Organisation} (NPO), \emph{Logiciel Nautilus}

%
% $RCSfile: open_galen.tex,v $
%
% Copyright (C) 2002-2008. Christian Heller.
%
% Permission is granted to copy, distribute and/or modify this document
% under the terms of the GNU Free Documentation License, Version 1.1 or
% any later version published by the Free Software Foundation; with no
% Invariant Sections, with no Front-Cover Texts and with no Back-Cover
% Texts. A copy of the license is included in the section entitled
% "GNU Free Documentation License".
%
% http://www.cybop.net
% - Cybernetics Oriented Programming -
%
% http://www.resmedicinae.org
% - Information in Medicine -
%
% Version: $Revision: 1.1 $ $Date: 2008-08-19 20:41:08 $ $Author: christian $
% Authors: Christian Heller <christian.heller@tuxtax.de>
%

\subsubsection{OpenGALEN}
\label{open_galen_heading}
\index{Generalised Architecture for Languages, Encyclopedias and Nomenclatures in Med.}
\index{GALEN}
\index{GALEN Common Reference Model}
\index{GALEN CRM}
\index{GALEN Representation and Integration Language}
\index{GRAIL}
\index{OpenGALEN}
\index{Synergy on the Extranet}
\index{SynEx}
\index{Synapses}

The \emph{Generalised Architecture for Languages, Encyclopaedias and
Nomenclatures in Medicine} (GALEN) is trying to construct a:
\textit{semantically sound model of clinical terminology} -- the GALEN
\emph{Common Reference Model} (CRM). The formal rules (representation scheme)
for manipulating its concepts are provided by the \emph{GALEN Representation
and Integration Language} (GRAIL).

The original GALEN project was sponsored by the \emph{European Union} (EU) and
open-sourced and renamed into \emph{OpenGALEN}, in 1999. It later continued as
part of the \emph{Synergy on the Extranet} (SynEx) project \cite{synex}, which
arose from the \emph{Synapses} project aiming at implementing a federated
healthcare record server.

Scheme: compositional\\
Maintainer: OpenGALEN \emph{Non-Profit Organisation} (NPO)

%
% $RCSfile: umls.tex,v $
%
% Copyright (C) 2002-2008. Christian Heller.
%
% Permission is granted to copy, distribute and/or modify this document
% under the terms of the GNU Free Documentation License, Version 1.1 or
% any later version published by the Free Software Foundation; with no
% Invariant Sections, with no Front-Cover Texts and with no Back-Cover
% Texts. A copy of the license is included in the section entitled
% "GNU Free Documentation License".
%
% http://www.cybop.net
% - Cybernetics Oriented Programming -
%
% http://www.resmedicinae.org
% - Information in Medicine -
%
% Version: $Revision: 1.1 $ $Date: 2008-08-19 20:41:09 $ $Author: christian $
% Authors: Christian Heller <christian.heller@tuxtax.de>
%

\subsubsection{UMLS}
\label{umls_heading}
\index{Unified Medical Language System}
\index{UMLS}
\index{UMLS Metathesaurus}
\index{Medical Subject Headings}
\index{MeSH}
\index{UMLS Semantic Network}
\index{UMLS Specialist Lexicon}
\index{UMLS Knowledge Source Server}
\index{UMLSKS}
\index{National Library of Medicine}
\index{NLM}

The \emph{Unified Medical Language System} (UMLS) consists of knowledge sources
(databases) and associated software tools (programs). By design, the knowledge
sources are multi-purpose, that is: \textit{they are not optimised for particular
applications, but can be applied in systems that perform a range of functions
involving one or more types of information, e.g. patient records, scientific
literature, guidelines, public health data.} \cite{umls} There are three UMLS
knowledge sources:

\begin{itemize}
    \item[-] \emph{Metathesaurus:} a very large, multi-purpose, and
        multi-lingual vocabulary database containing information about
        biomedical and health-related concepts, their various names, and the
        relationships among them; its source vocabularies are many different
        thesauri, classifications, code sets, and lists of controlled terms, of
        which it cross-references over 79 \cite{rogers}, often by deriving from
        lexical analysis of the terms; a core thesaurus are the
        \emph{Medical Subject Headings} (MeSH)
    \item[-] \emph{Semantic Network:} a consistent categorisation of all
        concepts represented in the UMLS Metathesaurus (currently 135 semantic
        types) and a set of useful relationships between these (currently 54
        semantic links)
    \item[-] \emph{Specialist Lexicon:} a general English lexicon that includes
        many biomedical terms; records the syntactic, morphological, and
        orthographic information for each word or term
\end{itemize}

To its associated software belongs the \emph{UMLS Knowledge Source Server}
(UMLSKS), which is: \textit{a set of Web-based interactive tools and a
programmer interface to allow users and developers access to the UMLS knowledge
sources, including the vocabularies within the Metathesaurus.} \cite{umls}

Scheme: lexical\\
Maintainer: \emph{United States} (US) \emph{National Library of Medicine} (NLM)

%
% $RCSfile: others.tex,v $
%
% Copyright (C) 2002-2008. Christian Heller.
%
% Permission is granted to copy, distribute and/or modify this document
% under the terms of the GNU Free Documentation License, Version 1.1 or
% any later version published by the Free Software Foundation; with no
% Invariant Sections, with no Front-Cover Texts and with no Back-Cover
% Texts. A copy of the license is included in the section entitled
% "GNU Free Documentation License".
%
% http://www.cybop.net
% - Cybernetics Oriented Programming -
%
% http://www.resmedicinae.org
% - Information in Medicine -
%
% Version: $Revision: 1.1 $ $Date: 2008-08-19 20:41:08 $ $Author: christian $
% Authors: Christian Heller <christian.heller@tuxtax.de>
%

\subsubsection{Others}
\label{others_heading}
\index{Oxford Medical Information System}
\index{OXMIS}
\index{International Classification of Health Problems in Primary Care}
\index{ICHPPC}
\index{International Classification of Primary Care}
\index{ICPC}
\index{International Classification of Functioning, Disability and Health}
\index{ICF}
\index{Universal Medical Device Nomenclature System}
\index{UMDNS}

Numerous other terminology systems exist, only some of which are listed below:

\begin{itemize}
    \item[-] \emph{Oxford Medical Information System} (OXMIS) Dictionary
    \item[-] \emph{International Classification of Health Problems in Primary Care} (ICHPPC)
    \item[-] \emph{International Classification of Primary Care} (ICPC)
    \item[-] \emph{International Classification of Functioning, Disability and Health} (ICF)
    \item[-] \emph{Universal Medical Device Nomenclature System} (UMDNS)
\end{itemize}

