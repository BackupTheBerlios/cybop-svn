%
% $RCSfile: edifact.tex,v $
%
% Copyright (C) 2002-2008. Christian Heller.
%
% Permission is granted to copy, distribute and/or modify this document
% under the terms of the GNU Free Documentation License, Version 1.1 or
% any later version published by the Free Software Foundation; with no
% Invariant Sections, with no Front-Cover Texts and with no Back-Cover
% Texts. A copy of the license is included in the section entitled
% "GNU Free Documentation License".
%
% http://www.cybop.net
% - Cybernetics Oriented Programming -
%
% http://www.resmedicinae.org
% - Information in Medicine -
%
% Version: $Revision: 1.1 $ $Date: 2008-08-19 20:41:06 $ $Author: christian $
% Authors: Christian Heller <christian.heller@tuxtax.de>
%

\subsubsection{EDIFACT}
\label{edifact_heading}
\index{Electronic Data Interchange for Administration, Commerce and Transport}
\index{EDIFACT}
\index{United Nations Standard}
\index{UN Standard}
\index{European Board of EDI Standardisation}
\index{EBES}
\index{EBES Expert Group 9}
\index{EEG9}
\index{ISO 9735}

The \emph{Electronic Data Interchange for Administration, Commerce and Transport}
(EDIFACT) \cite{edifact} is a standard maintained by committees of the
\emph{United Nations} (UN). It was defined to ease the electronic exchange of
general business data, but is widely used for the transmission of healthcare
information between organisations, too. \cite{kalra1998}

The \emph{European Board of EDI Standardisation} (EBES) participates in the
development and distribution of EDIFACT standards. For the healthcare sector,
this task falls to the \emph{EBES Expert Group} (EEG) 9. It has specified many
message formats, for instance for referral letters or electronic prescriptions.
Although some countries like Denmark, Norway or Austria make use of these
standards, they have not gained wider currency. \cite{atgexpertsreport}

Messages in form of the EDIFACT protocol are based on syntax elements which are
described in another standard, the \emph{ISO 9735}. Their data elements are
contained in a well-defined collection of segments, in a well-defined sequence.
\cite{edifactory}
