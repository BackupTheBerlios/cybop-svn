%
% $RCSfile: base_and_meta_level.tex,v $
%
% Copyright (C) 2002-2008. Christian Heller.
%
% Permission is granted to copy, distribute and/or modify this document
% under the terms of the GNU Free Documentation License, Version 1.1 or
% any later version published by the Free Software Foundation; with no
% Invariant Sections, with no Front-Cover Texts and with no Back-Cover
% Texts. A copy of the license is included in the section entitled
% "GNU Free Documentation License".
%
% http://www.cybop.net
% - Cybernetics Oriented Programming -
%
% http://www.resmedicinae.org
% - Information in Medicine -
%
% Version: $Revision: 1.1 $ $Date: 2008-08-19 20:41:05 $ $Author: christian $
% Authors: Christian Heller <christian.heller@tuxtax.de>
%

\subsection{Base- and Meta Level}
\label{base_and_meta_level_heading}
\index{Base Level}
\index{Meta Level}
\index{Reflective Technique}
\index{System- and Application Functionality}
\index{Bidirectional Dependency}

Reflective techniques as described in section \ref{reflection_heading} make use
of one so-called \emph{Base Level} and one or more \emph{Meta Levels}. The
reason for splitting a system's architecture in this way is the hope to be able
to move rather general \emph{System Functionality} into a meta level, while
leaving domain-specific \emph{Application Functionality} in the base level.
(Well, in his book \emph{Analysis Patterns -- Reusable Object Models}
\cite{fowler1997}, Fowler used meta levels to model general classes containing
not exclusively system- but also domain-specific functionality.) The conflicts
a design decision of that kind can bring with were described in section
\ref{broken_type_system_heading}, which -- above all -- criticised the
bidirectional dependencies.

However, what the proposition of reflective software patterns shows, is the
existence of a wish among software developers, to separate general system- from
more specific application functionality. And, as was shown in section
\ref{virtual_and_real_world_heading}, nature does exactly that. Yet while
reflective mechanisms use the same implementation techniques for system- as
well as for application-specific functionality, nature always treats passive
knowledge strictly separate from active system control (section
\ref{virtual_and_real_world_heading}). Bidirectional dependencies do not exist
between the both.
