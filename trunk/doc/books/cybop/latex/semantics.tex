%
% $RCSfile: semantics.tex,v $
%
% Copyright (C) 2002-2008. Christian Heller.
%
% Permission is granted to copy, distribute and/or modify this document
% under the terms of the GNU Free Documentation License, Version 1.1 or
% any later version published by the Free Software Foundation; with no
% Invariant Sections, with no Front-Cover Texts and with no Back-Cover
% Texts. A copy of the license is included in the section entitled
% "GNU Free Documentation License".
%
% http://www.cybop.net
% - Cybernetics Oriented Programming -
%
% http://www.resmedicinae.org
% - Information in Medicine -
%
% Version: $Revision: 1.1 $ $Date: 2008-08-19 20:41:08 $ $Author: christian $
% Authors: Christian Heller <christian.heller@tuxtax.de>
%

\subsection{Semantics}
\label{semantics_heading}
\index{Semantics of a Language}
\index{CYBOL Semantics}
\index{State Knowledge Modelling}
\index{Logic Knowledge Modelling}
\index{Extensible Markup Language}
\index{XML}
\index{XML Tag}
\index{XML Attribute}

The meaning expressed by terms and sentences is their \emph{Semantics}
\cite{duden}.

CYBOL files can be used to model either \emph{State-} or \emph{Logic Knowledge}
(chapter \ref{state_and_logic_heading}). In both cases, the \emph{same} syntax
(document structure) with \emph{identical} vocabulary (XML tags and -attributes)
is applied. It is the attribute \emph{Values} that make a difference in meaning.

The double hierarchy proposed by CYBOP's knowledge schema (section
\ref{knowledge_representation_heading}) is put into static CYBOL knowledge
templates, by using XML \emph{Attributes} for representing the whole-part
hierarchy, and XML \emph{Tags} for representing the additional meta information
that a whole model keeps about its part models.

%
% $RCSfile: attributes.tex,v $
%
% Copyright (C) 2002-2008. Christian Heller.
%
% Permission is granted to copy, distribute and/or modify this document
% under the terms of the GNU Free Documentation License, Version 1.1 or
% any later version published by the Free Software Foundation; with no
% Invariant Sections, with no Front-Cover Texts and with no Back-Cover
% Texts. A copy of the license is included in the section entitled
% "GNU Free Documentation License".
%
% http://www.cybop.net
% - Cybernetics Oriented Programming -
%
% http://www.resmedicinae.org
% - Information in Medicine -
%
% Version: $Revision: 1.1 $ $Date: 2008-08-19 20:41:05 $ $Author: christian $
% Authors: Christian Heller <christian.heller@tuxtax.de>
%

\subsubsection{Attributes}
\label{attributes_heading}
\index{CYBOL Attributes}
\index{CYBOL 'name' Attribute}
\index{CYBOL 'channel' Attribute}
\index{CYBOL 'abstraction' Attribute}
\index{CYBOL 'model' Attribute}

Normally, an XML \emph{Attribute} keeps meta information about the contents of
an XML \emph{Tag}. In CYBOL, however, three attributes keep meta information
about a fourth attribute. The attributes, altogether, are:

\begin{itemize}
    \item[-] name
    \item[-] channel
    \item[-] abstraction
    \item[-] model
\end{itemize}

The attribute of greatest interest is \emph{model}. It contains a model either
directly, or a path to one. The \emph{channel} attribute indicates whether the
\emph{model} attribute's value is to be read from:

\begin{itemize}
    \item[-] inline
    \item[-] file
    \item[-] ftp
    \item[-] http
\end{itemize}

The \emph{abstraction} attribute specifies how to interpret the model pointed
to by the \emph{model} attribute's value. A model may be given in formats like
for example:

\newpage

\begin{itemize}
    \item[-] cybol (a state- or logic compound model encoded in CYBOL format)
    \item[-] operation (a primitive logic model)
    \item[-] string
    \item[-] double
    \item[-] integer
    \item[-] boolean
\end{itemize}

The \emph{name} attribute, finally, provides the referenced model with a unique
identifier.

While the interpretation of the \emph{model} attribute's value depends on the
\emph{channel-} and \emph{abstraction} attributes, the other three attributes
(\emph{name}, \emph{channel}, \emph{abstraction}) themselves always get
interpreted as character string.

%
% $RCSfile: tags.tex,v $
%
% Copyright (C) 2002-2008. Christian Heller.
%
% Permission is granted to copy, distribute and/or modify this document
% under the terms of the GNU Free Documentation License, Version 1.1 or
% any later version published by the Free Software Foundation; with no
% Invariant Sections, with no Front-Cover Texts and with no Back-Cover
% Texts. A copy of the license is included in the section entitled
% "GNU Free Documentation License".
%
% http://www.cybop.net
% - Cybernetics Oriented Programming -
%
% http://www.resmedicinae.org
% - Information in Medicine -
%
% Version: $Revision: 1.1 $ $Date: 2008-08-19 20:41:09 $ $Author: christian $
% Authors: Christian Heller <christian.heller@tuxtax.de>
%

\subsubsection{Tags}
\label{tags_heading}
\index{CYBOL Tags}
\index{CYBOL 'model' Tag}
\index{CYBOL 'part' Tag}
\index{CYBOL 'property' Tag}
\index{CYBOL 'constraint' Tag}

There are many kinds of meta information besides the above-mentioned
attributes, that may be known about a model. These are given in special XML
tags called \emph{property} and \emph{constraint}. As defined in section
\ref{vocabulary_heading}, a CYBOL knowledge template may use four kinds of XML
tags:

\begin{itemize}
    \item[-] model
    \item[-] part
    \item[-] property
    \item[-] constraint
\end{itemize}

The \emph{model} tag appears just once. It is the root node which makes a CYBOL
knowledge template a valid XML document.

Of actual interest are the \emph{part} tags. They identify the models that the
\emph{whole} model described by the CYBOL knowledge template consists of.

A \emph{whole} model may know a lot more about its \emph{part} models, than is
given by a part model's XML attributes. A spatial state model may know about
the \emph{position} and \emph{size} of its parts, in space. A temporal model
(such as a workflow) may have to know about the \emph{position} of its parts in
time, in order to be able to execute them in the correct order. Further, the
temporal model needs to know about the \emph{input/output} (i/o) state models
which are to be manipulated by the corresponding logic operation (part model).
The number of parts within a whole (compound) model may be limited. And so on.
These additional information are provided by \emph{property} tags whose number
is conceptually unlimited.

Not only parts need additional meta information; properties may need such
information, too. The position or size as properties of a part may have to be
constrained to certain values, such as a \emph{minimum} or \emph{maximum}. The
values of the \emph{colour} property of a part model may have to be chosen out
of a pre-defined set called \emph{choice}. Information of that kind are stated
in \emph{constraint} tags.

Since the number of possible meta information implementable in CYBOL is already
quite large and steadily growing, as the development continues, this section
cannot list them all. At a future point in time, a more-or-less complete CYBOL
specification document may be found at the CYBOP project's website \cite{cybop}.

