%
% $RCSfile: kurzfassung.tex,v $
%
% Copyright (C) 2002-2008. Christian Heller.
%
% Permission is granted to copy, distribute and/or modify this document
% under the terms of the GNU Free Documentation License, Version 1.1 or
% any later version published by the Free Software Foundation; with no
% Invariant Sections, with no Front-Cover Texts and with no Back-Cover
% Texts. A copy of the license is included in the section entitled
% "GNU Free Documentation License".
%
% http://www.cybop.net
% - Cybernetics Oriented Programming -
%
% http://www.resmedicinae.org
% - Information in Medicine -
%
% Version: $Revision: 1.1 $ $Date: 2008-08-19 20:41:07 $ $Author: christian $
% Authors: Christian Heller <christian.heller@tuxtax.de>
%

\section{Kurzfassung}
\label{kurzfassung_heading}

Informationen und Wissen gewinnen in der heutigen Gesellschaft zunehmend an
Bedeutung. Software als eine Form der Abstraktion von Wissen spielt dabei eine
entscheidende Rolle. Die Hauptschwierigkeit beim Erstellen von Software besteht
in der \"Uberbr\"uckung der Diskrepanz zwischen menschlichen Denkkonzepten und
den Erfordernissen einer maschinellen Darstellung.

Herk\"ommliche Paradigmen des Software-Designs haben ihr Abstraktionsniveau in
der Vergangenheit erheblich steigern k\"onnen, weisen allerdings immer noch
etliche Schw\"achen auf. Diese Arbeit vergleicht und verbessert traditionelle
Konzepte der Software-Entwicklung durch Denkans\"atze anderer Wissenschaftsgebiete
bzw. Ph\"anomene der Natur -- daher ihre Bezeichnung: \emph{Kybernetik-orientiert}.

Im Ergebnis dieser interdisziplin\"aren Herangehensweise stehen dreierlei
Empfehlungen: (1) eine strikte Trennung aktiver Systemkontroll-Software von purem,
passivem Wissen; (2) die Verwendung eines Schemas zur Wissens-Repr\"asentation,
welches auf einer Doppel-Hierarchie zur kombinierten Darstellung von
Teil-Ganzes-Beziehungen und Meta-Informationen beruht; (3) eine getrennte
Behandlung jener Wissens-Modelle, die einen Zustand verk\"orpern, von solchen,
die Logik enthalten.

Zur Darstellung von Wissen gem\"a\ss{} dem vorgeschlagenen Schema wurde eine
XML-basierende Sprache namens \emph{CYBOL} definiert und ein dazugeh\"origer
Interpreter genannt \emph{CYBOI} entwickelt. Trotz ihrer Schlichtheit ist CYBOL
in der Lage, Wissen komplett zu beschreiben. Als Prototyp zum Nachweis der
prinzipiellen Funktionsf\"ahigkeit des CYBOP-Ansatzes wurde \emph{Res Medicinae},
ein \emph{Free-/ Open Source Software} Projekt, ins Leben gerufen.

CYBOP bietet eine neue Theorie des Programmierens, die durchaus vielversprechend
zu sein scheint, da sie nicht nur Mankos bestehender Paradigmen beseitigt,
sondern vor allem flexiblere, zukunftssichere Anwendungen erm\"oglicht. Durch
das leicht zu verstehende Hierarchie-Konzept werden Fach-Experten in die Lage
versetzt, selbst aktiv an der Anwendungs-Entwicklung mitzuwirken. Die in
klassischen Software-Entwicklungs-Prozessen zu findende Implementierungsphase
entf\"allt.

\subsubsection{Schlagworte}

Kybernetik-Orientierte Programmierung (CYBOP),
Wissens-Schema,
Ontologie,
XML-basierende Programmierung,
Freie und Quell-offene Software (FOSS),
Res Medicinae,
Elektronische Kranken-Akte (EHR),
Software Muster,
Programmier-Paradigma,
Software Entwicklungs-Prozess (SEP)

\subsubsection{Informationen}

Autor: Dipl.-Ing. \authormacro, Technische Universit\"at Ilmenau\\
Gutachter 1: Prof. Dr.-Ing. habil. Ilka Philippow (Mentorin), Technische Universit\"at Ilmenau\\
Gutachter 2: Prof. Dr.-Ing. habil. Dietrich Reschke, Technische Universit\"at Ilmenau\\
Gutachter 3: Mark Lycett (PhD), Brunel University, Gro\ss{}britannien\\
%Inventarisierungsnummer: 12345 / 2006 / 1;
Einreichung: 2005-12-12; Verteidigung: 2006-10-04
