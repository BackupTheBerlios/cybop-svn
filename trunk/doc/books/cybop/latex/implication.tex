%
% $RCSfile: implication.tex,v $
%
% Copyright (C) 2002-2008. Christian Heller.
%
% Permission is granted to copy, distribute and/or modify this document
% under the terms of the GNU Free Documentation License, Version 1.1 or
% any later version published by the Free Software Foundation; with no
% Invariant Sections, with no Front-Cover Texts and with no Back-Cover
% Texts. A copy of the license is included in the section entitled
% "GNU Free Documentation License".
%
% http://www.cybop.net
% - Cybernetics Oriented Programming -
%
% http://www.resmedicinae.org
% - Information in Medicine -
%
% Version: $Revision: 1.1 $ $Date: 2008-08-19 20:41:07 $ $Author: christian $
% Authors: Christian Heller <christian.heller@tuxtax.de>
%

\subsection{Implication}
\label{implication_heading}
\index{Medical Informatics Standards in Res Medicinae}

The number of standards for medical informatics is huge. The fields covered by
these standards are manifold. Popular standardisation efforts dealing with the
EHR structure are \emph{Open EHR} and \emph{CEN 13606}.

The borders to messaging and communication standards are blurred. Although
\emph{HL7}'s focus lies on message exchange, it created data structures in form
of its \emph{RIM} framework, too; a newer result for document exchange is their
\emph{CDA} specification. The former two standards (\emph{CEN 13606} and
\emph{Open EHR}), on the other hand, focus on the EHR structure but offer a
communication format as well; it is called \emph{Transaction} or
\emph{Composition}, respectively. Beale concludes in \cite{openhealth}:
\textit{\ldots\ all efforts have converged independently on at least one solid
concept -- the unit of change and committal in the EHR.}

OMG's \emph{HDTF} defines interfaces for the exchange of messages, which are
grouped into special services. Some national efforts have defined their own
data exchange formats, like the \emph{xDT} standard (to become \emph{SCIPHOX})
in Germany. Yet other standards recommendations for electronic data interchange
in medicine are \emph{EDIFACT}, worked out by the UN, and \emph{HXP}, defined
by a number of medical OSS projects.

To what concerns the field of medical terminology, there exist longer-lasting
efforts like \emph{ICD}, \emph{LOINC}, \emph{SNOMED CT}, \emph{OpenGALEN} or
\emph{UMLS}. Depending on their scheme of organisation, they may be grouped
into the three categories: \emph{enumerative}, \emph{compositional} and
\emph{lexical}. A lot of time and money has been invested into them, yet only
recently, their results have been adopted by increasingly more systems. Good
acceptance and popularity was reached for the \emph{ICD} codes classification
system.

Other standards for related fields exist, among them being \emph{DICOM} for
clinical imaging and -device communication, \emph{NCPDP} for the transmission
of pharmacy data, \emph{CLSI} for clinical laboratory testing, \emph{ADA}
delivering guidelines for dental informatics or \emph{CDISC} for the exchange
of large amounts of various data between information systems.

For the purpose of this work, with a minimalistic implementation of a prototype
application, the considered (de facto) standards specifications mainly had a
helper function, giving some architectural guidance. Concerning the record
architecture, CYBOL applications follow the purely compositional principles of
CYBOP anyway, so that record modelling advices had only few implications.
CYBOI's architecture, however, is flexible enough to support many messaging
standards in the future, by simply adding the corresponding translator modules.
Existing terminologies can partly be used by associating terms appearing in
CYBOL knowledge templates with their pendants in common terminology systems.

A promising trial, in this context, would be to use CYBOL for building up new,
or structuring existing terminologies. CYBOL innately supports compositional
structures, which makes it a perfect match for compositional schemes. Further,
it allows to add meta information as well as to integrate constraints. The meta
information, which is contained in so-called \emph{property} tags of a term
(chapter \ref{cybernetics_oriented_language_heading}), at system runtime called
\emph{details}, may link to more than one superior (parent) category, thereby
placing the term simultaneously under different categories that are valid.
Thus, some problems of current terminologies (section \ref{schemes_heading})
\emph{might} get solved. But this remains to be figured out in future works
(chapter \ref{summary_and_outlook_heading}).

Standards for imaging, pharmacy- or laboratory data transfer, guidelines for
dental informatics, health card usage and related specifications will be
considered closer as soon as more application modules are developed within
\emph{Res Medicinae}.
