%
% $RCSfile: formality.tex,v $
%
% Copyright (C) 2002-2008. Christian Heller.
%
% Permission is granted to copy, distribute and/or modify this document
% under the terms of the GNU Free Documentation License, Version 1.1 or
% any later version published by the Free Software Foundation; with no
% Invariant Sections, with no Front-Cover Texts and with no Back-Cover
% Texts. A copy of the license is included in the section entitled
% "GNU Free Documentation License".
%
% http://www.cybop.net
% - Cybernetics Oriented Programming -
%
% http://www.resmedicinae.org
% - Information in Medicine -
%
% Version: $Revision: 1.1 $ $Date: 2008-08-19 20:41:06 $ $Author: christian $
% Authors: Christian Heller <christian.heller@tuxtax.de>
%

\section{Formality}
\label{formality_heading}
\index{Formality}
\index{Informal Language}
\index{Semi-Formal Diagrams}
\index{Formal Programming Language}
\index{Machine Language}
\index{Knowledge Modelling Language}

Abstract models can be described in different ways, for example \cite{philippow}:

\begin{itemize}
    \item[-] \emph{informally} by natural language
    \item[-] \emph{semi-formally} by diagrams
    \item[-] \emph{formally} by a programming language
\end{itemize}

The use of a programming language eases model abstraction for human programmers.
Special tools exist that break down models given in form of programming language
code into their binary form, into sequences of \emph{0} and \emph{1}. These
sequences are called \emph{Machine Language} because they are understood by
computers.

Classical programming languages have the linguistic means to express high-level
\emph{Knowledge} as well as low-level \emph{System Control} operations, such as
those for \emph{input/ output} (i/o), necessary for communication. The use of
such languages inevitably leads to a mess in program code because knowledge and
system control are mixed up. Inflexible, overly complex systems with numerous
interdependencies are the result. Part \ref{basics_heading} of this work
criticised some of the weak points of traditional programming language concepts.

This work makes the necessary split: Knowledge gets \emph{separated} from
system control. Chapter \ref{statics_and_dynamics_heading} already discussed
this separation giving manifold examples, taken from several sciences. The CYBOL
language being described in the next sections is just another form of storing
knowledge. It can therefore also be called a \emph{Knowledge Modelling Language}.
Any other low-level system control functionality belongs to the
\emph{Cybernetics Oriented Interpreter} (CYBOI), which gets introduced in the
later chapter \ref{cybernetics_oriented_interpreter_heading}.
