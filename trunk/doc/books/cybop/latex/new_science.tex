%
% $RCSfile: new_science.tex,v $
%
% Copyright (C) 2002-2008. Christian Heller.
%
% Permission is granted to copy, distribute and/or modify this document
% under the terms of the GNU Free Documentation License, Version 1.1 or
% any later version published by the Free Software Foundation; with no
% Invariant Sections, with no Front-Cover Texts and with no Back-Cover
% Texts. A copy of the license is included in the section entitled
% "GNU Free Documentation License".
%
% http://www.cybop.net
% - Cybernetics Oriented Programming -
%
% http://www.resmedicinae.org
% - Information in Medicine -
%
% Version: $Revision: 1.1 $ $Date: 2008-08-19 20:41:07 $ $Author: christian $
% Authors: Christian Heller <christian.heller@tuxtax.de>
%

\section*{New Science}
\label{new_science_heading}
%\addcontentsline{toc}{section}{New Science}

It was end of October 2004 that I discovered Stephen Wolfram's book
\emph{A New Kind of Science} \cite{wolfram} (published in 2002), through a link
in Wikipedia \cite{wikipedia}. By that time, I was already heavily writing on
my own work.

During those years of thinking about software systems, nature, the universe --
I felt pretty similar to how Wolfram describes it in the preface of his book.
Starting with an inspection of state-of-the-art techniques, diving deeper and
deeper into several topics, I soon realised that they all could not deliver a
\emph{coherent}, \emph{conclusive} solution to software modelling. Each had its
own drawbacks that made workarounds necessary. And, the more I dived into the
different technologies, the more complex, complicated, intransparent they got
-- but still, none seemed to provide an \emph{overall} solution.

It was only when I got more and more distance to existing solutions and moved
away from current thinking, towards a more universal approach and a view at
software systems through the eyes of nature, that I found the basic principles
described in this work.

Now, after having read \emph{A New Kind of Science}, I am glad that Wolfram did
not already write down everything I want to say, so that there is something left
for me to contribute, by delivering this work :-) There is one difference that
soon became obvious to me: Wolfram argues, that it is possible to study the
abstract world of simple programs, and take lessons from what kinds of things
occur there and have them in mind when investigating natural systems
\cite{wikipedia}. My work follows the exact opposite way, in that it observes
phenomenons of nature and concepts used in other sciences, and tries to apply
them to the design of software systems.

This is \emph{not} to say that \emph{CYBOP} does provide \emph{the} overall
solution. But what it surely wants to reach is to encourage people to think in
more general terms, across disciplines, to possibly find new concepts. And for
that, this work hopes to deliver some ideas. And I certainly do hope that the
more you, as readers, think about these ideas, the more sense they will make to
you, too.
