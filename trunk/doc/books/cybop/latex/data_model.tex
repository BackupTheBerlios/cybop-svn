%
% $RCSfile: data_model.tex,v $
%
% Copyright (C) 2002-2008. Christian Heller.
%
% Permission is granted to copy, distribute and/or modify this document
% under the terms of the GNU Free Documentation License, Version 1.1 or
% any later version published by the Free Software Foundation; with no
% Invariant Sections, with no Front-Cover Texts and with no Back-Cover
% Texts. A copy of the license is included in the section entitled
% "GNU Free Documentation License".
%
% http://www.cybop.net
% - Cybernetics Oriented Programming -
%
% http://www.resmedicinae.org
% - Information in Medicine -
%
% Version: $Revision: 1.1 $ $Date: 2008-08-19 20:41:06 $ $Author: christian $
% Authors: Christian Heller <christian.heller@tuxtax.de>
%

\subsection{Data Model}
\label{data_model_heading}

A number of historical and more current domain modelling concepts are described
in brief in the following sections, some of them known from the field of database
technology.

- kind of relationships between the data and rules (knowledge)

%
% $RCSfile: hierarchical_data_model.tex,v $
%
% Copyright (C) 2002-2008. Christian Heller.
%
% Permission is granted to copy, distribute and/or modify this document
% under the terms of the GNU Free Documentation License, Version 1.1 or
% any later version published by the Free Software Foundation; with no
% Invariant Sections, with no Front-Cover Texts and with no Back-Cover
% Texts. A copy of the license is included in the section entitled
% "GNU Free Documentation License".
%
% http://www.cybop.net
% - Cybernetics Oriented Programming -
%
% http://www.resmedicinae.org
% - Information in Medicine -
%
% Version: $Revision: 1.1 $ $Date: 2008-08-19 20:41:07 $ $Author: christian $
% Authors: Christian Heller <christian.heller@tuxtax.de>
%

\subsubsection{Hierarchical Data Model}
\label{hierarchical_data_model_heading}

One of the first models to structure domain data was a simple \emph{Hierarchy},
being used in \emph{Hierarchical Databases} such as ... (VSAM),
in the 1960s and early 1970s. Many of them are still running now,
for instance in insurance companies who have not yet migrated their
systems to modern technologies.

Knowledge Engineering Systems make use of hierarchical data, today.

- Domain Engineering in general, vertical and horizontal separation
- see \cite{inpulse} paper

The following techniques of domain modelling are neither sorted historically,
nor after the SEP phase they are used in (analysis, design, implementation).
The order of their appearance is determined by similarities they have. New
techniques are added stepwise, in a didactic manner, one building on the
other.

The feature modelling, for example, is a domain analysis- and not an
implementation technique but mentioned here because it is based on a special
technique of abstraction worth considering. Moreover it is not exclusively used
for analysis, but also it represents the beginning of design in a software
engineering process, as mentioned in section \ref{abstraction_gaps_heading}.

- add Feature Model to domain modelling, because it is a hierarchical model

\input{network_data_model}
\input{entity_relationship_model}
%
% $RCSfile: object_oriented_model.tex,v $
%
% Copyright (C) 2002-2008. Christian Heller.
%
% Permission is granted to copy, distribute and/or modify this document
% under the terms of the GNU Free Documentation License, Version 1.1 or
% any later version published by the Free Software Foundation; with no
% Invariant Sections, with no Front-Cover Texts and with no Back-Cover
% Texts. A copy of the license is included in the section entitled
% "GNU Free Documentation License".
%
% http://www.cybop.net
% - Cybernetics Oriented Programming -
%
% http://www.resmedicinae.org
% - Information in Medicine -
%
% Version: $Revision: 1.1 $ $Date: 2008-08-19 20:41:07 $ $Author: christian $
% Authors: Christian Heller <christian.heller@tuxtax.de>
%

\subsubsection{Object Oriented Model}
\label{object_oriented_model_heading}

?? frame \cite{sowa}, as ancestor of OO and others

The introduction of object oriented programming (section
\ref{object_oriented_programming_heading}) made another software design
philosophy popular:
Every entity was now treated as \emph{Object} being a runtime-instance of a
\emph{Class} that was capable of inheriting \emph{Attributes} and \emph{Methods}
from a parent class. A \emph{Relation} was now called \emph{Association}.
Multiplicity was called \emph{Cardinality}.

The primary new ideas of \emph{Inheritance} between classes and classes owning not
only attributes but also \emph{Methods} could not be modeled well in entity relationship
models (section \ref{entity_relationship_model}) so that \emph{Data Mapper} layers
(section \ref{data_mapper_heading}) became necessary. In general, object oriented
models look not much different from entity relationship models.

