%
% $RCSfile: unproven_specifications.tex,v $
%
% Copyright (C) 2002-2008. Christian Heller.
%
% Permission is granted to copy, distribute and/or modify this document
% under the terms of the GNU Free Documentation License, Version 1.1 or
% any later version published by the Free Software Foundation; with no
% Invariant Sections, with no Front-Cover Texts and with no Back-Cover
% Texts. A copy of the license is included in the section entitled
% "GNU Free Documentation License".
%
% http://www.cybop.net
% - Cybernetics Oriented Programming -
%
% http://www.resmedicinae.org
% - Information in Medicine -
%
% Version: $Revision: 1.1 $ $Date: 2008-08-19 20:41:09 $ $Author: christian $
% Authors: Christian Heller <christian.heller@tuxtax.de>
%

\paragraph{Unproven Specifications}
\label{unproven_specifications_heading}

The operativeness of technical specifications and designs produced by
\emph{Information Technology} (IT) \emph{Standards Development Organisations}
(SDO) were \emph{never proven in practice}, yet the best way to test a design
was to try to build it. Current standards specifications rather reminded on
what software engineers would call \emph{Requirements Analysis Models}.
