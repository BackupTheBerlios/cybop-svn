%
% $RCSfile: unified_modeling_language.tex,v $
%
% Copyright (C) 2002-2008. Christian Heller.
%
% Permission is granted to copy, distribute and/or modify this document
% under the terms of the GNU Free Documentation License, Version 1.1 or
% any later version published by the Free Software Foundation; with no
% Invariant Sections, with no Front-Cover Texts and with no Back-Cover
% Texts. A copy of the license is included in the section entitled
% "GNU Free Documentation License".
%
% http://www.cybop.net
% - Cybernetics Oriented Programming -
%
% http://www.resmedicinae.org
% - Information in Medicine -
%
% Version: $Revision: 1.1 $ $Date: 2008-08-19 20:41:09 $ $Author: christian $
% Authors: Christian Heller <christian.heller@tuxtax.de>
%

\subsubsection{Unified Modeling Language}
\label{unified_modeling_language_heading}
\index{Unified Modeling Language}
\index{UML}
\index{Class Diagram}
\index{Activity Diagram}
\index{Sequence Diagram}
\index{Use Case Diagram}
\index{State Machine Diagram}
\index{State Chart Diagram}
\index{Component Diagram}
\index{Deployment Diagram}
\index{Object Diagram}
\index{Instance Diagram}
\index{Package Diagram}
\index{Communication Diagram}
\index{Collaboration Diagram}
\index{Composite Structure Diagram}
\index{Interaction Overview Diagram}
\index{Timing Diagram}
\index{UML Diagram Type}
\index{Object Constraint Language}
\index{OCL}
\index{Structure Diagram}
\index{Behaviour Diagram}
\index{Interaction Diagram}
\index{Functional Model}
\index{Object Model}
\index{Dynamic Model}
\index{UML Tool}
\index{Computer Aided Software Engineering Tool}
\index{CASE Tool}
\index{Object Process Diagram}
\index{OPD}
\index{Entity Relationship Diagram}
\index{ERD}

Meanwhile, the probably most famous modelling- and specification language is
the \emph{Unified Modeling Language} (UML) \cite{uml, booch}. It uses a
graphical notation defining a number of diagrams. UML 2.x specifications
\cite{uml} extend the number of different diagram types from 9 (UML 1.x) to 13.
A good overview is given by Ambler in \cite{ambler2005}, which table
\ref{diagrams_table} reproduces in adapted form, showing only \emph{some}
diagram elements. The column \emph{Importance} contains a certainly subjective
recommendation of Ambler, indicating the \emph{Learning Priority} the single
diagram types have in his opinion (which the author of this work supports).

\begin{table}[ht]
    \begin{center}
        \begin{footnotesize}
        \begin{tabular}{| p{35mm} | p{55mm} | p{15mm} |}
            \hline
            \textbf{Diagram} & \textbf{Elements} & \textbf{Importance}\\
            \hline
            Class (CsD) & Class, Inheritance, Association & High\\
            \hline
            Activity (AD) & Activity, Flow, Fork/ Join, Condition, Decision/ Merge & High\\
            \hline
            Sequence (SD) & Object, Lifeline, Activation Box (Method-Invocation Box), Message & High\\
            \hline
            Use Case (UCD) & Use Case, Actor, Association & Medium\\
            \hline
            State Machine (SMD), formerly State Chart Diagram & State, Transition & Medium\\
            \hline
            Component (CmD) & Component, Interface, Dependency & Medium\\
            \hline
            Deployment (DD) & Node, Connection & Medium\\
            \hline
            Object (ObD), also referred to as Instance Diagram & Object, Relationship & Low\\
            \hline
            Package (PD) & Package, Dependency & Low\\
            \hline
            Communication (CoD), formerly Collaboration Diagram & Object, Association & Low\\
            \hline
            Composite Structure (CSD) & Collaboration, Object, Role & Low\\
            \hline
            Interaction Overview (IOD) & Interaction Frame, Interaction Occurrence Frame & Low\\
            \hline
            Timing (TiD) & Object, Lifeline, State, Timing Constraint & Low\\
            \hline
        \end{tabular}
        \end{footnotesize}
        \caption{UML 2.x Diagram Types \cite{ambler2005}}
        \label{diagrams_table}
    \end{center}
\end{table}

One extension to the UML that is now also part of the corresponding de-facto
standard, is the \emph{Object Constraint Language} (OCL). Being a declarative
language, it describes rules applying to UML models, in a precise text format.
This is because not all rules can be expressed by diagrammatic notation
\cite{wikipedia}. The range of possible rules comprises constraints like pre-
and post-conditions or object query expressions. \cite{ocl}

A common classification distinguishes UML diagrams as follows \cite{ambler2005}:

\begin{enumerate}
    \item \emph{Structure:} CsD, CmD, CSD, DD, ObD, PD
    \item \emph{Behaviour:} AD, SMD, UCD
    \item \emph{Interaction:} CoD, IOD, SD, TiD
\end{enumerate}

Others share the information represented by the diagrams according to an
underlying, independently existing model \cite{wikipedia}:

\begin{itemize}
    \item[-] \emph{Functional Model} (UCD): Functionality of the system from
        the user's point of view
    \item[-] \emph{Object Model} (CsD): Structure and substructure of the
        system using objects, attributes, operations, and associations
    \item[-] \emph{Dynamic Model} (AD, SD, SCD): Internal behaviour of the
        system
\end{itemize}

A program working with UML diagrams is called \emph{UML Tool}, or more exactly
\emph{Computer Aided Software Engineering} (CASE) tool. Many of these programs
have developed and matured, over the past decade of years. Besides the standard
UML diagram types, they offer source code parsing and -generation,
documentation creation and more. Some tools introduced their own extensions to
the UML de-facto standard, for example: \emph{Object Process Diagram} (OPD)
\cite{burkhardt} and \emph{Entity Relationship Diagram} (ERD) \cite{otw}. The
description of a hypothetic design tool suggested for the language being
introduced in chapter \ref{cybernetics_oriented_language_heading} will refer
back to the UML diagrams as mentioned in this section, and suggest a different
way to categorise them. Further, chapter \ref{cybernetics_oriented_language_heading}
will try to define four diagram types to be used in conjunction with the
language described in it.
