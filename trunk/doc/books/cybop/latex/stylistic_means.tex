%
% $RCSfile: stylistic_means.tex,v $
%
% Copyright (C) 2002-2008. Christian Heller.
%
% Permission is granted to copy, distribute and/or modify this document
% under the terms of the GNU Free Documentation License, Version 1.1 or
% any later version published by the Free Software Foundation; with no
% Invariant Sections, with no Front-Cover Texts and with no Back-Cover
% Texts. A copy of the license is included in the section entitled
% "GNU Free Documentation License".
%
% http://www.cybop.net
% - Cybernetics Oriented Programming -
%
% http://www.resmedicinae.org
% - Information in Medicine -
%
% Version: $Revision: 1.1 $ $Date: 2008-08-19 20:41:09 $ $Author: christian $
% Authors: Christian Heller <christian.heller@tuxtax.de>
%

\section*{Stylistic Means and Notation}
\label{stylistic_means_heading}
%\addcontentsline{toc}{section}{Stylistic Means and Notation}

The language of choice in this document is \emph{British English}, more
precisely known as \emph{Commonwealth English}. Exceptions are citations or
proper names like \emph{Unified Modeling Language}, stemming from
\emph{American English} sources. (In Oxford English, \emph{Modelling} would be
written with double letter \emph{l}). I am thinking about writing a
\emph{German} version of this document, but am not sure if it will be worth the
effort. If you as reader are interested in a translation, send me a short note!
The more emails I receive, the more convinced I will be.

% Als Autor dieser Arbeit konnte ich mir die Freiheit nehmen, viele Textpassagen
% nicht wortgetreu, sondern vielmehr sinngem�� aus der englischen Originalfassung
% zu �bersetzen. Fachbegriffe aus der Computerwelt wurden in Englisch belassen,
% um keine Zweideutigkeiten und redundante Abk�rzungen entstehen zu lassen.

Correctly, masculine \emph{and} feminine forms are used in a work. When
describing a patient's record, for example, one would write: \textit{his or her
record}. In order to improve readability, and exclusively because of this
reason, only masculine forms are used in this work.

The document sticks to the widespread \emph{Unified Modeling Language} (UML)
\cite{uml} standard notation for describing classical software concepts in
diagrams, wherever suitable. Minor simplifications are applied wherever these
result in a clearer illustration with better overview.

Pieces of software source code are displayed in \texttt{Typewriter Typeface}.
Emphasised words are \emph{italicised}.

Footnotes are not used on purpose. In my opinion, they only interrupt the
flow-of-reading. Remarks are placed in context instead, sometimes enclosed in
parentheses.

To all authors and contributors of the Wikipedia Encyclopedia:\\
I have cited so many Wikipedia articles in this work, that it would
not have been possible to create an extra bibliography entry for each
of them, without letting the frame of this work explode. Therefore,
I have just referenced Wikipedia in general, whenever one of its
articles was used.

Some scientists still label Wikipedia a \emph{Pseudo Encyclopedia} not worth
being mentioned in scientific works. However, it is my firm believe that this
will change in the near future and one day, it will be hard to write any work
without referencing Wikipedia knowledge, which will then (if not already now)
be of best quality.
