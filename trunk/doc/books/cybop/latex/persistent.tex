%
% $RCSfile: persistent.tex,v $
%
% Copyright (C) 2002-2008. Christian Heller.
%
% Permission is granted to copy, distribute and/or modify this document
% under the terms of the GNU Free Documentation License, Version 1.1 or
% any later version published by the Free Software Foundation; with no
% Invariant Sections, with no Front-Cover Texts and with no Back-Cover
% Texts. A copy of the license is included in the section entitled
% "GNU Free Documentation License".
%
% http://www.cybop.net
% - Cybernetics Oriented Programming -
%
% http://www.resmedicinae.org
% - Information in Medicine -
%
% Version: $Revision: 1.1 $ $Date: 2008-08-19 20:41:08 $ $Author: christian $
% Authors: Christian Heller <christian.heller@tuxtax.de>
%

\subsubsection{Persistent}
\label{persistent_heading}
\index{Persistent Communication}
\index{Mediums for Knowledge Storage}
\index{Indirect Communication}
\index{Knowledge Carrier}
\index{Hard Disk Drive}
\index{HDD}
\index{Random Access Memory}
\index{RAM}

One great advantage of human beings is to be able to help each other, to
cooperate in order to reach a common aim, to form a society which is to fill
exactly these aims. Main tasks of a \emph{State} as one form of organisation of
society are: \emph{Security}, \emph{Education}, \emph{Social Welfare}. While
all of them depend upon \emph{Politics}, there is an additional factor playing
an increasingly important role: the \emph{Availability of Knowledge}. Knowledge
cannot only be exchanged between current citizens of earth; fortunately, it can
be forwarded over \emph{Generations}.

For this to become possible, mankind had to make use of different mediums for
external storage, such as: \emph{Rock Painting}, \emph{Stone Tablet},
\emph{Papyrus Roll}, \emph{Paper Book}, \emph{Chemical Film},
\emph{Electronic File}. It also had to invent technologies for the
dissemination of knowledge: \emph{Monks' copying by hand}, \emph{Library},
\emph{Printing}, \emph{Radio and TV}, \emph{Internet}. The following example
does therefore not deal with \emph{direct} inter-systems communication, but
rather its \emph{indirect} counterpart -- the interaction between a system and
mediums in its environment. Of course, that environment could be treated as
system, too; but for reasons of simplification it is not here.

One human being (\emph{System}) wants to send a message to another, which is
not near the same location, but at some remote place. The sender has to decide
for a persistent message, and to choose a non-transient medium to store that
message. He takes a piece of paper, writes down or paints some information and
finally sends that paper as letter by (snail) mail. Paper and letter act as
\emph{Knowledge Carriers}. The receiver may then perceive the message optically
and process it similarly to the transient communication explained in the
previous section.

Because the information in this example is permanently available and
reproducable from the external medium, communication processes of that kind may
be called \emph{Persistent Communication}. In computing, persistent information
can be stored in files on a \emph{Hard Disk Drive} (HDD), for example; data in
\emph{Random Access Memory} (RAM) or those sent over a network, on the other
hand, are transient.

The interpreter system introduced in chapter
\ref{cybernetics_oriented_interpreter_heading} is capable of using transient-
as well as persistent communication.
