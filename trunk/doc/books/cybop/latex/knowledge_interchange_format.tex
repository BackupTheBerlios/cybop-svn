%
% $RCSfile: knowledge_interchange_format.tex,v $
%
% Copyright (C) 2002-2008. Christian Heller.
%
% Permission is granted to copy, distribute and/or modify this document
% under the terms of the GNU Free Documentation License, Version 1.1 or
% any later version published by the Free Software Foundation; with no
% Invariant Sections, with no Front-Cover Texts and with no Back-Cover
% Texts. A copy of the license is included in the section entitled
% "GNU Free Documentation License".
%
% http://www.cybop.net
% - Cybernetics Oriented Programming -
%
% http://www.resmedicinae.org
% - Information in Medicine -
%
% Version: $Revision: 1.1 $ $Date: 2008-08-19 20:41:07 $ $Author: christian $
% Authors: Christian Heller <christian.heller@tuxtax.de>
%

\subsubsection{Knowledge Interchange Format}
\label{knowledge_interchange_format_heading}
\index{Knowledge Interchange Format}
\index{KIF}
\index{PostScript}
\index{PS}

The \emph{Knowledge Interchange Format} (KIF), as described in \cite{kif}, is:

\begin{itemize}
    \item[-] a language designed for use in the interchange of knowledge among
        disparate computer systems
    \item[-] not intended as a primary language for interaction with human users
    \item[-] not intended as an internal representation for knowledge within
        computer systems
    \item[-] in its purpose, roughly analogous to \emph{PostScript} (PS)
        (section \ref{page_description_language_heading})
    \item[-] not as efficient as a specialised representation for knowledge,
        but more general and programmer-readable
\end{itemize}

The idea behind KIF is that \cite{kif}: \textit{a computer system reads a
knowledge base in KIF, (and) converts the data into its own internal form
(pointer structures, arrays, etc.). All computation is done using these
internal forms. When the computer system needs to communicate with another
computer system, it maps its internal data structures into KIF.} KIF's design
is characterised by three features:

\begin{enumerate}
    \item \emph{Declarative Semantics:} independent from specific interpreters,
        as opposed to e.g. \emph{Prolog}
    \item \emph{Logically Comprehensive:} may express arbitrary logical
        sentences, as opposed to \emph{SQL} or \emph{Prolog}
    \item \emph{Meta Knowledge:} permits the introduction of new knowledge
        representation constructs, without changing the language
\end{enumerate}

The following syntax example \cite{kif} shows a logical term involving the
\emph{if} operator. If the object constant \emph{a} denotes a number, then the
term denotes the absolute value of that number:

\begin{scriptsize}
    \begin{verbatim}
    (if (> a 0) a (- a))
    \end{verbatim}
\end{scriptsize}

The language introduced in chapter \ref{cybernetics_oriented_language_heading}
may not only serve as interchange format between systems, but also for the
definition of user interfaces, workflows and domain models, altogether. It
treats state- and logic models as separate, composable concepts (chapter
\ref{state_and_logic_heading}), which KIF does not. Further, it provides the
means to express meta knowledge.
