%
% $RCSfile: read.tex,v $
%
% Copyright (C) 2002-2008. Christian Heller.
%
% Permission is granted to copy, distribute and/or modify this document
% under the terms of the GNU Free Documentation License, Version 1.1 or
% any later version published by the Free Software Foundation; with no
% Invariant Sections, with no Front-Cover Texts and with no Back-Cover
% Texts. A copy of the license is included in the section entitled
% "GNU Free Documentation License".
%
% http://www.cybop.net
% - Cybernetics Oriented Programming -
%
% http://www.resmedicinae.org
% - Information in Medicine -
%
% Version: $Revision: 1.1 $ $Date: 2008-08-19 20:41:08 $ $Author: christian $
% Authors: Christian Heller <christian.heller@tuxtax.de>
%

\subsubsection{READ}
\label{read_heading}
\index{Read Codes}
\index{READ}
\index{Clinical Terms Version 3}
\index{CTV3}
\index{ICD-10}
\index{OPCS-4}
\index{National Health Service Information Authority}
\index{NHSIA}

The \emph{Read Codes} (READ), as their older name \emph{Clinical Terms Version 3}
(CTV3) says, are a: \textit{list of terms describing the care and treatment of
patients}. They: \textit{cover a wide range of topics in categories such as
signs and symptoms, treatments and therapies, investigations, occupations,
diagnoses and drugs and appliances.} Further, they: \textit{provide cross maps
to both ICD-10 and OPCS-4 classification codes.} \cite{read}

Scheme: enumerative\\
Maintainer: \emph{United Kingdom} (UK)
\emph{National Health Service Information Authority} (NHSIA)
