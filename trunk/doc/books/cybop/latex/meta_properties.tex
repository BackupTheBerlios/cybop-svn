%
% $RCSfile: meta_properties.tex,v $
%
% Copyright (C) 2002-2008. Christian Heller.
%
% Permission is granted to copy, distribute and/or modify this document
% under the terms of the GNU Free Documentation License, Version 1.1 or
% any later version published by the Free Software Foundation; with no
% Invariant Sections, with no Front-Cover Texts and with no Back-Cover
% Texts. A copy of the license is included in the section entitled
% "GNU Free Documentation License".
%
% http://www.cybop.net
% - Cybernetics Oriented Programming -
%
% http://www.resmedicinae.org
% - Information in Medicine -
%
% Version: $Revision: 1.1 $ $Date: 2008-08-19 20:41:07 $ $Author: christian $
% Authors: Christian Heller <christian.heller@tuxtax.de>
%

\subsubsection{Meta Properties}
\label{meta_properties_heading}
\index{CYBOL Meta Property Example}
\index{Graphical User Interface}
\index{GUI}
\index{Java Swing Framework}
\index{GUI Layouts}

When modelling \emph{Graphical User Interfaces} (GUI), a speciality to take
care about is the \emph{Position} of GUI elements within their surrounding
container. GUI components may have very different orderings and positions. The
\emph{Java Swing} framework \cite{java}, for example, offers \emph{BorderLayout},
\emph{BoxLayout}, \emph{CardLayout}, \emph{FlowLayout}, \emph{GridBagLayout} etc.

The following example of a GUI \emph{Dialogue} assumes that an interpreter
knows how to handle \emph{Compass} layouts, which are the pendant of the
above-mentioned \emph{BorderLayout}:

\begin{scriptsize}
    \begin{verbatim}
<model>
    <part name="title" channel="inline" abstraction="string" model="Prescription Dialogue"/>
    <part name="menu_bar" channel="file" abstraction="cybol" model="menu_bar.cybol">
        <property name="position" channel="inline" abstraction="string" model="north"/>
    </part>
    <part name="tool_bar" channel="file" abstraction="cybol" model="tool_bar.cybol">
        <property name="position" channel="inline" abstraction="string" model="west"/>
    </part>
    <part name="contents_panel" channel="file" abstraction="cybol" model="contents_panel.cybol">
        <property name="position" channel="inline" abstraction="string" model="centre"/>
    </part>
    <part name="status_bar" channel="file" abstraction="cybol" model="status_bar.cybol">
        <property name="position" channel="inline" abstraction="string" model="south"/>
    </part>
</model>
    \end{verbatim}
\end{scriptsize}

Further meta information such as the \emph{Colour} or \emph{Size} of a GUI
component may be given. The following example shows how a GUI \emph{Button} may
be modelled as part of some GUI panel. Again, properties like \emph{size} are
not modelled as part, because the button does not \emph{consist} of them, in a
structural way of thinking:

\begin{scriptsize}
    \begin{verbatim}
<model>
    <part name="exit_button" channel="file" abstraction="cybol" model="exit_button.cybol">
        <property name="position" channel="inline" abstraction="integer" model="0"/>
        <property name="size" channel="inline" abstraction="vector" model="80,20,1"/>
        <property name="colour" channel="inline" abstraction="rgb" model="127,127,127"/>
        <property name="action" channel="inline" abstraction="string" model="exit.cybol"/>
    </part>
</model>
    \end{verbatim}
\end{scriptsize}
