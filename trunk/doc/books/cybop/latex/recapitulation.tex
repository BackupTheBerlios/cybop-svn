%
% $RCSfile: recapitulation.tex,v $
%
% Copyright (C) 2002-2008. Christian Heller.
%
% Permission is granted to copy, distribute and/or modify this document
% under the terms of the GNU Free Documentation License, Version 1.1 or
% any later version published by the Free Software Foundation; with no
% Invariant Sections, with no Front-Cover Texts and with no Back-Cover
% Texts. A copy of the license is included in the section entitled
% "GNU Free Documentation License".
%
% http://www.cybop.net
% - Cybernetics Oriented Programming -
%
% http://www.resmedicinae.org
% - Information in Medicine -
%
% Version: $Revision: 1.2 $ $Date: 2008-09-07 15:36:07 $ $Author: christian $
% Authors: Christian Heller <christian.heller@tuxtax.de>
%

\section{Recapitulation}
\label{recapitulation_heading}

The concepts that were found by considering other scientific disciplines, reveal
a number of state-of-the-art software design solutions that do not comply with
their original in nature, for example the:

\begin{enumerate}
    \item Mix of static application knowledge and instructions for dynamic
        system control (chapter \ref{statics_and_dynamics_heading})
    \item False combination of information ignoring hierarchical structure and
        mixing in meta information (chapter \ref{knowledge_schema_heading})
    \item Bundling of state- and logic knowledge (chapter
        \ref{state_and_logic_heading})
\end{enumerate}

\newpage

These discrepancies are the major reason for the issues mentioned in section
\ref{motivation_heading}. They become clearer only later in this work (part
\ref{contribution_heading}), where more background knowledge will be provided.
Almost all problems they cause have their root in \emph{Dependencies}. As a
system grows, the inter-dependencies between its single parts grow with. Why
does this happen? Simply because a clear architecture is missing. Even if
developers really try to follow a such -- on some point in the software's
lifetime, compromises have to be made due to unforeseen requirements and
dependencies:

\begin{itemize}
    \item[-] \emph{Meta Techniques} are used to provide basic functionality
    \item[-] \emph{Static Managers} accessible by any other parts in the system
        are introduced
    \item[-] \emph{Multiple Interfaces} are implemented to realise new
        properties (\emph{Mix-In})
    \item[-] \emph{Redundant Code} needs to be written to avoid too many
        unwanted inter-dependencies
    \item[-] \emph{Varying Mechanisms} are applied to plugin new software layers
\end{itemize}

It seems that today's software models rarely abstract the real world correctly.
This is \emph{not} general criticism on software development as it exists today,
\emph{nor} is it criticism on the abilities of application developers who use
current concepts and languages. It is just the neutral, unbiased realisation
that there are a few concepts in use which cause unclear, unnecessary, wrong
dependencies within software systems. The application of principles of other
scientific disciplines might have the potential to solve that.

It was early that, in the style of \emph{Bionics}, parallels between computing
machines and the human brain were seen, yet unfortunately do both not function
in exactly the same manner. Concepts like \emph{Artificial Neural Networks} (ANN)
%(section \ref{artificial_neural_networks_heading})
that try to imitate the
physical structure of the human brain exist, but are today's computers with
deterministic behaviour not built like that; they often have a
\emph{von Neumann Architecture} \cite{philippow}. This forces human programmers
to \emph{adapt} their thinking to the machine concepts.

Traditional programming languages and design solutions try to ease application
development by bridging the gap between concepts of human thinking and those of
the machine. Software developers are given tools to design programs in a more
abstract way, independently from the source code which gets generated later.
But as long as the underlying concepts of abstraction are insufficient, design
problems are to be expected. The kind and quality of abstractions is so
important, because it influences -- and \emph{is} influenced by -- all aspects
of software development (part \ref{basics_heading}) dealing with
\emph{Knowledge}:

\begin{itemize}
    \item[-] the \emph{Software Engineering Process} specifies static knowledge
        models (abstractions resulting from process phases), to be later
        dynamically processed in a computer system
    \item[-] the \emph{Physical Architecture} requires the translation of
        knowledge models (communication) between systems
    \item[-] the \emph{Logical Architecture} provides the means to represent
        knowledge models (by languages and various techniques) within a system
\end{itemize}

They all, consciously or not, are trials to apply human patterns. The structure
of knowledge models, for example, is based on concepts of \emph{Human Thinking},
the logical \emph{Mind} -- as opposed to the above-mentioned neural networks
that want to imitate the functioning of the physical \emph{Brain}. Because of
the central importance of knowledge, one aim of this work is to investigate new
techniques for its abstraction, to thereby revise state-of-the-art software
development. However, probably not all traditional concepts will be thrown
away. Basic things like control structures (looping, branching etc.)
abstracting logic knowledge in form of algorithms are still of importance but
appear in a different form (as will be shown in chapter
\ref{cybernetics_oriented_language_heading}). It therefore seems to be more
suitable to say that the new concepts will complement (and not revise
completely) existing development techniques, as was planned at the beginning of
this work (figure \ref{method_figure}).
