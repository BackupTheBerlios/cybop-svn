%
% $RCSfile: attributes.tex,v $
%
% Copyright (C) 2002-2008. Christian Heller.
%
% Permission is granted to copy, distribute and/or modify this document
% under the terms of the GNU Free Documentation License, Version 1.1 or
% any later version published by the Free Software Foundation; with no
% Invariant Sections, with no Front-Cover Texts and with no Back-Cover
% Texts. A copy of the license is included in the section entitled
% "GNU Free Documentation License".
%
% http://www.cybop.net
% - Cybernetics Oriented Programming -
%
% http://www.resmedicinae.org
% - Information in Medicine -
%
% Version: $Revision: 1.1 $ $Date: 2008-08-19 20:41:05 $ $Author: christian $
% Authors: Christian Heller <christian.heller@tuxtax.de>
%

\subsubsection{Attributes}
\label{attributes_heading}
\index{CYBOL Attributes}
\index{CYBOL 'name' Attribute}
\index{CYBOL 'channel' Attribute}
\index{CYBOL 'abstraction' Attribute}
\index{CYBOL 'model' Attribute}

Normally, an XML \emph{Attribute} keeps meta information about the contents of
an XML \emph{Tag}. In CYBOL, however, three attributes keep meta information
about a fourth attribute. The attributes, altogether, are:

\begin{itemize}
    \item[-] name
    \item[-] channel
    \item[-] abstraction
    \item[-] model
\end{itemize}

The attribute of greatest interest is \emph{model}. It contains a model either
directly, or a path to one. The \emph{channel} attribute indicates whether the
\emph{model} attribute's value is to be read from:

\begin{itemize}
    \item[-] inline
    \item[-] file
    \item[-] ftp
    \item[-] http
\end{itemize}

The \emph{abstraction} attribute specifies how to interpret the model pointed
to by the \emph{model} attribute's value. A model may be given in formats like
for example:

\newpage

\begin{itemize}
    \item[-] cybol (a state- or logic compound model encoded in CYBOL format)
    \item[-] operation (a primitive logic model)
    \item[-] string
    \item[-] double
    \item[-] integer
    \item[-] boolean
\end{itemize}

The \emph{name} attribute, finally, provides the referenced model with a unique
identifier.

While the interpretation of the \emph{model} attribute's value depends on the
\emph{channel-} and \emph{abstraction} attributes, the other three attributes
(\emph{name}, \emph{channel}, \emph{abstraction}) themselves always get
interpreted as character string.
