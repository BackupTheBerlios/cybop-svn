%
% $RCSfile: building_blocks.tex,v $
%
% Copyright (C) 2002-2008. Christian Heller.
%
% Permission is granted to copy, distribute and/or modify this document
% under the terms of the GNU Free Documentation License, Version 1.1 or
% any later version published by the Free Software Foundation; with no
% Invariant Sections, with no Front-Cover Texts and with no Back-Cover
% Texts. A copy of the license is included in the section entitled
% "GNU Free Documentation License".
%
% http://www.cybop.net
% - Cybernetics Oriented Programming -
%
% http://www.resmedicinae.org
% - Information in Medicine -
%
% Version: $Revision: 1.1 $ $Date: 2008-08-19 20:41:05 $ $Author: christian $
% Authors: Christian Heller <christian.heller@tuxtax.de>
%

\subsection{Building Blocks}
\label{building_blocks_heading}
\index{Term}
\index{Concept}
\index{Semantic Link}
\index{Code of a Concept}

As for the word \emph{Ontology}, there are differing definitions for the
meanings of the words used in the field of \emph{Terminology}. The ones given
in \cite{metaterminology} differ only slightly from those of Jeremy Rogers, who
has assembled a very useful website \cite{rogers}. The following explanations
are based on it. They are necessary background knowledge for the investigations
on \emph{Human Thinking} and the relations within the new knowledge schema
introduced in chapter \ref{knowledge_schema_heading}.

An elementary building block is the word \emph{Term}, which is a word or phrase
(many words) labelling some idea. Another word for idea is \emph{Concept}.
Commonly distinguished concepts are:

\begin{itemize}
    \item[-] \emph{Primitive Concept} (Atomic): cannot be
        completely expressed in terms of other concepts
    \item[-] \emph{Composed Concept:} can be expressed in terms of other concepts
    \item[-] \emph{Pre-coordinated Concept} (Composed): has position in concept
        system that gets determined before the concept is supplied to end users
    \item[-] \emph{Post-coordinated Concept} (Composed): did not exist in the
        concept system as delivered to the user
\end{itemize}

Special kinds of terms are:

\begin{itemize}
    \item[-] \emph{Synonym:} two different terms that mean the same thing
    \item[-] \emph{Homonym:} two terms that sound the same but are spelled differently
    \item[-] \emph{Eponym:} a term that includes a proper name (like \emph{Murphy's Law})
\end{itemize}

Concepts can be related to each other by a \emph{Link}. Flavours of
\emph{Semantic Links} are:

\begin{itemize}
    \item[-] \emph{IS-KIND-OF:} diabetes \emph{is-a} disease
    \item[-] \emph{IS-PART-OF:} upper limb \emph{has-a} hand
    \item[-] \emph{CAUSES:} smoking \emph{causes} cancer
\end{itemize}

A \emph{Code} is an abstract identifier for either a link, or a concept or a
term. Rogers \cite{rogers} writes on this:

\begin{quote}
    If the concepts and the terms in a system are represented separately, then
    each concept and each term are \emph{unique}. Therefore, each can have a
    unique code assigned to it. By this mechanism, a single concept may be
    associated with more than one term (e.g. synonyms or foreign language
    translations) and a given term might be associated with two quite different
    concepts (homonyms e.g. \emph{cool} meaning \emph{cold} and \emph{cool}
    meaning \emph{groovy}).
\end{quote}

The language introduced in chapter \ref{cybernetics_oriented_language_heading}
has primitive concepts (state- and logic primitives, as mentioned in the
previous section) and it can express composed concepts. Knowledge templates
defined in that language represent pre-coordinated concepts which become
post-coordinated knowledge models when instantiated and altered at runtime.
Only \emph{IS-PART-OF} relations (links) are of importance in the language.
Knowledge templates written in it can hold many different codes, may they be
part of various terminologies or translations into foreign natural languages.
