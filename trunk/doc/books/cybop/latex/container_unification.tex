%
% $RCSfile: container_unification.tex,v $
%
% Copyright (C) 2002-2008. Christian Heller.
%
% Permission is granted to copy, distribute and/or modify this document
% under the terms of the GNU Free Documentation License, Version 1.1 or
% any later version published by the Free Software Foundation; with no
% Invariant Sections, with no Front-Cover Texts and with no Back-Cover
% Texts. A copy of the license is included in the section entitled
% "GNU Free Documentation License".
%
% http://www.cybop.net
% - Cybernetics Oriented Programming -
%
% http://www.resmedicinae.org
% - Information in Medicine -
%
% Version: $Revision: 1.1 $ $Date: 2008-08-19 20:41:06 $ $Author: christian $
% Authors: Christian Heller <christian.heller@tuxtax.de>
%

\subsection{Container Unification}
\label{container_unification_heading}
\index{Container Unification}
\index{Container}
\index{Collection as Container}
\index{Array as Container}
\index{Vector as Container}
\index{Stack as Container}
\index{Set as Container}
\index{List as Container}
\index{Map as Container}
\index{Hash Map as Container}
\index{Hash Table as Container}
\index{Tree as Container}

Section \ref{falsifying_polymorphism_heading} demonstrated how container
inheritance, due to polymorphism, may cause unpredictable behaviour leading to
\emph{falsified} container contents. The sections of this chapter introduced a
knowledge schema which they claimed to be \emph{general}. But that also means
that all kinds of containers must be representable by the suggested schema. But
why are there so many different kinds of containers? What actually is a
container?

It is a concept expressing that some model \emph{contains} some other model(s).
Types of containers that were introduced in section \ref{container_heading} are
\emph{Collections} (Array, Vector, Stack, Set, List), \emph{Maps} (Hash Map,
Hash Table) and the \emph{Tree}. They all are containers. What differs is just
the meta information they store about their elements. A list, for example,
holds position information about each of its elements. A map relates the name
of an element to its model (1:1). A tree links one model to many others (1:n).

But does the different meta information a container holds about its elements
justify the existence of different container models? If a knowledge schema was
general enough to represent a container structure on one hand, and to express
different kinds of meta information on the other, it might be able to behave
like \emph{any} of the known container types.

The schema proposed in this work claims to be this kind of knowledge schema. It
has a container structure by default, and can thus hold many parts in a
\emph{Tree}-like manner. It holds standard meta information about its parts:
their \emph{Name}, \emph{Model}, kind of \emph{Abstraction} and further meta
information called \emph{Details} -- and is therefore able to link the name of
an element to its model, in a \emph{Map}-like manner. To the additional meta
information (details) may belong the \emph{Position} of an element within its
model, in a \emph{List}-like manner. A \emph{Table} structure can be represented
as well, by splitting it into a hierarchical (tree-like) representation, as
known from markup languages (section \ref{markup_language_heading}).

Chapter \ref{cybernetics_oriented_language_heading} will introduce a language
capable of expressing all aspects of the knowledge schema as proposed in this
chapter.
