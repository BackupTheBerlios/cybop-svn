%
% $RCSfile: applicability_of_cybop.tex,v $
%
% Copyright (C) 2002-2008. Christian Heller.
%
% Permission is granted to copy, distribute and/or modify this document
% under the terms of the GNU Free Documentation License, Version 1.1 or
% any later version published by the Free Software Foundation; with no
% Invariant Sections, with no Front-Cover Texts and with no Back-Cover
% Texts. A copy of the license is included in the section entitled
% "GNU Free Documentation License".
%
% http://www.cybop.net
% - Cybernetics Oriented Programming -
%
% http://www.resmedicinae.org
% - Information in Medicine -
%
% Version: $Revision: 1.1 $ $Date: 2008-08-19 20:41:05 $ $Author: christian $
% Authors: Christian Heller <christian.heller@tuxtax.de>
%

\paragraph{The Applicability of CYBOP in Low-Level Systems}
\label{applicability_of_cybop_heading}

The focus of CYBOP is standard business application development. Knowledge
provided in form of CYBOL templates has a rather high abstraction level. It
would be interesting to know in how far CYBOP concepts, the CYBOL language and
the CYBOI interpreter are applicable in the development of \emph{Real Time}
(RT) systems, of control units in \emph{Automation Engineering} (AE), and
others more.

Surely, CYBOI's signal scheduling mechanism would have to be touched in such an
investigation. In addition to the position property of a logic compound model's
part, being used to place part signals in the correct order into the signal
memory, a \emph{Timestamp} (current time) could be used for this. It could
determine the scheduled execution time. Possibly, such a timestamp could also
serve as signal id.

In case CYBOI does not match RT requirements and -performance, at least CYBOL
might be suitable for representing configuration knowledge appropriately. This
may also count for the knowledge encoded by \emph{Hardware Description Languages}
(HDL).
