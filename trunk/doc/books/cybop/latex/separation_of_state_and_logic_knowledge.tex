%
% $RCSfile: separation_of_state_and_logic_knowledge.tex,v $
%
% Copyright (C) 2002-2008. Christian Heller.
%
% Permission is granted to copy, distribute and/or modify this document
% under the terms of the GNU Free Documentation License, Version 1.1 or
% any later version published by the Free Software Foundation; with no
% Invariant Sections, with no Front-Cover Texts and with no Back-Cover
% Texts. A copy of the license is included in the section entitled
% "GNU Free Documentation License".
%
% http://www.cybop.net
% - Cybernetics Oriented Programming -
%
% http://www.resmedicinae.org
% - Information in Medicine -
%
% Version: $Revision: 1.1 $ $Date: 2008-08-19 20:41:08 $ $Author: christian $
% Authors: Christian Heller <christian.heller@tuxtax.de>
%

\subsection{Separation of State- and Logic Knowledge}
\label{separation_of_state_and_logic_knowledge_heading}
\index{CYBOP Separation of State- and Logic Knowledge}

A third aspect causing troubles in software system design is the bundling of
state- and logic knowledge, known from object-oriented programming. It results
in:

\begin{itemize}
    \item[a] Difficult handling and repeated implementation of the same
        communication mechanisms (section \ref{misleading_tiers_heading})
    \item[b] Differing patterns complicating the handling of communication
        (section \ref{pattern_heading})
    \item[c] Bidirectional dependencies (circular references) between classes/
        objects in object-oriented systems, due to attribute-method bundling
        (section \ref{bidirectional_dependency_heading})
    \item[d] Pre-defined logic concepts in structured/ procedural- as well as
        object-oriented programming
\end{itemize}

Chapter \ref{state_and_logic_heading} therefore suggested a separation of
state- and logic concepts, in order to eliminate unnecessary inter-dependencies
and to be able to apply a unified translator architecture.

\paragraph{a}

Since low-level communication mechanisms are implemented in CYBOI,
application developers writing CYBOL knowledge templates do not have to bother
with these anymore.

\paragraph{b}

Since standard communication patterns are unified, the handling of
communication is simplified. Thanks to this unification, an extensible
translator architecture can be applied. Using it, any kind of abstract
knowledge model can be translated into any other. Universal communication
becomes possible.

\paragraph{c}

Since state- are split from logic concepts, many (partly bidirectional)
dependencies between knowledge models disappear, which reduces the coupling
between- and increases cohesion within models. Both kinds use exclusively
unidirectional relations. Additionally, logic- may access state models and each
other, but always unidirectionally.

\paragraph{d}

Since logic concepts (algorithms, workflows) are themselves modelled as CYBOL
knowledge templates, they become configurable. Traditionally, only structures
representing states are manipulatable at runtime; procedures representing logic
are fixed and cannot be altered.
