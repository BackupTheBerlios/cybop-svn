%
% $RCSfile: presentation_and_content.tex,v $
%
% Copyright (C) 2002-2008. Christian Heller.
%
% Permission is granted to copy, distribute and/or modify this document
% under the terms of the GNU Free Documentation License, Version 1.1 or
% any later version published by the Free Software Foundation; with no
% Invariant Sections, with no Front-Cover Texts and with no Back-Cover
% Texts. A copy of the license is included in the section entitled
% "GNU Free Documentation License".
%
% http://www.cybop.net
% - Cybernetics Oriented Programming -
%
% http://www.resmedicinae.org
% - Information in Medicine -
%
% Version: $Revision: 1.1 $ $Date: 2008-08-19 20:41:08 $ $Author: christian $
% Authors: Christian Heller <christian.heller@tuxtax.de>
%

\subsubsection{Presentation and Content}
\label{presentation_and_content_heading}
\index{CYBOL Presentation and Content Example}
\index{Mathematical Markup Language}
\index{MathML}

The \emph{Mathematical Markup Language} (MathML) \cite{mathml} provides means
for representing mathematical expressions. Its specification document states:

\begin{quote}
    A fundamental challenge in defining a markup language for mathematics on
    the Web is reconciling the need to encode both the \emph{Presentation} of a
    mathematical notation and the \emph{Content} of the mathematical idea or
    object which it represents.\\
    The relationship between a mathematical notation and a mathematical idea is
    subtle and deep. On a formal level, the results of mathematical logic raise
    unsettling questions about the correspondence between systems of symbolic
    logic and the phenomena they model. At a more intuitive level, anyone who
    uses mathematical notation knows the difference that a good choice of
    notation can make; the symbolic structure of the notation suggests the
    logical structure.
\end{quote}

This observation is very important because it helps avoid mixing \emph{Content}
and \emph{Presentation} of data. Both are discrete models, comparable to the
\emph{Domain} and \emph{User Interface} (UI) of a software application, which
can be translated into each other (sections \ref{translator_architecture_heading}
and \ref{logic_manipulates_state_heading}). The good side of MathML is that it
\cite{mathml}: \textit{allows authors to encode both the notation which
represents a mathematical object and the mathematical structure of the object
itself.} Its bad side, however, is that: \textit{authors can mix both kinds of
encoding in order to specify both the presentation and content of a
mathematical idea.} Another disadvantage is that different tags are used for
presentation and contents of a mathematical model.

CYBOL uses just four tags (section \ref{semantics_heading}) but can represent
mathematical expressions as well. What MathML calls \emph{Content}, becomes a
\emph{Logic} knowledge template in CYBOL. The mathematical content of the
formula $(a + b)^{2}$ would be modelled as follows:

\begin{scriptsize}
    \begin{verbatim}
<model>
    <part name="addition" channel="inline" abstraction="operation" model="add">
        <property name="summand_1" channel="inline" abstraction="knowledge" model="domain.a"/>
        <property name="summand_2" channel="inline" abstraction="knowledge" model="domain.b"/>
        <property name="sum" channel="inline" abstraction="knowledge" model="domain.c"/>
    </part>
    <part name="exponentiation" channel="inline" abstraction="operation" model="power">
        <property name="base" channel="inline" abstraction="knowledge" model="domain.c"/>
        <property name="power" channel="inline" abstraction="integer" model="2"/>
        <property name="result" channel="inline" abstraction="knowledge" model="domain.r"/>
    </part>
</model>
    \end{verbatim}
\end{scriptsize}

And the formula's \emph{Presentation} would be defined by the following two
CYBOL \emph{State} knowledge templates, of which the second one represents the
\emph{Base} that is referenced by the first one:

\begin{scriptsize}
    \begin{verbatim}
<model>
    <part name="base" channel="inline" abstraction="file" model="domain/base.cybol">
        <property name="fence" channel="inline" abstraction="boolean" model="true"/>
    </part>
    <part name="power" channel="inline" abstraction="integer" model="2">
        <property name="superscript" channel="inline" abstraction="boolean" model="true"/>
    </part>
</model>

<model>
    <part name="summand_1" channel="inline" abstraction="string" model="a"/>
    <part name="operator" channel="inline" abstraction="string" model="+"/>
    <part name="summand_2" channel="inline" abstraction="string" model="b"/>
</model>
    \end{verbatim}
\end{scriptsize}
