%
% $RCSfile: nature_as_pool.tex,v $
%
% Copyright (C) 2002-2008. Christian Heller.
%
% Permission is granted to copy, distribute and/or modify this document
% under the terms of the GNU Free Documentation License, Version 1.1 or
% any later version published by the Free Software Foundation; with no
% Invariant Sections, with no Front-Cover Texts and with no Back-Cover
% Texts. A copy of the license is included in the section entitled
% "GNU Free Documentation License".
%
% http://www.cybop.net
% - Cybernetics Oriented Programming -
%
% http://www.resmedicinae.org
% - Information in Medicine -
%
% Version: $Revision: 1.1 $ $Date: 2008-08-19 20:41:07 $ $Author: christian $
% Authors: Christian Heller <christian.heller@tuxtax.de>
%

\paragraph{Nature as Pool of Ideas for Software Design}
\label{nature_as_pool_heading}

Further analogies between nature, sciences and software design/ informatics in
general could be worth investigating. In particular the human brain and -mind
seem to use interesting concepts which are not fully considered in software
models to now. The \emph{Hippocampus}, for example, is a part of the human
brain that filters information by their importance and meaning
\cite{schoenhofer}. In form of \emph{Priorities}, a similar mechanism is
already used by \emph{Operating Systems} (OS). It is to be investigated in how
far these filter mechanisms are suitable for handling general \emph{Security}
issues in software systems.

After all, software can only be as good as the models behind it. CYBOP's
principles base on phenomenons of nature. Structures and their relations in
space, time and further dimensions can only be implemented as well as they are
currently known. It will be the task of natural sciences, philosophy, psychology
and other fields to steadily improve their concepts so that software modelling
can continue to learn from them.
