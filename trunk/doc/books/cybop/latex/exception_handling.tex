%
% $RCSfile: exception_handling.tex,v $
%
% Copyright (C) 2002-2008. Christian Heller.
%
% Permission is granted to copy, distribute and/or modify this document
% under the terms of the GNU Free Documentation License, Version 1.1 or
% any later version published by the Free Software Foundation; with no
% Invariant Sections, with no Front-Cover Texts and with no Back-Cover
% Texts. A copy of the license is included in the section entitled
% "GNU Free Documentation License".
%
% http://www.cybop.net
% - Cybernetics Oriented Programming -
%
% http://www.resmedicinae.org
% - Information in Medicine -
%
% Version: $Revision: 1.1 $ $Date: 2008-08-19 20:41:06 $ $Author: christian $
% Authors: Christian Heller <christian.heller@tuxtax.de>
%

\subsection{Exception Handling}
\label{exception_handling_heading}

- Exceptions (errors) should be logged and can then be sent as signal, to be
displayed on console or in a graphical dialogue
- exception handling of languages like Java is not good, since part objects
need to know about their whole object, in order to forward exceptions to it

jos-general@lists.sourceforge.net

There is one thing about Exceptions that disturbs me in general:

Exceptions are just normal signals, only that they indicate an unnormal
system behaviour. If an exception occurs, it is forwarded back to the
calling method/ an object's parent object and so forth until an exception
handling ("catch" block) is found. To me, this is insecure. A child object
should _never_ know about its parent object (parent in the sense of
instance creation, not inheritance). And yes, this is possible!

The flow of information should always go from the absolute root object
of all systems through the whole object tree, not vice-versa. Somewhere
in this root object (belonging to the operating system), there must be
some loop catching all signals/ events/ actions (interrupt handling),
may it be a keyboard, mouse, network etc. event or also an exception!

(Application or Web Servers do nothing else, they just have a loop
catching/ filtering out certain signals to react to.)

Wouldn't it be possible that applications send an "ExceptionSignal"
instead of using the "throw" mechanism of the Java language? This would
be much more proper, since the exception signal could be catched by
the top loop and sent through the whole system, top-to-bottom. If all
objects in an operating system and all applications get forwarded a
SignalMemory/ EventQueue, they can place any kind of signals there.
Exception signals would simply have to have higher priority then.

Just some thoughts I am currently musing about ...
Christian

jos-general@lists.sourceforge.net

--

jos-general@lists.sourceforge.net

Date:
Thu, 8 May 2003 09:28:02 +0200

> > Exceptions are just normal signals, only that they indicate an unnormal
> > system behaviour.
>
> By signals do you mean h/w interrupts ?

Not exactly. I mean a signal object (instance) that gets created when a
h/w IRQ occurs. This signal would encapsulate all necessary information.

> Exceptions are not normal, they are exceptional.
> In particular the performance is _not_ to be relied upon.
>
> A specific difference is in the handling - with (shared) interrupts
> all handlers may need to be run, with exceptions the handlers are nested
> with only the innermost handler running.

O.k., but that also depends on the framework's philosophy. There are
abstract layers with different responsibilities, e.g. one part doing some
calculation, another part organizing the graphical display of results.
If now an exception occurs in the calculation part, I do not want that
part to handle (and perhaps display in form of a graphical dialogue) the
exception. I'd prefer to send that Exception as signal with high priority
starting from the system root, so that the part which is responsible for
catching Exception signals (or/and for graphical display) can handle it.

If many handlers may need to be run, depends on how many there are.
I have to admit not to know exactly yet how this is done on the base
OS level. In my abstract thinking, however, every system, also the OS,
inherits from a common parent class. Every such system has a controller
which handles any signals. If the OS's controller does not handle a
signal, the signal gets forwarded to the sub systems (applications),
so that their controllers can filter out and handle the signal, if they
like.

If, on the other hand, an exception is thrown back to the calling method,
it may only be catched and handled by some parent method but not by
a sister node (to where the exception occurred) of the system tree.
O.k., the parent method might send the exception down another branch
of the system tree - but perhaps that branch is not "high" enough?
That is, perhaps another main branch more up in the tree would have
been the right one. Finally, this can only be assured if the exception
signal is really sent from top to bottom.

> > If an exception occurs, it is forwarded back to the
> > calling method/ an object's parent object and so forth until an exception
> > handling ("catch" block) is found. To me, this is insecure. A child
> > object should _never_ know about its parent object (parent in the sense
> > of instance creation, not inheritance). And yes, this is possible!
>
> Java does not include a restart option in its exception handling.
> How does an object throwing an exception learn anything about its
> creator ?

What do you mean by "restart option"?

An object does not need to learn anything about its creator. No child
object knows anything about any of its parent (creator) objects.
The system root object owns a SignalHandler (EventQueue) and this one
is sent through the whole system. That way, children can place signals
in the queue without knowing about their parent objects.
A loop in the system root object then frequently checks the queue for
new signals to be handled.

> > The flow of information should always go from the absolute root object
> > of all systems through the whole object tree, not vice-versa.
>
> I think passing command line arguments to a program is going down the tree.

Yes. But I don't understand that sentence in the context of our discussion
about "Exceptions".

> > (Application or Web Servers do nothing else, they just have a loop
> > catching/ filtering out certain signals to react to.)
> >
> > Wouldn't it be possible that applications send an "ExceptionSignal"
> > instead of using the "throw" mechanism of the Java language? This would
> > be much more proper, since the exception signal could be catched by
> > the top loop and sent through the whole system, top-to-bottom.
>
> I disagree that top-to-bottom is correct for exceptions.
>
> I like the L4 idea of treating interrupts as IPC/method calls and using
> threads to have multiple handlers.

What is the "L4 idea"?

> > If all
> > objects in an operating system and all applications get forwarded a
> > SignalMemory/ EventQueue, they can place any kind of signals there.
> > Exception signals would simply have to have higher priority then.
>
> Perhaps we are talking at different levels of abstraction - something has
> to map h/w interrupts to the code to enqueue event objects.

Possibly. What you mean seems to be one level nearer to the h/w. In my
opinion, it would be ideal to have one standard signal format (Class)
that could be used for system internal messages as well as for h/w
interrupts etc. As soon as a h/w interrupt occurs, it would have to get
packed into a new signal object which can then be sent through the
system. This would bring great flexibility and allow for OO techniques
to be used.

Christian
