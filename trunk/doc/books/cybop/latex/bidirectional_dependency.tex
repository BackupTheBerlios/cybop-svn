%
% $RCSfile: bidirectional_dependency.tex,v $
%
% Copyright (C) 2002-2008. Christian Heller.
%
% Permission is granted to copy, distribute and/or modify this document
% under the terms of the GNU Free Documentation License, Version 1.1 or
% any later version published by the Free Software Foundation; with no
% Invariant Sections, with no Front-Cover Texts and with no Back-Cover
% Texts. A copy of the license is included in the section entitled
% "GNU Free Documentation License".
%
% http://www.cybop.net
% - Cybernetics Oriented Programming -
%
% http://www.resmedicinae.org
% - Information in Medicine -
%
% Version: $Revision: 1.1 $ $Date: 2008-08-19 20:41:05 $ $Author: christian $
% Authors: Christian Heller <christian.heller@tuxtax.de>
%

\subsubsection{Bidirectional Dependency}
\label{bidirectional_dependency_heading}
\index{Bidirectional Dependency}
\index{Inter-Dependency}
\index{Endless Loop}
\index{Tree}
\index{Directed Acyclic Graph}
\index{DAG}
\index{Oriented Acyclic Graph}
\index{Process Tree}
\index{Object Tree}
\index{Database Data Structure}
\index{File System Structure}
\index{Syntax Tree}
\index{Acyclic Graph}
\index{Circular Reference}

\emph{Bidirectional References} are a nightmare for every software developer.
They cause \emph{Inter-Dependencies} so that changes in one part of a system can
affect multiple other parts which in turn affect the originating part, which may
finally lead to cycles or even endless loops. Also, the actual program flow and
effects of changes to a system become very hard to trace. Therefore, the avoidance
of such dependencies belongs to the core principles of good software design.

A \emph{Tree}, in mathematics, is defined as \textit{Directed Acyclic Graph}
(DAG), also known as \emph{Oriented Acyclic Graph} \cite{nist}. It has a
\emph{Root Node} and \emph{Child Nodes} which can become \emph{Parent Nodes}
when having children themselves; otherwise they are called \emph{Leaves}.
Children of the same node are \emph{Siblings}. \textit{A common constraint is
that no node can have more than one parent}, as \cite{foldoc} writes and
continues: \textit{Moreover, for some applications, it is necessary to consider
a node's children to be an ordered list, instead of merely a set.} A graph is
\emph{acyclic} if every node can be reached via exactly one path, which then
also is the shortest possible.

In computing, trees are used in many forms, for example as \emph{Process Tree}
of an operating system \cite{debian, gnu, linux} or as \emph{Object Tree} of an
object-oriented application. They represent \emph{Data Structures} in databases
or file systems and also the \emph{Syntax Tree} of languages. The violation of
the principle of the \emph{Acyclic Graph} can lead to the same loops, also
called \emph{Circular References}, as mentioned above, which can result in the
crossing of memory limits and is a potential security risk. Therefore, one of
the main aims in the creation of the new concepts introduced in part
\ref{contribution_heading} of this work was the avoidance of bidirectional
relations.
