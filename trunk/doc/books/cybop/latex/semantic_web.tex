%
% $RCSfile: semantic_web.tex,v $
%
% Copyright (C) 2002-2008. Christian Heller.
%
% Permission is granted to copy, distribute and/or modify this document
% under the terms of the GNU Free Documentation License, Version 1.1 or
% any later version published by the Free Software Foundation; with no
% Invariant Sections, with no Front-Cover Texts and with no Back-Cover
% Texts. A copy of the license is included in the section entitled
% "GNU Free Documentation License".
%
% http://www.cybop.net
% - Cybernetics Oriented Programming -
%
% http://www.resmedicinae.org
% - Information in Medicine -
%
% Version: $Revision: 1.1 $ $Date: 2008-08-19 20:41:08 $ $Author: christian $
% Authors: Christian Heller <christian.heller@tuxtax.de>
%

\subsection{Semantic Web}
\label{semantic_web_heading}
\index{Semantic Web}
\index{Extensible Markup Language}
\index{XML}
\index{XML Schema}
\index{XML Data}
\index{Document Content Description}
\index{DCD}
\index{Schema for Object Oriented XML}
\index{SOX}
\index{XML Metadata Interchange}
\index{XMI}
\index{Resource Description Framework}
\index{RDF}
\index{Web Ontology Language}
\index{OWL}

As mentioned in section \ref{extensible_markup_language_heading}, the
\emph{Extensible Markup Language} (XML) \cite{rdfowlrelease} provides a:
\textit{set of rules for creating vocabularies that can bring structure to both
documents and data on the Web} and it: \textit{gives clear rules for syntax.}
XML Schemas \cite{xmlschema} served as: \textit{a method for composing XML
vocabularies.} Yet although XML were a powerful, flexible surface syntax for
structured documents, it imposed no \emph{Semantic Constraints} on the
\emph{Meaning} of these documents. Having investigated the usefulness of XML
for a \emph{meaningful} sharing of information units at the semantic level,
Robin Cover writes \cite{xmlsemantics}:

\begin{quote}
    \ldots\ the use of XML for \emph{Data Interchange} may already outweigh its
    use for \emph{Document Display}. For messaging and other transaction data,
    specifications approaching the level of formal semantics (e.g. KIF or KQML)
    are desirable, governing not just common (atomic) data types in business
    objects, but complex objects used by computer agents in large-scale business
    transactions. XML vocabularies supporting these applications will need to be
    defined in terms of precise object semantics.
\end{quote}

He lists a number of efforts dealing with the support for generic XML semantics,
that is \emph{Semantic Transparency} of XML in a broader sense, to provide
unambiguous semantic specification:

\begin{itemize}
    \item[-] \emph{XML Data} \cite{xmldata}
    \item[-] \emph{Document Content Description} (DCD) for XML \cite{dcd}
    \item[-] \emph{Schema for Object Oriented XML} (SOX) \cite{sox}
    \item[-] \emph{XML Metadata Interchange} (XMI) \cite{mda}
    \item[-] \emph{Resource Description Framework} (RDF) \cite{rdf}
    \item[-] \emph{Web Ontology Language} (OWL) \cite{owl}
\end{itemize}

The RDF and OWL as well-known efforts are mentioned in the following two
subsections. Both are often comprised under the umbrella term
\emph{Semantic Web}. Much of what is written about the semantic web sounds as
if it was a replacement technology for the Web as known today. Yet Eric Miller,
leader of W3C's semantic web activity, means \cite{rdfowlrelease}:

\begin{quote}
    In reality, it's more Web Evolution than Revolution. The Semantic Web is
    made through incremental changes, by bringing machine-readable descriptions
    to the data and documents already on the Web. XML, RDF and OWL enable the
    Web to be a global infrastructure for sharing both, documents and data which
    make searching and reusing information easier and more reliable as well.
\end{quote}

%
% $RCSfile: resource_description_framework.tex,v $
%
% Copyright (C) 2002-2008. Christian Heller.
%
% Permission is granted to copy, distribute and/or modify this document
% under the terms of the GNU Free Documentation License, Version 1.1 or
% any later version published by the Free Software Foundation; with no
% Invariant Sections, with no Front-Cover Texts and with no Back-Cover
% Texts. A copy of the license is included in the section entitled
% "GNU Free Documentation License".
%
% http://www.cybop.net
% - Cybernetics Oriented Programming -
%
% http://www.resmedicinae.org
% - Information in Medicine -
%
% Version: $Revision: 1.1 $ $Date: 2008-08-19 20:41:08 $ $Author: christian $
% Authors: Christian Heller <christian.heller@tuxtax.de>
%

\subsubsection{Resource Description Framework}
\label{resource_description_framework_heading}
\index{Resource Description Framework}
\index{RDF}
\index{Extensible Markup Language}
\index{XML}
\index{RDF Schema}
\index{XML Schema}
\index{OWL}

The \emph{Resource Description Framework} (RDF) \cite{rdf} as part of the
\emph{Semantic Web} provides a standard way for simple descriptions to be made.
It is: \textit{a simple data model for referring to objects (resources) and how
they are related. An RDF-based model can be represented in XML syntax.}
\cite{wikipedia}

RDF wants to achieve for \emph{Semantics} what XML has achieved for
\emph{Syntax} -- to provide a clear set of rules for creating descriptive
information. Both follow a special schema, \emph{RDF Schema} \cite{rdf} and
\emph{XML Schema} \cite{xmlschema}, respectively, which defines the structure
and vocabulary that may be used in the corresponding documents.

Many applications that use XML as syntax for data interchange, may apply the
RDF specifications to better support the exchange of actual knowledge on the
web. The RDF data framework is used \cite{rdfowlrelease} in: asset management,
enterprise integration and the sharing and reuse of data on the web. Example
applications combining information from multiple sources on the web
\cite{rdfowlrelease} include: library catalogs, world-wide directories, news-
and content aggregation, collections of music or photos.

In the words of Brian McBride \cite{rdfowlrelease}, chair of the RDF core
working group, his group had: \textit{turned the RDF specifications into both a
practical and mathematically precise foundation on which OWL and the rest of
the semantic web can be built.}

Chapter \ref{cybernetics_oriented_language_heading} will come back to RDF once
more, and compare it with the new language then introduced.

%
% $RCSfile: web_ontology_language.tex,v $
%
% Copyright (C) 2002-2008. Christian Heller.
%
% Permission is granted to copy, distribute and/or modify this document
% under the terms of the GNU Free Documentation License, Version 1.1 or
% any later version published by the Free Software Foundation; with no
% Invariant Sections, with no Front-Cover Texts and with no Back-Cover
% Texts. A copy of the license is included in the section entitled
% "GNU Free Documentation License".
%
% http://www.cybop.net
% - Cybernetics Oriented Programming -
%
% http://www.resmedicinae.org
% - Information in Medicine -
%
% Version: $Revision: 1.1 $ $Date: 2008-08-19 20:41:09 $ $Author: christian $
% Authors: Christian Heller <christian.heller@tuxtax.de>
%

\subsubsection{Web Ontology Language}
\label{web_ontology_language_heading}
\index{Web Ontology Language}
\index{OWL}
\index{Uniform Resource Indicator}
\index{URI}
\index{Resource Description Framework}
\index{RDF}
\index{DARPA Agent Markup Language}
\index{Ontology Inference Layer}
\index{DAML+OIL}
\index{Ontology}

The \emph{Web Ontology Language} (OWL) is \cite{owl}: \textit{a semantic markup
language for publishing and sharing ontologies on the world wide web \ldots\
which delivers richer integration and interoperability of data among descriptive
communities.} It uses \emph{Uniform Resource Indicators} (URI) for naming and
is an extension of the \emph{Resource Description Framework} (RDF), adding more
vocabulary for describing properties and classes, for example relations between
classes, cardinality, richer typing of properties, or enumerated classes. OWL
was originally derived from the \emph{DARPA Agent Markup Language} +
\emph{Ontology Inference Layer} (DAML+OIL) web ontology language (section
\ref{agent_communication_language_heading}).

In the understanding of OWL, an ontology is a subject- or domain specific
vocabulary which defines the terms used to describe and represent an area of
knowledge \cite{rdfowlrelease}. However, there are other definitions of the term
\emph{Ontology} which are given in section \ref{conceptual_network_heading}.
OWL aims to add to ontologies capabilities like \cite{rdfowlrelease}:

\begin{itemize}
    \item[-] Ability to be distributed across many systems
    \item[-] Scalability to web needs
    \item[-] Compatibility with web standards for accessibility and internationalisation
    \item[-] Openness and extensibility
\end{itemize}

It introduces keywords for the use of \emph{Classification}, \emph{Subclassing}
with \emph{Inheritance} and further abstraction principles. RDF is neutral
enough to permit such extensions. Also the language introduced in chapter
\ref{cybernetics_oriented_language_heading} may be extended with meta
properties, such as one for inheritance.

