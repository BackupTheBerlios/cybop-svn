%
% $RCSfile: continuity_of_care.tex,v $
%
% Copyright (C) 2002-2008. Christian Heller.
%
% Permission is granted to copy, distribute and/or modify this document
% under the terms of the GNU Free Documentation License, Version 1.1 or
% any later version published by the Free Software Foundation; with no
% Invariant Sections, with no Front-Cover Texts and with no Back-Cover
% Texts. A copy of the license is included in the section entitled
% "GNU Free Documentation License".
%
% http://www.cybop.net
% - Cybernetics Oriented Programming -
%
% http://www.resmedicinae.org
% - Information in Medicine -
%
% Version: $Revision: 1.1 $ $Date: 2008-08-19 20:41:06 $ $Author: christian $
% Authors: Christian Heller <christian.heller@tuxtax.de>
%

\subsection{Continuity of Care}
\label{continuity_of_care_heading}
\index{Continuity of Care Record}
\index{CCR}
\index{Personal Health Project}
\index{PHP}

A main result of the opinion stated in the previous section was the realisation
that a major challenge for EHR design will be to overcome the difference
between an organisation's evidential record management process with emphasis on
\emph{legal/ financial aspects} and the record keeping as
\emph{medical/ health documentation}, that an individual would do.

This is exactly the issue that Philippe Ameline and his French colleagues
address in their \emph{Nautilus/ Odyssee} project \cite{nautilus}. It
distinguishes between three levels of data:

\begin{itemize}
    \item[-] \emph{Individual}: personal, various local
    \item[-] \emph{Group}: professional, 24 hour availability
    \item[-] \emph{Collective}: dedicated to continuity of care
\end{itemize}

The latter is called \emph{Personal Health Project} (PHP). Its health management
data can be shared between a \emph{Patient} and his \emph{Care Team}, with the
EHR \emph{passing by} institutions. Ameline writes in \cite{openehrtechnical}
that the management of these two referentials -- health professional and patient
-- meant that applications now had to handle differently the \emph{history data}
with a time duration (which may get changed by someone else) and the data of the
\emph{instantaneous picture} kind (what one noticed and reported at a given time).

A similar effort with U.S. American roots is called \emph{Continuity of Care Record}
(CCR) \cite{ccr}. Just like the PHP, it does not want to be a complete EHR, but
rather: \textit{organise and make transportable a set of basic patient
information consisting of the most relevant and timely facts about a patient's
condition.} Through specified XML code, the CCR becomes interoperable.
