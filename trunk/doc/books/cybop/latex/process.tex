%
% $RCSfile: process.tex,v $
%
% Copyright (C) 2002-2008. Christian Heller.
%
% Permission is granted to copy, distribute and/or modify this document
% under the terms of the GNU Free Documentation License, Version 1.1 or
% any later version published by the Free Software Foundation; with no
% Invariant Sections, with no Front-Cover Texts and with no Back-Cover
% Texts. A copy of the license is included in the section entitled
% "GNU Free Documentation License".
%
% http://www.cybop.net
% - Cybernetics Oriented Programming -
%
% http://www.resmedicinae.org
% - Information in Medicine -
%
% Version: $Revision: 1.1 $ $Date: 2008-08-19 20:41:08 $ $Author: christian $
% Authors: Christian Heller <christian.heller@tuxtax.de>
%

\section{Process}
\label{process_heading}
\index{Process}
\index{Resource Grouping}
\index{Execution}
\index{Address Space}
\index{Thread of Execution}
\index{Central Processing Unit}
\index{CPU}
\index{Program Counter}
\index{Registers}
\index{Stack}
\index{Abstract System Concepts}
\index{Operating System}
\index{OS}
\index{Session}
\index{Process Group}
\index{Job}
\index{System}
\index{Application}
\index{Task}
\index{Lightweight Process}
\index{Work Queue}
\index{Task Farm}
\index{Task Bag}

The most common word used to describe a running computer program is
\emph{Process}. Tanenbaum \cite{tanenbaum2001} defines it as an abstract model
based on two independent concepts: \emph{Resource Grouping} (space) and
\emph{Execution} (time).

He writes that \emph{Resource Grouping} meant that a process had an address
space containing program text and data, as well as other resources. A
\emph{Thread of Execution}, on the other hand, were the entity scheduled for
execution on the \emph{Central Processing Unit} (CPU). It had a program counter
(keeping track of which instruction to execute next), registers (holding its
current working variables) and a stack (containing the execution history, with
one frame for each procedure called but not yet returned from). Although a
thread would have to execute in some process, the thread and its process were
different concepts and could be treated separately.

A slightly different explanation is given in \cite{iseries}:

\begin{quote}
    A thread is the path a program takes while it runs, the steps it performs,
    and the order in which it performs the steps. A thread runs code from its
    starting location in an ordered, predefined sequence for a given set of
    inputs. The term \emph{Thread} is shorthand for \emph{Thread of Control}.
    (One) can use multiple threads to improve application performance by
    running different application tasks simultaneously.
\end{quote}

\begin{table}[ht]
    \begin{center}
        \begin{footnotesize}
        \begin{tabular}{| p{25mm} | p{45mm} | p{35mm} |}
            \hline
            \textbf{Abstract Concept} & \textbf{Explanation} & \textbf{Synonyms}\\
            \hline
            Session & Bundle of processes of one user &\\
            \hline
            Process Group & Collection of one or more processes & Job\\
            \hline
            Process & Container for related Resources & System, Application, Task\\
            \hline
            Thread & Schedulable Entity & Lightweight Process\\
            \hline
        \end{tabular}
        \end{footnotesize}
        \caption{Systematics of Abstract System Concepts}
        \label{concepts_table}
    \end{center}
\end{table}

There are other abstract concepts which are of importance, especially in an
\emph{Operating System} (OS) context. A terminal in the \emph{Linux} OS
\cite{johnson}, for example, may control a \emph{Session} consisting of
\emph{Process Groups} which in turn contain many \emph{Processes} providing
resources for the threads running in them. Table \ref{concepts_table} shows one
possible systematics of these concepts.

Some ambiguities exist, however. The term \emph{Job} which, some decades ago,
still stood for a program or set of programs, is nowadays used to label a
process group in \emph{Windows 2000} \cite[p. 7, 796]{tanenbaum2001} and
similarly in \emph{Linux} \cite[p. 125, 237]{johnson}. The notion of a
\emph{Task} is sometimes used equivalent to thread \cite{daene}, but other
times refers to a process or even process group \cite[p. 113]{johnson}.
Additionally, some sources use the term in the meaning of a signal or event
belonging to a work queue called \emph{Task Farm} or \emph{Task Bag}
\cite[p. 548, 606]{tanenbaum1999}.

This document uses the more general word \emph{System} to write about a process
that manages the input, storage, processing and output of data in a computer.
This is contrary to some other works which mean a whole computer, including its
hardware and software programs running on it, when talking about systems. In
the understanding of this work, once again, a \emph{System} is a \emph{Process}
(software system) running on a \emph{Computer} (hardware system).
