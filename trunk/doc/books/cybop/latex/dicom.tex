%
% $RCSfile: dicom.tex,v $
%
% Copyright (C) 2002-2008. Christian Heller.
%
% Permission is granted to copy, distribute and/or modify this document
% under the terms of the GNU Free Documentation License, Version 1.1 or
% any later version published by the Free Software Foundation; with no
% Invariant Sections, with no Front-Cover Texts and with no Back-Cover
% Texts. A copy of the license is included in the section entitled
% "GNU Free Documentation License".
%
% http://www.cybop.net
% - Cybernetics Oriented Programming -
%
% http://www.resmedicinae.org
% - Information in Medicine -
%
% Version: $Revision: 1.1 $ $Date: 2008-08-19 20:41:06 $ $Author: christian $
% Authors: Christian Heller <christian.heller@tuxtax.de>
%

\subsubsection{DICOM}
\label{dicom_heading}
\index{Digital Imaging and Communications in Medicine}
\index{DICOM}
\index{Computer Tomograph}
\index{CT}
\index{DICOM Message Service Element}
\index{DIMSE}
\index{Transfer Control Protocol}
\index{TCP}
\index{Simple Object Access Protocol}
\index{SOAP}
\index{Service Oriented Architecture}
\index{SOA}
\index{Remote Procedure Call}
\index{RPC}
\index{American College of Radiology}
\index{ACR}
\index{National Electrical Manufacturers Association}
\index{NEMA}

\emph{Digital Imaging and Communications in Medicine} (DICOM) is a:
\textit{multi-part standard produced to facilitate the interchange of
information between digital imaging computer systems in medical environments.}
\cite{dicom} Medical devices like \emph{Computer Tomographs} (CT), manufactured
by various vendors, produce a variety of digital image formats, which explains
the need to standardise their transfer.

DICOM not only defines its own file format containing meta data and the actual
image data (in compressed or uncompressed form), but also a transport protocol
called \emph{DICOM Message Service Element} (DIMSE), which is based on the
\emph{Transfer Control Protocol} (TCP). Just like the
\emph{Simple Object Access Protocol} (SOAP), DICOM uses the
\emph{Service Oriented Architecture} (SOA) -- a simple principle describing a
service as \emph{Remote Procedure Call} (RPC). \cite{kleinschmidt}

Maintainer: \emph{American College of Radiology} (ACR),
\emph{National Electrical Manufacturers Association} (NEMA)
