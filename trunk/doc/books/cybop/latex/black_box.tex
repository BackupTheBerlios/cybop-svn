%
% $RCSfile: black_box.tex,v $
%
% Copyright (C) 2002-2008. Christian Heller.
%
% Permission is granted to copy, distribute and/or modify this document
% under the terms of the GNU Free Documentation License, Version 1.1 or
% any later version published by the Free Software Foundation; with no
% Invariant Sections, with no Front-Cover Texts and with no Back-Cover
% Texts. A copy of the license is included in the section entitled
% "GNU Free Documentation License".
%
% http://www.cybop.net
% - Cybernetics Oriented Programming -
%
% http://www.resmedicinae.org
% - Information in Medicine -
%
% Version: $Revision: 1.1 $ $Date: 2008-08-19 20:41:05 $ $Author: christian $
% Authors: Christian Heller <christian.heller@tuxtax.de>
%

\subsubsection{Black Box}
\label{black_box_heading}
\index{Black Box}
\index{Operation}
\index{Functional Elements}
\index{Complexity Hiding}
\index{Block Diagram}
\index{Unified Modeling Language}
\index{UML}

An \emph{Operation} can be well treated as system: it contains rules of logic
after which its input gets transformed into its output. But not all systems are
as easy as a simple operation; many are \emph{composed} of yet smaller systems.
Biological systems, for example, are extremely difficult to describe in their
entirety, with a simple mathematical formula.

A system may be seen as a number of interacting \emph{Functional Elements}. It
is these elements and their interactions which determine the specific
properties and behaviour of a system. However, for modelling the behaviour of a
transmission system, its inner structure is not important. Systems theory
focusses on the time-dependent progression of input- and output signals as well
as their relation.

A common technique in systems engineering is to reduce complexity by
\emph{hiding} functionality which is unimportant in the given context, inside a
system. One then talks of a system as \emph{Black Box} since only its input,
output and their relation, but not its inside, are considered (figure
\ref{blackbox_figure}). The black box provides an encapsulation towards the
infinite microcosm, and it knows nothing about its usage within a greater
macrocosm (section \ref{knowledge_ontology_heading}).

The usual way to illustrate system elements and their relations is the
\emph{Block Diagram}. It is an important instrument for system analysis. Many
structures and processes can be described in that manner. In software
engineering, the \emph{Unified Modeling Language} (UML) has become the de-facto
modelling standard instead.
