%
% $RCSfile: intrinsic_or_extrinsic_properties.tex,v $
%
% Copyright (C) 2002-2008. Christian Heller.
%
% Permission is granted to copy, distribute and/or modify this document
% under the terms of the GNU Free Documentation License, Version 1.1 or
% any later version published by the Free Software Foundation; with no
% Invariant Sections, with no Front-Cover Texts and with no Back-Cover
% Texts. A copy of the license is included in the section entitled
% "GNU Free Documentation License".
%
% http://www.cybop.net
% - Cybernetics Oriented Programming -
%
% http://www.resmedicinae.org
% - Information in Medicine -
%
% Version: $Revision: 1.1 $ $Date: 2008-08-19 20:41:07 $ $Author: christian $
% Authors: Christian Heller <christian.heller@tuxtax.de>
%

\subsection{Intrinsic or Extrinsic Properties}
\label{intrinsic_or_extrinsic_properties_heading}
\index{Intrinsic Property}
\index{Extrinsic Property}
\index{Position as Point}
\index{Size as Difference}
\index{Benediktine Approach}
\index{Spatial Dimension}

Properties may not only represent the \emph{Position} of a part in space/ mass/
time, but also its \emph{Size} in the same dimensions. For space, it may be
called \emph{Expansion}, for mass \emph{Massiness}, and for time \emph{Duration}.
While a size is always represented by the \emph{Difference} of two values, a
position is represented by a \emph{Point}. No matter what the kind of property
-- all of them are stored as meta information in the compound model, that is
external to the parts.

The abstraction principles used in this work thereby \emph{differ} from the
\emph{Benediktine Approach} as introduced by Michael Benedikt in his article on
the structure of cyberspace \cite{benedikt}. That distinguishes \emph{extrinsic}
and \emph{intrinsic} spatial dimensions, to which the properties (attributes)
of an item (object) may be mapped. While extrinsic dimensions in a graphical
dialogue, for example, would be the positions of the buttons contained in it
(location in space), intrinsic dimensions would be those that are contained
directly in the buttons which they describe (shape, size, colour).

In other words, the \emph{Benediktine Approach} proposes to keep some
properties \emph{outside} the described model (meta knowledge) and others
\emph{inside} the model (self-knowledge). Having reflected on the principles of
human thinking, and speaking in Benediktine's terminology, this work, however,
proposes a knowledge schema (section \ref{schema_heading}) which uses solely
\emph{extrinsic} properties.
