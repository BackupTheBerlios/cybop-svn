%
% $RCSfile: structured_dna.tex,v $
%
% Copyright (C) 2002-2008. Christian Heller.
%
% Permission is granted to copy, distribute and/or modify this document
% under the terms of the GNU Free Documentation License, Version 1.1 or
% any later version published by the Free Software Foundation; with no
% Invariant Sections, with no Front-Cover Texts and with no Back-Cover
% Texts. A copy of the license is included in the section entitled
% "GNU Free Documentation License".
%
% http://www.cybop.net
% - Cybernetics Oriented Programming -
%
% http://www.resmedicinae.org
% - Information in Medicine -
%
% Version: $Revision: 1.1 $ $Date: 2008-08-19 20:41:09 $ $Author: christian $
% Authors: Christian Heller <christian.heller@tuxtax.de>
%

\paragraph{Structured DNA}
\label{structured_dna_heading}
\index{Structured DNA}
\index{Desoxy Ribo Nucleic Acid}
\index{DNA}
\index{Gene}
\index{Knowledge Schema}

A \emph{Desoxy Ribo Nucleic Acid} (DNA) molecule represents serialised
knowledge. If it was possible to bring structure into the sequence of chemical
bases that the DNA consists of, that is to recognise typical elements which all
DNA strings contain (similar to the special tokens in a markup language), a
corresponding knowledge schema might get created. One may assume that such a
schema underlies each DNA molecule. It would help identify the meaning of the
single DNA elements (\emph{Genes}).
