%
% $RCSfile: validation.tex,v $
%
% Copyright (C) 2002-2008. Christian Heller.
%
% Permission is granted to copy, distribute and/or modify this document
% under the terms of the GNU Free Documentation License, Version 1.1 or
% any later version published by the Free Software Foundation; with no
% Invariant Sections, with no Front-Cover Texts and with no Back-Cover
% Texts. A copy of the license is included in the section entitled
% "GNU Free Documentation License".
%
% http://www.cybop.net
% - Cybernetics Oriented Programming -
%
% http://www.resmedicinae.org
% - Information in Medicine -
%
% Version: $Revision: 1.1 $ $Date: 2008-08-19 20:41:09 $ $Author: christian $
% Authors: Christian Heller <christian.heller@tuxtax.de>
%

\section{Validation}
\label{validation_heading}
\index{CYBOP Validation}

The state-of-the-art chapters \ref{software_engineering_process_heading},
\ref{physical_architecture_heading} and \ref{logical_architecture_heading}, at
the beginning of this work, dealt with the \emph{Software Engineering Process}
(SEP), the \emph{Physical-} and \emph{Logical Architecture} of information
systems. A rather large number of existing software design concepts were
investigated, and some of their aspects criticised, before chapter
\ref{extended_motivation_heading} suggested a new approach for their
improvement. Many of its new concepts and ideas stem from nature or other
disciplines of science, which is why that programming approach was given the
attribute \emph{cybernetics-oriented} (CYBOP). Part \ref{contribution_heading}
then proposed a slightly different view on how to abstract knowledge in form of
software, which part \ref{proof_heading} tried to prove by introducing a
language and interpreter, as well as an application prototype using both.

In order to validate the results of this work, the following sub sections
explain once again in short why many of the problems identified in today's
programming language concepts are solved when applying CYBOP principles.

%
% $RCSfile: distinction_of_statics_and_dynamics.tex,v $
%
% Copyright (C) 2002-2008. Christian Heller.
%
% Permission is granted to copy, distribute and/or modify this document
% under the terms of the GNU Free Documentation License, Version 1.1 or
% any later version published by the Free Software Foundation; with no
% Invariant Sections, with no Front-Cover Texts and with no Back-Cover
% Texts. A copy of the license is included in the section entitled
% "GNU Free Documentation License".
%
% http://www.cybop.net
% - Cybernetics Oriented Programming -
%
% http://www.resmedicinae.org
% - Information in Medicine -
%
% Version: $Revision: 1.1 $ $Date: 2008-08-19 20:41:06 $ $Author: christian $
% Authors: Christian Heller <christian.heller@tuxtax.de>
%

\subsection{Distinction of Statics and Dynamics}
\label{distinction_of_statics_and_dynamics_heading}
\index{CYBOP Distinction of Statics and Dynamics}

A first major mistake in current language concepts and design solutions is the
mix-up of static and dynamic parts of software, that is pure application-domain
knowledge and its processing, close to hardware. It is the reason for:

\begin{itemize}
    \item[a] Abstraction gap between designed system architecture and implemented
        source code, in a SEP (section \ref{abstraction_gaps_heading})
    \item[b] Global data access via static class methods being insecure
        (section \ref{global_access_heading})
    \item[c] Bidirectional dependencies caused by some software patterns
        (section \ref{bidirectional_dependency_heading})
    \item[d] Usage of reflective techniques which are based on bidirectional
        dependencies and often cause broken type systems with circular
        references between super- and sub platform (section \ref{reflection_heading})
    \item[e] Spread functionality through crosscutting concerns and complicated
        handling of aspects (section \ref{aspect_oriented_programming_heading})
    \item[f] Memory leaks
    \item[g] Repeated implementation of the same platform-dependent functionality
    \item[h] Repeated usage and copying of the same software patterns (section
        \ref{pattern_systematics_heading})
\end{itemize}

Chapter \ref{statics_and_dynamics_heading} therefore recommended a strict distinction
of high-level static knowledge and low-level dynamic system control functionality.
The CYBOL language (chapter \ref{cybernetics_oriented_language_heading}) was
defined to express and specify knowledge in the most general sense; the CYBOI
interpreter (chapter \ref{cybernetics_oriented_interpreter_heading}) was
created to control a system based on CYBOL input.

\paragraph{a}

Since CYBOL knowledge templates are a complete formal description of an
application's architecture, they represent its implementation at the same time.
But that also means that the transfer of a system's design into a programming
language, as known from classical application development, becomes superfluous.
The \emph{Design-} and \emph{Implementation} phases are merged, so that a gap
does not exist anymore.

\paragraph{b}

Since CYBOI holds all knowledge in one single instance tree whose nodes can
be accessed along well-defined paths, data are not globally accessible anymore.

\paragraph{c}

Since parts of a knowledge model are accessed unidirectionally, bidirectional
dependencies are not an issue any longer.

\paragraph{d}

Since CYBOI is based on one standardised knowledge schema providing a
well-defined type structure which does not have to be changed at runtime, there
is neither a need nor a possibility for workarounds like reflective mechanisms
causing a broken type system. Because the knowledge schema's whole-part
hierarchy already provides meta information such as a part model's name and
kind of abstraction (comparable to an attribute's name and type hold by the
meta class of a class), one main reason for using reflective techniques like
meta classes thereby falls apart. A second reason that does not count any
longer is the provision of basic features (like persistence or communication)
through meta techniques; CYBOI already contains these features and may act as
universal communicator.

\paragraph{e}

Since CYBOI does provide all necessary low-level mechanisms, crosscutting
concerns become superfluous. These concerns usually want to achieve the same as
reflection in that they provide basic functionality to all parts of a system.
However, since CYBOL knowledge templates are free from low-level system control
information and contain pure domain knowledge instead, crosscutting concerns
and aspects are not a topic of interest any longer.

\paragraph{f}

Since CYBOI concentrates all knowledge models (instances) in one place, as
branches of one single knowledge tree, forgotten models can get smoothly
destroyed at application shutdown. Traditionally, special mechanisms like
\emph{Garbage Collectors} (GC) had to be applied to achieve this, because
systems written in classical languages leave it up to the programmer to
properly reference all instances. If a reference to one instance was lost, it
could not get destroyed and resided as leak in memory. Often, more and more
memory space got blocked that way, until all RAM space was taken and a computer
hung (crashed). In CYBOI, a reference to any instance is always available, via
the root of the knowledge tree.

\paragraph{g}

Since CYBOI contains all hardware-dependent functionality, the application
knowledge encoded in CYBOL templates or serialised models is truly
platform-neutral, easily switchable and exchangeable among systems. The
low-level system gets uninteresting; high-level knowledge is what application
developers can now concentrate on. Finally, the old dream of having knowledge
engineers (domain experts) working independently from software system engineers
might possibly be coming true.

\paragraph{h}

Since CYBOI already implements all necessary patterns, application developers
and domain experts are freed from the burden to learn and apply the same
software patterns again and again; they can now develop application systems
considering just one concept: that of hierarchical \emph{Composition}.

%
% $RCSfile: usage_of_a_double_hierarchy_knowledge_schema.tex,v $
%
% Copyright (C) 2002-2008. Christian Heller.
%
% Permission is granted to copy, distribute and/or modify this document
% under the terms of the GNU Free Documentation License, Version 1.1 or
% any later version published by the Free Software Foundation; with no
% Invariant Sections, with no Front-Cover Texts and with no Back-Cover
% Texts. A copy of the license is included in the section entitled
% "GNU Free Documentation License".
%
% http://www.cybop.net
% - Cybernetics Oriented Programming -
%
% http://www.resmedicinae.org
% - Information in Medicine -
%
% Version: $Revision: 1.1 $ $Date: 2008-08-19 20:41:09 $ $Author: christian $
% Authors: Christian Heller <christian.heller@tuxtax.de>
%

\subsection{Usage of a Double-Hierarchy Knowledge Schema}
\label{usage_of_a_double_hierarchy_knowledge_schema_heading}
\index{CYBOP Usage of a Double-Hierarchy Knowledge Schema}

A further problem that was identified in this work is the missing concept of
hierarchy, which is not inherent in types of the corresponding languages.
Moreover, knowledge structures are mixed up with meta information leading to:

\begin{itemize}
    \item[a] Inflexible static typing in system programming languages (section
        \ref{system_programming_heading})
    \item[b] Fragile base class problem when using inheritance (section
        \ref{fragile_base_class_heading})
    \item[c] Overly large source code due to encapsulation without sense
        (section \ref{encapsulation_heading})
    \item[d] Unpredictable behaviour and falsified contents due to container
        inheritance (section \ref{falsifying_polymorphism_heading})
    \item[e] Redundant code caused by concerns and difficult application of an
        ontological structure (section \ref{spread_functionality_heading})
    \item[f] Complicated, partly impossible serialisation of knowledge models
\end{itemize}

Chapter \ref{knowledge_schema_heading} therefore proposed a new knowledge
schema which considers structural- as well as meta information, in two
different hierarchies.

\paragraph{a}

Since type information is not fixed statically, it gets dynamically
configurable at runtime, leading to highly flexible application systems. There
is only one static data structure -- the standardised knowledge schema. It
holds meta information about the kind of abstraction (type) of the data
contained in it.

\paragraph{b}

Since runtime knowledge models in CYBOI rely on composition only, the fragile
base class problem caused by runtime type inheritance in object-oriented systems
does not occur.

\paragraph{c}

Since the CYBOI-internal knowledge structure is a container by default, it
also provides all necessary access procedures. Thousands of useless access
methods as known from object-oriented programming are avoided. The partial
security they provided can be replaced with other mechanisms. Since all
knowledge resides in just one instance tree, it is easy to apply any kind of
security checks, whenever a part of the knowledge tree gets accessed.

\paragraph{d}

Since each CYBOP knowledge template or -model is constructed as hierarchy, so
that containers of any kind can be emulated, problematic container inheritance
belongs to the past.

\paragraph{e}

Since CYBOP knowledge models are purely hierarchical, it gets easier to apply
ontological structures which bundle functionality, instead of spreading it in a
concern-like manner.

\paragraph{f}

Since all knowledge is modelled hierarchically, it is easily serialisable and
hence exchangeable.

%
% $RCSfile: separation_of_state_and_logic_knowledge.tex,v $
%
% Copyright (C) 2002-2008. Christian Heller.
%
% Permission is granted to copy, distribute and/or modify this document
% under the terms of the GNU Free Documentation License, Version 1.1 or
% any later version published by the Free Software Foundation; with no
% Invariant Sections, with no Front-Cover Texts and with no Back-Cover
% Texts. A copy of the license is included in the section entitled
% "GNU Free Documentation License".
%
% http://www.cybop.net
% - Cybernetics Oriented Programming -
%
% http://www.resmedicinae.org
% - Information in Medicine -
%
% Version: $Revision: 1.1 $ $Date: 2008-08-19 20:41:08 $ $Author: christian $
% Authors: Christian Heller <christian.heller@tuxtax.de>
%

\subsection{Separation of State- and Logic Knowledge}
\label{separation_of_state_and_logic_knowledge_heading}
\index{CYBOP Separation of State- and Logic Knowledge}

A third aspect causing troubles in software system design is the bundling of
state- and logic knowledge, known from object-oriented programming. It results
in:

\begin{itemize}
    \item[a] Difficult handling and repeated implementation of the same
        communication mechanisms (section \ref{misleading_tiers_heading})
    \item[b] Differing patterns complicating the handling of communication
        (section \ref{pattern_heading})
    \item[c] Bidirectional dependencies (circular references) between classes/
        objects in object-oriented systems, due to attribute-method bundling
        (section \ref{bidirectional_dependency_heading})
    \item[d] Pre-defined logic concepts in structured/ procedural- as well as
        object-oriented programming
\end{itemize}

Chapter \ref{state_and_logic_heading} therefore suggested a separation of
state- and logic concepts, in order to eliminate unnecessary inter-dependencies
and to be able to apply a unified translator architecture.

\paragraph{a}

Since low-level communication mechanisms are implemented in CYBOI,
application developers writing CYBOL knowledge templates do not have to bother
with these anymore.

\paragraph{b}

Since standard communication patterns are unified, the handling of
communication is simplified. Thanks to this unification, an extensible
translator architecture can be applied. Using it, any kind of abstract
knowledge model can be translated into any other. Universal communication
becomes possible.

\paragraph{c}

Since state- are split from logic concepts, many (partly bidirectional)
dependencies between knowledge models disappear, which reduces the coupling
between- and increases cohesion within models. Both kinds use exclusively
unidirectional relations. Additionally, logic- may access state models and each
other, but always unidirectionally.

\paragraph{d}

Since logic concepts (algorithms, workflows) are themselves modelled as CYBOL
knowledge templates, they become configurable. Traditionally, only structures
representing states are manipulatable at runtime; procedures representing logic
are fixed and cannot be altered.

