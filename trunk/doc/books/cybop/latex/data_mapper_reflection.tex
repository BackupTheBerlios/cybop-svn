%
% $RCSfile: data_mapper_reflection.tex,v $
%
% Copyright (C) 2002-2008. Christian Heller.
%
% Permission is granted to copy, distribute and/or modify this document
% under the terms of the GNU Free Documentation License, Version 1.1 or
% any later version published by the Free Software Foundation; with no
% Invariant Sections, with no Front-Cover Texts and with no Back-Cover
% Texts. A copy of the license is included in the section entitled
% "GNU Free Documentation License".
%
% http://www.cybop.net
% - Cybernetics Oriented Programming -
%
% http://www.resmedicinae.org
% - Information in Medicine -
%
% Version: $Revision: 1.1 $ $Date: 2008-08-19 20:41:06 $ $Author: christian $
% Authors: Christian Heller <christian.heller@tuxtax.de>
%

\subsubsection{Data Mapper Reflection}
\label{data_mapper_reflection_heading}
\index{Data Mapper Pattern}
\index{Domain Model}
\index{Entity Relationship Model}
\index{ERM}
\index{Database Management System}
\index{DBMS}
\index{Data Source Layer}

The most important idea of the \emph{Data Mapper} pattern is to abolish the
interdependency between domain model and data source (persistence medium). All
information about where a data source like a \emph{File} or \emph{Database}
(DB) is located, how to talk to it (File Stream, JDBC with SQL etc.) and how to
map \emph{Domain-} to \emph{Entity Relationship Model} (ERM) data is moved away
from the domain, into the data mapper layer.

This separation contributes to a clear architecture; it is not enough, though.
The data mapper layer often concentrates not only \emph{Mapping-}, but also
\emph{Communication} functionality. \emph{Database Management Systems} (DBMS)
such as PostgreSQL \cite{hartwig, postgresql, postgresql2002} or MySQL
\cite{mysql} are often treated different than normal servers. Frequently, they
are assigned a logical \emph{Data Source} layer (figure \ref{logical_figure}).
But in fact, DBMS are \emph{Systems}, as their name says, and as such need to
be addressed using special communication mechanisms (like JDBC or ODBC).

It therefore seems useful to extract all communication functionality from the
data mapper, and put it directly into the system control layer. Chapter
\ref{statics_and_dynamics_heading} explained why it is favourable to have
application knowledge separated from system control mechanisms. While
persistence/ communication mechanisms as such do not contain any domain
knowledge, mapper (translator) modules do. The remaining data mapper layer
would hence contain application-related logic knowledge, for translating data
from the domain model to the corresponding persistence model and vice-versa.

The \emph{Cybernetics Oriented Interpreter} (CYBOI) that will be introduced in
chapter \ref{cybernetics_oriented_interpreter_heading} is a system able to
handle local- as well as remote communication mechanisms. Applications written
in the \emph{Cybernetics Oriented Language} (CYBOL), introduced in chapter
\ref{cybernetics_oriented_language_heading}, will have to deliver the necessary
logic for model translation, but application developers are freed from
implementing the same low-level system communication functionality (like
sockets) again and again, leading to clearer code with greatly reduced size.
CYBOL application developers are offered a number of communication mechanisms
to choose from.
