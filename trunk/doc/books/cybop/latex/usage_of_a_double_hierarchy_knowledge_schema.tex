%
% $RCSfile: usage_of_a_double_hierarchy_knowledge_schema.tex,v $
%
% Copyright (C) 2002-2008. Christian Heller.
%
% Permission is granted to copy, distribute and/or modify this document
% under the terms of the GNU Free Documentation License, Version 1.1 or
% any later version published by the Free Software Foundation; with no
% Invariant Sections, with no Front-Cover Texts and with no Back-Cover
% Texts. A copy of the license is included in the section entitled
% "GNU Free Documentation License".
%
% http://www.cybop.net
% - Cybernetics Oriented Programming -
%
% http://www.resmedicinae.org
% - Information in Medicine -
%
% Version: $Revision: 1.1 $ $Date: 2008-08-19 20:41:09 $ $Author: christian $
% Authors: Christian Heller <christian.heller@tuxtax.de>
%

\subsection{Usage of a Double-Hierarchy Knowledge Schema}
\label{usage_of_a_double_hierarchy_knowledge_schema_heading}
\index{CYBOP Usage of a Double-Hierarchy Knowledge Schema}

A further problem that was identified in this work is the missing concept of
hierarchy, which is not inherent in types of the corresponding languages.
Moreover, knowledge structures are mixed up with meta information leading to:

\begin{itemize}
    \item[a] Inflexible static typing in system programming languages (section
        \ref{system_programming_heading})
    \item[b] Fragile base class problem when using inheritance (section
        \ref{fragile_base_class_heading})
    \item[c] Overly large source code due to encapsulation without sense
        (section \ref{encapsulation_heading})
    \item[d] Unpredictable behaviour and falsified contents due to container
        inheritance (section \ref{falsifying_polymorphism_heading})
    \item[e] Redundant code caused by concerns and difficult application of an
        ontological structure (section \ref{spread_functionality_heading})
    \item[f] Complicated, partly impossible serialisation of knowledge models
\end{itemize}

Chapter \ref{knowledge_schema_heading} therefore proposed a new knowledge
schema which considers structural- as well as meta information, in two
different hierarchies.

\paragraph{a}

Since type information is not fixed statically, it gets dynamically
configurable at runtime, leading to highly flexible application systems. There
is only one static data structure -- the standardised knowledge schema. It
holds meta information about the kind of abstraction (type) of the data
contained in it.

\paragraph{b}

Since runtime knowledge models in CYBOI rely on composition only, the fragile
base class problem caused by runtime type inheritance in object-oriented systems
does not occur.

\paragraph{c}

Since the CYBOI-internal knowledge structure is a container by default, it
also provides all necessary access procedures. Thousands of useless access
methods as known from object-oriented programming are avoided. The partial
security they provided can be replaced with other mechanisms. Since all
knowledge resides in just one instance tree, it is easy to apply any kind of
security checks, whenever a part of the knowledge tree gets accessed.

\paragraph{d}

Since each CYBOP knowledge template or -model is constructed as hierarchy, so
that containers of any kind can be emulated, problematic container inheritance
belongs to the past.

\paragraph{e}

Since CYBOP knowledge models are purely hierarchical, it gets easier to apply
ontological structures which bundle functionality, instead of spreading it in a
concern-like manner.

\paragraph{f}

Since all knowledge is modelled hierarchically, it is easily serialisable and
hence exchangeable.
