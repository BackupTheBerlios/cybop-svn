%
% $RCSfile: kernel_concepts.tex,v $
%
% Copyright (C) 2002-2008. Christian Heller.
%
% Permission is granted to copy, distribute and/or modify this document
% under the terms of the GNU Free Documentation License, Version 1.1 or
% any later version published by the Free Software Foundation; with no
% Invariant Sections, with no Front-Cover Texts and with no Back-Cover
% Texts. A copy of the license is included in the section entitled
% "GNU Free Documentation License".
%
% http://www.cybop.net
% - Cybernetics Oriented Programming -
%
% http://www.resmedicinae.org
% - Information in Medicine -
%
% Version: $Revision: 1.1 $ $Date: 2008-08-19 20:41:07 $ $Author: christian $
% Authors: Christian Heller <christian.heller@tuxtax.de>
%

\subsection{Kernel Concepts}
\label{kernel_concepts_heading}
\index{Operating System Concepts in CYBOI}
\index{OS Concepts in CYBOI}
\index{Kernel Concepts in CYBOI}
\index{CYBOI as Monolithic Kernel}
\index{CYBOI as Microkernel}
\index{CYBOI as Hybrid Kernel}
\index{CYBOI as Exokernel}
\index{CYBOI as Hardware Abstraction Layer}
\index{CYBOI avoiding Inter Process Communication}

Although certainly not qualifying as \emph{Operating System} (OS), CYBOI uses
some similar concepts. However, it is not easy to assign CYBOI to one of the four
broad categories of OS kernels, which the Wikipedia Encyclopedia \cite{wikipedia}
describes as follows:

\begin{enumerate}
    \item \emph{Monolithic Kernels:} providing rich and powerful abstractions
        of the underlying hardware
    \item \emph{Microkernels:} providing a small set of simple hardware abstractions
        and using applications called servers to provide more functionality
    \item \emph{Hybrid Kernels:} being much like pure microkernels, except that
        they include some additional code in kernelspace to increase performance
    \item \emph{Exokernels:} providing minimal abstractions but allowing the
        use of library operating sytems to provide more functionality via
        direct or nearly direct access to hardware
\end{enumerate}

Just like a \emph{Monolithic Kernel}, CYBOI encapsulates low-level functionality
like memory management and provides a high-level virtual interface (operations),
over which CYBOL applications may access it. Although the processing of signals
happens in its main process, CYBOI has to rely on a few communication services,
running in their own threads, and control their data exchange. This is (roughly)
comparable to the \emph{Microkernel} design which was mentioned as pattern in
section \ref{pattern_merger_heading} before. But these services sharing a common
address space (\emph{Internals Memory} and \emph{Signal Memory}) with the CYBOI
kernel, actually makes CYBOI a \emph{Hybrid Kernel}. The only existing data in
user address space (\emph{Knowledge Memory}) are knowledge models that have been
created from CYBOL knowledge templates. The kernel category coming closest to
CYBOI's design, however, is the \emph{Exokernel}. This is because CYBOI, though
serving as \emph{Hardware Abstraction Layer} (HAL) to CYBOL applications (what
is actually known from classic monolithic- and microkernels), also has a central
\emph{Signal Checker} control loop calling subroutines managing a part of the
hardware or software, which, after \cite{wikipedia}, were one of the simplest
methods of creating an exokernel.

The above-mentioned services provide \emph{Input/ Output} (i/o) functionality
to CYBOI so that it can communicate (\emph{virtual world}) ideas with other
(human or technical) systems, across (\emph{real world}) hardware. The services
may be configured, started up, interrupted and shutdown via CYBOL operations.
Similar to the \emph{Self Awareness} of human systems (mentioned in section
\ref{self_awareness_heading}), a CYBOL application has to know about available
i/o services, in order to actually use them. This configuration information may
be stored in CYBOL files as well.

Another thought turns around the \emph{Process} concept \cite{tanenbaum2001},
which is used to share computing time of the \emph{Central Processing Unit}
(CPU) as well as resource space in \emph{Random Access Memory} (RAM) between
applications. One then says: \textit{Every application runs in a separate
process.} The example environment in chapter \ref{physical_architecture_heading}
contained many different kinds of inter-communicating systems, not all of which
have to run on a separate physical machine, but surely in a separate process,
if on one-and-the-same machine. That way, one machine may host multiple
application systems.

However, the existence of many program processes running concurrently in an OS
holds conflicts. It necessitates ways for \emph{Inter-Process Communication}
(IPC), that assure the integrity of data in memory and avoid deadlocks
(blocking of the system). Common IPC methods include \cite{ipc, tanenbaum2001}:
\emph{Pipes} and \emph{Named Pipes}, \emph{Message Queueing}, \emph{Semaphores},
\emph{Shared Memory} and \emph{Sockets}.

A different approach was chosen in CYBOI: It is the only process running. CYBOL
applications reside as separate knowledge models in the \emph{Knowledge Memory}
(user address space), and CYBOI controls them all. This is the exact opposite
of running one process per application. However, it is unclear, at first, how
multiple applications running in parallel receive their corresponding signals.
CYBOI needs to evaluate incoming signals and then call the right logic of the
right application. But how to do that?

The application to which a signal is sent can be identified, depending on the
communication mechanism used. A \emph{keyboard\_pressed} or \emph{mouse\_clicked}
event, for example, always references the top-most window in a GUI environment.
Menu-, toolbar- and other buttons of the top-most window in turn reference a
logic (algorithm) whose dot-separated name starts with that of the application.
Within CYBOI, applications have to have a unique name, of course, so that
signals can be addressed correctly, to the right receiver. Once a logic routine
is identified, it can be sent as new signal to be processed by the signal loop.
Another example would be signals arriving over network. Also here, applications
can be identified, for example by a special \emph{Port} that was assigned to
them beforehand. A remote call references a specific logic by name so that it
can be processed locally. Equally named local procedures are distinguished by
the application name identified before. Within an application, logic names have
to be unique, of course.
