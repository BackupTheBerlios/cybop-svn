%
% $RCSfile: rdf.tex,v $
%
% Copyright (C) 2002-2008. Christian Heller.
%
% Permission is granted to copy, distribute and/or modify this document
% under the terms of the GNU Free Documentation License, Version 1.1 or
% any later version published by the Free Software Foundation; with no
% Invariant Sections, with no Front-Cover Texts and with no Back-Cover
% Texts. A copy of the license is included in the section entitled
% "GNU Free Documentation License".
%
% http://www.cybop.net
% - Cybernetics Oriented Programming -
%
% http://www.resmedicinae.org
% - Information in Medicine -
%
% Version: $Revision: 1.1 $ $Date: 2008-08-19 20:41:08 $ $Author: christian $
% Authors: Christian Heller <christian.heller@tuxtax.de>
%

\subsection{RDF}
\label{rdf_heading}
\index{Resource Description Framework}
\index{RDF}
\index{CYBOL-RDF Comparison}
\index{Uniform Resource Indicator reference}
\index{URIref}

Using the \emph{Resource Description Framework} (RDF) described in section
\ref{resource_description_framework_heading}, a catalogue of products available
at a certain domain \emph{www.example.com} might be encoded as in the following
example \cite{rdfprimer}:

\begin{scriptsize}
    \begin{verbatim}
<?xml version="1.0"?>
<!DOCTYPE rdf:RDF [<!ENTITY xsd "http://www.w3.org/2001/XMLSchema#">]>
<rdf:RDF xmlns:rdf="http://www.w3.org/1999/02/22-rdf-syntax-ns#"
    xmlns:exterms="http://www.example.com/terms/">
    <rdf:Description rdf:about="http://www.example.com/2002/04/products#item10245">
        <rdf:type rdf:resource="http://www.example.com/terms/Tent"/>
        <exterms:model rdf:datatype="&xsd;string">Overnighter</exterms:model>
        <exterms:sleeps rdf:datatype="&xsd;integer">2</exterms:sleeps>
        <exterms:weight rdf:datatype="&xsd;decimal">2.4</exterms:weight>
        <exterms:packedSize rdf:datatype="&xsd;integer">784</exterms:packedSize>
    </rdf:Description>
    ... other product descriptions ...
</rdf:RDF>
    \end{verbatim}
\end{scriptsize}

This piece of code contains just one product description: that of
\emph{item10245}, which is typed as a \emph{Tent}. It owns properties like its
\emph{model}, \emph{sleeps}, \emph{weight} and \emph{packed\_size}. The XML
namespace declaration in line four specifies that the namespace
\emph{Uniform Resource Indicator reference} (URIref)
\emph{http://www.example.org/terms/} is to be associated with the
\emph{exterms:} prefix. URIrefs beginning with the string
\emph{http://www.example.org/terms/} are used for terms from the vocabulary
defined by the example organization, \emph{example.org}. The \emph{ENTITY}
declaration specified as part of the \emph{DOCTYPE} declaration in the second
line defines the entity \emph{xsd} to be the string representing the namespace
URIref for XML Schema datatypes. This declaration allows the full namespace
URIref to be abbreviated elsewhere in the XML document by the entity reference
\emph{\&xsd;}, so that data types like \emph{string} or \emph{integer} may be
written as \emph{\&xsd;string} and \emph{\&xsd;integer}, respectively.

The same product catalogue example written in CYBOL would look like this:

\begin{scriptsize}
    \begin{verbatim}
<?xml version="1.0"?>
<model>
    <part name="item10245" channel="http" abstraction="cybol"
        model="www.example.com/2002/04/products#item10245">
        <property name="type" channel="http" abstraction="rdf" model="www.example.com/terms/Tent"/>
        <property name="model" channel="inline" abstraction="string" model="overnighter"/>
        <property name="sleeps" channel="inline" abstraction="integer" model="2"/>
        <property name="weight" channel="inline" abstraction="decimal" model="2.4"/>
        <property name="packed_size" channel="inline" abstraction="integer" model="784"/>
    </part>
    ... other products and their properties ...
</model>
    \end{verbatim}
\end{scriptsize}

One can find the product identifiable by the name \emph{item10245}, as one part
of the catalogue, as well as its properties representing meta (descriptive)
information. The product's model has the abstraction \emph{cybol} which means
that the corresponding resource is available in CYBOL format. The resource's
channel is \emph{http} which means that it has to be read using that protocol
and communication mechanism. Other abstractions are possible, of course. The
\emph{type} property in the example is available in RDF format, as indicated by
its abstraction attribute.

The meaning of the single XML attributes was explained in previous sections. Up
to now, there was no need to apply domains for building attribute- or tag
names. CYBOL's tags, that is their number as well as their names, are fixed.
The same counts for its attributes. What is changeable, are attribute
\emph{values} alone.

While RDF's main focus is on providing the means for making \emph{descriptive}
(meta) statements \emph{about} a subject, CYBOL provides these meta information
together with structural (whole-part) information, encoded in form of a double
hierarchy (section \ref{double_hierarchy_heading}).
