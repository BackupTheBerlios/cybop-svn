%
% $RCSfile: definition.tex,v $
%
% Copyright (C) 2002-2008. Christian Heller.
%
% Permission is granted to copy, distribute and/or modify this document
% under the terms of the GNU Free Documentation License, Version 1.1 or
% any later version published by the Free Software Foundation; with no
% Invariant Sections, with no Front-Cover Texts and with no Back-Cover
% Texts. A copy of the license is included in the section entitled
% "GNU Free Documentation License".
%
% http://www.cybop.net
% - Cybernetics Oriented Programming -
%
% http://www.resmedicinae.org
% - Information in Medicine -
%
% Version: $Revision: 1.1 $ $Date: 2008-08-19 20:41:06 $ $Author: christian $
% Authors: Christian Heller <christian.heller@tuxtax.de>
%

\section{Definition}
\label{definition_heading}
\index{CYBOL Definition}
\index{Language Analysis}
\index{Linguistics}
\index{Context of a Language}
\index{Semantics of a Language}
\index{Pragmatics of a Language}
\index{Lexico-Grammar of a Language}
\index{Phonology of a Language}
\index{Graphology of a Language}
\index{Orthography of a Language}
\index{Vocabulary of a Language}
\index{Terms of a Language}
\index{Whole Part Hierarchy}
\index{Meta Information Hierarchy}
\index{Extensible Markup Language}
\index{XML}

When defining a new language, several linguistic aspects have to be considered.
One way to systematise \emph{Language Analysis} as part of \emph{Linguistics}
is to stratify it \cite{odonnell} into the four fields:

\begin{itemize}
    \item[-] \emph{Context:} topic described by the language, relationships
        between discussants, channel of communication
    \item[-] \emph{Semantics:} meaning of language symbols and character strings;
        includes what is usually called \emph{Pragmatics} (\emph{Use} of language)
    \item[-] \emph{Lexico-Grammar:} syntactic organisation of words into utterances
    \item[-] \emph{Phonology-Graphology:} study of the sound system and its
        phonemes \cite{wordnet}; judging a person's character from his
        handwriting \cite{websters}
\end{itemize}

Many more (sub) fields like \emph{Orthography} and also other systematics
\cite{teachsam} exist that will not be considered further in this document.
\emph{Phonology} and \emph{Graphology} are not considered either, and the
\emph{Context} of CYBOL is very clear: It is to become a language for knowledge
modelling. The remaining fields \emph{Syntax} and \emph{Semantics} are
important for the definition of CYBOL and will get their own sections
following. Another point receiving attention is the language \emph{Vocabulary}
(\emph{Terms}) whose background in abstraction was discussed in sections
\ref{language_heading} and \ref{knowledge_abstraction_and_manipulation_heading}.

The CYBOP knowledge schema (section \ref{knowledge_representation_heading}) is
based upon two kinds of hierarchies, one representing \emph{Whole-Part}
relations and the other the \emph{Meta Information} which a whole keeps about
its parts. The syntax and semantics of CYBOL as new language must be rich
enough to express abstract models using these kinds of hierarchies.

Yet before inventing a completely new language definition, it seems useful to
make use of existing technologies and solutions. An interesting candidate is
the \emph{Extensible Markup Language} (XML), as introduced in section
\ref{extensible_markup_language_heading}.

%
% $RCSfile: syntax.tex,v $
%
% Copyright (C) 2002-2008. Christian Heller.
%
% Permission is granted to copy, distribute and/or modify this document
% under the terms of the GNU Free Documentation License, Version 1.1 or
% any later version published by the Free Software Foundation; with no
% Invariant Sections, with no Front-Cover Texts and with no Back-Cover
% Texts. A copy of the license is included in the section entitled
% "GNU Free Documentation License".
%
% http://www.cybop.net
% - Cybernetics Oriented Programming -
%
% http://www.resmedicinae.org
% - Information in Medicine -
%
% Version: $Revision: 1.1 $ $Date: 2008-08-19 20:41:09 $ $Author: christian $
% Authors: Christian Heller <christian.heller@tuxtax.de>
%

\subsection{Syntax}
\label{syntax_heading}
\index{CYBOL Syntax}
\index{Syntax of a Language}
\index{Grammar of a Language}
\index{Extensible Markup Language}
\index{XML}
\index{XML Tag}
\index{XML Attribute}
\index{Discrimination}
\index{Composition}

Every language has a special \emph{Syntax}, that is a \emph{Grammar} with rules
for combining terms and symbols \cite{foldoc}. CYBOL could define its own
syntax or use an already existing one, of another language. Because of its
popularity, clear text representation, flexibility, extensibility and ease of
use, \emph{XML} was chosen to deliver the syntax for CYBOL.

To mention just two of the syntactical elements of XML, \emph{Tag} and
\emph{Attribute} are considered shortly here. Tags are special, arbitrary
keywords that have to be defined by the system working with an XML document.
Attributes keep additional information about the contents enclosed by two tags.
Two examples:

\begin{scriptsize}
    \begin{verbatim}
    <tag attribute="value">
        contents
    </tag>
    \end{verbatim}
\end{scriptsize}

\begin{scriptsize}
    \begin{verbatim}
    <tag attribute1="value" attribute2="contents"/>
    \end{verbatim}
\end{scriptsize}

An XML document carries a name and can such represent a \emph{Discrete Item},
as suggested by the principles of human thinking (section
\ref{human_thinking_heading}). Being a \emph{Compound}, it consists of parts --
and, it can link to other documents treated as its parts. That way, a whole
hierarchy can be formed. Tag attributes can keep additional information about
the linked parts. Most importantly, XML documents have a hierarchical structure
based on tags, which may be used to store meta information about a part.

Considering these properties of XML, it seems predestinated for formally
representing abstract models using the CYBOP concepts. CYBOL, finally, is XML
\emph{plus} a defined set of tags, attributes and values, used to structure and
link documents meaningfully.

%
% $RCSfile: vocabulary.tex,v $
%
% Copyright (c) 2001-2004. Christian Heller. All rights reserved.
%
% No copying, altering, distribution or any other actions concerning this
% document, except after explicit permission by the author!
% At some later point in time, this document is planned to be put under
% the GNU FDL license. For now, _everything_ is _restricted_ by the author.
%
% http://www.cybop.net
% - Cybernetics Oriented Programming -
%
% http://www.resmedicinae.org
% - Information in Medicine -
%
% @author Christian Heller <christian.heller@tuxtax.de>
%

\subsection{Vocabulary}
\label{vocabulary_heading}

XML allows to define and exchange the whole vocabulary of a language. It offers
two ways in which a list of legal elements can be defined: The traditional
\emph{Document Type Definition} (DTD) and the more modern \emph{XML Schema
Definition} (XSD). Besides the vocabulary, DTD and XSD define the structure of
an XML document and allow to typify, constrain and validate items. The CYBOL DTD
and XSD can be found at \cite{cybop}.

%
% $RCSfile: semantics.tex,v $
%
% Copyright (c) 2002-2007. Christian Heller. All rights reserved.
%
% Permission is granted to copy, distribute and/or modify this document
% under the terms of the GNU Free Documentation License, Version 1.1 or
% any later version published by the Free Software Foundation; with no
% Invariant Sections, with no Front-Cover Texts and with no Back-Cover
% Texts. A copy of the license is included in the section entitled
% "GNU Free Documentation License".
%
% http://www.cybop.net
% - Cybernetics Oriented Programming -
%
% Version: $Revision: 1.1 $ $Date: 2007-08-01 13:59:00 $ $Author: christian $
% Authors: Christian Heller <christian.heller@tuxtax.de>
%

\section{Semantics}
\label{semantics_heading}
\index{Semantics}
\index{State Knowledge Modelling}
\index{Logic Knowledge Modelling}
\index{Extensible Markup Language}
\index{XML}
\index{XML Tag}
\index{XML Attribute}

The meaning expressed by terms and sentences is their \emph{Semantics}
\cite{duden}.

CYBOL files can be used to model \emph{State Knowledge} (like a graphical
window or a person's address) and \emph{Logic Knowledge} (like an operation or
algorithm or workflow) \cite{cybopbook}. In both cases, the \emph{same} syntax
(document structure) with \emph{identical} vocabulary (XML tags and -attributes)
is applied. It is the attribute \emph{Values} that make a difference in meaning.

The double hierarchy mentioned before is realised in CYBOL knowledge templates
by using XML \emph{Attributes} for representing the whole-part hierarchy, and
XML \emph{Tags} for representing the additional meta data that a whole model
keeps about its part models.

%
% $RCSfile: attributes.tex,v $
%
% Copyright (C) 2002-2008. Christian Heller.
%
% Permission is granted to copy, distribute and/or modify this document
% under the terms of the GNU Free Documentation License, Version 1.1 or
% any later version published by the Free Software Foundation; with no
% Invariant Sections, with no Front-Cover Texts and with no Back-Cover
% Texts. A copy of the license is included in the section entitled
% "GNU Free Documentation License".
%
% http://www.cybop.net
% - Cybernetics Oriented Programming -
%
% http://www.resmedicinae.org
% - Information in Medicine -
%
% Version: $Revision: 1.1 $ $Date: 2008-08-19 20:41:05 $ $Author: christian $
% Authors: Christian Heller <christian.heller@tuxtax.de>
%

\subsubsection{Attributes}
\label{attributes_heading}
\index{CYBOL Attributes}
\index{CYBOL 'name' Attribute}
\index{CYBOL 'channel' Attribute}
\index{CYBOL 'abstraction' Attribute}
\index{CYBOL 'model' Attribute}

Normally, an XML \emph{Attribute} keeps meta information about the contents of
an XML \emph{Tag}. In CYBOL, however, three attributes keep meta information
about a fourth attribute. The attributes, altogether, are:

\begin{itemize}
    \item[-] name
    \item[-] channel
    \item[-] abstraction
    \item[-] model
\end{itemize}

The attribute of greatest interest is \emph{model}. It contains a model either
directly, or a path to one. The \emph{channel} attribute indicates whether the
\emph{model} attribute's value is to be read from:

\begin{itemize}
    \item[-] inline
    \item[-] file
    \item[-] ftp
    \item[-] http
\end{itemize}

The \emph{abstraction} attribute specifies how to interpret the model pointed
to by the \emph{model} attribute's value. A model may be given in formats like
for example:

\newpage

\begin{itemize}
    \item[-] cybol (a state- or logic compound model encoded in CYBOL format)
    \item[-] operation (a primitive logic model)
    \item[-] string
    \item[-] double
    \item[-] integer
    \item[-] boolean
\end{itemize}

The \emph{name} attribute, finally, provides the referenced model with a unique
identifier.

While the interpretation of the \emph{model} attribute's value depends on the
\emph{channel-} and \emph{abstraction} attributes, the other three attributes
(\emph{name}, \emph{channel}, \emph{abstraction}) themselves always get
interpreted as character string.

%
% $RCSfile: tags.tex,v $
%
% Copyright (C) 2002-2008. Christian Heller.
%
% Permission is granted to copy, distribute and/or modify this document
% under the terms of the GNU Free Documentation License, Version 1.1 or
% any later version published by the Free Software Foundation; with no
% Invariant Sections, with no Front-Cover Texts and with no Back-Cover
% Texts. A copy of the license is included in the section entitled
% "GNU Free Documentation License".
%
% http://www.cybop.net
% - Cybernetics Oriented Programming -
%
% http://www.resmedicinae.org
% - Information in Medicine -
%
% Version: $Revision: 1.1 $ $Date: 2008-08-19 20:41:09 $ $Author: christian $
% Authors: Christian Heller <christian.heller@tuxtax.de>
%

\subsubsection{Tags}
\label{tags_heading}
\index{CYBOL Tags}
\index{CYBOL 'model' Tag}
\index{CYBOL 'part' Tag}
\index{CYBOL 'property' Tag}
\index{CYBOL 'constraint' Tag}

There are many kinds of meta information besides the above-mentioned
attributes, that may be known about a model. These are given in special XML
tags called \emph{property} and \emph{constraint}. As defined in section
\ref{vocabulary_heading}, a CYBOL knowledge template may use four kinds of XML
tags:

\begin{itemize}
    \item[-] model
    \item[-] part
    \item[-] property
    \item[-] constraint
\end{itemize}

The \emph{model} tag appears just once. It is the root node which makes a CYBOL
knowledge template a valid XML document.

Of actual interest are the \emph{part} tags. They identify the models that the
\emph{whole} model described by the CYBOL knowledge template consists of.

A \emph{whole} model may know a lot more about its \emph{part} models, than is
given by a part model's XML attributes. A spatial state model may know about
the \emph{position} and \emph{size} of its parts, in space. A temporal model
(such as a workflow) may have to know about the \emph{position} of its parts in
time, in order to be able to execute them in the correct order. Further, the
temporal model needs to know about the \emph{input/output} (i/o) state models
which are to be manipulated by the corresponding logic operation (part model).
The number of parts within a whole (compound) model may be limited. And so on.
These additional information are provided by \emph{property} tags whose number
is conceptually unlimited.

Not only parts need additional meta information; properties may need such
information, too. The position or size as properties of a part may have to be
constrained to certain values, such as a \emph{minimum} or \emph{maximum}. The
values of the \emph{colour} property of a part model may have to be chosen out
of a pre-defined set called \emph{choice}. Information of that kind are stated
in \emph{constraint} tags.

Since the number of possible meta information implementable in CYBOL is already
quite large and steadily growing, as the development continues, this section
cannot list them all. At a future point in time, a more-or-less complete CYBOL
specification document may be found at the CYBOP project's website \cite{cybop}.

%
% $RCSfile: tag_attribute_swapping.tex,v $
%
% Copyright (C) 2002-2008. Christian Heller.
%
% Permission is granted to copy, distribute and/or modify this document
% under the terms of the GNU Free Documentation License, Version 1.1 or
% any later version published by the Free Software Foundation; with no
% Invariant Sections, with no Front-Cover Texts and with no Back-Cover
% Texts. A copy of the license is included in the section entitled
% "GNU Free Documentation License".
%
% http://www.cybop.net
% - Cybernetics Oriented Programming -
%
% http://www.resmedicinae.org
% - Information in Medicine -
%
% Version: $Revision: 1.1 $ $Date: 2008-08-19 20:41:09 $ $Author: christian $
% Authors: Christian Heller <christian.heller@tuxtax.de>
%

\subsection{Tag-Attribute Swapping}
\label{tag_attribute_swapping_heading}
\index{CYBOL Tag-Attribute Swapping}

CYBOL swaps the meaning attributes and tags traditionally have in XML
documents, where tags represent elements that may be nested infinitely and
attributes hold additional (meta) information about a tag. Following an example
of how CYBOL might have looked that way:

\begin{scriptsize}
    \begin{verbatim}
<model>
    <part>
        <name="title"/>
        <channel="inline"/>
        <abstraction="character"/>
        <model="Res Medicinae"/>
    </part>
    <part layout="compass" position="north">
        <name="menu_bar"/>
        <channel="file"/>
        <abstraction="cybol"/>
        <model="gui/menu_bar.cybol"/>
    </part>
</model>
    \end{verbatim}
\end{scriptsize}

The current final specification of CYBOL, on the contrary, uses attributes to
define a nested element (part) and tags to give properties (meta information)
about such a nested element, in the following way:

\begin{scriptsize}
    \begin{verbatim}
<model>
    <part name="title" channel="inline" abstraction="character" model="Res Medicinae"/>
    <part name="menu_bar" channel="file" abstraction="cybol" model="gui/menu_bar.cybol">
        <property name="layout" channel="inline" abstraction="character" model="compass"/>
        <property name="position" channel="inline" abstraction="character" model="north"/>
    </part>
</model>
    \end{verbatim}
\end{scriptsize}

This is because:

\begin{enumerate}
    \item the number of attributes specifying a part in CYBOL is fixed, whereas
        the number of tags specifying a property of a part is not, and the
        number of XML tags is easier extensible than that of attributes;
    \item that way it is also possible to specify a part without any properties
        in just one CYBOL code line, while otherwise four tags would always
        have to be given;
    \item not only a part may be nested (consist of smaller parts), but also a
        property may be (for example a position consisting of three coordinates
        given as parts), which necessitates the four standard attributes to be
        given for properties and constraints as well.
\end{enumerate}


%
% $RCSfile: tag_attribute_swapping.tex,v $
%
% Copyright (C) 2002-2008. Christian Heller.
%
% Permission is granted to copy, distribute and/or modify this document
% under the terms of the GNU Free Documentation License, Version 1.1 or
% any later version published by the Free Software Foundation; with no
% Invariant Sections, with no Front-Cover Texts and with no Back-Cover
% Texts. A copy of the license is included in the section entitled
% "GNU Free Documentation License".
%
% http://www.cybop.net
% - Cybernetics Oriented Programming -
%
% http://www.resmedicinae.org
% - Information in Medicine -
%
% Version: $Revision: 1.1 $ $Date: 2008-08-19 20:41:09 $ $Author: christian $
% Authors: Christian Heller <christian.heller@tuxtax.de>
%

\subsection{Tag-Attribute Swapping}
\label{tag_attribute_swapping_heading}
\index{CYBOL Tag-Attribute Swapping}

CYBOL swaps the meaning attributes and tags traditionally have in XML
documents, where tags represent elements that may be nested infinitely and
attributes hold additional (meta) information about a tag. Following an example
of how CYBOL might have looked that way:

\begin{scriptsize}
    \begin{verbatim}
<model>
    <part>
        <name="title"/>
        <channel="inline"/>
        <abstraction="character"/>
        <model="Res Medicinae"/>
    </part>
    <part layout="compass" position="north">
        <name="menu_bar"/>
        <channel="file"/>
        <abstraction="cybol"/>
        <model="gui/menu_bar.cybol"/>
    </part>
</model>
    \end{verbatim}
\end{scriptsize}

The current final specification of CYBOL, on the contrary, uses attributes to
define a nested element (part) and tags to give properties (meta information)
about such a nested element, in the following way:

\begin{scriptsize}
    \begin{verbatim}
<model>
    <part name="title" channel="inline" abstraction="character" model="Res Medicinae"/>
    <part name="menu_bar" channel="file" abstraction="cybol" model="gui/menu_bar.cybol">
        <property name="layout" channel="inline" abstraction="character" model="compass"/>
        <property name="position" channel="inline" abstraction="character" model="north"/>
    </part>
</model>
    \end{verbatim}
\end{scriptsize}

This is because:

\begin{enumerate}
    \item the number of attributes specifying a part in CYBOL is fixed, whereas
        the number of tags specifying a property of a part is not, and the
        number of XML tags is easier extensible than that of attributes;
    \item that way it is also possible to specify a part without any properties
        in just one CYBOL code line, while otherwise four tags would always
        have to be given;
    \item not only a part may be nested (consist of smaller parts), but also a
        property may be (for example a position consisting of three coordinates
        given as parts), which necessitates the four standard attributes to be
        given for properties and constraints as well.
\end{enumerate}

%%
% $RCSfile: expressiveness.tex,v $
%
% Copyright (C) 2002-2008. Christian Heller.
%
% Permission is granted to copy, distribute and/or modify this document
% under the terms of the GNU Free Documentation License, Version 1.1 or
% any later version published by the Free Software Foundation; with no
% Invariant Sections, with no Front-Cover Texts and with no Back-Cover
% Texts. A copy of the license is included in the section entitled
% "GNU Free Documentation License".
%
% http://www.cybop.net
% - Cybernetics Oriented Programming -
%
% http://www.resmedicinae.org
% - Information in Medicine -
%
% Version: $Revision: 1.1 $ $Date: 2008-08-19 20:41:06 $ $Author: christian $
% Authors: Christian Heller <christian.heller@tuxtax.de>
%

\subsection{Expressiveness}
\label{expressiveness_heading}

- show how the issues criticised in \ref{extensible_markup_language_heading} are addressed
- for example, the semantics is not contained in tags/attributes, but \emph{inherent}
in the hierarchical meta structure of CYBOL

Q: Why doesn't CYBOL use the standard structure as intended by XML,
that is tags for parts and attributes for meta information (properties)?

A1: The parts belonging to one knowledge template may have an unpredictable
number of properties (meta information). If attributes were used to model
these, thousands, may be millions of attribute names of all knowledge templates
ever to be created would have to be known beforehand. But this would break
CYBOL's flexibility and would not be future-proof.

A2: The CYBOP knowledge schema relies on the fixed numbers of four different
tags and four different attributes. Only because these numbers are fixed, it
was possible at all to create a general schema covering all kinds of knowledge.

A3: Properties (meta information) are themselves models that have a name,
channel, abstraction (type) and primitive or compound value. If properties were
put into XML attributes, these information could not be given.

A4: Moreover, it is possible to keep meta information, namely constraints,
about properties, too. This wouldn't be possible with properties as attributes.

- merger of all classes into just one (knowledge schema) in CYBOI
OOP Klassen merge meta information and actual content (part models)
- big disadvantage since information is not expressive (semantic info is missing)
- therefore so many efforts are dealing with this problem (semantic web, ontologies etc.)

==

\cite{w3c}
An XSLT stylesheet specifies the presentation of a class of XML documents by
describing how an instance of the class is transformed into an XML document
that uses a formatting vocabulary, such as (X)HTML or XSL-FO.

XSL is a language for expressing style sheets. An XSL style sheet is, like
with CSS, a file that describes how to display an XML document of a given type.
XSL shares the functionality and is compatible with CSS2 (although it uses a
different syntax). It also adds:
- A transformation language for XML documents: XSLT. Originally intended to
perform complex styling operations, like the generation of tables of contents
and indexes, it is now used as a general purpose XML processing language.
XSLT is thus widely used for purposes other than XSL, like generating HTML web
pages from XML data.
- Advanced styling features, expressed by an XML document type which defines a
set of elements called Formatting Objects, and attributes (in part borrowed from
CSS2 properties and adding more complex ones.

XML for Ontology:
XML is widely predicted to improve the degree of interoperation between commerce
agents on the Internet. Yet XML does not address ontology and provides only a
syntactic representation of knowledge. For this reason, many Internet commerce
initiatives are developing taxonomies to support XML-based interoperation.
These developments mostly focus upon the identification of standard "tags",
and not the underlying ontology. Interoperation is therefore dependent upon
each trading partner agreeing to use particular tag sets and using these
consistently. It is not clear that this strategy will achieve the degree of
interoperation and flexibility sought by Ontology.Org.

>"x++ is the world's first object-oriented language that is entirely
>based on XML's syntactical structure. x++ conforms with the XML version
>1.0 specification..."
>
>  http://xplusplus.sourceforge.net/index.htm

http://avalon.apache.org/sandbox/merlin/meta/
- XML partly used for description/ specification/ configuration of components
- compare meta model to CYBOP!
- also: Intershop
- XML in general used by configuration/ property files of more and more applications

http://www.ontology.org/main/papers/faq.html
How does ontology relate to the eXtensible Markup Language (XML)?

\emph{Extensible Stylesheet Language} (XSL) is a family of recommendations for
defining XML document transformation and presentation. It consists of three parts:
\emph{XSL Transformations} (XSLT), \emph{XML Path Language} (XPath) and
\emph{XSL Formatting Objects} (XSL-FO).

%http://www.oasis-open.org/committees/tc\_home.php?wg_abbrev=cam
%see document /edu/grad/tmp/oasis\_cam\_technical\_brochure.pdf
%for short description (on first page) of XSD, XSLT, XPath, RDF, OWL

FORMATS: XML, DTD, SGML, DOCBOOK, RELAX NG, Schematron, XML Schema, XSLT
%\cite{} [http://xml.ascc.net/resource/schematron/schematron.html]
%The Schematron (description + code example)

Q: How to put required relations into CYBOL models (see \ref{feature_model_heading})?
Example: If the application has GUI capabilities, then use graphical dialogues
to report log messages!
A: It does not make sense to put this constraint into the GUI model.
It should be stored in the user configuration settings instead.
At system startup, a logic model would then normally read the user settings
and set a flag for the logger to use graphical dialogues instead of textual messages.
--> That means, as a recommendation to application developers, that not all
constraints should be put into the \emph{constraint} tag! Rather, the developer
should take some time to bethink, and sort all information where it actually belongs,
leaving the single knowledge models as independent from each other as possible.

