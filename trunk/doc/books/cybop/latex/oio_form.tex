%
% $RCSfile: oio_form.tex,v $
%
% Copyright (C) 2002-2008. Christian Heller.
%
% Permission is granted to copy, distribute and/or modify this document
% under the terms of the GNU Free Documentation License, Version 1.1 or
% any later version published by the Free Software Foundation; with no
% Invariant Sections, with no Front-Cover Texts and with no Back-Cover
% Texts. A copy of the license is included in the section entitled
% "GNU Free Documentation License".
%
% http://www.cybop.net
% - Cybernetics Oriented Programming -
%
% http://www.resmedicinae.org
% - Information in Medicine -
%
% Version: $Revision: 1.1 $ $Date: 2008-08-19 20:41:07 $ $Author: christian $
% Authors: Christian Heller <christian.heller@tuxtax.de>
%

\subsection{OIO Form}
\label{oio_form_heading}

\emph{OIO Form}
Andrew Ho; OpenHealth Mailing List, December 2003 \cite{openhealth}

<form>
<item>
  <name>sBP</name>
  <description>systemic arterial blood pressure,supine</description>
  <prompt>sBP?</prompt>
  <itemtype>pressure</itemtype>
</item>
<itemtype>
  <name>pressure</name>
  <description>mmHg</description>
  <action>number</action>
  <choice></choice>
</itemtype>
</form>

OpenHealth Mailing List, December 2003 (relates to the OIO form above) \cite{openhealth}
Andrew Ho answers to Thomas Beale on the topic of OpenEHR Archetypes vs. OIO Forms

this form does not say anything about:
- systolic or diastolic (or other phases) of blood pressure
>
Just add it. Instead of "sBP" - call it "Systolic BP".
>
- what data type blood pressure is recorded as
>
The itemtype is "pressure". That is the "data type".
>
- what will you do if you want the patient standing - another form with
'standing' instead of 'supine'?
>
You can use another form or another item (a posture item).
>
- what if you want to record a time-series of BPs, say 3, with the patient
in a different position prior to each?
>
Just have 3 question items on the same form - or have 3 forms. Your choice.
>
- what if you want to record the protocol, i.e. instrument, cuff size
>
Add a question item called "protocol".
>
- how will you provide validation on data input, to e.g.
+ to ensure that
valid data types are used for each entered item (e.g. a Date is not entered
for diastolic BP)?
>
Use action=number
>
+ ensure that the systolic bp is in some sane range, e.g. 0-500mmHg?

Use action=number and choice='>20,<500', for example.
>
+ ensure that sensble positions of patient are offered to the user?

Communicate the "sensible positions" via the question prompt.
>
- how will you indicate that the data created by this form was in fact
created by this form (what is its id?), so that my software can process
the data according to the form's model of BP?
>
All data created via this OIO form are stored and labeled with the name of
this OIO form.
