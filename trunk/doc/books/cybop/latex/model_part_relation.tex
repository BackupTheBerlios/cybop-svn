%
% $RCSfile: model_part_relation.tex,v $
%
% Copyright (C) 2002-2008. Christian Heller.
%
% Permission is granted to copy, distribute and/or modify this document
% under the terms of the GNU Free Documentation License, Version 1.1 or
% any later version published by the Free Software Foundation; with no
% Invariant Sections, with no Front-Cover Texts and with no Back-Cover
% Texts. A copy of the license is included in the section entitled
% "GNU Free Documentation License".
%
% http://www.cybop.net
% - Cybernetics Oriented Programming -
%
% http://www.resmedicinae.org
% - Information in Medicine -
%
% Version: $Revision: 1.1 $ $Date: 2008-08-19 20:41:07 $ $Author: christian $
% Authors: Christian Heller <christian.heller@tuxtax.de>
%

\subsubsection{Model-Part Relation}
\label{model_part_relation_heading}
\index{CYBOL Model-Part Relation Example}
\index{DocBook DTD}
\index{The Linux Documentation Project}
\index{TLDP}

The DocBook DTD \cite{docbook} is one of many well-known specifications for
structuring documents. The Linux Documentation Project \cite{linuxdoc} makes
heavy use of it. DocBook is based on numerous XML tags with defined meaning.

The following example shows how parts of a \emph{Text Document} can be modelled
differently, with at most four tags, using CYBOL:

\begin{scriptsize}
    \begin{verbatim}
<model>
    <part name="title" channel="inline" abstraction="string" model="Quo Vadis"/>
    <part name="author" channel="inline" abstraction="string" model="Henryk Sienkiewicz"/>
    <part name="date" channel="inline" abstraction="date" model="1896-01-01"/>
    <part name="contents" channel="file" abstraction="cybol" model="contents.cybol"/>
    <part name="chapter_1" channel="file" abstraction="cybol" model="chapter_1.cybol"/>
    <part name="chapter_2" channel="file" abstraction="cybol" model="chapter_2.cybol"/>
    <part name="chapter_3" channel="file" abstraction="cybol" model="chapter_3.cybol"/>
    <part name="appendix" channel="file" abstraction="cybol" model="appendix.cybol"/>
</model>
    \end{verbatim}
\end{scriptsize}
