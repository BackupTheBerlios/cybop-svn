%
% $RCSfile: tools.tex,v $
%
% Copyright (C) 2002-2008. Christian Heller.
%
% Permission is granted to copy, distribute and/or modify this document
% under the terms of the GNU Free Documentation License, Version 1.1 or
% any later version published by the Free Software Foundation; with no
% Invariant Sections, with no Front-Cover Texts and with no Back-Cover
% Texts. A copy of the license is included in the section entitled
% "GNU Free Documentation License".
%
% http://www.cybop.net
% - Cybernetics Oriented Programming -
%
% http://www.resmedicinae.org
% - Information in Medicine -
%
% Version: $Revision: 1.1 $ $Date: 2008-08-19 20:41:09 $ $Author: christian $
% Authors: Christian Heller <christian.heller@tuxtax.de>
%

\subsection{Tools}
\label{tools_heading}
\index{Res Medicinae Development Tools}

Classical application development relies on tools like a \emph{UML Designer}, for
creating \emph{Unified Modeling Language} (UML) diagrams, a \emph{Text Editor},
\emph{Compiler} and \emph{Debugger}. Nowadays, these and other tools are
offered in one package, as \emph{Integrated Development Environment} (IDE).

Because the CYBOI interpreter is written in the system programming language
\emph{C}, its development requires a compiler. CYBOL applications, on the other
hand, do not have to be compiled. They base on interpreted XML code which can
be written in every text editor; nothing else is needed. An adapted editor was
proposed in section \ref{template_editor_heading}.

Res Medicinae development could certainly be speeded up by using graphical
diagrams in the style of the UML. But unfortunately, design tools that directly
support CYBOP do not exist yet. As section \ref{knowledge_designer_heading}
tried to show, some UML diagrams could be used with only minor adaptations for
CYBOL modelling. For the time being, standard XML editors have to suffice.

For running and testing CYBOL applications, of course, the CYBOI interpreter is
needed.
