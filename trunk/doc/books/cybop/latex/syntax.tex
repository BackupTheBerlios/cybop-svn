%
% $RCSfile: syntax.tex,v $
%
% Copyright (C) 2002-2008. Christian Heller.
%
% Permission is granted to copy, distribute and/or modify this document
% under the terms of the GNU Free Documentation License, Version 1.1 or
% any later version published by the Free Software Foundation; with no
% Invariant Sections, with no Front-Cover Texts and with no Back-Cover
% Texts. A copy of the license is included in the section entitled
% "GNU Free Documentation License".
%
% http://www.cybop.net
% - Cybernetics Oriented Programming -
%
% http://www.resmedicinae.org
% - Information in Medicine -
%
% Version: $Revision: 1.1 $ $Date: 2008-08-19 20:41:09 $ $Author: christian $
% Authors: Christian Heller <christian.heller@tuxtax.de>
%

\subsection{Syntax}
\label{syntax_heading}
\index{CYBOL Syntax}
\index{Syntax of a Language}
\index{Grammar of a Language}
\index{Extensible Markup Language}
\index{XML}
\index{XML Tag}
\index{XML Attribute}
\index{Discrimination}
\index{Composition}

Every language has a special \emph{Syntax}, that is a \emph{Grammar} with rules
for combining terms and symbols \cite{foldoc}. CYBOL could define its own
syntax or use an already existing one, of another language. Because of its
popularity, clear text representation, flexibility, extensibility and ease of
use, \emph{XML} was chosen to deliver the syntax for CYBOL.

To mention just two of the syntactical elements of XML, \emph{Tag} and
\emph{Attribute} are considered shortly here. Tags are special, arbitrary
keywords that have to be defined by the system working with an XML document.
Attributes keep additional information about the contents enclosed by two tags.
Two examples:

\begin{scriptsize}
    \begin{verbatim}
    <tag attribute="value">
        contents
    </tag>
    \end{verbatim}
\end{scriptsize}

\begin{scriptsize}
    \begin{verbatim}
    <tag attribute1="value" attribute2="contents"/>
    \end{verbatim}
\end{scriptsize}

An XML document carries a name and can such represent a \emph{Discrete Item},
as suggested by the principles of human thinking (section
\ref{human_thinking_heading}). Being a \emph{Compound}, it consists of parts --
and, it can link to other documents treated as its parts. That way, a whole
hierarchy can be formed. Tag attributes can keep additional information about
the linked parts. Most importantly, XML documents have a hierarchical structure
based on tags, which may be used to store meta information about a part.

Considering these properties of XML, it seems predestinated for formally
representing abstract models using the CYBOP concepts. CYBOL, finally, is XML
\emph{plus} a defined set of tags, attributes and values, used to structure and
link documents meaningfully.
