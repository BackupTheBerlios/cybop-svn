%
% $RCSfile: compositional_scheme.tex,v $
%
% Copyright (C) 2002-2008. Christian Heller.
%
% Permission is granted to copy, distribute and/or modify this document
% under the terms of the GNU Free Documentation License, Version 1.1 or
% any later version published by the Free Software Foundation; with no
% Invariant Sections, with no Front-Cover Texts and with no Back-Cover
% Texts. A copy of the license is included in the section entitled
% "GNU Free Documentation License".
%
% http://www.cybop.net
% - Cybernetics Oriented Programming -
%
% http://www.resmedicinae.org
% - Information in Medicine -
%
% Version: $Revision: 1.1 $ $Date: 2008-08-19 20:41:06 $ $Author: christian $
% Authors: Christian Heller <christian.heller@tuxtax.de>
%

\subsubsection{Compositional Scheme}
\label{compositional_scheme_heading}
\index{Compositional Scheme}
\index{Compositional Conceptual Scheme}
\index{Dictionary}
\index{Generalised Architecture for Languages, Encyclopedias and Nomenclatures in Med.}
\index{GALEN}
\index{Systematized Nomenclature of Medicine}
\index{SNOMED}
\index{Enumerative-compositional Scheme}
\index{Logical Observation Identifiers, Names and Codes}
\index{LOINC}
\index{International Classification of Nursing Procedures}
\index{ICNP}
\index{Combinatorial Explosion}

A \emph{Compositional Conceptual Scheme} typically contains a \emph{controlled}
and \emph{fixed} list (\emph{Dictionary}) of a relatively small number (a few
ten-thousand) of \emph{primitive} terms, each of which can have a unique code.
These primitives may be combined together by users to form more complex terms,
including those which might be found in an existing enumerative scheme but also
other, sometimes trivial, variations and expansions \cite{rogers}.

Examples of compositional schemes include the \emph{Generalised Architecture
for Languages, Encyclopaedias and Nomenclatures in Medicine} (GALEN) and the
\emph{Systematized Nomenclature of Medicine} (SNOMED). Hybrid
enumerative-compositional schemes are
\emph{Logical Observation Identifiers, Names and Codes} (LOINC) and the
\emph{International Classification of Nursing Procedures} (ICNP).

The sheer unlimited number of possible combinations, when seen as a problem, is
called \emph{Combinatorial Explosion}. Much worse problems, however, are the:

\begin{itemize}
    \item[-] \emph{Nonsense} combinations that may be constructed
        (avoidable with a set of semantic links, a grammar and constraints)
    \item[-] \emph{Redundancy} which occurs when more than one combination of
        terms express the same concept
        (avoidable with formal algorithms helping to identify redundant compositions)
    \item[-] \emph{Post-hoc Classification} (unforeseeable addition of new,
        unknown concepts) that may prevent a meaningful data analysis
        (avoidable with a type hierarchy of primitives and of semantics links)
    \item[-] \emph{Intractability} of data due to \emph{exploding} computer
        algorithms so that the computer will never find an answer
\end{itemize}
