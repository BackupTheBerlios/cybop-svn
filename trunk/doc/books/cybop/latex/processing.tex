%
% $RCSfile: processing.tex,v $
%
% Copyright (C) 2002-2008. Christian Heller.
%
% Permission is granted to copy, distribute and/or modify this document
% under the terms of the GNU Free Documentation License, Version 1.1 or
% any later version published by the Free Software Foundation; with no
% Invariant Sections, with no Front-Cover Texts and with no Back-Cover
% Texts. A copy of the license is included in the section entitled
% "GNU Free Documentation License".
%
% http://www.cybop.net
% - Cybernetics Oriented Programming -
%
% http://www.resmedicinae.org
% - Information in Medicine -
%
% Version: $Revision: 1.1 $ $Date: 2008-08-19 20:41:08 $ $Author: christian $
% Authors: Christian Heller <christian.heller@tuxtax.de>
%

\subsection{Processing}
\label{processing_heading}
\index{Processing of Knowledge}
\index{Signal}
\index{Event}
\index{Signal Memory}
\index{Event Queue}
\index{Signal Loop}
\index{Waiting Loop}
\index{Priority of a Signal}
\index{Prioritising}
\index{Operating System}
\index{OS}
\index{Intra System Communication}
\index{Inter System Communication}
\index{Message}

While knowledge as such is static at a given time instant, its \emph{Processing}
and manipulation over time are dynamic. The processing is triggered by some
\emph{Signal} (also called \emph{Event}), which is a state change known to the
system. Such \textit{signs with defined meaning}, as the Duden Encyclopedia
\cite{duden} calls them, can be most different in their appearance and
communication channel used.

Signals are commonly stored in a \emph{Signal Memory} (also called
\emph{Event Queue}), as mentioned in the previous section. An endlessly running
\emph{Signal Loop} (also called \emph{Waiting Loop}) as illustrated in figure
\ref{system_figure} is constantly checking the signal memory for new signals.
Once a signal is detected, it gets removed from the signal memory and handled
by the system. The signal with highest \emph{Priority} is processed first. The
later chapter \ref{cybernetics_oriented_interpreter_heading} will explain
further details and deliver a more functional illustration (figure
\ref{dependencies_figure}).

Section \ref{brain_regions_heading} mentioned the \emph{Hypothalamus} and
\emph{Limbic System} as parts of the human brain producing emotions. Section
\ref{information_processing_model_heading} wrote that the processing of a signal
may be greatly influenced by the meaningfulness or \emph{Emotional Content} of
an item. Well, software systems do not work with emotions, but signals can be
assigned a \emph{Priority}, which is somewhat comparable. \emph{Prioritising}
as technique stems from \emph{Operating System} (OS) research and can be well
applied in the described knowledge-processing system: Signals can be filtered
in a way that unimportant signals get discarded; urgent signals get processed
right away; less important but meaningful signals get queued for later handling.

All \emph{intra-system} and \emph{inter-system} communication is based on the
exchange of knowledge via signals. A signal can transport simple or more complex
\emph{Messages}, mostly in encoded form. The communication details, including
encoding and decoding procedures for knowledge model translation, and the logic
after which an input state gets transferred into an output state are the topic
of chapter \ref{state_and_logic_heading}.

Besides the \emph{declarative} \emph{Long Term Memory} (LTM), section
\ref{short_and_long_term_memory_heading} mentioned the \emph{procedural}
(non-declarative) LTM, enabling humans to carry out a \emph{Background Task},
without having to consciously control it. A similar principle is applied for
input/ output (i/o) handling, in the described knowledge processing system.
Independent \emph{Threads} running their own loops control a special i/o
mechanism (like \emph{UNIX socket etc.}), each (figure \ref{system_figure}).
