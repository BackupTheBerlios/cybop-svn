%
% $RCSfile: reference_and_archetype_model.tex,v $
%
% Copyright (C) 2002-2008. Christian Heller.
%
% Permission is granted to copy, distribute and/or modify this document
% under the terms of the GNU Free Documentation License, Version 1.1 or
% any later version published by the Free Software Foundation; with no
% Invariant Sections, with no Front-Cover Texts and with no Back-Cover
% Texts. A copy of the license is included in the section entitled
% "GNU Free Documentation License".
%
% http://www.cybop.net
% - Cybernetics Oriented Programming -
%
% http://www.resmedicinae.org
% - Information in Medicine -
%
% Version: $Revision: 1.1 $ $Date: 2008-08-19 20:41:08 $ $Author: christian $
% Authors: Christian Heller <christian.heller@tuxtax.de>
%

\subsection{Reference- and Archetype Model}
\label{reference_and_archetype_model_heading}
\index{Reference Model}
\index{RM}
\index{Archetype Model}
\index{AM}
\index{Archetype}
\index{Archetype Definition Language}
\index{ADL}
\index{Dual Model Approach}

The \emph{Archetype} concept as introduced in section \ref{archetype_heading}
\emph{does} provide an independent implementation technique (language) for the
definition of application-specific domain knowledge: the
\emph{Archetype Definition Language} (ADL). The documents written in it,
altogether, are referred to as \emph{Archetype Model} (AM). They get parsed and
instantiated at runtime. These instances are then used to constrain instances
of a \emph{Reference Model} (RM). Because of the existence of two models
implemented with two independent techniques, this method of programming is
called \emph{Dual Model Approach} (section \ref{dual_model_approach_heading}).

It wants to solve the dilemma of lacking domain semantics in classical
information models. Archetypes are the corresponding knowledge documents
carrying semantic information. They provide the structures and rules after
which instances of an RM can be combined meaningfully. Despite its drawbacks
mentioned in section \ref{dual_model_approach_heading}, the dual model approach
animated this work to pay attention to two things:

\begin{enumerate}
    \item the usage of different implementation technologies for domain
        knowledge (AM) and underlying system-level functionality (RM)
    \item the need to provide constraint information with knowledge models
\end{enumerate}

The distinction between domain knowledge and system-level functionality is
realised by providing a knowledge modelling language (chapter
\ref{cybernetics_oriented_language_heading}) and a corresponding interpreter
(chapter \ref{cybernetics_oriented_interpreter_heading}). The language is
capable of expressing structural- as well as meta information, to which also
belong constraints.
