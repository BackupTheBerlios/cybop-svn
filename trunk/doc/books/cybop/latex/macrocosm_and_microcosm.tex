%
% $RCSfile: macrocosm_and_microcosm.tex,v $
%
% Copyright (C) 2002-2008. Christian Heller.
%
% Permission is granted to copy, distribute and/or modify this document
% under the terms of the GNU Free Documentation License, Version 1.1 or
% any later version published by the Free Software Foundation; with no
% Invariant Sections, with no Front-Cover Texts and with no Back-Cover
% Texts. A copy of the license is included in the section entitled
% "GNU Free Documentation License".
%
% http://www.cybop.net
% - Cybernetics Oriented Programming -
%
% http://www.resmedicinae.org
% - Information in Medicine -
%
% Version: $Revision: 1.1 $ $Date: 2008-08-19 20:41:07 $ $Author: christian $
% Authors: Christian Heller <christian.heller@tuxtax.de>
%

\subsubsection{Macrocosm and Microcosm}
\label{macrocosm_and_microcosm_heading}
\index{Macrocosm as Part of an Ontology}
\index{Microcosm as Part of an Ontology}
\index{Astronomical Particles as Ontology}
\index{Universe}
\index{Top Level Model}
\index{Concept}
\index{Electronic Health Record}
\index{EHR}
\index{Electronic Insurance Record}
\index{EIR}

Table \ref{astronomical_table} lists \emph{Astronomical Particles}
\cite{fernandezdavid, arnett}. It ends with the \emph{Universe} and an
undefinable \emph{Macrocosm}.

\begin{table}[ht]
    \begin{center}
        \begin{footnotesize}
        \begin{tabular}{| p{40mm} | p{65mm} |}
            \hline
            \textbf{Category} & \textbf{Example Model}\\
            \hline
            Macrocosm & (Infinity)\\
            \hline
            Universe & Our Universe with its Laws of Nature\\
            \hline
            Heap of Galaxies & Local Group, Heap of Virgo\\
            \hline
            Galaxy & Milky Way (our), Andromeda, Magellan's Clouds\\
            \hline
            Planetary (Solar) System & Sol (Sun), 51 Pegasi\\
            \hline
            Star/ Planet & Beta Pictoris, Mercury, Venus, Earth, Mars\\
            \hline
        \end{tabular}
        \end{footnotesize}
        \caption{Astronomical Particles}
        \label{astronomical_table}
    \end{center}
\end{table}

When trying to abstract things (in software), there has to be some limit, a
kind of \emph{Top Level Model}. It represents the \emph{Concept} to be
described. For a medical information system, one such top level model will be
the \emph{Electronic Health Record} (EHR); for an insurance application, it
will be the \emph{Electronic Insurance Record} (EIR); and so on.

Models do not only have to be limited \emph{upwards}; the same holds true for
modelling towards \emph{Microcosm}. Table \ref{physical_table} organises
particles, as used by natural sciences, into several categories.

\begin{table}[ht]
    \begin{center}
        \begin{footnotesize}
        \begin{tabular}{| p{50mm} | p{55mm} |}
            \hline
            \textbf{Category} & \textbf{Example Model}\\
            \hline
            Physical Compound & Air, Water, Fire, Ground\\
            \hline
            Chemical Compound/ Molecule & H$_{2}$, O$_{2}$, O$_{3}$, H$_{2}$O\\
            \hline
            Crystal & C (Diamond)\\
            \hline
            Atom (Chemical Element) & H, He, O\\
            \hline
            Elementary Particle & Quark, Lepton (Electron, Neutrino)\\
            \hline
            Urelement & (Primary Particle)\\
            \hline
            Microcosm & (Infinity)\\
            \hline
        \end{tabular}
        \end{footnotesize}
        \caption{Physical Particles}
        \label{physical_table}
    \end{center}
\end{table}

Although the real world seems to be built like that (infinite, nobody knowing
what comes beneath the \emph{Quark} particles) -- in software modelling it
makes no sense (and is actually impossible) to neverendingly introduce lower
and lower levels, towards \emph{Microcosm}. On some point, the hierarchy has to
be stopped, to be able to abstract it in software. The later chapter
\ref{state_and_logic_heading} gives an overview of common knowledge primitives.
