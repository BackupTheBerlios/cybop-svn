%
% $RCSfile: healthcare_domain_task_force.tex,v $
%
% Copyright (C) 2002-2008. Christian Heller.
%
% Permission is granted to copy, distribute and/or modify this document
% under the terms of the GNU Free Documentation License, Version 1.1 or
% any later version published by the Free Software Foundation; with no
% Invariant Sections, with no Front-Cover Texts and with no Back-Cover
% Texts. A copy of the license is included in the section entitled
% "GNU Free Documentation License".
%
% http://www.cybop.net
% - Cybernetics Oriented Programming -
%
% http://www.resmedicinae.org
% - Information in Medicine -
%
% Version: $Revision: 1.1 $ $Date: 2008-08-19 20:41:07 $ $Author: christian $
% Authors: Christian Heller <christian.heller@tuxtax.de>
%

\subsubsection{Healthcare Domain Task Force}
\label{healthcare_domain_task_force_heading}
\index{Object Management Group}
\index{OMG}
\index{Model Driven Architecture}
\index{MDA}
\index{Common Object Request Broker Architecture}
\index{CORBA}
\index{Interface Definition Language}
\index{IDL}
\index{Object Request Broker}
\index{ORB}
\index{Internet Inter ORB Protocol}
\index{IIOP}
\index{Healthcare Domain Taskforce}
\index{HDTF}
\index{CORBAmed}
\index{Person (Patient) Identification Service}
\index{PIDS}
\index{Lexicon (Terminology) Query Service}
\index{LQS}
\index{Clinical Observations Access Service}
\index{COAS}
\index{Resource Access Decision Service}
\index{RADS}
\index{Clinical Image Access Service}
\index{CIAS}

The \emph{Object Management Group} (OMG) whose mission is: \textit{to help
computer users solve integration problems by supplying open, vendor-neutral
interoperability specifications}, is the creator of widely-used de facto
standards (section \ref{model_driven_architecture_heading}) like UML, MOF, XMI,
CWM or CORBA, all now belonging to the \emph{Model Driven Architecture} (MDA).

Published in the 1990s, the \emph{Common Object Request Broker Architecture}
(CORBA) was one of the first standards specifications created by the OMG. It
tried to separate interfaces of programming objects (components) from their
implementation, to improve communication between programs, independent of which
programming language, operating system, computer architecture or network were
used. The specification includes the neutral \emph{Interface Definition Language}
(IDL), the \emph{Object Request Broker} (ORB) middleware functionality and a
corresponding \emph{Internet Inter ORB Protocol} (IIOP).

CORBA has been adopted by many applications, and been adapted for many domains,
one of them being \emph{Healthcare}. The OMG working group dealing with that
field is the \emph{Healthcare Domain Taskforce} (HDTF) \cite{omghdtf} (formerly
called \emph{CORBAmed}). It defined IDL interfaces for a number of different
healthcare services, for example:

\begin{itemize}
    \item[-] \emph{Person (Patient) Identification Service} (PIDS)
    \item[-] \emph{Lexicon (Terminology) Query Service} (LQS)
    \item[-] \emph{Clinical Observations Access Service} (COAS)
    \item[-] \emph{Resource Access Decision Service} (RADS)
    \item[-] \emph{Clinical Image Access Service} (CIAS)
\end{itemize}

HDTF specifications are helpful in that they standardise certain functionality
calls that all systems implementing the corresponding interfaces may rely on.
However, they do not make any assertions about how medical knowledge should be
structured.
