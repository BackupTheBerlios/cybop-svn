%
% $RCSfile: lifecycle_management.tex,v $
%
% Copyright (C) 2002-2008. Christian Heller.
%
% Permission is granted to copy, distribute and/or modify this document
% under the terms of the GNU Free Documentation License, Version 1.1 or
% any later version published by the Free Software Foundation; with no
% Invariant Sections, with no Front-Cover Texts and with no Back-Cover
% Texts. A copy of the license is included in the section entitled
% "GNU Free Documentation License".
%
% http://www.cybop.net
% - Cybernetics Oriented Programming -
%
% http://www.resmedicinae.org
% - Information in Medicine -
%
% Version: $Revision: 1.1 $ $Date: 2008-08-19 20:41:07 $ $Author: christian $
% Authors: Christian Heller <christian.heller@tuxtax.de>
%

\subsection{Lifecycle Management}
\label{lifecycle_management_heading}
\index{CYBOI Lifecycle Management}

To the startup routine belong the creation of the three containers: knowledge
memory, signal memory and internals memory, and the creation of a startup model
which is placed as first signal into the signal memory. Additional meta
information given are the signal's model, its kind of abstraction and priority.

Typical synonyms for \emph{Signal} are \emph{Event} or \emph{Action} -- and
even an \emph{Operating System} (OS) \emph{Interrupt} is some form of signal,
only on a lower system level, closer to hardware. In CYBOI, a signal is simply
a reference to a logic model, which may be either a composed algorithm, or a
primitive operation.

With the startup signal being placed in the signal memory, the system enters
the \emph{check} procedure (\emph{checker} module). On shutdown, the system
runs through similar procedures in opposite direction, only that then startup
signal, memories and global variables are destroyed.
