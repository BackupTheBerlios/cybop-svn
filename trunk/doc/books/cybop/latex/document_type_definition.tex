%
% $RCSfile: document_type_definition.tex,v $
%
% Copyright (C) 2002-2008. Christian Heller.
%
% Permission is granted to copy, distribute and/or modify this document
% under the terms of the GNU Free Documentation License, Version 1.1 or
% any later version published by the Free Software Foundation; with no
% Invariant Sections, with no Front-Cover Texts and with no Back-Cover
% Texts. A copy of the license is included in the section entitled
% "GNU Free Documentation License".
%
% http://www.cybop.net
% - Cybernetics Oriented Programming -
%
% http://www.resmedicinae.org
% - Information in Medicine -
%
% Version: $Revision: 1.1 $ $Date: 2008-08-19 20:41:06 $ $Author: christian $
% Authors: Christian Heller <christian.heller@tuxtax.de>
%

\subsubsection{Document Type Definition}
\label{document_type_definition_heading}
\index{CYBOL DTD}
\index{Document Type Definition}
\index{DTD}
\index{Standard Generalized Markup Language}
\index{SGML}
\index{Extensible Markup Language}
\index{XML}
\index{Markup Tag}
\index{CYBOP 'model' Tag}
\index{CYBOP 'part' Tag}
\index{CYBOP 'property' Tag}
\index{CYBOP 'constraint' Tag}
\index{CYBOP 'name' Attribute}
\index{CYBOP 'channel' Attribute}
\index{CYBOP 'abstraction' Attribute}
\index{CYBOP 'model' Attribute}

A DTD represents the type definition of an SGML or XML document. It consists of
a set of \emph{Markup Tags} and their \emph{Interpretation} \cite{foldoc}. DTDs
can be declared inline, within a document, or as an external reference
\cite{w3schools}. Figure \ref{dtd_figure} shows the DTD of the CYBOL language.

\begin{figure}[ht]
    \bigskip
    \begin{scriptsize}
        \begin{verbatim}
<!ELEMENT model (part*)>
<!ELEMENT part (property*)>
<!ELEMENT property (constraint*)>
<!ELEMENT constraint EMPTY>

<!ATTLIST part
    name CDATA #REQUIRED
    channel CDATA #REQUIRED
    abstraction CDATA #REQUIRED
    model CDATA #REQUIRED>
<!ATTLIST property
    name CDATA #REQUIRED
    channel CDATA #REQUIRED
    abstraction CDATA #REQUIRED
    model CDATA #REQUIRED>
<!ATTLIST constraint
    name CDATA #REQUIRED
    channel CDATA #REQUIRED
    abstraction CDATA #REQUIRED
    model CDATA #REQUIRED>
        \end{verbatim}
    \end{scriptsize}
    \caption{Recommended CYBOL DTD}
    \label{dtd_figure}
\end{figure}

Following the pure hierarchical structure of the CYBOP knowledge schema (section
\ref{knowledge_representation_heading}), it would actually suffice to use a DTD
as simple as the one shown in figure \ref{simpledtd_figure}. Since the three
elements \emph{part}, \emph{property} and \emph{constraint} (compare figure
\ref{dtd_figure}) have the same list of required attributes, they could be
summarised under the name \emph{part}, for example. Because the structure of a
CYBOL model is non-ambiguous, the meaning of its elements can be guessed from
their position within the model.

\begin{figure}[ht]
    \bigskip
    \begin{scriptsize}
        \begin{verbatim}
<!ELEMENT part (part*)>

<!ATTLIST part
    name CDATA #REQUIRED
    channel CDATA #REQUIRED
    abstraction CDATA #REQUIRED
    model CDATA #REQUIRED>
        \end{verbatim}
    \end{scriptsize}
    \caption{Simplified CYBOL DTD}
    \label{simpledtd_figure}
\end{figure}

For the purpose of expressing knowledge in accordance with the schema suggested
by CYBOP, an XML document does not need to have a root element. The document's
(file) name clearly identifies a model. For reasons of XML conformity, however,
an extra root element called \emph{model} was defined (figure \ref{dtd_figure}).
And for reasons of better readability and programmability, the three kinds of
embedded elements were given distinct names.
