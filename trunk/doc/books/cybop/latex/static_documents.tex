%
% $RCSfile: static_documents.tex,v $
%
% Copyright (C) 2002-2008. Christian Heller.
%
% Permission is granted to copy, distribute and/or modify this document
% under the terms of the GNU Free Documentation License, Version 1.1 or
% any later version published by the Free Software Foundation; with no
% Invariant Sections, with no Front-Cover Texts and with no Back-Cover
% Texts. A copy of the license is included in the section entitled
% "GNU Free Documentation License".
%
% http://www.cybop.net
% - Cybernetics Oriented Programming -
%
% http://www.resmedicinae.org
% - Information in Medicine -
%
% Version: $Revision: 1.1 $ $Date: 2008-08-19 20:41:09 $ $Author: christian $
% Authors: Christian Heller <christian.heller@tuxtax.de>
%

\paragraph{Static Documents}
\label{static_documents_heading}

Modern developers understood the idea of a \emph{Living Documentation} -- one
which were never finished and always under modification due to feedback from
implementation and actual use. That is why systems got rebuilt two or three
times before they were really good. Yet this didn't happen with standards.
They were published as \emph{static} documents, and the available feedback
processes were so slow as to be nearly useless. But feedback had crucial value
in validating and improving specifications. Many standards processes continued
as talk-/ documentation fests for years, before anyone seriously tried to
validate the models or designs.
