%
% $RCSfile: generative_programming.tex,v $
%
% Copyright (C) 2002-2008. Christian Heller.
%
% Permission is granted to copy, distribute and/or modify this document
% under the terms of the GNU Free Documentation License, Version 1.1 or
% any later version published by the Free Software Foundation; with no
% Invariant Sections, with no Front-Cover Texts and with no Back-Cover
% Texts. A copy of the license is included in the section entitled
% "GNU Free Documentation License".
%
% http://www.cybop.net
% - Cybernetics Oriented Programming -
%
% http://www.resmedicinae.org
% - Information in Medicine -
%
% Version: $Revision: 1.1 $ $Date: 2008-08-19 20:41:06 $ $Author: christian $
% Authors: Christian Heller <christian.heller@tuxtax.de>
%

\subsection{Generative Programming}
\label{generative_programming_heading}
\index{Generative Programming}
\index{GP}
\index{Aspect Oriented Programming}
\index{AOP}
\index{Generic Programming}
\index{Domain Specific Language}
\index{DSL}
\index{Feature Model}
\index{Generator}
\index{Model Driven Architecture}
\index{MDA}

\emph{Generative Programming} (GP), as proposed by Czarnecki \cite{czarnecki},
is \textit{a comprehensive software development paradigm to achieving high
intentionality, reusability, and adaptability without the need to compromise
the runtime performance and computing resources of the produced software.} It
encompasses techniques of the following, previously described paradigms:

\begin{itemize}
    \item[-] \emph{Aspect Oriented Programming} (AOP) (section
        \ref{aspect_oriented_programming_heading}): used to achieve separation
        of concerns
    \item[-] \emph{Generic Programming} (section \ref{generics_heading}): used
        to parameterise over types
    \item[-] \emph{Domain Specific Language} (DSL) (section
        \ref{domain_specific_language_heading}): used to improve
        intentionality, optimisation and error checking of program code
    \item[-] \emph{Feature Model} (section \ref{feature_model_heading}): used
        as configuration knowledge, to map between problem- and solution space
\end{itemize}

Czarnecki's work contributes to the formal specification and extension of the
\emph{Feature Model}, but does not itself deliver new forms of knowledge
abstraction. GP, however, is mentioned here because of its idea of applying
\emph{Generators} (or generative techniques) producing implementation source
code for a software system from the higher-level specifications defined in the
design phase. Similar techniques are used in the
\emph{Model Driven Architecture} (see next section). GP is a trial to automate
the process of crossing abstraction gap number \emph{2} (with reference to
figure \ref{gaps_figure}), and it is often quite successful. However, the gap
between architecture design models and program source code remains. Chapter
\ref{knowledge_schema_heading} will introduce a \emph{Knowledge Schema} serving
as universal type, so that differing type-based architectures do not have to be
designed anymore.
