%
% $RCSfile: object_oriented_model.tex,v $
%
% Copyright (C) 2002-2008. Christian Heller.
%
% Permission is granted to copy, distribute and/or modify this document
% under the terms of the GNU Free Documentation License, Version 1.1 or
% any later version published by the Free Software Foundation; with no
% Invariant Sections, with no Front-Cover Texts and with no Back-Cover
% Texts. A copy of the license is included in the section entitled
% "GNU Free Documentation License".
%
% http://www.cybop.net
% - Cybernetics Oriented Programming -
%
% http://www.resmedicinae.org
% - Information in Medicine -
%
% Version: $Revision: 1.1 $ $Date: 2008-08-19 20:41:07 $ $Author: christian $
% Authors: Christian Heller <christian.heller@tuxtax.de>
%

\subsubsection{Object Oriented Model}
\label{object_oriented_model_heading}

?? frame \cite{sowa}, as ancestor of OO and others

The introduction of object oriented programming (section
\ref{object_oriented_programming_heading}) made another software design
philosophy popular:
Every entity was now treated as \emph{Object} being a runtime-instance of a
\emph{Class} that was capable of inheriting \emph{Attributes} and \emph{Methods}
from a parent class. A \emph{Relation} was now called \emph{Association}.
Multiplicity was called \emph{Cardinality}.

The primary new ideas of \emph{Inheritance} between classes and classes owning not
only attributes but also \emph{Methods} could not be modeled well in entity relationship
models (section \ref{entity_relationship_model}) so that \emph{Data Mapper} layers
(section \ref{data_mapper_heading}) became necessary. In general, object oriented
models look not much different from entity relationship models.
