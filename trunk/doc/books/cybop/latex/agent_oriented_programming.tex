%
% $RCSfile: agent_oriented_programming.tex,v $
%
% Copyright (C) 2002-2008. Christian Heller.
%
% Permission is granted to copy, distribute and/or modify this document
% under the terms of the GNU Free Documentation License, Version 1.1 or
% any later version published by the Free Software Foundation; with no
% Invariant Sections, with no Front-Cover Texts and with no Back-Cover
% Texts. A copy of the license is included in the section entitled
% "GNU Free Documentation License".
%
% http://www.cybop.net
% - Cybernetics Oriented Programming -
%
% http://www.resmedicinae.org
% - Information in Medicine -
%
% Version: $Revision: 1.1 $ $Date: 2008-08-19 20:41:05 $ $Author: christian $
% Authors: Christian Heller <christian.heller@tuxtax.de>
%

\subsection{Agent Oriented Programming}
\label{agent_oriented_programming_heading}
\index{Agent Oriented Programming}
\index{AGOP}
\index{Component Oriented Programming}
\index{COP}
\index{Inversion of Control Pattern}
\index{IoC}
\index{Application Programming Interface}
\index{API}
\index{Active Component}
\index{Passive Component}
\index{Agent}
\index{Multi Agent System}
\index{MAS}
\index{Agent Communication Language}
\index{ACL}
\index{Language Paradigm}
\index{Ontology}
\index{Mobility of an Agent}
\index{Object Oriented Programming}
\index{OOP}
\index{Speech Act Theory}
\index{Belief of an Agent}
\index{Capability of an Agent}
\index{Decision of an Agent}
\index{Mental State of an Agent}
\index{Knowledge Base of an Agent}
\index{Agent0 Language}

Components created after the principles of \emph{Component Oriented Programming}
(COP) are \emph{passive}, because they follow the \emph{Inversion of Control}
(IoC) pattern. The functionality, or \emph{Service}, they offer is called by a
surrounding container, via a well-defined \emph{Application Programming Interface}
(API). \emph{Active} components, on the other hand, act alone. An \emph{Agent}
is a self-acting component. It runs \emph{autonomically} or
\emph{semi-autonomically}, is \emph{proactive}, \emph{reactive} and
\emph{social} \cite[p. 330]{sowa}. Many individual communicative software
agents may form a \emph{Multi Agent System} (MAS) \cite{wikipedia}.
Communication happens by some \emph{Agent Communication Language} (ACL)
(section \ref{agent_communication_language_heading}). David Parks, who calls
\emph{Agent Oriented Programming} (AGOP) a \emph{Language Paradigm}, writes
\cite{parks}:

\begin{quote}
    In AGOP, objects known as agents interact to achieve individual goals.
    Agents can exist in a structure as complex as a global internet or one as
    simple as a module of a common program. Agents can be autonomous entities,
    deciding their next step without the interference of a user, or they can be
    controllable, serving as a mediary between the user and another agent.
\end{quote}

In search for a uniform definition of the term \emph{Agent}, Ralf Kuehnel
investigated numerous sources of literature but finally comes to the conclusion
\cite[p. 203]{kuehnel} that the term is just a \emph{Metaphor} standing for
different properties, depending on the field it is used in. Typically mentioned
means of agents, however, are \cite[p. 11]{kuehnel}: \emph{Distribution}, common
\emph{Language} and \emph{Ontology} (section \ref{conceptual_network_heading}),
\emph{Cooperation} and \emph{Coordination}, \emph{Security} and \emph{Mobility}.

Comparing \emph{Agents} of AGOP with \emph{Objects} known from OOP, Parks
\cite{parks} writes: \textit{It is not clear, for example, what the concepts of
inheritance and dynamic dispatch mean when discussing an agent.} He points out
the following significant differences:

\begin{itemize}
    \item[-] The fields of an agent are restricted. The state of an agent is
        described in terms of \emph{Beliefs}, \emph{Capabilities} and
        \emph{Decisions} (\emph{Obligation} / \emph{Commitment}). These ideas
        are built into the syntax of the language.
    \item[-] Each message is also defined in terms of mental activities. An agent
        may engage another (or itself) with messaging activities from a restricted
        class of categories. In Shoham's formalism \cite{shoham}, the categories
        of messages are taken from \emph{Speech-Act Theory}; they are:
        \emph{Informing}, \emph{Requesting}, \emph{Offering}, \emph{Accepting},
        \emph{Rejecting}, \emph{Competing} and \emph{Assisting}.
\end{itemize}

Yoav Shoham, who presented AGOP as a new way to describe intelligent agents
\cite{shoham}, suggests that an AGOP system needs three elements to be
complete, a:

\begin{enumerate}
    \item[-] \emph{Formal Language} with clear syntax for describing the mental state
    \item[-] \emph{Programming Language} in which to define agents
    \item[-] \emph{Method} for converting neutral applications into agents
\end{enumerate}

To the \emph{Mental State} of an agent belong information \cite{kuehnel} about its:

\begin{itemize}
    \item[-] \emph{Environment} (constraints)
    \item[-] \emph{Expertise} (capabilities) and \emph{Motivations} (aims)
    \item[-] \emph{Actions} and \emph{Plans}
\end{itemize}

Tim Finin et al. \cite{kqml} classify the statements in a knowledge base into
two categories: \emph{Beliefs} and \emph{Goals}. After them, an agent's beliefs
encoded information it has about itself (capabilities) and its external
environment (constraints), including the knowledge bases of other agents. An
agent's goals encoded states of its external environment that the agent would
act to achieve.

To a running \emph{Agent} system belong the following modules \cite{kuehnel}:

\begin{itemize}
    \item[-] \emph{Knowledge Base:} mental state, as described above
    \item[-] \emph{Controller:} task controller, scheduler and option selection algorithm
    \item[-] \emph{Executor:} task runner and security
    \item[-] \emph{Interaction:} communication handler, sender and receiver
    \item[-] \emph{Management:} lifecycle manager, startup and shutdown
\end{itemize}

While early research in AGOP used special languages like Shoham's \emph{Agent0}
\cite{shoham}, agent-oriented systems created later were also built upon OOP-
and other contemporary programming paradigms \cite[p. 237]{kuehnel}. Ralf
Kuehnel \cite{kuehnel} calls \emph{Agent0} alone a \textit{very limited
programming language} and takes this as evidence for supporting both, the
development of agents and the representation of knowledge with a framework based
on OOP principles. For the implementation of this framework, his choice fell on
Java as system programming language (section \ref{system_programming_heading}).

That is, although AGOP suggests the separation of a system's \emph{Knowledge}
(mental state) from its internal runtime processing and \emph{Control} (agent)
and sees them both as separate elements that should be implemented in
\emph{different} languages (\emph{formal} vs. \emph{programming}), as mentioned
by Shoham (see above), many agent-oriented systems use just \emph{one} language
for implementing both. Even if they are kept in different modules, the conceptual
differences between \emph{high-level} application knowledge and \emph{low-level}
system control cannot be honoured sufficiently. This \emph{Mix-up} puts them on
the same level like traditional systems. \emph{Cybernetics Oriented Programming}
(CYBOP) as described in this work therefore defines a knowledge modelling
language (chapter \ref{cybernetics_oriented_language_heading}) which is
independent from the implementation language of its underlying interpreter.

Furthermore, if OO concepts like \emph{Composition} or \emph{Inheritance} were
present in knowledge models, the usage of an OOP language to implement the actual
agent system could \emph{not} be justified any longer. In such a case, lower-level
\emph{Structured and Procedural Programming} (SPP) languages would suffice, and
work much more efficiently. Chapter \ref{cybernetics_oriented_interpreter_heading}
of this work introduces a knowledge interpreter that is written in the \emph{C}
programming language. The interpreter owns a knowledge base keeping all
application knowledge, and it has modules for lifecycle management, signal
(event) processing, communication etc., just like the definition of an agent
(see above) suggests.
