\chapter{Thesen}

  These 1: \\
  CYBOL entspricht dem  XML - Standard, da dieser die hierarchische Beschreibungssprache abdeckt.\\
  
  These 2: \\
  Mit CYBOL sind die Grundoperationen, Sequential, Selection, Iteration, als Grundbestandteil 
  einer Beschreibungssprache realisierbar. \\
  
  These 3: \\
  Die Integration eines Webservers in CYBOI und somit die Realisierbarkeit von Webanwendungen 
  f�r CYBOP ist m�glich. \\
  
  These 4: \\
  Die Darstellung von Webanwendungen unter CYPOP ist mit XHTML am sinnvollsten. 
  Damit wird auch das hierachische System von Human Thinking konsequent weitergef�hrt.  \\
  
  These 5: \\
  Durch die Verwendung von Threads und blockierenden Sockets ist die Belastung der Hardware und 
  der damit verbundene Ressourcenverbrauch  sehr gering, wodurch  die 
  Performance von CYBOI wesentlich erh�ht wird.  \\
  
  These 6: \\
  Die Grundstruktur von Threads und blockierenden Kommunikationskanal, wie sie 
  bei dem Webserver eingesetzt wird, ist analog auf andere  Services 
  �bertragbar. \\
  
  
  These 7: \\
  Erweiterungen von Operationen sind ohne Anpassungen in CYBOL m�glich. 
  Diese sind  nur in CYBOI zu implementieren.\\
  
  These 8: \\
  Durch Verwendung von speziellen Editoren, die die spezifischen Besonderheiten von 
  CYBOP ber�cksichtigen (Knowledge Memory, Hierarchiedarstellung der Templates), 
  ist die Programmentwicklung  effizienter zu gestalten.  \\

\par
\vspace{5cm}

\begin{tabbing}
  Ilmenau, den \Abgabetermin \=\hspace{4cm}\= --------------------------- \\
  \> \> \hspace{0.5cm} \= Rolf Holzm�ller\\
\end{tabbing}