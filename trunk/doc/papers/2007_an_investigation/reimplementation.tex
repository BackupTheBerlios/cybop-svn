%
% $RCSfile$
%
% Copyright (c) 2005-2006. Christian Heller. All rights reserved.
%
% Permission is granted to copy, distribute and/or modify this document
% under the terms of the GNU Free Documentation License, Version 1.1 or
% any later version published by the Free Software Foundation; with no
% Invariant Sections, with no Front-Cover Texts and with no Back-Cover
% Texts. A copy of the license is included in the section entitled
% "GNU Free Documentation License".
%
% http://www.cybop.net
% - Cybernetics Oriented Programming -
%
% http://www.resmedicinae.org
% - Information in Medicine -
%
% Version: $Revision$ $Date$ $Author$
% Authors: Christian Heller <christian.heller@tuxtax.de>
%

\subsubsection{Reimplementation}
\label{reimplementation_heading}

The architecture-advanced prototype of the \emph{Record} module had \emph{much}
less functionality than earlier ones, in fact not much more than starting a
graphical frame with menu bar and exiting the application again. This was so,
because yet before all domain knowledge could be extracted into CYBOL, another
issue turned up:

CYBOP modelling concepts like \emph{Itemisation} or \emph{Composition} are an
integral part of the CYBOL knowledge representation language. Other concepts
like the \emph{Bundling} of attributes and methods, property- and container
\emph{Inheritance}, as known from \emph{Object Oriented Programming} (OOP),
were considered unfavourable (section \ref{existing_problems_heading}) and
neither to be used in CYBOL, nor in the CYBOI interpreter. Consequently,
OOP languages like Java or C++ were not suitable for CYBOI any longer. A slim
and fast language, close to hardware and fast in processing CYBOL was needed.

Having such requirements, one of the first candidates coming to mind was the
\emph{C} programming language. It is \emph{high-level} enough to permit fast
programming and \emph{low-level} enough to connect efficiently to hardware or
an \emph{Operating System} (OS). Many OS are written in C themselves, anyway.
CYBOI was therefore reimplemented in C, which hasn't changed since. What has
changed and is changing all the time is its functionality, an overview of which
was given in section \ref{cyboi_heading}.

One problem that had to be solved was \emph{Graphical User Interface} (GUI)
handling. While the Java-implemented CYBOI could make use of the
\emph{Abstract Windowing Toolkit} (AWT)/ Swing, the C-implemented CYBOI did not
have such functionality at first. Toolkit candidates like \emph{Qt} \cite{qt}
or \emph{wxWindows} \cite{wxwidgets}, being implemented in C++, were out. Other
GUI frameworks like the \emph{Gimp Toolkit} (GTK) \cite{gtk}, written in C,
were considered cumbersome to cope with so that finally, the decision was taken
to use low-level graphics drawing routines. For CYBOI, being developed on a
\emph{GNU/Linux} OS \cite{linux}, that meant using \emph{XFree86's}
\cite{xfree86} \emph{X-Library} (Xlib) functionality directly. The necessary
effort for transforming hierarchical CYBOL models into GTK- or other toolkit
structures was estimated to be equal or even higher than translating them into
Xlib functionality right away. At the time of writing this, implementation is
in progress but not completed.

Similar implementations are necessary for \emph{Textual User Interfaces} (TUI),
\emph{Web User Interfaces} (WUI) and \emph{Socket Communication Mechanisms},
the latter two being already finished in a first version. Further development
activities may for instance enable CYBOI to run on other platforms and integrate
more hardware-driving functionality, to get independent from underlying OS.

While the CYBOL specification can be considered quite mature, CYBOI, as could
be seen, will need plenty of extensions and additions in future, in order to
leave its prototype stage and become fully usable.
