%
% $RCSfile$
%
% Copyright (c) 2002-2006. Christian Heller. All rights reserved.
%
% Permission is granted to copy, distribute and/or modify this document
% under the terms of the GNU Free Documentation License, Version 1.1 or
% any later version published by the Free Software Foundation; with no
% Invariant Sections, with no Front-Cover Texts and with no Back-Cover
% Texts. A copy of the license is included in the section entitled
% "GNU Free Documentation License".
%
% http://www.cybop.net
% - Cybernetics Oriented Programming -
%
% http://www.resmedicinae.org
% - Information in Medicine -
%
% Version: $Revision$ $Date$ $Author$
% Authors: Christian Heller <christian.heller@tuxtax.de>
%

\subsubsection{Platform Specific and -Independent}
\label{platform_specific_and_independent_heading}

The \emph{Model Driven Architecture} (MDA) \cite{mda} took a first step into
the right direction, by distinguishing \emph{Platform Independent Models}
(PIM), that is domain- and application logic, and \emph{Platform Specific Models}
(PSM), that is implementation technology. It encourages the use of automated
tools for defining and transforming these models.

While the definition, organisation and management of architectures (PIM) mostly
happen in the analysis- and design phase of a \emph{Software Engineering Process}
(SEP) (section \ref{abstraction_gaps_heading}), the generation of source code
(PSM) can be assigned to the implementation phase. The approach still has
weaknesses, and tools which can truly generate running systems are rare or not
existent, at least to what concerns more complex software systems -- not to
talk of the so-called \emph{Roundtrip Engineering}, which is managed by even
less tools.

Nevertheless, the trend clearly goes towards more model-centric approaches. The
aim of this work was to supply domain experts and application developers with a
\emph{Model Only} technology, allowing to create application systems that do
\emph{not} have to be transformed into classical implementation code any longer,
whereby the SEP abstraction gap number \emph{2} (figure \ref{gaps_figure})
could be closed conclusively. The knowledge schema introduced in section
\ref{knowledge_schema_heading} is a necessary prerequisite therefor.
