%
% $RCSfile: recommendation.tex,v $
%
% Copyright (c) 2004. Christian Heller. All rights reserved.
%
% No copying, altering, distribution or any other actions concerning this
% document, except after explicit permission by the author!
% At some later point in time, this document is planned to be put under
% the GNU FDL license. For now, _everything_ is _restricted_ by the author.
%
% http://www.cybop.net
% - Cybernetics Oriented Programming -
%
% http://www.resmedicinae.org
% - Information in Medicine -
%
% @author Christian Heller <christian.heller@tuxtax.de>
%

\subsection{Recommendation}
\label{recommendation_heading}

The first category \emph{Itemization} (objectification) is the base of any
modelling activity and clearly necessary.

The next three categories \emph{1:1 Association}, \emph{1:n Association} and
\emph{Recursion} are special kinds of associations that rely exclusively on
\emph{unidirectional} relations and result in a clean architecture which is why
their usage is strongly recommended.

\emph{Bidirectionalism}, on the other hand, is an \emph{ill} variant of the three
aforementioned categories and should be avoided wherever possible. Patterns in
this category are one reason for endless loops and unpredictable behaviour since
it becomes very difficult to trace the effects that changes in one place of a
system have on others (section \ref{bidirectional_dependency_heading}).

\emph{Polymorphism} is a good thing. It relies on categorization and due to
inheritance can avoid a tremendous amount of otherwise redundant source code.
However, it also makes understanding a system more difficult, since the whole
architecture must be understood before being able to manipulate code correctly.
Unwanted source code changes caused by inheritance dependencies are often
described with the term \emph{Fragile Base Class Problem}
\cite[section \emph{Layers}]{buschmann}. They are just the opposite of what
inheritance was actually intended to be for: \emph{Reusability}
\cite[Vorwort]{gruhn}.

\emph{Grouping} models is essential to keep overview in a complex software
system. A very promising technology to support this are \emph{Ontologies}
\cite{hellerkunze}. A lot of thought-work has to go into them but if they are
well thought-out, they are clearly recommended.

The habit of globally accessing models is banned since OOP became popular.
However, it is not banned completely. Patterns like \emph{Singleton}
encapsulate and bundle global access but they still permit it. They disregard
any dependencies and relations in a system, such are a security risk and reason
for untraceable data changes. This paper sees the whole category of
\emph{Global Access} as potentially dangerous and can \emph{not} recommend its
patterns.
