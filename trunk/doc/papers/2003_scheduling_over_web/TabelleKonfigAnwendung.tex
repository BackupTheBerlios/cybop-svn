\begin{table}[hbtp!]
  \caption{Eintr�ge der Konfigurationsdatei Sektion "`logging"'}
  \centering

  \begin{tabular} {|p{3.6cm}|p{10cm}|} 

    \hline
    
    \textbf{Schl�ssel}  & \textbf{Beschreibung} \\ \hline\hline

    log\_level          & Loglevel der Application (1..15)      \\ \hline
    log\_file           & Datei der Logeintr�ge                 \\ \hline
  \end{tabular}
\end{table}


\begin{table}[hbtp!]
  \caption{Eintr�ge der Konfigurationsdatei Sektion "`database"'}
  \centering
  \begin{tabular} {|p{3.6cm}|p{10cm}|} 

    \hline
    
    \textbf{Schl�ssel}  & \textbf{Beschreibung} \\ \hline\hline

    driver              & 
    Treiber der JDBC-Verbindung
    (org.gjt.mm.mysql.Driver)             \\ \hline
    url                 & 
    Verwendungs-URL f�r die JDBC-Verbing 
    (jdbc:mysql://localhost/resdata)                             \\ \hline
    user                & 
    Verwendete Benutzer der Datenbank   
    (resdata)                                                   \\ \hline
    pwd                 & 
    Verwendetes Pa�wort der Datenbank  
    (resdata)                                                   \\ \hline
  \end{tabular}
\end{table}

    
\begin{table}[hbtp!]
  \caption{Eintr�ge der Konfigurationsdatei Sektion "`mail"'}
  \centering
  \begin{tabular} {|p{3.6cm}|p{10cm}|} 

    \hline
    
    \textbf{Schl�ssel}  & \textbf{Beschreibung} \\ \hline\hline

    protocol            & 
    Protokoll der eMail-Verschickung  
    (smtp)                                                       \\ \hline
    host                & 
    Host f�r die eMail-Verschickung   
    (mail.gmx.de)                                               \\ \hline
    user                & 
    Login f�r das eMail-Konto   
    (xxx)                                                       \\ \hline
    pwd                 & 
    Pa�wort f�r das eMail-Konto   
    (???)                                                       \\ \hline
    fromadresse         & 
    Absenderadresse f�r die eMail-Verschickung  
    (xxx@yyy.net)                                               \\ \hline
    subject\_desired    & 
    Betreff f�r eMail-Verschickung Terminwunsch   
    (Web Terminwunsch)                                           \\ \hline
    subject\_confirm    & 
    Betreff f�r eMail-Verschickung Terminbest�tigung 
    (Terminbest�tigung)                                           \\ \hline
    subject\_cancel     & 
    Betreff f�r eMail-Verschickung Terminabsage  
    (Terminabsage)                                               \\ \hline
  \end{tabular}
\end{table}

    
\begin{table}[hbtp!]
  \caption{Eintr�ge der Konfigurationsdatei Sektion "`path"'}
  \centering
  \begin{tabular} {|p{3.6cm}|p{10cm}|} 

    \hline
    
    \textbf{Schl�ssel}  & \textbf{Beschreibung} \\ \hline\hline

    servlet\_path       & 
    Pfad f�r das Servlet \newline
    (http://localhost:8080/resdata/servlet/
    org.resmedicinae.application.healthcare.resdata.controller)\\ \hline
    jsp\_path           & 
    Pfad f�r JSP-Dateien  
    (/jsp/)                                                     \\ \hline
    css\_path           & 
    Pfad f�r CSS-Dateien \newline  
    (http://servername:8080/resdata/css/)                         \\ \hline
    image\_path         & 
    Pfad f�r Bild-Dateien \newline  
    (http://servername:8080/resdata/images/)                     \\ \hline
  \end{tabular}
\end{table}


\begin{table}[hbtp!]
  \caption{Eintr�ge der Konfigurationsdatei Sektion "`navigation"'}
  \centering
  \begin{tabular} {|p{3.6cm}|p{10cm}|} 

    \hline
    
    \textbf{Schl�ssel}  & \textbf{Beschreibung} \\ \hline\hline

    resmedicinae\_home  & 
    URL der Home-Seite f�r Resmedicinae  \newline
    (http://resmedicinae.sourceforge.net/index.html)             \\ \hline
    resdata\_home       & 
    URL der Home-Seite f�r ResData \newline 
    (http://servername:8080/resdata/jsp/ResDataStart.jsp)         \\ \hline
  \end{tabular}
\end{table}


\begin{table}[hbtp!]
  \caption{Eintr�ge der Konfigurationsdatei Sektion "`parameter"'}
  \centering
  \begin{tabular} {|p{3.6cm}|p{10cm}|} 

    \hline
    
    \textbf{Schl�ssel}  & \textbf{Beschreibung} \\ \hline\hline

    control\_param      & 
    Instanzbezeichnung f�r den Controller 
    (myController)                                               \\ \hline
    loginmodel\_param   & 
    Instanzbezeichnung f�r das Modell Login  
    (myModelLogin)                                              \\ \hline
    doctormodel\_param  & 
    Instanzbezeichnung f�r das Modell Doctor 
    (myModelDoctor)                                             \\ \hline
    datemodel\_param    & 
    Instanzbezeichnung f�r das Modell Date  
    (myModeldate)                                               \\ \hline
    mailmodel\_param    & 
    Instanzbezeichnung f�r das Modell Mail  
    (myModelMail)                                               \\ \hline
  \end{tabular}
\end{table}

\begin{table}[hbtp!]
  \caption{Eintr�ge der Konfigurationsdatei Sektion "`color"'}
  \centering
  \begin{tabular} {|p{3.6cm}|p{10cm}|} 

    \hline
    
    \textbf{Schl�ssel}  & \textbf{Beschreibung} \\ \hline\hline

    status\_no          & 
    Farbe f�r freie Termine  
    (blue)                                                       \\ \hline
    status\_request     & 
    Farbe f�r angeforderte Termine
    (red)                                                       \\ \hline
    status\_confirmed   & 
    Farbe f�r best�tigte Termine   
    (yellow)                                                     \\ \hline    
  \end{tabular}
\end{table}
