%
% $RCSfile: abstract.tex,v $
%
% Copyright (c) 2001-2004. Christian Heller. All rights reserved.
%
% Permission is granted to copy, distribute and/or modify this document
% under the terms of the GNU Free Documentation License, Version 1.1
% or any later version published by the Free Software Foundation;
% with no Invariant Sections, with no Front-Cover Texts and with no Back-Cover
% Texts. A copy of the license is included in the section entitled
% "GNU Free Documentation License".
%
% http://www.cybop.net
% - Cybernetics Oriented Programming -
%
% http://www.resmedicinae.org
% - Information in Medicine -
%
% @author Christian Heller <christian.heller@tuxtax.de>
% @author Jens Bohl <info@jens-bohl.de>
%

\begin{center}
    \textbf{\large{Abstract}}
\end{center}
\normalsize
\textit{
This document describes how existing design patterns can be combined to merge
their advantages into one domain- independent software framework. This framework,
called Cybernetics Oriented Programming (CYBOP), is characterized by flexibility
and extensibility. Further, the concept of Ontology is used to structure the
software architecture as well as to keep it maintainable. A Component Lifecycle
ensures the proper startup and shutdown of any systems built on top of CYBOP.\\
The practical proof of these new concepts was accomplished within the diploma
thesis of Jens Bohl which consisted of designing and developing a module called
Record, of the Open Source Software (OSS) project Res Medicinae. The major task
of this module is to provide a user interface for creating medical documentation.
New structure models such as Episodes were considered and implemented. In this
context, the integration of a graphical tool for Topological Documentation was
also highly demanded. The tool allows documentation with the help of anatomical
images and the setting of markers for pathological findings.\\
\textbf{Keywords.} Design Pattern, Framework, Component Lifecycle, Ontology,
CYBOP, Res Medicinae, Episode Based Documentation, Topological Documentation
} \rm

