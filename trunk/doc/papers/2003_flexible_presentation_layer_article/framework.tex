\section{CYBOP}
Section two introduced essential design patterns that represent
the main structure of the CYBOP framework. Section three explained the
Component Lifecycle and section four the well-known idea of ontology.
Now these design principles and conceptual architectures will be combined
to comprise their advantages and to increase the demanded quality characteristics:
high flexibility and maintainability.\\
\emph{Structure by Hierarchy} -- this is the basic idea behind CYBOP.
Extending the concept of Hierarchical Model-View-Controller to whole
software architectures, CYBOP was designed to be the domain-independent
backbone for information systems of any kind. Originally designed for
medical purposes, it should also be usable for insurance, financial or any
other standard applications in future.

\subsection{Class Item}
As shown, tree-like structures can be realized by the Composite pattern.
In CYBOP, this pattern can be found in class {\tt Item} which is super type
of all other classes. References, respectively relations to child elements
are held within a hashmap. No attributes were used except of this hashmap.
Every element of the map can be accessed by a special key value. So, no
particular {\tt get}- or {\tt set}-methods were needed for attributes.
\includepicture{6}{eps/item.eps}{Class Item}{Class Item}{Class Item}{Class Item}

\subsection{Basic Structure}
Comprising the design patterns \emph{Composite}, \emph{Layers}, and
\emph{Chain of Responsibility}, the framework CYBOP is comparable to a big
tree containing objects organized in different levels. Figure \ref{Basic Structure}
shows the object tree and the different levels of granularity.
\includepicture{8}{eps/framework-structure.eps}{Basic Structure}{Basic Structure}{Basic Structure}{Basic Structure}
