%
% $RCSfile: summary.tex,v $
%
% Copyright (c) 2001-2004. Christian Heller. All rights reserved.
%
% Permission is granted to copy, distribute and/or modify this document
% under the terms of the GNU Free Documentation License, Version 1.1
% or any later version published by the Free Software Foundation;
% with no Invariant Sections, with no Front-Cover Texts and with no Back-Cover
% Texts. A copy of the license is included in the section entitled
% "GNU Free Documentation License".
%
% http://www.cybop.net
% - Cybernetics Oriented Programming -
%
% http://www.resmedicinae.org
% - Information in Medicine -
%
% @author Christian Heller <christian.heller@tuxtax.de>
% @author Jens Bohl <info@jens-bohl.de>
%

\section{Summary}
\label{summary_heading}

Software design patterns are essential elements of frameworks. They can be
combined to comprise their advantages and to realize hierarchical structures.
These structures can be created and destroyed in the lifecycle of components.
In that lifecycle, object relations become more transparent and are easier to
control and to maintain.\\
Ontologies can help to model particular domains and to layer software. Every level
of these ontologies has a particular supertype, whereby these types depend on each
other by inheritance. This concept supports the modelling and logical separation
of software into hierarchical architectures. The granularity of the ontology
(number of ontological levels) can be adapted to particular requests.\\
By applying the new concepts introduced in this document, the quality of software
can be greatly increased. The time for building systems can be reduced to a minimum.
The clear architecture avoids common confusion as the systems grow.

