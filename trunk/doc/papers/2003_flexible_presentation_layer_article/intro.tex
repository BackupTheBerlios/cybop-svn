%
% $RCSfile: intro.tex,v $
%
% Copyright (c) 2001-2004. Christian Heller. All rights reserved.
%
% No copying, altering, distribution or any other actions concerning this
% document, except after explicit permission by the author!
% At some later point in time, this document is planned to be put under
% the GNU FDL license. For now, _everything_ is _restricted_ by the author.
%
% http://www.cybop.net
% - Cybernetics Oriented Programming -
%
% http://www.resmedicinae.org
% - Information in Medicine -
%
% @author Christian Heller <christian.heller@tuxtax.de>
%

\section{Introduction}
\label{introduction_heading}

Quality of software is often defined by its maintainability, extensibility and
flexibility. \emph{Object Oriented Programming} (OOP) should help to achieve
these goals -- but this wasn't possible by only introducing another programming
paradigm.\\
So, major research objectives are to find concepts and principles to increase
the reusability of software architectures and the resulting code. \emph{Frameworks}
shall prevent code duplication and development efforts. Recognizing recurring
structures means finding \emph{Design Patterns} for application on similar problems.
These two concepts -- frameworks and design patterns -- depend on each other and
provide a higher flexibility of software components \cite{pree}.\\
The aim of this work was to find suitable combinations of design patterns to
compose a framework that is characterized by a strict hierarchical architecture.
Everything in universe is organized within a hierarchy of elements -- the human
body for example consists of organs, organs consist of regions, regions consist
of cells and so on. This very simple idea can also be mapped on software
architectures -- and basically, this is what this document is about.\\
What kind of techniques to realize such a concept of strict hierarchy does software
engineering provide? The following chapters first introduce common design patterns
and the lifecycle of components as templates for own ideas and then show how the
resulting framework \emph{Cybernetics Oriented Programming} (CYBOP) \cite{cybop}
is designed.

