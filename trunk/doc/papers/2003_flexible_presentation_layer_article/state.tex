\section{Design Principles}

\subsection{Essential Design Patterns \cite{ch:entwurfsmuster}}
Design patterns are elements of reusable software. They can be
used for solving recurrent design problems and are recommendations
on how to build software in an elegant way. With the help of these
patterns, software shall be more extensible, flexible and easy to
maintain in respect to further enhancements. The following
patterns are essential within CYBOP.

\subsubsection{Composite}
This design pattern allows creating tree-like object structures.
One object is child of another object and has exactly one parent.
This pattern is often used to realize \emph{whole-part} relations:
one object is \emph{part} of another one.
\includepicture{8}{eps/compositum.eps}{Composite Pattern}{Composite Pattern}{Composite Pattern}{Composite Pattern}

\subsubsection{Layers}
With the help of this pattern, software can be organized in
horizontal layers. Modules and applications can be separated into
logical levels, whereby these levels should be as independent from
each other as possible, to ensure a high substitutability.
\includepicture{5}{eps/3-Tier-1.eps}{Layer Pattern}{Layer Pattern}{Layer Pattern}{Layer Pattern}

\subsubsection{Chain Of Responsibility}
Messages initiated by a particular object can be sent over a chain of instances to the receiving
object. So, either the message will be transmitted over a bunch of objects or evaluated immediately
by the target object.
\includepicture{6}{eps/chain.eps}{Chain Of Responsibility Pattern}{Chain Of Responsibility Pattern}{Chain Of Responsibility Pattern}{Chain Of Responsibility Pattern}

\subsubsection{Model-View-Controller}
Dividing the presentation layers into the logical components \emph{Model},
\emph{View} and \emph{Controller}, is a very approved way for designing software
for user interfaces. The model encapsulates the data presented by the view and
manipulated by the \emph{controller}.
\includepicture{8}{eps/mvc-1.eps}{Model-View-Controller Pattern}{Model-View-Controller Pattern}{Model-View-Controller Pattern}{Model-View-Controller Pattern}

\subsubsection{Hierarchical Model-View-Controller \cite{hmvc}}
The Hierarchical Model-View-Controller combines the essential design patterns
\emph{Composite}, \emph{Layers}, and \emph{Chain of Responsibility} into
one conceptual architecture. It consists of hierarchical layers containing
\emph{MVC-Triads}. These triads conventionally separate the presentation layer
into model, view, and controller. Triads communicate with each other by
relating over the controller object. Here is a short explanation of this
concept, using a practical example: The upper-most triad could represent
a dialog, the middle one a container such as a panel. In this container,
a third triad, for example a button, can be held.\\
All Triads communicate with each other by using the controller component.
The basic idea behind this concept is to divide the presentation layer
into hierarchical sections.
\includepicture{7}{eps/hmvc.eps}{Hierarchical Model-View-Controller Pattern}{Hierarchical Model-View-Controller Pattern}{Hierarchical Model-View-Controller Pattern}{Hierarchical Model-View-Controller Pattern}

\subsection{Component Lifecycle \cite{avalon}}

Each \emph{Component} lives in a system that is responsible for the component's
creation, destruction etc. When talking about components, this article sticks
to the definition of Apache-Jakarta-Avalon \cite{avalon}, which considers
components to be \emph{a passive entity that performs a specific role}.\\
A component has a number of methods which need to be called in a certain order.
The order of method calls is what is known as \emph{Component Lifecycle}.
An outside, active entity is responsible for calling the lifecycle methods
in the right order. In other words, such an entity or \emph{Component Container}
can control and use the component. The Avalon documentation \cite{avalon} says:

\begin{quotation}
"It is up to each container to indicate which lifecycle methods it will honor.
This should be clearly documented together with the description of the container."
\end{quotation}
