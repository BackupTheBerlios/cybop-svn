%
% $RCSfile: document_type_definition.tex,v $
%
% Copyright (c) 2005-2006. Christian Heller. All rights reserved.
%
% Permission is granted to copy, distribute and/or modify this document
% under the terms of the GNU Free Documentation License, Version 1.1 or
% any later version published by the Free Software Foundation; with no
% Invariant Sections, with no Front-Cover Texts and with no Back-Cover
% Texts. A copy of the license is included in the section entitled
% "GNU Free Documentation License".
%
% http://www.cybop.net
% - Cybernetics Oriented Programming -
%
% http://www.resmedicinae.org
% - Information in Medicine -
%
% Version: $Revision: 1.1 $ $Date: 2006-01-03 08:21:45 $ $Author: christian $
% Authors: Christian Heller <christian.heller@tuxtax.de>
%

\subsection{Document Type Definition}
\label{document_type_definition_heading}

Since CYBOL is based on the \emph{Extensible Markup Language} (XML), a
\emph{Document Type Definition} (DTD) can be given (figure \ref{dtd_figure}).

\begin{figure}[ht]
    \bigskip
    \begin{scriptsize}
        \begin{verbatim}
<!ELEMENT model (part*)>
<!ELEMENT part (property*)>
<!ELEMENT property (constraint*)>
<!ELEMENT constraint EMPTY>

<!ATTLIST part
    name CDATA #REQUIRED
    channel CDATA #REQUIRED
    abstraction CDATA #REQUIRED
    model CDATA #REQUIRED>

<!ATTLIST property
    name CDATA #REQUIRED
    channel CDATA #REQUIRED
    abstraction CDATA #REQUIRED
    model CDATA #REQUIRED>

<!ATTLIST constraint
    name CDATA #REQUIRED
    channel CDATA #REQUIRED
    abstraction CDATA #REQUIRED
    model CDATA #REQUIRED>
        \end{verbatim}
        \caption{CYBOL DTD}
        \label{dtd_figure}
    \end{scriptsize}
\end{figure}

One can recognise the purely hierarchical structure as described by the CYBOP
knowledge schema (section \ref{schema_heading}). The three elements
\emph{part}, \emph{property} and \emph{constraint} have the same list of
required attributes.
