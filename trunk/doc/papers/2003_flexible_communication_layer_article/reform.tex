\section{ReForm Module}
\label{reform_module_section}

The practical background for the application of CYBOP is \emph{Res Medicinae}.
A modern clinical information system is the aim of all efforts in this project.
In the future, it shall serve medical documentation, laboratory data, billing etc.
\emph{Res Medicinae} is separated into single modules solving different tasks.\\
One of these modules is \emph{Record} -- an application for documenting medical
information. In addition to new documentation models, it also contains a tool
for topological documentation (figure \ref{record_module_figure}).

\begin{figure}[ht]
    \begin{center}
       \includegraphics[scale=0.27]{images/record_modul.eps}
       \caption{Record Module \cite{urban}}
       \label{record_module_figure}
    \end{center}
\end{figure}

A second module in which the \emph{CYBOP/Layer PerCom} concepts were applied,
is \emph{ReForm} (figure \ref{reform_module_figure}). It offers a number of
medical forms that can be filled in and printed out. Since one of the main reasons
to implement this module was the testing and proof of the new persistence and
communication concepts, it includes a dialog for choosing the communication protocol
or persistence mechanism, respectively. This is the only remaining part where users
have to care about the underlaying techniques. They also have to decide whether to
use the local file system via XML-format or to store the data in a central database.
In the future, the same XML file format may as well be used for remote communication,
e.g. via SOAP.\\
Because of the component-based architecture of Res Medicinae, it is possible to
start more than one instance of ReForm. In this way, the data exchange between
modules can be tested. A module X looks for another registered module Y at the
naming service (of RMI, CORBA or some other). X gets the address of the remote
service Y (depending on the communication mechanism). The stub and skeleton of X
and Y marshal, send and unmarshal the data for further working.

\begin{figure}[ht]
    \begin{center}
       \includegraphics[scale=0.24]{images/reform_module.eps}
       \caption{ReForm Module}
       \label{reform_module_figure}
    \end{center}
\end{figure}
