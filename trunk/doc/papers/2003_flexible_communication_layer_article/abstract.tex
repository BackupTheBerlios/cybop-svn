\begin{center}\bf \large Abstract
\end{center}
\normalsize \it This document describes how existing design patterns can be extended and combined
to merge their advantages into one, domain-independent software framework. This framework, called
\emph{Cybernetics Oriented Programming} (CYBOP), is characterized by flexibility and extensibility.
Further, the concept of \emph{Ontology} is used to structure the software architecture as well as
to keep it maintainable. A \emph{Component Lifecycle} ensures the proper startup and shutdown of
any systems built on top of CYBOP.\\
One core component of the framework is \emph{Layer PerCom} - an advanced architecture for seemless
integration of communication paradigms and persistence mechanisms. Overcoming the classical scheme
of thinking in terms of \emph{Domain}, \emph{Frontend}, \emph{Backend} and \emph{Communication},
PerCom treats them all similar, as passive data models which can be translated into each other - as
opposed to the classical approach that unnecessarily complicates their design.\\
The practical proof of this combined architectural approach was accomplished
within the diploma work of Torsten Kunze \cite{Diplomarbeit_Torsten_Kunze} which
consisted of an (ongoing) effort to design and develop a module called \emph{ReForm},
for the \emph{Open Source Software} (OSS) project \emph{Res Medicinae}.
The major task for this module is to provide a user interface for printing medical
forms. It was used to examine the communication between modules and to find a
structure for effective implementation and easy expansion.\\
\textbf{Keywords.} Design Pattern, Framework, Persistence, Communication, User Interface, Frontend,
Backend, CYBOP, Res Medicinae \rm
