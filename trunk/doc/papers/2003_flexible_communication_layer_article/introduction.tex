\section{Introduction}
\label{introduction_section}

Quality of software is often defined by its maintainability, extensibility and
flexibility. \emph{Object Oriented Programming} (OOP) was to help to achieve these goals
-- but this wasn't possible by only introducing another programming paradigm.\\
So, major research objectives are to find concepts and principles to increase the
reusability of software architectures and of the resulting code. \emph{Frameworks}
shall prevent code duplication and development efforts. Recognizing recurring
structures means finding \emph{Design Patterns} for application on similar problems.
These two concepts -- frameworks and design patterns -- depend on each other and
provide higher flexibility for software components \cite{ch:pree}.\\
This paper is not about frameworks but our solutions are extracted and used in one
called \emph{Cybernetics Oriented Programming} (CYBOP) \cite{cybop}. Its main concept
is based on the hierarchical structure of the universe. This very simple idea can
perfectly be mapped on software systems.\\
The aim of this work was to find an architectural approach that simplifies and
unifies the implementation of any kind of communication mechanism, may it be
for persistence, communication with remote systems or user interaction.\\
The first chapters elucidate three common design patterns that were used as a basis
to build on. The same sections describe all suggested modifications to these patterns.
Following chapters show details of the new design and how to integrate them into
physical architectures. Practical proof was given in form of the \emph{ReForm}
module described before the final summary.
