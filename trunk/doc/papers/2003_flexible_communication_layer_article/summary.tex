\section{Summary}
\label{summary_section}

Persistence, communication and user interface mechanisms have common properties.
Classical system architectures treat them as \emph{Backend}, \emph{DataTransfer}
and \emph{Frontend} and use different methods and design patterns (\emph{DataMapper},
\emph{DataTransferObject}, \emph{ModelViewController}) to implement them.\\
This paper proposes to sum up their common properties and behaviour in parent
classes and to merge them all into one architecture, avoiding redundant parts.
The resulting design is a good solution for the implementation of highly flexible,
easily extensible and maintainable, reusable source code. The interdependency of
domain data, persistence layer, communication layer and user interface is abolished.\\
The time needed to create such an architecture (like in form of the CYBOP framework)
is more than for the classical way. But once the architecture is there -- it can
save a tremendous amount of time when deriving modules being capable of communicating
across various mechanisms. Due to its flexibility and low dependencies, it also
ensures that extensions (e.g. new communication mechanisms) and modifications
can be done anytime later without destroying already existing solutions.
