\chapter*{Vorwort}
Die vorliegende Diplomarbeit entstand im Zusammenhang mit einem medizinischen Projekt des Instituts
f�r technische Informatik der TU-Ilmenau. Es tr�gt den Namen Res Medicinae. Hierbei soll eine
komponentenbasierte Anwendung realisiert werden, die es erm�glicht, Patientendaten zu verwalten und
unter anderem auch
graphisch auszuwerten.\\
Die einzelnen Komponenten sollen autonom als eigenst�ndige Applikation arbeiten, aber auch
miteinander interagieren k�nnen.\\
Eine Teilaufgabe der Diplomarbeit war das Erstellen einer dieser Komponenten, die das Drucken
medizinischer Formulare realisiert. Die Verwaltung von Patienten- und Metadaten stellte einen
weiteren Schwerpunkt dar. Hierzu war es insbesondere notwendig, sich mit der Entwicklung von
Java-Applikationen vertraut zu machen. Es mussten eine XML-Dateiverwaltung und eine
JDBC-Datenbankanbindung realisiert werden. Im letzten Abschnitt der Diplomarbeit wurde untersucht,
ob eine Anbindung verschiedener Middleware zur Interprozesskommunikation in das Res Medicinae
-Projekt m�glich ist. So wie dem Anwender bei der Datenverwaltung der momentan verwendete
Persitenzmechanismus transparent erscheint, so soll auch verborgen bleiben, welches
Kommunikationsparadigma im Augenblick zum Einsatz kommt.\\
Besonderes Augenmerk wurde auf die Umsetzung verschiedener Software-Muster gelegt, mit denen eine
Strukturierung der Anwendung erfolgt, wodurch diese eine flexiblere Gestalt erh�lt.\\\par

Nach einem kurzen Vorwort wird im zweiten Kapitel mit den Grundlagen f�r die in der Diplomarbeit
verwendeten Software-Technologien begonnen. Es gibt eine Einteilung von Mustern im Allgemeinen,
bevor auf einige spezielle Beispiele eingegangen wird. Der anschlie�ende Abschnitt besch�ftigt sich
mit der Einf�hrung in die oben bereits angesprochenen Persistenzmechanismen und belegt deren
Notwendigkeit. Das vierte und das f�nfte Kapitel diskutieren die Umsetzung der in vorangegangenen
Abschnitten beschriebenen Technologien. Ein letzter Punkt zieht noch einmal R�ckschl�sse aus der
Umsetzung und gibt Empfehlungen f�r weiterf�hrende Realisierungsm�glichkeiten.\\\par

Im laufenden Text sind einige Begriffe durch \emph{kursive Schrift}  hervorgehoben. Inklusive einer
Erl�uterung findet man diese im Glossar des Anhangs wieder.\\
Der Stil \tt Typewriter Typeface \rm wurde verwendet, um Java-Programmzeilen hervorzuheben. Es wird
zus�tzlich an den entsprechenden Stellen gesondert darauf hingewiesen.

\par
\vspace{2cm}

Ilmenau, \heute
\hfill Torsten Kunze\\
