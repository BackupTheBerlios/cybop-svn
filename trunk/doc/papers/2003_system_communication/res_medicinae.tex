%
% $RCSfile: res_medicinae.tex,v $
%
% Copyright (c) 2001-2004. Christian Heller. All rights reserved.
%
% No copying, altering, distribution or any other actions concerning this
% document, except after explicit permission by the author!
% At some later point in time, this document is planned to be put under
% the GNU FDL license. For now, _everything_ is _restricted_ by the author.
%
% http://www.cybop.net
% - Cybernetics Oriented Programming -
%
% http://www.resmedicinae.org
% - Information in Medicine -
%
% @author Christian Heller <christian.heller@tuxtax.de>
%

\subsection{Res Medicinae}
\label{res_medicinae_heading}

The practical background for the application of CYBOP is \emph{Res Medicinae}
\cite{resmedicinae}. A modern clinical information system is the aim of all
efforts in this project. In the future, it shall serve medical documentation,
laboratory data, billing etc.\\
\emph{Res Medicinae} is separated into single modules solving different tasks.
One module in which the CYBOP communication concepts were applied is \emph{ReForm}
(figure \ref{reform_module_figure}). It offers a medical form that can be filled
in and printed out. Since one of the main reasons to implement this module was
the testing and proof of the new persistence and communication concepts, it
includes a dialog for choosing the communication protocol or persistence mechanism,
respectively. This is the only remaining part where users have to care about the
underlaying techniques. They also have to decide whether to use the local file
system via \emph{Extensible Markup Language} (XML) format or to store the data
in a central database. In the future, an XML file format may as well be used for
remote communication, e.g. via \emph{Simple Object Access Protocol} (SOAP).\\
Because of the component-based design of \emph{Res Medicinae}, it is possible
to start more than one instance of \emph{ReForm} at the same time. In this way,
the data exchange between modules can be tested.
A module \emph{X} looks for another registered module \emph{Y} at the naming service
of \emph{Remote Method Invocation} (RMI), \emph{Common Object Request Broker
Architecture} (CORBA) or some other. \emph{X} gets the address of the remote service
\emph{Y} (depending on the communication mechanism). The stub and skeleton of
\emph{X} and \emph{Y} marshal, send and unmarshal the data for further working.

\begin{figure}[ht]
    \begin{center}
        \includegraphics[scale=0.24]{vector/reform_module.eps}
        \caption{ReForm Module}
        \label{reform_module_figure}
    \end{center}
\end{figure}

