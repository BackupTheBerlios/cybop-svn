%
% $RCSfile: abstract.tex,v $
%
% Copyright (c) 2001-2004. Christian Heller. All rights reserved.
%
% No copying, altering, distribution or any other actions concerning this
% document, except after explicit permission by the author!
% At some later point in time, this document is planned to be put under
% the GNU FDL license. For now, _everything_ is _restricted_ by the author.
%
% http://www.cybop.net
% - Cybernetics Oriented Programming -
%
% http://www.resmedicinae.org
% - Information in Medicine -
%
% @author Christian Heller <christian.heller@tuxtax.de>
%

\begin{center}
    \textbf{\large{Abstract}}
\end{center}
\normalsize
\textit{
This paper introduces an improved architecture for system communication.
The architecture is based on the Translator pattern which is derived from
existing design patterns. Hierarchical abstraction and ontologies are used
to combine these basic patterns and to merge their advantages into one,
domain- and language-independent software framework.\\
This conceptual framework, called Cybernetics Oriented Programming (CYBOP),
has its roots in the layered architecture pattern and is characterized by
flexibility and extensibility. It helps to structure software as well as to
keep it maintainable. A Component Lifecycle ensures the proper startup and
shutdown of any systems built on top of CYBOP.\\
Great influence was exerted by the biological model of information processing
in the human brain. It provided the idea of a seemless integration of communication
paradigms and persistence mechanisms. Overcoming the classical scheme of thinking
in terms of Domain, Frontend, Backend and Communication, this architecture treats
them all similar, as passive data models which can be translated into each other
-- as opposed to the classical approach that unnecessarily complicates their design.\\
The practical proof of this combined architectural approach was accomplished
within an (ongoing) effort to design and develop a module called ReForm, for the
Open Source Software (OSS) project Res Medicinae. The main task for this module
is to provide a user interface for printing medical forms. It was used to examine
the communication between modules and to find a structure for effective
implementation and easy expansion.\\
\textbf{Keywords.} CYBOP, Design Pattern, Hierarchy, Ontology, Translator,
Assembler, Mapper, Communication, Backend, Persistence, Frontend, User Interface,
Res Medicinae
} \rm

