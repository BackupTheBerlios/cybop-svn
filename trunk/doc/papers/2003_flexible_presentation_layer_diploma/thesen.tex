 %
 % $RCSfile: thesen.tex,v $
 %
 % Copyright (c) 1999-2002. Jens Bohl. All rights reserved.
 %
 % This software is published under the GPL GNU General Public License.
 % This program is free software; you can redistribute it and/or
 % modify it under the terms of the GNU General Public License
 % as published by the Free Software Foundation; either version 2
 % of the License, or (at your option) any later version.
 %
 % This program is distributed in the hope that it will be useful,
 % but WITHOUT ANY WARRANTY; without even the implied warranty of
 % MERCHANTABILITY or FITNESS FOR A PARTICULAR PURPOSE. See the
 % GNU General Public License for more details.
 %
 % You should have received a copy of the GNU General Public License
 % along with this program; if not, write to the Free Software
 % Foundation, Inc., 59 Temple Place - Suite 330, Boston, MA  02111-1307, USA.
 %
 % http://www.resmedicinae.org
 % - Information in Medicine -

 %The theses.

\section*{Thesen}
\addcontentsline{toc}{section}{\protect\numberline{Anhang D}{\hspace{1.2cm}Thesen}}\lhead{Anhang D \hspace{2mm} Thesen}
\begin{itemize}
    \item{Die Einhaltung softwareergonomischer Prinzipien gew�hrleistet Akzeptanz einer Software durch den Benutzer und
    bedingt gleichzeitig ihren effektiven Einsatz.}
    \item{Benutzerfreundlichkeit einer Oberfl�che wird durch den Einsatz eingabeunterst�tzender Komponenten, die
    ihrerseits ein Ma� an ''Intelligenz'' besitzen, erh�ht (Word Wheel).}
    \item{Konzepte der Natur (Lebenszyklus, Ontologien) sind bis zu einem gewissen Grad auch auf Probleme der
    Softwaretechnik �bertragbar.}
    \item{Der Lebenszyklus von Softwarekomponenten ist eines dieser Prinzipien, die die Natur zum Vorbild nimmt. Durch
    ihn werden Objekte einem geordneten Kreislauf unterworfen, der die Kontrolle von Status und Abh�ngigkeiten dieser
    Entit�ten erm�glicht.}
    \item{Die Entwicklung �ber Frameworks beschleunigt die Realisierung grosser Systeme. Gleichzeitig ist aber die
    genaue Analyse und Planung von Frameworks notwendig, da nachtr�gliche Modifikationen auch zu �nderungen der Systeme f�hren,
    die auf Frameworks basieren. Dieses vorausschauende Denken ist �u�erst schwierig.}
    \item{Kaskadierung und Aggregation von Entwurfsmustern kann die Vorteile dieser zusammenf�hren und Wartbarkeit und
    Erweiterbarkeit von Software erh�hen. Ontologien sind ein Beispiel f�r ein solches Vorgehen.}
    \item{Ontologien, angelehnt an nat�rliche Hierarchien, gew�hrleisten die Kontrolle von Objektreferenzen durch eine
    eindeutig definierte Richtung der Assoziationen. Dies kann aber nur bei strikter Einhaltung der Regeln einer Ontologie
    der Fall sein.}
    \item{Ontologien machen den Gebrauch von Schnittstellen (Interfaces) sowie abstrakter Klassen aufgrund ihres
    streng hierarchischen Charakters �berfl�ssig. Aber hier liegt auch die Kritik dieses Denkmodells: Die Einhaltung
    dieser Hierarchie ist zwingend notwendig und stellt damit (wie jedes Framework) auch ein Einschr�nkung der eigenen
    Kreativit�t des Entwicklers dar.}
    \item{Open Source verbindet das Wissen vieler unterschiedlicher Entwickler und macht Software auf Dauer erschwinglich.
    Open Source stellt zuk�nftig -- aber auch schon teilweise heute -- eine Alternative zu den kommerziellen Produkten
    der aktuellen Marktf�hrer dar.}
    \item{Die triviale und sequentielle Auflistung von Patientendaten in Listen oder einfachen Tabellen offenbart Schw�chen
     in der �bersichtlichkeit der Informationen und l�sst zeitliche Zusammenh�nge zwischen Problemen ausser Acht.
     Die episodenbasierte Befundung pr�sentiert einen alternativen Weg, medizinische Informationen zu strukturieren.
    }
    \item{Topologische Befundung unterst�tzt den Mediziner visuell bei der Anfertigung von Befunden und ist ein
    hilfreiches Werkzeug zur Lokalisation �rtlich begrenzter pathologischer Erscheinungen.}
    \par
    \vspace{3cm}

    \begin{tabbing}
      Ilmenau, den \today \=\hspace{4cm}\= --------------------------- \\
      \> \> \hspace{0.5cm} \= Jens Bohl\\
    \end{tabbing}

    \end{itemize}
\clearpage
