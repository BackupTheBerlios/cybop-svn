\lhead{Anhang E \hspace{2mm} Thesen}
\section*{Thesen}
\addcontentsline{toc}{section}{\protect\numberline{Anhang E}{\hspace{1.2cm}Thesen}}

\begin{itemize}

\item{JDBC bietet eine gute M�glichkeit f�r den Zugriff auf Datenbanken innerhalb von Java-Programmen
und gewinnt zunehmend an Bedeutung.}\\

\item{SQL ist ein Standard, der in seinen Kernfunktionen eine gute Interoperabilit�t
mit Datenbanken liefert, aber aufgrund h�ufig auftretender Verwendung propriet�rer Operationen eine
sehr komplexe Portierarbeit nach sich zieht.}\\

\item{Die Extensible Markup Language (XML) als weltweit anerkannter Standard wird eine f�hrende Rolle beim Speichern
und Austauschen von Daten �ber Netzwerke einnehmen. Sie bietet eine robuste M�glichkeit, die Daten
auszutauschen und auf Ver�nderungen in Server- oder Clientstruktur zu reagieren, ohne mit einen
vollst�ndigen Datenverlust rechnen zu m�ssen, im Gegensatz zu bekannten
Objektserialisierungsverfahren.}\\

\item{Der Datenaustausch in einem XML-Format - anstelle einer bin�ren Serialisierung - bringt allerdings
einen gr��eren Datenverkehr mit sich.}\\

\item{Muster erlauben es, Software flexibler f�r Erweiterungen und robuster gegen�ber
�nderungen zu gestalten, beispielsweise bei Verwendung des Architekturmusters Model View Controller
f�r die Entwicklung einer vern�nftig strukturierten Pr�sentationsschicht.}\\

\item{Das Anbieten eines einzelnen Persistenzmechanismus reicht f�r die heutigen Anforderungen
an zuverl�ssige medizinische Software nicht aus.}\\

\pagebreak

\item{Die Verwendung des Data Mapper Musters h�lt Gesch�ftslogik und Persistenzmechanismen
unabh�ngig voneinander und erm�glicht daher die �nderung eines der beiden, ohne die Notwendigkeit
einer Modifikation des anderen zu verursachen.}\\

\item{Mehrere verschiedene Persistenzmechanismen k�nnen durch eine
Mappingschicht in einer Anwendung vereinigt werden und unabh�ngig voneinander operieren.}\\

\item{Zus�tzlich k�nnen in die selbe Schicht unterschiedliche Kommunikationsparadigmen eingebunden werden,
um somit ebenfalls eine Transparenz und Flexibilit�t bei der Interprozesskommunikation zu
gew�hrleisten.}\\

\item{Die Anwendung der Strukturmuster Remote Facade und Data Transfer Object bei
Interprozesskommunikation verbessert die Performance der Applikation.}\\

\item{Drei- und N-Tier-Modelle erh�hen die Flexibilit�t von Softwaresystemen und
vermeiden die Entstehung von Fat-Clients.}\\

\item{Andererseits entlasten Fat-Clients den Server, da alle grundlegenden Operationen bereits im Client
erfolgen. Daf�r m�ssen aber bei �nderungen in der Struktur der Daten oder des Programmes auf jedem
Client die selben Anpassungen vorgenommen werden.}\\

\end{itemize}

\vspace{2cm}

Ilmenau, den \Abgabetermin \hfill -------------------------------------------- \\
\hspace*{12.2cm} Torsten Kunze \\
