\chapter*{Kurzfassung}
Die vorliegende Diplomarbeit entstand im Zusammenhang mit einem informationstechnischen Projekt f�r
den Einsatz in der Medizin. Es tr�gt den Namen \emph{Res Medicinae}. Hierbei soll eine
komponentenbasierte Anwendung realisiert werden, die es erm�glicht, Patientendaten zu verwalten und
unter anderem auch graphisch auszuwerten.\\
Die einzelnen Komponenten sollen autonom als eigenst�ndige Applikationen arbeiten, aber auch
miteinander interagieren k�nnen.

\emph{Res Medicinae} existierte zu Beginn der Diplomarbeit nur in Grundz�gen. Unter anderem fehlte
die M�glichkeit, Daten persistent zu sichern. Daher war ein Schwerpunkt der Diplomarbeit die
Modellierung einer Persistenzschicht zu einem anzupassenden Domain-Modell, welches s�mtliche
Gesch�ftsdaten enth�lt. Eine zus�tzliche, dem Anwender transparente Schicht zwischen Domain-Modell
und Persistenzschicht verhindert die direkte Abh�ngigkeit der Persistenzlogik von den Domain-Daten
und bildet je nach Funktion (Speichern oder Laden) jeweils den einen auf den anderen Teil ab. Diese
Schicht wurde durch Umsetzung und Anpassung des Musters \emph{Data Mapper} modelliert. In
Voraussicht einer Verlagerung dieser Mapping-Schicht in einen separaten Prozess auf einen
m�glicherweise entfernten Rechner wurden die Untersuchungen f�r die Integration verschiedener
Kommunikationsparadigmen wie Java RMI, JMS, CORBA in die Mapping-Schicht erweitert. Als Schwerpunkt
und wesentliche Neuerung gegen�ber fr�heren Ans�tzen erhielt sie eine eigene Bezeichnung:
\emph{Layer PerCom} und kennzeichnet damit die Verwaltung bzw. Abbildung von Persistenz und
Kommunikation in einer einzigen Schicht. F�r die Persistenzmechanismen besteht bereits eine
Implementation als Umsetzung in \emph{Res Medicinae}. Die Interprozesskommunikation ist lediglich
durch die abstrakte Definition der Klassen des zugeh�rigen Modells im Programmcode enthalten. Damit
ist die Basis f�r eine sp�tere praktische Realisierung geschaffen.\\
Zum Testen der Persistenzmechanismen und der Funktionsf�higkeit von \emph{Layer PerCom} wurde der
Prototyp eines Moduls f�r \emph{Res Medicinae} entwickelt, der gespeicherte Informationen l�dt bzw.
neue Daten ablegen und modifizieren l�sst und diese f�r den Druck von medizinischen Formularen
aufbereitet.

Die Komplexit�t und der Umfang der Architektur des gesamten Softwaresystems \emph{Res Medicinae}
haben sich durch Hinzuf�gen der neuen Elemente um etwa ein Drittel vergr��ert.\\
Mit Hilfe der entworfenen Modelle und der darauf basierenden, teilweise praktischen Umsetzung ist
es nun m�glich, Daten lokal in XML-Dateien oder in einer entfernten relationalen Datenbank
abzuspeichern und somit verteilt auf den selben Daten zu arbeiten. Die erneut vom jeweiligen
Persistenzmedium geladen Informationen k�nnen in vorgefertigte medizinische Formulare ausgedruckt
werden. Andere Module nutzen ebenfalls bereits \emph{Layer PerCom} und die Persistenzschicht. In
ankn�pfenden Arbeiten, nach Implementation der modellierten Klassen zu den
Kommunikationsparadigmen, kann die Funktionalit�t f�r eine gegenseitige Interaktion der einzelnen
Module erweitert werden. Weiterhin ist abzusehen, dass eine Vielzahl zus�tzlicher Informationen in
die Datenbank und in das XML-Dateisystem aufgenommen werden muss, so dass die Erweiterung um
Tabellen bzw. Tags ein Bestandteil weiterf�hrender Untersuchungen sein k�nnte.

%Die Verwaltung von Patienten- und Metadaten stellte einen weiteren Schwerpunkt dar. Hierzu war es
%insbesondere notwendig, sich mit der Entwicklung von Java-Applikationen vertraut zu machen. Es
%m�ssen eine XML-Dateiverwaltung und eine JDBC-Datenbankanbindung entworfen werden. Im letzten
%Abschnitt der Diplomarbeit wurde untersucht, ob eine Anbindung verschiedener Middleware zur
%Interprozesskommunikation in das Res Medicinae -Projekt m�glich ist. So wie dem Anwender bei der
%Datenverwaltung der momentan verwendete Persitenzmechanismus transparent erscheint, so soll auch
%verborgen bleiben, welches
%Kommunikationsparadigma im Augenblick zum Einsatz kommt.\\
%Besonderes Augenmerk wurde auf die Umsetzung verschiedener Software-Muster gelegt, mit denen eine
%Strukturierung der Anwendung erfolgt, wodurch diese eine flexiblere Gestalt erh�lt.

\par
\vspace{2cm}

Ilmenau, den \Abgabetermin
\hfill Torsten Kunze\\
