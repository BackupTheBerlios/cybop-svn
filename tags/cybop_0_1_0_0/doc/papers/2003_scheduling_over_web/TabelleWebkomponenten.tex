\begin{table}[ht]
  \caption{Tabelle der verwendeten Komponenten}
  \centering
  \begin{tabular}{|l|p{10cm}|}
    \hline
    \textbf{Komponente} & \textbf{Beschreibung} \\ \hline
    \hline
    
    Webserver &
      Ein Server, der auf Anforderung Web-Seiten zu einem 
      HTML-Browser mittels dem Protokoll HTTP �bertr�gt.              
    \\ \hline
    Applicationserver &
      Erm�glicht die Ausf�hrung komplexerer
      Aufgaben an einem entfernten Rechner oder �ber das Internet. Auf diese 
      Weise wird die Kommunikation zwischen Server und Benutzer einer 
      Webseite verbessert, so dass dieser z. B. eine Datenbank abfragen  
      oder Programme ausf�hren kann, die auf dem Server installiert sind.
      Application-Server bieten h�ufig zus�tzlich Sicherheitsmerkmale, 
      Lastenverteilung (load balancing) und Ausfallmechanismen (failover mechanism) 
      sowie Skalierungs- und Integrationsfunktionen.        
    \\ \hline
    Datenbankserver: &
      Die Aufgabe des Servers ist die Verwaltung und 
      Organisation der Daten, die schnelle Suche, das Einf�gen 
      und das Sortieren von Datens�tzen. Auf diesem Server l�uft eine
      Datenbank, die diese Aufgaben �bernimmt.
    \\ \hline
    eMail-Server: &
      Dieser Server stellt Dienste f�r das Verschicken und Empfangen  
      von eMails  bereit.
    \\ \hline
    Browser: &
      Ein Browser ist in erster Linie ein Anwendungsprogramm zur Anzeige 
      von HTML-Seiten. Mit zunehmender Verbreitung der Internet-Technologien 
      entwickelt sich der Browser zum Universal-Client und dient als 
      Schnittstelle f�r s�mtliche Informationen und Anwendungen, 
      die auf Internet-Technologien basieren. 
    \\ \hline

  \end{tabular}
\end{table}
