\section{JDBC}

  \subsection{�berblick}


		JDBC steht f�r Java Database Connectivity und ist die Datenbankschnittstelle
		f�r Java. Ab JDK 1.1 bietet Java durch JDBC eine vollst�ndige 
		Datenbankunterst�tzung zum Zugriff auf relationale Datenbanken an.
		Dazu stellt JDBC folgende Funktionalit�t zur Verf�gung:\cite{hajdbc}
		
		\begin{itemize}
			\item Objekte f�r die Verbindung mit Datenbanken
			\item Ausf�hren von SQL-Anweisungen mit gespeicherten Prozeduren
			\item Methoden zur Ermittlung von Informationen �ber Objekte
			      in der Datenbank
		\end{itemize}
		
		Mittels JDBC wird ein Datenbankzugriff von Java-Anwendungen
		�ber verschiedene Treiber m�glich. Es gibt sowohl spezielle Treiber f�r 
		einen native-Zugriff auf Datenbanken, als auch allgemeine Treiber 
		f�r den Datenbankzugriff �ber ODBC (siehe Treibertypen). Durch 
		die JDBC-Spezifikation
		wird eine maximale Flexibilit�t sowohl f�r Datenbankbenutzer
		als auch f�r Datenbankanbieter gew�hrleistet. 
		
		Um eine Austauschbarkeit der Datenbanken halbwegs zu gew�hrleisten, 
		fordert SUN von den JDBC-Treiberherstellern, mindestens den 
		SQL-2 Entry-Level-Standard von 1992 zu erf�llen. 
		Verwendet der Programmierer dar�ber hinaus propriet�re SQL-Anweisungen
		der Datenbanken, so ist eine Austauschbarkeit nur mit erheblichem Mehraufwand 
		zu erreichen.
		
	
	
  \subsection{Treibertypen}
  
    JDBC ist keine eigene Datenbank, sondern eine Schnittstelle zwischen einer 
    SQL-Datenbank und der Applikation, die sie benutzen will. Bez�glich der 
    Architektur der zugeh�rigen Verbindungs-, Anweisungs- und Ergebnisklassen 
    unterscheidet man vier Typen von JDBC-Treibern \cite{gkhjp}: 
	
		\begin{itemize}
		  \item Steht bereits ein ODBC-Treiber zur Verf�gung, so kann er mit Hilfe der 
		  			im Lieferumfang enthaltenen JDBC-ODBC-Bridge in Java-Programmen 
		  			verwendet werden. Diese Konstruktion bezeichnet man als Typ-1-Treiber. 
		  			Mit seiner Hilfe k�nnen alle Datenquellen, f�r die ein ODBC-Treiber 
		  			existiert, in Java-Programmen genutzt werden. 
		  \item Zu vielen Datenbanken gibt es neben ODBC-Treibern auch spezielle 
		  			Treiber des jeweiligen Datenbankherstellers. Setzt ein JDBC-Treiber 
		  			auf einem solchen propriet�ren Treiber auf, bezeichnet man ihn 
		  			als Typ-2-Treiber.
		  \item Wenn ein JDBC-Treiber komplett in Java geschrieben und auf dem Client 
		  			keine spezielle Installation erforderlich ist, der Treiber zur 
		  			Kommunikation mit einer Datenbank aber auf eine funktionierende Middleware 
		  			angewiesen ist, handelt es sich um einen Typ-3-Treiber. 
			\item Falls ein JDBC-Treiber komplett in Java geschrieben ist und die JDBC-Calls 
						direkt in das erforderliche Protokoll der jeweiligen Datenbank umsetzt, 
						handelt es sich um einen Typ-4-Treiber. 
		\end{itemize}
	 
