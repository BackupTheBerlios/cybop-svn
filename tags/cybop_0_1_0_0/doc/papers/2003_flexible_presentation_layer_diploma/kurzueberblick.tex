%
 % $RCSfile: kurzueberblick.tex,v $
 %
 % Copyright (c) 1999-2002. Jens Bohl. All rights reserved.
 %
 % This software is published under the GPL GNU General Public License.
 % This program is free software; you can redistribute it and/or
 % modify it under the terms of the GNU General Public License
 % as published by the Free Software Foundation; either version 2
 % of the License, or (at your option) any later version.
 %
 % This program is distributed in the hope that it will be useful,
 % but WITHOUT ANY WARRANTY; without even the implied warranty of
 % MERCHANTABILITY or FITNESS FOR A PARTICULAR PURPOSE. See the
 % GNU General Public License for more details.
 %
 % You should have received a copy of the GNU General Public License
 % along with this program; if not, write to the Free Software
 % Foundation, Inc., 59 Temple Place - Suite 330, Boston, MA  02111-1307, USA.
 %
 % http://www.resmedicinae.org
 % - Information in Medicine -

 %A preface.
\newpage
\chapter*{Vorwort}
    Das vorliegende Dokument stellt die Diplomarbeit von Jens Bohl zur Erlangung des Diplomtitels der Studienrichtung
    Informatik der Technischen Universit�t Ilmenau dar. \\
    Es ist das Ergebnis einer mehrmonatigen, intensiven, theoretischen wie auch praktischen Auseinandersetzung mit
    einem Thema, welches sich aufgrund seines interdisziplin�ren Charakters nicht nur einem konkreten
    wissenschaftlichen Forschungsbereich zuordnen l�sst.
    Vielmehr umfasst der Inhalt dieser Arbeit mehrere unterschiedliche Aspekte der Informatik:
    Komponentenbasierte Softwareentwicklung, Entwurfsmuster und Softwarearchitekturen, Medizinische Informatik und
    Softwareergonomie. Dieses breite Spektrum an Betrachtungen, bestehend aus den verschiedensten Bereichen der
    Informatik, zeichnet den vielseitigen Charakter dieser Arbeit aus.\\
    Die Wahl des Themas der Diplomarbeit liegt in den unterschiedlichen Interessen des Autors begr�ndet. Zum Einen sind
    diese klar im Bereich der Informationstechnologie -- genauer des Softwareentwurfes und -design  -- angesiedelt. Zum
    Anderen stand nach Erlangung der Allgemeinen Hochschulreife die Entscheidung zwischen einem Studium der Informatik
     -- oder der Medizin. Trotz des Entschlusses, Informatik den Vorzug zu geben, ist das Interesse an medizinischen Sachverhalten
    seitdem nicht verloren gegangen und spiegelt sich nun im praktischen Abschnitt dieser Diplomarbeit wider.\\
    Der Kern des theoretischen Teils manifestiert sich in neuesten Ideen und Forschungsergebnissen auf dem
    Gebiet der komponentenbasierten Softwareentwicklung, und hierbei im Speziellen im Entwurf eines ontologischen
    Software-Frameworks. Dieses Framework basiert auf der Idee, jedes Element eines Softwaresystems einer streng
    hierarchischen Struktur unterzuordnen. Es gleicht einem Schichtensystem, wobei die Granularit�t von Ebene zu Ebene
    zunimmt. Jede objektorientierte Klasse einer Anwendung kann in eine dieser Schichten eingeordnet werden.\\
    Ein solches Framework bezeichnet man als {\it Ontologie}.\\
    Existentielle Objektbeziehungen werden dabei durch den erweiterten Lebenszyklus von Softwarekomponenten
    beschrieben. Ein solcher Kreislauf st�tzt sich auf die Festlegung, dass jede Instanz
    genau ein Elternobjekt besitzt, durch das es initialisiert wird und von dem es existenziell abh�ngt.
    Dabei wurden bereits bestehende Konzepte wie bekannte Architekturmuster oder der {\it Component Lifecycle}
     von Apache \cite{avalon} um eigene Ideen erweitert.\\
    All diese theoretischen Betrachtungen vereinigen sich im praktischen Teil dieser
    Diplomarbeit.
    Vor dem Hintergrund der Entwicklung einer medizinischen Software fanden die beschriebenen theoretischen
    Konzepte eine anwendungsorientierte Umsetzung.
    Das entstandene Programm ist ein Prototyp, da es weder komplett in Funktionsumfang, noch voll einsetzbar ist.
    Dennoch l�uft es keineswegs Gefahr, nach relativ kurzer Zeit in Vergessenheit zu
    geraten. Die Software ist Teil eines klinischen Informationssystems namens {\it Res Medicinae}\footnote{Res Medicinae ({\it lat.}):
    die Medizin betreffend, Sache der Medizin}, welches in der Zukunft einzigartig in Umfang und
    Umsetzung sein wird und l�ngerfristig eine Alternative zu existierenden kommerziellen Produkten darstellen soll.
    Dabei ist nicht nur die konkrete Realisierung des Prototypen, sondern auch die in ihm vorhandene Umsetzung neuester
    medizinischer Modelle zur Diagnostik und Befundung von Interesse. Letzteres spielt in dieser technischen Diplomarbeit
    zwar eine eher untergeordnete Rolle, wird aber im Kontext des vorliegenden Dokumentes aus Gr�nden des besseren Verst�ndnisses
    fachlicher Zusammenh�nge nicht unerw�hnt bleiben.\par
    Die Struktur und Reihenfolge der folgenden Kapitel ist so angelegt, dass dem Leser ein unkomplizierter Zugang
    zum Thema geboten und er dann Schritt f�r Schritt �ber die softwaretheoretischen Grundlagen der Arbeit
    zu den enthaltenen eigenen Ideen und ihrer Umsetzung gef�hrt wird.\\
    Kapitel zwei bildet im Anschluss an die Einleitung mit Betrachtungen zur Gestaltung von Pr�sentationsschichten
    den Einstieg in diese Arbeit. Nach einer kurzen Vorstellung der unterschiedlichen Arten von Benutzerschnittstellen
    werden ergonomische Gesichtspunkte bei der Konzeption von grafischen Oberfl�chen dargelegt und
    technische M�glichkeiten der Programmierung solcher Schnittstellen gezeigt.\\
    Im Anschluss an das zweite folgen drei Kapitel, die das Kernst�ck der theoretischen Arbeit darstellen: Muster, Frameworks
    und komponentenbasierte Softwareentwicklung werden vorgestellt, um diese ''State-Of-The-Art''-Technologien als
    Ausgangspunkt der selbsterarbeiteten Konzepte zu betrachten. Diese eigenen Ideen werden
    auch an passender Stelle in den drei Kapiteln dargelegt -- und das immer in Bezug auf den praktischen Teil
    dieser Diplomarbeit.\\ Diesem ist dann auch mit Kapitel sechs ein eigener Abschnitt gewidmet, dessen Inhalt
    zun�chst noch einmal die Darlegung der Ziele dieser Arbeit ist.
    Daraufhin werden die zugrundeliegenden medizinischen Modelle pr�sentiert und dann die praktische Umsetzung im Rahmen des bereits
    erw�hnten Prototypen besprochen.\\
    Das letzte Kapitel
    enth�lt abschlie�ende Bemerkungen und gibt einen Ausblick auf zuk�nftige Arbeiten zu diesem Thema.\par
    Ausz�ge aus dem Programmcode werden in der Schriftart {\tt Typewriter Typeface},
    fremdsprachige Bezeichnungen sowie Eigennamen und Zitate
    {\it kursiv} dargestellt.
    Begriffe, die einer zus�tzlichen Erl�uterung bed�rfen, sind lediglich als Fu�noten
    angef�hrt, da ein Glossar den Lesefluss der Arbeit beeintr�chtigen w�rde. Wichtige Abk�rzungen und �bersetzungen
    finden sich im Anhang wieder.
    \par
    \vspace{2cm}

    Ilmenau, den \today
    \hfill Jens Bohl\\
