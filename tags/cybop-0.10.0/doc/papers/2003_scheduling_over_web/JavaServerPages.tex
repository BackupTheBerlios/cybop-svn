\section{Java Server Pages (JSP)}

  \subsection{�berblick}
  
  	Java Server Pages basieren auf der Servlettechnologie
  	von Sun. In diesem Kapitel wird die JSP-Technologie 
  	kurz vorgestellt. 
    Es werden aber keine Vergleiche zu anderen Techniken
    behandelt. Es wird hier auch keine Sprachbeschreibung f�r JSP 
    geben. An dieser Stelle wird nur das prinzipielle Vorgehen der 
    JSP-Technologie in Hinblick auf die Trennung der
    Ansicht und der Applikation erkl�rt. Dazu geh�ren die Techniken 
    JavaBeans und Tags.
    
  \subsection{Beschreibung}      
    JSP sind eine verbreitete M�glichkeit, Webseiten dynamisch zu
    gestalten. Fokussiert wurde dabei ein m�glichst einfaches 
    Zusammenspiel zwischen HTML-Code 
    und des leistungsstarken Java API. Zusammen mit der erleichterten 
    Einbindung von JavaBeans lassen sich mit erheblich verringertem 
    Aufwand selbst komplexe Webapplikationen erstellen.
    Syntaktisch gesehen handelt es sich bei einer JSP-Seite um 
    ein HTML-Dokument mit eingebetteten Anweisungen einer 
    Programmiersprache und zus�tzlichen Tags. 
    
  \subsection{Typischer Ablauf einer JSP-Anfrage}
    JSP-Seiten sind eine Mischung aus HTML und eingebettetem Java-Code, die durch 
    JSP-Tags eingef�gt werden. Ruft ein Benutzer eine JSP-Seite auf, sendet
    der Webserver nicht wie bei einer HTML-Seite einfach den Quelltext
    an den Browser. Stattdessen reicht er den Aufruf an eine JSP-Engine weiter.
    Sie �bersetzt das JSP mit einem JSP-Compiler in ein Java-Servlet. 
    Die �bersetzung des JSP in ein Servlet findet nur statt, wenn das Servlet 
    noch nicht existiert oder der JSP-Quelltext ge�ndert wurde. Dieses Servlet
    wird bearbeitet, und das Ergebnis wird in den Tags der JSP-Seite 
    hinzugef�gt. Erst diese generierte Seite wird dem Benutzer auf seine 
    Anfrage zur�ck geschickt. 
    
    \begin{center}
      \includegraphics[  width=15cm]{AblaufJSP}
    \end{center}
    Dies ist dem Buch \cite{swjsp} entnommen.


  \subsection{Java Beans}
     
    Um ein Bean in JSP nutzen zu k�nnen, mu� es in der JSP-Seite
    definiert sein.
\begin{verbatim}
<jsp:useBean 
  id="myModelDoctor" 
  scope="session"
  class="org.resmedicinae.application.healthcare.resdata.model.ResDataDoctor"
/>
\end{verbatim}
    Nach der Definition k�nnen �ber das Bean die Methoden der Klasse
    genutzt werden. Auf die Beans kann auf folgende Weise zugegriffen werden:
    \begin{itemize}
      \item Normaler Zugriff auf die Beans z.B. mit 
\begin{verbatim}
<\% bean.getDoc_name(); \%>
\end{verbatim}
      \item Zugriff auf die Beans mittels Tags, z.B. 
\begin{verbatim}
<jsp:getProperty name="myModelDoctor"  property="doc_name" />
\end{verbatim}
    \end{itemize}
    
    Leider haben Beans auch Nachteile. Sie k�nnen nicht auf implizite Objekte
    wie 'response' oder 'session' zugreifen, und sie k�nnen keine direkten Ausgabe
    mittels 'out.println()' machen. 
    
  \subsection{Tags}
  
		Eine wirkliche gelungene Sache zur Trennung von Ansicht und Verarbeitungslogik ist die
		Integration von TagLibs in JSP ab der Version 1.1. JSP's sind nun durch eigene Tags und 	
		Funktionen erweiterbar. Der in den ersten Versionen m�gliche Zugriff auf JavaBeans bezog
		sich auf Properties. Mit den TagLibs lassen sich Elemente f�r Funktionen ebenfalls
		integrieren. Die Tags sind XML-konforme Erweiterungen der bisherigen Standard-Tags. 
		Diese Syntax erleichtert die Zusammenarbeit zwischen HTML-Designer und Entwicklern. 
		Es k�nnen leicht spezielle Vereinfachungen f�r die dynamischen Inhalte der Site zur
		Verf�gung gestellt werden, ohne dass Hunderte von Zeilen Codes oder mehrere
		include-Anweisungen in die Seite einzubauen w�ren. 
		Der Entwickler kann eine neue Eigenschaft oder einen Fehler 
		korrigieren, und das Ergebnis wird sofort in allen Seiten aktiviert, ohne sie 
		zu modifizieren. 
  
    Tags dienen also dazu, Anwendungslogik und Darstellung zu trennen. 
    Sie unterliegen nicht der Beschr�nkung wie die JavaBeans. Tags k�nnen 
    auf implizite Objekte zugreifen und k�nnen HTML-Text �ber
    out.println() ausgeben. 
    
    Ein Tag ist ein St�ck Text in der Form
\begin{verbatim}
  <Pr�fix:Tagname attr1="wert1" attr2="wert2" ... > </Pr�fix:Tagname>  
  bzw. die Kurzform
  <Pr�fix:Tagname attr1="wert1" attr2="wert2" ... />
\end{verbatim}
    
    Dabei steht der Pr�fix f�r den Namensraum, in dem der Tagname G�ltigkeit hat. 
    Dies dient dazu, Namenskonflikte zu vermeiden, weil der Tagname f�r 
    einen speziellen Namensraum definiert wird. 
    
    