\section{JavaBeans}
  
	\subsection{�berblick}  	

    Spricht man von Java Beans, so ist die komponentenbasierte
    Anwendungsentwicklung in der Sprache von SUN Microsystems gemeint. 
    Ausgehend von der von JavaSoft vorgestellten Programmiersprache "`JAVA"'
    wurde die Objektorientierung zu einer Komponentenorientierung
    weiter entwickelt.
    
    Grundidee des �bergangs zu Komponenten ist der verst�rkte Wunsch 
    nach einer Wiederverwendung bestimmter, allgemeing�ltiger
    Programmteile. So schreibt M. Morrieson: "`Eine Komponente
    ist ein wiederverwendbares St�ck Software, das leicht mit anderen 
    zusammengebaut werden kann, um auf diese Weise Anwendungen
    viel effizienter zu erstellen."' \cite{mmjavabeans} 
    Ziel der Wiederverwendung ist also 
    eine schnellere kosteng�nstigere Softwareentwicklung.
  
  \subsection{Beschreibung}
    Beans wurden vor allem f�r die visuelle Softwareentwicklung
    entwickelt. Die wichtigsten Konzepte von JavaBeans sind 
    Ereignisse (events) und Eigenschaften (properties). 
    Diese werden hier im folgenden kurz beschrieben.

    \begin{itemize}
      \item Ereignisse \\
				    JavaBeans definiert ein Standardmodell zur Kommunikation von 
				    Komponenten �ber einen Ereignismechanismus. Der Ereignismechanismus erlaubt
				    die weitgehende Entkopplung von Sender- und Empf�ngerkomponenten. 
				    Die Sender definieren eine Ereignisschnittstelle, welche die Anmeldung
				    von Empf�ngern f�r ein Ereignis erlaubt. 
				    Empf�nger definieren Methodenschnittstellen, die bei Auftreten der
				    Ereignisse angesprochen werden.
      \item Eigenschaften \\
						Eigenschaften sind benannte Attribute eines Beans, 
						die das Erscheinungsbild und/oder das Verhalten 
						beeinflussen k�nnen. Sie sind Teil des Zustands von Beans.     
		\end{itemize}
		
    Um auf die JavaBeans zugreifen zu k�nnen, m�ssen Informationen
    zu diesen ermittelbar sein. Dies geschieht durch ein 
    Introspektionsmechanismus. Das kann bei Einhaltung der
    Namenskonvention durch den Reflectionmechanismus von Java oder bei 
    Nichteinhaltung �ber die BeanInfo-Schnittstelle geschehen.
    
    Weitergehende Informationen zu JavaBeans sind z. B. unter \cite{dfjava} 
    und \cite{afbeans} nachzulesen.
    
 
	\subsection{Konventionen}
    JavaBeans folgen im allgemeinen einem speziellen 
    Design- und Namensmuster. F�r Java Server Pages 
    sind vor allem die Eigenschaften von JavaBeans
    von Bedeutung. 
    F�r die Verwendung von JavaBeans in den Java
    Server Pages sind folgende 
    Konventionen einzuhalten: 
    \begin{itemize}
      \item m�ssen im Package sein
      \item Default-Konstruktor (Konstruktor ohne Parameter)
      \item Definieren einer Menge von Properties
      \item f�r den lesenden und schreibenden Zugriff auf die
            Properties werden get- und set-Methoden definiert 
    \end{itemize}
		Werden die get- bzw. set-Methoden weggelassen, so hat man entweder 
		nur lesend oder nur schreibend auf die Eigenschaften (Properties) 
		Zugriff.

		Weitere Informationen zu Konventionen zu JavaBeans entnehmen Sie bitte
		der Spezifikation von Sun. Diese ist unter
		 http://java.sun.com/beans/spec.html 
		erreichbar.
    
