\section{Einf�hrung}
  
  \subsection{Ziel}
    Im Rahmen der vorliegenden Studienarbeit soll untersucht werden,
    wie intuitive Frontends �ber einen Webserver mit Hilfe der 
    JSP-Technologie realisiert werden k�nnen. Der Aufgabenbereich ist dabei 
    die Verwaltung administrativer Daten und die Terminplanung mit 
    Anbindung an ein medizinisches Informationssystem. Daf�r soll 
    eine Referenzimplementierung umgesetzt werden.
    
    %Technologien%
    Zum besseren Verst�ndnis wird in den ersten Kapiteln auf allgemeine
    Technologien f�r Webanwendungen eingegangen. Dies umfa�t eine
    allgemeine Web-Architektur sowie die verwendeten 
    Technologien (Servlets, JSP, JDBC, Java Beans, Tags).
    
    %Model-View-Controller%
    Des weiteren ist auf ein flexibles Design und die Erweiterbarkeit 
    der Anwendung Wert gelegt worden. Daf�r ist es sinnvoll,
    auf Erfahrungen von anderen zur�ck zu greifen. Diese Erfahrungen
    sind in Entwurfsmustern enthalten. Ein g�ngiges Entwurfsmuster
    ist das Model-View-Controller Muster (MVC-Muster). Dabei wird eine 
    Trennung von Inhalt, Darstellung und Interaktion gemacht. Diese Trennung 
    ist wichtig, um schnell �nderungen an einzelnen Komponenten, wie z.B.
    der Ansicht, zu realisieren. 
    
    %Prototyp%
    Um die hier beschriebenen Verfahren und Technologien zu 
    veranschaulichen, geh�rt zu dieser Arbeit eine 
    Referenzimplementierung einer Terminvergabe f�r
    eine Arztpraxis. Diese wurde nach dem 
    Model-View-Controller Muster und den hier vorgestellten 
    Technologien realisiert.
    
  \subsection{Umfeld}
    Die Studienarbeit wird im Rahmen des Projektes Res Medicinae
    realisiert. Dieses Open Source Projekt verfolgt das Ziel,
    medizinische Anwendung als freie Software zur Verf�gung
    zu stellen. Sie soll in Debian als ein Paket integriert werden. 
    Dazu ist es notwendig, dass die erstellten Programme und 
    Dokumentationen unter die GNU-Lizenz gestellt sind.
    