\section{Web-Architektur}
  \subsection{�berblick}
    In diesem Abschnitt wird eine einfache Hardwarearchitektur
    f�r die Webanwendung beschrieben. Die Web-Architektur geh�rt
    zu den Client-Server-Architekturen. Da f�r eine Webanwendung 
    verschiedene 
    Komponenten zusammenarbeiten, werden diese Komponenten hier vorgestellt
    und das Zusammenwirken der Komponenten beschrieben. Da es eine Vielzahl 
    von M�glichkeiten und Kombinationen der Komponenten gibt, kann dies hier 
    nicht umfassend dargestellt, sondern nur angerissen werden. 

	\subsection{Client-Server-Architektur}
		
		Hier ist die Definition aus dem Buch \cite{hwinf}
		\begin{quotation}
			"`Unter der Client-Server-Architektur (engl.: client-server architecture) 
			versteht man eine kooperative Informationsverarbeitung, bei der die 
			Aufgaben zwischen Programmen auf verbundenen Rechnern aufgeteilt werden. 
			In einem solchen Verbundsystem k�nnen Rechner aller Art zusammenarbeiten. 
			Server (= Dienstleister; Backend) bieten �ber das Netz Dienstleistungen 
			an, Clients (= Kunden; Frontend) fordern diese bei Bedarf an. Die 
			Kommunikation zwischen einem Client-Programm und dem Server-Programm 
			basiert auf Transaktionen, die vom Client generiert und dem Server 
			zur Verarbeitung �berstellt werden. Eine Transaktion (engl.: transaction) 
			ist eine Folge logisch zusammengeh�riger Aktionen, beispielsweise 
			zur Verarbeitung eines Gesch�ftsvorfalls. Client und Server k�nnen �ber 
			ein lokales Netz verbunden sein oder sie k�nnen �ber gro�e Entfernungen 
			hinweg, zum Beispiel �ber eine Satellitenverbindung, miteinander kommunizieren. 
			Dabei kann es sich um Systeme jeglicher Gr��enordnung handeln; das 
			Leistungsverm�gen des Clients kann das des Servers also durchaus �bersteigen. 
			Grundidee der Client-Server-Architektur ist eine optimale Ausnutzung der 
			Ressourcen der beteiligten Systeme."'
		\end{quotation}
  
  \subsection{Verwendetet Komponenten}
    Ein Server ist ein Prozess, Programm oder Computer zur Bearbeitung 
    der Anforderungen eines Clients bzw. zur Bereitstellung von Diensten, 
    die von einem Client genutzt werden k�nnen.
    
    Um die typische Webanwendung umzusetzen, bedarf es folgender Komponenten:
    \begin{enumerate}
      \item Webserver
      \item Applicationserver 
      \item Datenbankserver
      \item eMail-Server
      \item Browser
    \end{enumerate}
    
    \begin{table}[ht]
  \caption{Tabelle der verwendeten Komponenten}
  \centering
  \begin{tabular}{|l|p{10cm}|}
    \hline
    \textbf{Komponente} & \textbf{Beschreibung} \\ \hline
    \hline
    
    Webserver &
      Ein Server, der auf Anforderung Web-Seiten zu einem 
      HTML-Browser mittels dem Protokoll HTTP �bertr�gt.              
    \\ \hline
    Applicationserver &
      Erm�glicht die Ausf�hrung komplexerer
      Aufgaben an einem entfernten Rechner oder �ber das Internet. Auf diese 
      Weise wird die Kommunikation zwischen Server und Benutzer einer 
      Webseite verbessert, so dass dieser z. B. eine Datenbank abfragen  
      oder Programme ausf�hren kann, die auf dem Server installiert sind.
      Application-Server bieten h�ufig zus�tzlich Sicherheitsmerkmale, 
      Lastenverteilung (load balancing) und Ausfallmechanismen (failover mechanism) 
      sowie Skalierungs- und Integrationsfunktionen.        
    \\ \hline
    Datenbankserver: &
      Die Aufgabe des Servers ist die Verwaltung und 
      Organisation der Daten, die schnelle Suche, das Einf�gen 
      und das Sortieren von Datens�tzen. Auf diesem Server l�uft eine
      Datenbank, die diese Aufgaben �bernimmt.
    \\ \hline
    eMail-Server: &
      Dieser Server stellt Dienste f�r das Verschicken und Empfangen  
      von eMails  bereit.
    \\ \hline
    Browser: &
      Ein Browser ist in erster Linie ein Anwendungsprogramm zur Anzeige 
      von HTML-Seiten. Mit zunehmender Verbreitung der Internet-Technologien 
      entwickelt sich der Browser zum Universal-Client und dient als 
      Schnittstelle f�r s�mtliche Informationen und Anwendungen, 
      die auf Internet-Technologien basieren. 
    \\ \hline

  \end{tabular}
\end{table}


    \subsection{Zusammenspiel der Komponenten}
      Dies ist nur eine Beispielarchitektur. In dem Bild wird f�r jeden Server ein
      eigener Rechner verwendet. Dies wird in den meisten F�llen nicht so sein. 
      Es k�nnen auch mehrere Server (z.B. Webserver und Applicationserver) auf 
      einem Rechner betrieben werden.
      
    \begin{center}
      \includegraphics{Hardware-Architektur}
    \end{center}
