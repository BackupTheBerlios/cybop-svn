\section{Zusammenfassung, Ausblick}

  Zusammenfassend l��t sich sagen, dass die Umsetzung 
  von Webapplikationen mit JSP gut realisierbar sind. 
  Dazu tr�gt einerseits die plattformunabh�ngige 
  Programmiersprache Java und ihre definierten 
  Schnittstellen (JDBC, Reflection, Tag) bei. Andererseits 
  bietet JSP die M�glichkeit, �ber JavaBeans und Tags
  die Anwendungslogik und die Darstellung zu trennen. 
  Damit ist eine getrennte Entwicklung (Programmierer,
  Designer) der Verarbeitungslogik und des Designs m�glich.
  Darauf mu� aber bei der Erstellung der JSP-Seiten
  R�cksicht genommen werden. Ziel ist es, so wenig wie m�glich 
  Quelltext in den JSP-Seiten zu verwenden. Dies ist aber nicht
  zwingend bei JSP vorgeschrieben. 
  Programmierer und Designer m�ssen sich auf eine Schnittstelle
  f�r die �bergabe der Parameter und den Zugriff auf Modell
  und Controller einigen. 
  
  Das MVC-Muster konnte teilweise f�r die Entwicklung
  der Webapplikationen umgesetzt werden. Eine vollst�ndige 
  Realisierung war aufgrund der speziellen Kommunikationswege
  der Webapplikation nicht m�glich. Es besteht keine M�glichkeit,
  Aktualisierungen der Darstellung nach �nderungen von Daten auf dem 
  Sever vom Server zu veranlassen. Damit kann kein Beobachtungsmuster
  implementiert werden. Die Umsetzung des MVC-Muster wurde f�r die 
  Webapplikation neu erstellt. Eine weitere M�glichkeit w�re, auf ein 
  bestehendes Framework die Applikation aufzubauen. Damit w�rde man auf
  die Ressourcen und Erfahrungen von mehreren Entwicklern zugreifen
  und Entwicklungsaufwand f�r ein eigenes Framework minimieren. 
  
  Durch die definierte Schnittstelle JDBC von SUN ist es m�glich,
  durch Einsatz entsprechender Treiber die Datenbank flexibel 
  auszutauschen. Die Anwendung ist 	SQL-2 Entry-Level-Standard von 1992
  konform umgesetzt worden, so da� keine datenbankspezifische Anpassungen 
  vorzunehmen sind. In diesem Prototyp ist die Speicherung der Daten nicht 
  in XML gemacht worden. Dies w�re eine weitere M�glichkeit, die
  Daten abzulegen.
  
  Im Prinzip ist der Prototyp in dem jetzigen Umfang einsetzbar.
  Zu beachten ist aber, das f�r die Konfiguration und Wartung
  der Server (Webserver, Applikationserver) der Administrator
  verantwortlich ist. F�r Webapplikationen sind gewisse
  Sicherheitsrichtlinien einzuhalten. Weiterhin ist eine Kontrolle 
  der Sicherheitsrichtlinien und Wartung des Servers Aufgabe 
  des Administrators. Dies ist nicht immer ganz einfach und deswegen 
  ist hier Vorsicht geboten.
  
