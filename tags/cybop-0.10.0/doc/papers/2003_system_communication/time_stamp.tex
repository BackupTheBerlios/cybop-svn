%
% $RCSfile: time_stamp.tex,v $
%
% Copyright (c) 2001-2004. Christian Heller. All rights reserved.
%
% No copying, altering, distribution or any other actions concerning this
% document, except after explicit permission by the author!
% At some later point in time, this document is planned to be put under
% the GNU FDL license. For now, _everything_ is _restricted_ by the author.
%
% http://www.cybop.net
% - Cybernetics Oriented Programming -
%
% http://www.resmedicinae.org
% - Information in Medicine -
%
% @author Christian Heller <christian.heller@tuxtax.de>
%

\subsubsection{Time Stamp}
\label{time_stamp_heading}

Most database developers will know this technique. Each table has a separate
column for storing the time at which the data were written into this table.
If someone requests information from the database, the time stamp is delivered
as well. After modifying the data, they have to be written back into the database.
At this time, both timestamps (the one in the database table and the one delivered
before) are compared. If there is a difference, the data were modified by another
user. Then, one has to care about the update without overwriting the new data in
the table.

