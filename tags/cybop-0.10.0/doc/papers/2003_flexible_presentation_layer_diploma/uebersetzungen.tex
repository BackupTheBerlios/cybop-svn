 %
 % $RCSfile: uebersetzungen.tex,v $
 %
 % Copyright (c) 1999-2002. Jens Bohl. All rights reserved.
 %
 % This software is published under the GPL GNU General Public License.
 % This program is free software; you can redistribute it and/or
 % modify it under the terms of the GNU General Public License
 % as published by the Free Software Foundation; either version 2
 % of the License, or (at your option) any later version.
 %
 % This program is distributed in the hope that it will be useful,
 % but WITHOUT ANY WARRANTY; without even the implied warranty of
 % MERCHANTABILITY or FITNESS FOR A PARTICULAR PURPOSE. See the
 % GNU General Public License for more details.
 %
 % You should have received a copy of the GNU General Public License
 % along with this program; if not, write to the Free Software
 % Foundation, Inc., 59 Temple Place - Suite 330, Boston, MA  02111-1307, USA.
 %
 % http://www.resmedicinae.org
 % - Information in Medicine -

 %A section including translations of important terms.

\section*{�bersetzungen wichtiger Begriffe}

\lhead{Anhang C \hspace{2mm} �bersetzungen wichtiger Begriffe}

\addcontentsline{toc}{section}{\protect\numberline{Anhang C}{\hspace{1.2cm}�bersetzungen wichtiger Begriffe}}

\begin{tabbing}
\hspace{1cm} \= \hspace{6cm} -- \= \hspace{1.5cm}\= \kill
\>{\bf Abstract Factory}\>Abstrakte Fabrik\\
\>{\bf Assessment}\>Bemerkung\\
\>{\bf Chain of Responsibility}\>Kette der Zust�ndigkeiten\\
\>{\bf Composite}\>Kompositum\\
\>{\bf Decorator}\>Dekorierer\\
\>{\bf Description}\>Beschreibung\\
\>{\bf Disease}\>Krankheit\\
\>{\bf Domain}\>Dom�ne\\
\>{\bf Electronic Health Record}\>Elektronische Patientenakte\\
\>{\bf Garbage Collection}\>Speicherbereinigung\\
\>{\bf Homunculus}\>K�nstlich erzeugter Mensch\\
\>{\bf Layer}\>Schicht\\
\>{\bf Objective}\>Befund\\
\>{\bf Observer}\>Beobachter\\
\>{\bf Overlay}\>�berzug\\
\>{\bf pathologisch}\>krankhaft\\
\>{\bf Plan}\>Vorgehen\\
\>{\bf Res Medicinae}\>Sache der Medizin\\
\>{\bf Separation of Concern}\>Trennung der Angelegenheiten\\
\>{\bf Strategy}\>Strategie\\
\>{\bf Subjective}\>Anamnese\\
\>{\bf TargetOverlay}\>Ziel-Overlay\\
\>{\bf Tier}\>Schicht, Ebene\\
\>{\bf Word Wheel}\>Wort-Rad\\

\end{tabbing}
\clearpage
