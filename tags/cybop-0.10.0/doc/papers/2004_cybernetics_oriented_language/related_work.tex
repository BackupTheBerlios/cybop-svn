%
% $RCSfile: related_work.tex,v $
%
% Copyright (c) 2001-2004. Christian Heller. All rights reserved.
%
% No copying, altering, distribution or any other actions concerning this
% document, except after explicit permission by the author!
% At some later point in time, this document is planned to be put under
% the GNU FDL license. For now, _everything_ is _restricted_ by the author.
%
% http://www.cybop.net
% - Cybernetics Oriented Programming -
%
% http://www.resmedicinae.org
% - Information in Medicine -
%
% @author Christian Heller <christian.heller@tuxtax.de>
%

\section{Related Work}
\label{related_work_heading}

There are a number of efforts that go into a similar direction like CYBOP \cite{cybop}.
Basically, every application that stores configuration data (colours, fonts)
does use some kind of knowledge model for the file or database to save in.
However, they all are limited to their corresponding field. What this paper
proposes is, in short, to store complete systems in special configuration files
in CYBOL format. CYBOP wants to show an overall approach and provide the means
(CYBOL/ CYBOI) to build abstract software models for any possible application layer,
may it be a domain, user interface, workflow, data transfer object or storage.

The two projects mentioned following do related work, with focus on the medical
domain.

\subsubsection{Open Infrastructure for Outcomes} (OIO) \cite{oio} is a Web-based
data management system that uses forms (and workflows) which are defined in XML.
Its most critical point is that OIO forms mix user interface with domain model
data. Moreover, it misses a clear theory behind and does not distinguish static
and dynamic models.

\subsubsection{Open Electronic Health Record} (OpenEHR) \cite{openehr} is a
standardization effort that arose from a European initiative. Its \emph{Dual Model
Approach} also influenced CYBOP. The project's main aim is the creation of knowledge
templates (which it calls \emph{Archetypes}), for which an own \emph{Archetype
Definition Language} (ADL) was defined. A lot of emphasis is placed on constraint
inclusion, to ensure correct models. However, OpenEHR's model concepts are not
based on abstraction as it happens in the human mind. They do not clearly
distinguish between constraints, positions and the actual model information.
There is no facility for translating between archetypes \cite{openhealth}.
It offers only static archetype models, no dynamic workflows. ADL seems overly
complex and difficult to understand. The whole project still lacks implementation
experiences and practical proof of workability.
