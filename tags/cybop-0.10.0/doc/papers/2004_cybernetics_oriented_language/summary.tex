%
% $RCSfile: summary.tex,v $
%
% Copyright (c) 2001-2004. Christian Heller. All rights reserved.
%
% No copying, altering, distribution or any other actions concerning this
% document, except after explicit permission by the author!
% At some later point in time, this document is planned to be put under
% the GNU FDL license. For now, _everything_ is _restricted_ by the author.
%
% http://www.cybop.net
% - Cybernetics Oriented Programming -
%
% http://www.resmedicinae.org
% - Information in Medicine -
%
% @author Christian Heller <christian.heller@tuxtax.de>
%

\section{Summary}
\label{summary_heading}

This paper means that wild \emph{Dependencies} are a major reason for error-prone,
unstable, unflexible, unmaintainable software systems. Two facts causing such
dependencies are the \emph{Bundling} of static and dynamic properties by
object-oriented languages and the \emph{Mix} of knowledge and hardware control
in traditional programming languages. This information mix additionally forces
software development projects to run through a course of different abstraction
steps which would not differ if one common knowledge abstraction were used.

As solution to the above's problems, this document suggests to build software
systems after the concepts of \emph{Human Thinking}. The approach, named CYBOP,
such follows the recommendations of the science of \emph{Cybernetics} and its
specialization \emph{Bionics}, whereby biological principles should be applied
to the study and design of engineering systems. An abstract model as formed in the
human mind represents an \emph{Item}, \emph{Category} and \emph{Compound}, at the
same time. Additionally, it contains \emph{Meta Information} about its parts.
This information often corresponds to physical dimensions and determines whether
the model is an abstraction of \emph{static} or \emph{dynamic} real-world aspects.

The introduced \emph{CYBOL} language has the semantics to express knowledge models
as used by human thinking. It allows to create complete application systems. Its
syntax is based on \emph{XML} which results in absolutely platform- independent
system definitions. CYBOL files get interpreted by the \emph{CYBOI} interpreter
and can be changed at runtime. CYBOI manages all hardware access. It concentrates
model instances and signal handling in one place and such avoids memory leaks and
endless loops.

CYBOL models could be displayed graphically, using special design tools. But their
\emph{formal definition} also allows them to be used as main abstraction throughout
all phases in a software project's lifetime. Analysts and experts can start their
work by creating rudimentary CYBOL models (defining static structures and dynamic
processes) which software designers can later complete and check for correctness.
The implementation phase becomes superfluous at all: CYBOL models already represent
the system to be built, no further code is needed! It is hard to imagine the amount
of saved time and costs for software projects. Even better: Experts are placed in
a position to, themselves, actively help creating systems.
