%
% $RCSfile: syntax.tex,v $
%
% Copyright (c) 2001-2004. Christian Heller. All rights reserved.
%
% No copying, altering, distribution or any other actions concerning this
% document, except after explicit permission by the author!
% At some later point in time, this document is planned to be put under
% the GNU FDL license. For now, _everything_ is _restricted_ by the author.
%
% http://www.cybop.net
% - Cybernetics Oriented Programming -
%
% http://www.resmedicinae.org
% - Information in Medicine -
%
% @author Christian Heller <christian.heller@tuxtax.de>
%

\subsection{Syntax}
\label{syntax_heading}

An XML document carries a name and can such represent a \emph{Discrete Item}. It
can also link to other documents, such as one being a \emph{Super Category} to
the item currently considered. Most importantly, XML documents have a hierarchical
structure based on \emph{Tags} which may be used to model \emph{Parts} of a
\emph{Compound}. Tag \emph{Attributes} keep \emph{Meta Information} about the tag
contents.

Considering these properties of XML, it seems predestinated for formally
representing abstract models using the CYBOP concepts. CYBOL, finally, is XML
\emph{plus} a defined set of tags and attributes used to structure and link
models meaningfully. The tags are: \(<\)model\(>\), \(<\)super\(>\), \(<\)part\(>\).
