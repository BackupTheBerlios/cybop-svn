%
% $RCSfile: introduction.tex,v $
%
% Copyright (c) 2001-2004. Christian Heller. All rights reserved.
%
% No copying, altering, distribution or any other actions concerning this
% document, except after explicit permission by the author!
% At some later point in time, this document is planned to be put under
% the GNU FDL license. For now, _everything_ is _restricted_ by the author.
%
% http://www.cybop.net
% - Cybernetics Oriented Programming -
%
% http://www.resmedicinae.org
% - Information in Medicine -
%
% @author Christian Heller <christian.heller@tuxtax.de>
%

\section{Introduction}
\label{introduction_heading}

One important area the science of \emph{Informatics} deals with is software --
the art of \emph{representing} and \emph{processing} information. As such, one of
its major aims is to find \emph{Abstract Models} which represent the real world
best. The better this is done and the better information can be stored and
processed, the better software can assist its human users.

Since about 40 years, the same, often unsatisfying concepts are used in informatics,
which caused some people to talk about an ongoing \emph{Software Crisis}.
Since about 20 years, the \emph{Free and Open Source Software} (FOSS) Movement
increasingly eases that pain by providing a tremendous amount of code containing
plenty of new concepts but still -- the dream of true componentization and
reusability has not been reached.
\emph{Structured} and \emph{Object Oriented Programming} (OOP) delivered some new
concepts, a major one of was the extension of data \emph{Type} to \emph{Class},
owning inheritable properties and methods. However, there is a number of problems
that still keep us away from clear, effective and above all flexible solutions,
in particular the:

\begin{itemize}
    \item[-]{false combination and grouping of information}
    \item[-]{mix of knowledge and system control information}
    \item[-]{bundling of static and dynamic aspects}
\end{itemize}

A more detailed analysis of point one is given in \cite{hellerbohl}.
\emph{Ontologies} are suggested as means for improvement in \cite{hellerkunze}.
They help structuring data models by dividing a domain into singular \emph{Concepts}
(as known from \emph{Knowledge Engineering}) which later get associated with each
other. Elements of a concept are organized in strict layers which ensures
flexibility for later extensions. This paper wants to use those results but also
concentrate on point two and three, as listed above, and investigate their
negative effects and possible solutions.

As a system grows, the interdependencies between its single parts grow with.
Why does this happen? Simply because a clear architecture is missing. Even if
developers really try to follow a such -- on some point in the software's
lifetime, compromises have to be made due to unforeseen requirements and
dependencies:

\begin{itemize}
    \item[-]{multiple interfaces are used to realize new properties (Mix-In)}
    \item[-]{static manager objects accessible by any other objects in the system
        are introduced}
    \item[-]{new layers are plugged in with varying mechanisms}
    \item[-]{redundant code needs to be written to avoid too many unwanted
        interdependencies}
\end{itemize}

These decisions, in turn, can lead to buggy code with: memory leaks, endless loops,
false results, weak performance.
Can all that be avoided? And if, then how? The author's opinion is yes and the
new concepts and language introduced in this document show ways out of the misery.
