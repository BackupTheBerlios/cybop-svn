%
% $RCSfile: bidirectional_dependency.tex,v $
%
% Copyright (c) 2004. Christian Heller. All rights reserved.
%
% No copying, altering, distribution or any other actions concerning this
% document, except after explicit permission by the author!
% At some later point in time, this document is planned to be put under
% the GNU FDL license. For now, _everything_ is _restricted_ by the author.
%
% http://www.cybop.net
% - Cybernetics Oriented Programming -
%
% http://www.resmedicinae.org
% - Information in Medicine -
%
% @author Christian Heller <christian.heller@tuxtax.de>
%

\subsection{Bidirectional Dependency}
\label{bidirectional_dependency_heading}

\emph{Bidirectional References} are a nightmare for every software developer.
They cause \emph{Inter-Dependencies} so that changes in one part of a system can
affect multiple other parts which in turn affect the originating part, which may
finally lead to cycles or even endless loops. Also, the actual program flow and
effects of changes to a system become very hard to trace. Therefore, the avoidance
of such dependencies belongs to the core principles of good software design.

A \emph{Tree}, in mathematics, is defined as \textit{Directed Acyclic Graph}
(DAG), also known as \emph{Oriented Acyclic Graph} \cite{nist}. It has a
\emph{Root Node} and \emph{Child Nodes} which can become \emph{Parent Nodes}
when having children themselves; otherwise they are called \emph{Leaves}.
Children of the same node are \emph{Siblings}. \textit{A common constraint is
that no node can have more than one parent.}, as \cite{foldoc} writes and
continues: \textit{Moreover, for some applications, it is necessary to consider
a node's children to be an ordered list, instead of merely a set.} A graph is
\emph{acyclic} if every node can be reached via exactly one path, which then
also is the shortest possible.

In computing, trees are used in many forms, for example as \emph{Process Tree}
of an \emph{Operating System} (OS) or as \emph{Object Tree} of an object-oriented
application. They represent \emph{Data Structures} in databases or file systems
and also the \emph{Syntax Tree} of languages.

The violation of the principle of the \emph{Acyclic Graph} can lead to the same
loops, also called \emph{Circular References}, as mentioned above, which can
result in the crossing of memory limits and is a potential security risk.
