%
% $RCSfile$
%
% Copyright (c) 2005-2006. Christian Heller. All rights reserved.
%
% Permission is granted to copy, distribute and/or modify this document
% under the terms of the GNU Free Documentation License, Version 1.1 or
% any later version published by the Free Software Foundation; with no
% Invariant Sections, with no Front-Cover Texts and with no Back-Cover
% Texts. A copy of the license is included in the section entitled
% "GNU Free Documentation License".
%
% http://www.cybop.net
% - Cybernetics Oriented Programming -
%
% http://www.resmedicinae.org
% - Information in Medicine -
%
% Version: $Revision$ $Date$ $Author$
% Authors: Christian Heller <christian.heller@tuxtax.de>
%

\section{Related Work}
\label{related_work_heading}

There exists a plethora of -- partly \emph{Open Source Software} (OSS) -- projects
promoting XML-based programming, using the \emph{Extensible Markup Language}
(XML) as replacement for a programming language. Many of these in fact follow
the principles of \emph{Structured- and Procedural Programming} (SPP) or
\emph{Object Oriented Programming} (OOP) with just another syntax, and try to
map the corresponding constructs to XML. They are far away from the system
control/ knowledge separation that CYBOP wants to reach.

Further, XML is often used for specifying user interfaces or workflows, the
latter mostly in commercial systems. These approaches come closer to what CYBOP
does with its language CYBOL, only that CYBOL can express not only user
interfaces and workflows, but also domain knowledge and algorithms.

The idea of separating system control and knowledge is used in the \emph{OpenEHR}
project \cite{openehr}, which inspired CYBOP in its beginnings. OpenEHR follows
a meta model approach (which it calls \emph{Dual Model}) that is based on
Fowler's \emph{Analysis Patterns} \cite{fowler1997} describing a kind of ad hoc
two-level modelling, using a \emph{Knowledge Level} and \emph{Operational Level}
-- as described by the \emph{Reflection} pattern, which calls the two levels
\emph{Meta Level} and \emph{Base Level}, respectively. The difference between
the dual model approach and classical meta architectures is that the latter
implement both, meta- and base level using the same technology (language).
OpenEHR, on the other hand, uses so-called \emph{Archetypes} for specifying
knowledge, written in a special language. Besides obvious benefits of OpenEHR's
approach in constraining domain knowledge, there are a number of weaknesses:

\begin{itemize}
    \item[-] mix of meta information (properties, constraints) and hierarchical whole-part structure
    \item[-] incomplete domain knowledge lacking logic (algorithms/ workflows) and user interfaces
    \item[-] inflexible structures due to runtime-dependency of instances from archetypes
    \item[-] use of object-oriented concepts with all their limits
\end{itemize}

Although the \emph{OpenEHR} project claims archetypes to be both:
\emph{domain-empowered} and \emph{future-proof}, the above-mentioned issues
prevent them from being so. The dual model approach in conjunction with
archetypes only partly fulfills the expectations of independent and complete
knowledge structures.
