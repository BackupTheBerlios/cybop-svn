%
% $RCSfile$
%
% Copyright (c) 2002-2006. Christian Heller. All rights reserved.
%
% Permission is granted to copy, distribute and/or modify this document
% under the terms of the GNU Free Documentation License, Version 1.1 or
% any later version published by the Free Software Foundation; with no
% Invariant Sections, with no Front-Cover Texts and with no Back-Cover
% Texts. A copy of the license is included in the section entitled
% "GNU Free Documentation License".
%
% http://www.cybop.net
% - Cybernetics Oriented Programming -
%
% http://www.resmedicinae.org
% - Information in Medicine -
%
% Version: $Revision$ $Date$ $Author$
% Authors: Christian Heller <christian.heller@tuxtax.de>
%

\subsection{Motivation}
\label{motivation_heading}

To the issues that the work described in this article had with some
state-of-the-art solutions belong three things:

\begin{enumerate}
    \item \emph{Abstraction Gaps} in Software Engineering Process (section
        \ref{abstraction_gaps_heading})
    \item \emph{Misleading Tiers} in Physical Architecture (section
        \ref{misleading_tiers_heading})
    \item \emph{Modelling Mistakes} in Logical Architecture (section
        \ref{modelling_mistakes_heading})
\end{enumerate}

The traversing of abstraction gaps in a software engineering process belongs to
the main difficulties in software development, and causes considerable cost-
and time effort. It necessitates a steady synchronisation between domain
experts and application system developers, because their responsibilities
cannot be clearly separated and interests often clash. A first objective was
therefore to contribute to closing these gaps, especially the one existing
between a designed system architecture and the implemented source code.

The misinterpretation of the physical tiers in an information technology
environment often leads to wrong-designed software architectures. Logical
layers are adapted to physical tiers (frontend, business logic and backend) and
differing patterns are used to implement them. Instead, systems should be
designed in a way that allows the usage of a unified translator architecture,
so to give every application system the capability to communicate universally
by default, which was the second objective.

Several well-known issues exist with the modelling of logical system
architectures, for example: fragile base class problem, container inheritance,
bidirectional dependencies, global data access. These and others more result
from using wrong principles of knowledge abstraction, like the bundling of
attributes and methods in one class, as suggested by
\emph{Object Oriented Programming} (OOP), or the equalising of
structural- and meta information in a model. A third aim was therefore to
closer investigate the basic principles and concepts after which current
software systems are created, and to search for new concepts, with the
objective of finding a universal type structure (knowledge schema).
