\chapter{Einleitung und Motivation}
F�r medizinische Sachverhalte gibt es bereits eine Vielzahl an kommerziellen Programmen, die beim
L�sen von Problemen oder Verwalten von Daten helfen sollen. Diese Diplomarbeit leistet einen Teil
zur Entwicklung einer Open Source Applikation f�r die Arbeit mit Patientendaten beliebiger Art.\\
Die Problematik bei den meisten Programmen besteht in der eingeschr�nkte Flexibilit�t in Sachen
Erweiterbarkeit und in der schlechten Robustheit gegen�ber �nderungen. Daher sollen Untersuchungen
angestellt werden, ob die Verwendung von Software-Mustern diese Probleme minimiert.\\
Die Patientendaten werden in sogenannten Electronic Health Records (EHR) abgespeichert und zur
Verarbeitung im Programm gehalten. Es handelt sich bei einem EHR also um einen eigens kreierten
Datentyp der aus mehreren unterschiedlichen, zusammengesetzten Datentypen besteht und
patientenspezifische, medizinische Informationen enth�lt.\\
Werden die Daten nur auf eine Art gespeichert so schr�nkt das die Flexibilit�t der Software sehr
ein. Bei einer lokalen Speicherung bedarf es bei Anforderung von elektronischen Informationen einer
direkten Kopie auf ein Wechselmedium und dessen physischen Transport zum anderen Rechner. Dieser
Vorgang ist oft sehr zeitaufwendig und h�ufig treten durch unsachgem��e Behandlung des
Wechselmediums, z.B. heruntergefallene Diskette oder zerkratzte CD, Fehler auf, die bis zur
v�lligen Unbrauchbarkeit der Daten f�hren k�nnen. Das etwaige Verlieren des Datentr�gers kommt als
Sicherheitsrisiko hinzu. Eine zentrale Datenbank innerhalb eines Netzwerkes schliesst derartige
Nachteile zwar aus, verhindert allerdings die M�glichkeit weiterzuarbeiten, falls das Netzwerk
ausf�llt. Daher wurde untersucht inwieweit sich mehrere unterschiedliche Persistenzmechanismen in
einer Anwendung vereinen lassen, ohne die Benutzerfreundlichkeit und Bedienbarkeit einzuschr�nken
oder spezielles Fachwissen vom Anwender voraussetzen zu m�ssen.\\
Zu den theoretischen Betrachtungen dieser Diplomarbeit z�hlt noch ein weitere Abschnitt, der sich
damit befasst, wie Daten zwischen Prozessen ausgetauscht werden k�nnen. Da, wie bereits erw�hnt,
bei der Entwicklung der Anwendung Flexibilit�t und Erweiterbarkeit im Vordergrund stehen, darf die
Einschr�nkung und Spezialisierung auf einige wenige Kommunikationsparadigmen nicht erfolgen.
Vielmehr soll eine M�glichkeit gefunden werden, den Anwender zur Laufzeit entscheiden zu lassen,
welches Kommunikationsprotokoll er verwenden m�chte, wiederum ohne das er �ber fachliche
Kompetenz verf�gen muss.\\
Eine Interprozesskommunikation ist f�r diese Anwendung notwendig, da sie aus einzelnen Modulen
bestehen soll, von denen jedes als separater Prozess agiert, aber �ber die M�glichkeit verf�gen
muss, auf den selben Daten zu arbeiten, wie ein anderes Modul. Die Patientendaten zwischen zwei
lokalen Prozessen auszutauschen ist der eine Aspekt. Ein weiterer ist der Datentransfer zwischen
mehreren Rechnern.\\
Der medizinische Formulardruck dient als Beispielanwendung, um persistent abgelegte Daten wieder zu
laden und sinnvoll zu verarbeiten. Mit m�glichst geringem Aufwand sollen die Patientendaten in
medizinische Formularvordrucke ausgegeben werden k�nnen.

\par

In den beiden nachfolgenden Kapiteln werden die Grundlagen f�r L�sung der zu Untersuchenden
Problematik erl�utert.
