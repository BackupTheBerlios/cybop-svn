\chapter{Einleitung}

  \section{Ziel} 
    
    In der vorliegenden Diplomarbeit wird CYBOP (Cybernetics Oriented Programming)  
    und dessen Realisierungsm�glichkeiten 
    einer Webanwendung untersucht. Dies basiert auf den Konzepten von CYBOP.
    CYBOP setzt sich aus der Beschreibungssprache CYBOL und dem Interpreter
    CYBOI zusammen. Der Hauptgedanke in CYBOL ist,
    dass die Programmiersprache nicht dem technischem Umfeld (Computer)
    bzw. programmtechnischen Umfeld (Programm)
    gen�gt, sondern sich an dem menschlichen Denken orientiert.
    Dazu ist es immer noch n�tig die Mensch-Maschine-Schnittstelle
    zu definieren, nur dass diese Schnittstelle dem Menschen auf 
    fundamentaler Ebene verst�ndlich ist.     
    Auf der Basis dieser Sprache wird ein kleines Anwendungsprogramm entwickelt, 
    welches im Webbrowser ablaufen soll. F�r die Realisierung dieser Aufgabe 
    n�hert man sich dem Problem auf zwei Wegen, der erste Weg 
    aus der Sicht von CYBOP und der zweite aus der Sicht von Webanwendungen.
    Aus der Sicht von CYBOP ist zu kl�ren, auf
    welche Prinzipien CYBOP basiert und wie sich die Prinzipien in CYBOL
    auswirken. Danach wird die aktuelle technische Umsetzung des Interpreters
    CYBOI, sowie dessen Designaufbau und Entwicklungsprinzipien erkl�rt.
    F�r den zweiten Weg sind die Darstellungsm�glichkeiten in Webanwendungen zu untersuchen, 
    eine Auswahl aus diesen M�glichkeiten zu treffen  
    und die erforderlichen technischen Voraussetzungen zu kl�ren.
    Am Schluss sind beide Wege zu integrieren. 
    Daf�r ist zu pr�fen, welche Erweiterungen CYBOL braucht, 
    um die Definition f�r die Webanwendung vollst�ndig abzudecken, 
    wie die Umsetzung in CYBOI eingebunden wird und welche Probleme dabei 
    zu beachten und zu l�sen sind.
    
    
  \section{Umfeld}
    Die Diplomarbeit wird im Rahmen des Projektes CYBOP 
    realisiert. Dieses Open Source Projekt verfolgt das Ziel,
    Softwareentwicklung auf Basis von nat�rlichen Konzepten zu entwickeln
    und dies als Debian-Paket zur Verf�gung zu stellen. 
    Dazu ist es notwendig, dass die erstellten Programme und 
    Dokumentationen unter die GNU-Lizenz gestellt sind. 
    Weitere Informationen f�r das Debian-Projekt sind unter 
    \cite{debian} und f�r das GNU-Projekt unter \cite{gnu}
    zu finden.
  
  \section{Praktische Aufgabe}
  
    % grobe Aufgabe %
    F�r die praktische Aufgabe der Diplomarbeit ist eine Anwendung auf Basis von CYBOL 
    zu entwickeln, die die folgenden Aufgaben abdeckt:    
    % Aufgabe f�r die Darstellung des Webinhaltes %
    \begin{itemize} 
      \item Darstellung von Adressdaten in einer Tabelle
      \item die dargestellten Adressdaten sollen l�schbar und editierbar sein 
      \item es sollen neue Adressen eingebbar sein
    \end{itemize}
    
    Als Voraussetzung f�r diese Webanwendung sind zu erstellen:
    \begin{itemize}
      \item Domain-Modell in CYBOL
      \item Anwendungsmodell in CYBOL
      \item Weboberfl�chenbeschreibung in CYBOL
      \item Prototyp eines Webservers mit Integration in CYBOI
      \item Erweiterung von CYBOI f�r die Bearbeitung von Web-spezifischen Anfragen
    \end{itemize}
     
 