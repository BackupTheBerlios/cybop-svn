%
% $RCSfile: code_reduction.tex,v $
%
% Copyright (C) 2002-2008. Christian Heller.
%
% Permission is granted to copy, distribute and/or modify this document
% under the terms of the GNU Free Documentation License, Version 1.1 or
% any later version published by the Free Software Foundation; with no
% Invariant Sections, with no Front-Cover Texts and with no Back-Cover
% Texts. A copy of the license is included in the section entitled
% "GNU Free Documentation License".
%
% http://www.cybop.net
% - Cybernetics Oriented Programming -
%
% http://www.resmedicinae.org
% - Information in Medicine -
%
% Version: $Revision: 1.1 $ $Date: 2008-08-19 20:41:05 $ $Author: christian $
% Authors: Christian Heller <christian.heller@tuxtax.de>
%

\subsection{Code Reduction}
\label{code_reduction_heading}
\index{Code Reduction}
\index{Picture Element}
\index{Pixel}

In his book \emph{Programming Pearls} \cite[page 128]{bentley}, Jon Bentley
demonstrates \emph{Code Reduction} on the following graphics program example:

\begin{scriptsize}
    \begin{verbatim}
� � for i = [17, 43] set(i, 68)
� � for i = [18, 42] set(i, 69)
� � for j = [81, 91] set(30, j)
� � for j = [82, 92] set(31, j)
    \end{verbatim}
\end{scriptsize}

He suggests to replace the \emph{set} procedures that switch a
\emph{Picture Element} (Pixel) with suitable functions for drawing horizontal
and vertical lines:

\begin{scriptsize}
    \begin{verbatim}
� � hor(17, 43, 68)
� � hor(18, 42, 69)
� � vert(81, 91, 30)
� � vert(82, 92, 31)
    \end{verbatim}
\end{scriptsize}

This code, finally, gets reduced to pure data stored in an array:

\begin{scriptsize}
    \begin{verbatim}
� � h 17 43 68
� � h 18 42 69
� � v 81 91 30
� � v 82 92 31
    \end{verbatim}
\end{scriptsize}

The data can be read by an interpreter program which knows about their meaning.

Bentley's example shows in a nice way how knowledge can be extracted from program
source code. The graphic application's actual data are represented by the values
in the array above. All other functionality accessing and manipulating Pixels
directly does belong to system control and remains in the interpreter program.
Chapter \ref{cybernetics_oriented_interpreter_heading} will introduce an
interpreter that is able to read and handle \emph{general} knowledge, only on a
much larger scale.
