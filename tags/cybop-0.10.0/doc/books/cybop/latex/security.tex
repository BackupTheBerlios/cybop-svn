%
% $RCSfile: security.tex,v $
%
% Copyright (C) 2002-2008. Christian Heller.
%
% Permission is granted to copy, distribute and/or modify this document
% under the terms of the GNU Free Documentation License, Version 1.1 or
% any later version published by the Free Software Foundation; with no
% Invariant Sections, with no Front-Cover Texts and with no Back-Cover
% Texts. A copy of the license is included in the section entitled
% "GNU Free Documentation License".
%
% http://www.cybop.net
% - Cybernetics Oriented Programming -
%
% http://www.resmedicinae.org
% - Information in Medicine -
%
% Version: $Revision: 1.1 $ $Date: 2008-08-19 20:41:08 $ $Author: christian $
% Authors: Christian Heller <christian.heller@tuxtax.de>
%

\subsection{Security}
\label{security_heading}
\index{Security in CYBOI}
\index{CYBOI as Secure Architecture}
\index{CYBOI as Capability Based System}
\index{CYBOI avoiding Access Control Lists}
\index{Capability Based System}
\index{Access Control List}
\index{ACL}
\index{Orthogonally Persistent Operating System}
\index{Flex Machine}
\index{Ten15}

Berin Loritsch of the former Apache Avalon Project \cite{avalon} writes that
system \emph{Security} has three distinct concerns, of which \emph{Encryption}
is only a part:

\begin{enumerate}
    \item \emph{Authentication:} authoritative validation of the identity of a
        party, such as a software component
    \item \emph{Authorisation:} deciding what access a component has to system
        resources
    \item \emph{Architecture:} usage of a proper, secure architecture
\end{enumerate}

Since this chapter is about CYBOI's architecture, what interests the most here
is point number three. But how to ensure a secure architecture? The avoidance
of global data access and bidirectional dependencies is clearly a requirement
(section \ref{recommendation_heading}), which CYBOI accomplishes through
disregard of the corresponding patterns (section \ref{pattern_merger_heading}).
Any kind of application knowledge resides in its \emph{Knowledge Memory}, whose
tree structure may be navigated along well-defined, unidirectional paths.

One may wonder how address spaces of the different applications are protected,
so that one application may not access another one's models? Traditionally, the
process concept assures that separation, but with CYBOI being the only process,
and all applications being part of it, another solution needs to be applied.
The exact mechanism to solve this in CYBOI yet has to be determined. However,
since all signals have to pass the same, central signal loop, they all can be
checked for permissions, before being processed, or filtered out, if necessary.
It would be imaginable and not difficult, to attach an application name as a
signal's origin, or a unique passphrase as additional meta information to a
signal memory's signals, that own an \emph{Identifier} (ID) anyway. The signal
loop or signal handler could then decide whether or not to send a signal to a
specific application.

Attached meta information of that kind would be comparable to a concept called
\emph{Capability}, which is known from \emph{Secure Computing}, a subfield of
\emph{Security Engineering} \cite{wikipedia}. It is the alternative to
\emph{Access Control Lists} (ACL), another means of enforcing firstly:
\emph{Mandatory Access Control}, and secondly: \emph{Privilege Separation}
(where an entity has only the privileges that are needed for its function).
Wikipedia \cite{wikipedia} writes on this:

\begin{quote}
    \emph{Capabilities} (also known as \emph{Key}) achieve their objective of
    improving system security by being used in place of \emph{Plain References}.
    A plain reference (for example, a path name) uniquely identifies an object,
    but does not specify which \emph{Access Rights} are appropriate for that
    object and the user program which holds that reference. Consequently, any
    attempt to access the referenced object must be validated by the operating
    system, typically via the use of an \emph{Access Control List} (ACL). In
    contrast, in a pure capability-based system, the mere fact that a user
    program possesses that capability entitles it to use the referenced object
    in accordance with the rights that are specified by that capability. In
    theory, a pure capability-based system removes the need for any ACL or
    similar mechanism, by giving all entities all and only the capabilities
    they will actually need.
\end{quote}

With almost all important \emph{Operating Systems} (OS) still using ACL, for
various historical reasons, CYBOI could be seen as chance to implement a pure
capability-based system. Since it concentrates all knowledge in one container
realised as tree structure, and processes all signals in one central loop,
security by design is given, and security checks of different shade can be
easily applied to all knowledge models and all signals. These checks of dynamic
runtime models are a necessary supplement to the checks of static CYBOL template
files. Traditional OS do the latter via \emph{File Descriptors} (also called
\emph{File Handles}), which are facilities very similar, but not equal to
capabilities.

In the opinion of Wikipedia \cite{wikipedia}, one main reason why the benefits
of a pure capability-based system could not be realised in a traditional OS
environment, were the fact that entities which might hold capabilities (such as
processes and files) cannot be made persistent in such a way that maintains the
integrity of the secure information that a capability represents. The OS could
not trust a user program to read back a capability and not tamper with the
object reference or the access rights. Consequently, when a program wished to
regain access to an object that is referenced on disk, the OS had to have some
way of validating that access request, and an ACL or similar mechanism were
mandated.

\emph{Orthogonally Persistent} OS like the \emph{Flex Machine} and its successor
\emph{Ten15}, as \cite{wikipedia} writes, were a novel approach to solving this
problem. They maintained the integrity and security of the capabilities contained
within all storage, both volatile and nonvolatile (dynamic and static), at all
times. Further, such OS were responsible for storage allocation, deallocation
and garbage collection, which immediately precluded a whole class of errors
arising from the misuse (deliberate or accidental) of pointers. Two other
features were:

\begin{itemize}
    \item[-] \emph{Tagged, Write-Once Filestore}, which allows arbitrary code
        and data structures to be written and retrieved transparently, without
        recourse to external encodings; data could thus be passed safely from
        program to program
    \item[-] \emph{Remote Capabilities}, which allow data and procedures on
        other machines to be accessed over a network connection, again without
        the application program being involved in external encodings of data,
        parameters or result values
\end{itemize}

This reads like a description of CYBOI, which is able to parse/ serialise all
knowledge from/ to CYBOL sources (e.g. files), and to handle low-level storage-
and communication mechanisms. More in section \ref{model_transition_heading}.
If, in this manner, user programs (CYBOL applications) were relieved of these
responsibilities, there would be no need to trust them to reproduce only legal
capabilities, nor to validate requests for access using an ACL-like mechanism.
However, these were just some thoughts on how to bring yet more security into
CYBOI's architecture. The details will have to be figured out in future works
(chapter \ref{summary_and_outlook_heading}).
