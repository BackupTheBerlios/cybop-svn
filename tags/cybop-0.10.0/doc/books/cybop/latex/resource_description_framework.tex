%
% $RCSfile: resource_description_framework.tex,v $
%
% Copyright (C) 2002-2008. Christian Heller.
%
% Permission is granted to copy, distribute and/or modify this document
% under the terms of the GNU Free Documentation License, Version 1.1 or
% any later version published by the Free Software Foundation; with no
% Invariant Sections, with no Front-Cover Texts and with no Back-Cover
% Texts. A copy of the license is included in the section entitled
% "GNU Free Documentation License".
%
% http://www.cybop.net
% - Cybernetics Oriented Programming -
%
% http://www.resmedicinae.org
% - Information in Medicine -
%
% Version: $Revision: 1.1 $ $Date: 2008-08-19 20:41:08 $ $Author: christian $
% Authors: Christian Heller <christian.heller@tuxtax.de>
%

\subsubsection{Resource Description Framework}
\label{resource_description_framework_heading}
\index{Resource Description Framework}
\index{RDF}
\index{Extensible Markup Language}
\index{XML}
\index{RDF Schema}
\index{XML Schema}
\index{OWL}

The \emph{Resource Description Framework} (RDF) \cite{rdf} as part of the
\emph{Semantic Web} provides a standard way for simple descriptions to be made.
It is: \textit{a simple data model for referring to objects (resources) and how
they are related. An RDF-based model can be represented in XML syntax.}
\cite{wikipedia}

RDF wants to achieve for \emph{Semantics} what XML has achieved for
\emph{Syntax} -- to provide a clear set of rules for creating descriptive
information. Both follow a special schema, \emph{RDF Schema} \cite{rdf} and
\emph{XML Schema} \cite{xmlschema}, respectively, which defines the structure
and vocabulary that may be used in the corresponding documents.

Many applications that use XML as syntax for data interchange, may apply the
RDF specifications to better support the exchange of actual knowledge on the
web. The RDF data framework is used \cite{rdfowlrelease} in: asset management,
enterprise integration and the sharing and reuse of data on the web. Example
applications combining information from multiple sources on the web
\cite{rdfowlrelease} include: library catalogs, world-wide directories, news-
and content aggregation, collections of music or photos.

In the words of Brian McBride \cite{rdfowlrelease}, chair of the RDF core
working group, his group had: \textit{turned the RDF specifications into both a
practical and mathematically precise foundation on which OWL and the rest of
the semantic web can be built.}

Chapter \ref{cybernetics_oriented_language_heading} will come back to RDF once
more, and compare it with the new language then introduced.
