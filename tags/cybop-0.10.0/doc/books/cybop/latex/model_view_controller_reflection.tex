%
% $RCSfile: model_view_controller_reflection.tex,v $
%
% Copyright (C) 2002-2008. Christian Heller.
%
% Permission is granted to copy, distribute and/or modify this document
% under the terms of the GNU Free Documentation License, Version 1.1 or
% any later version published by the Free Software Foundation; with no
% Invariant Sections, with no Front-Cover Texts and with no Back-Cover
% Texts. A copy of the license is included in the section entitled
% "GNU Free Documentation License".
%
% http://www.cybop.net
% - Cybernetics Oriented Programming -
%
% http://www.resmedicinae.org
% - Information in Medicine -
%
% Version: $Revision: 1.1 $ $Date: 2008-08-19 20:41:07 $ $Author: christian $
% Authors: Christian Heller <christian.heller@tuxtax.de>
%

\subsubsection{Model View Controller Reflection}
\label{model_view_controller_reflection_heading}
\index{Model View Controller Pattern}
\index{MVC}
\index{Frontend}
\index{Graphical User Interface}
\index{GUI}
\index{Domain Model}
\index{User Interface Model}
\index{UI Model}

After having had a closer look at two common software patterns for persistence
and communication, this section finally considers the so-called \emph{Frontend}
of an application, which is often realised in form of a
\emph{Graphical User Interface} (GUI). Nowadays, the well-known
\emph{Model View Controller} (MVC) pattern is used by a majority of standard
applications. Its principle is to have the \emph{Model} holding domain data,
the \emph{View} accessing and displaying these data and the \emph{Controller}
providing the workflow of the application by handling any signals (events/
actions) appearing on the view.

Since the view serves as means of communication between a software system and
its user (\emph{Human Being} as system), it is in fact just another kind of
communication model that should be assembled by a special \emph{Translator}.
Because there are many ways in which domain data can be displayed, different
types of user interfaces exist (textual, graphical, web, vocal, Braille). Each
of them has to have its very own translator that knows how to map data both
ways, from the \emph{Domain} model to the \emph{User Interface} (UI) model and
vice-versa.

Signalling and related mechanisms, as well as hardware-driving functionality
such as graphics adapter access belong into \emph{System Control} modules; UI
translators, on the other hand, are application-specific models containing
\emph{Logic Knowledge}.
