%
% $RCSfile: data_transfer_object_reflection.tex,v $
%
% Copyright (C) 2002-2008. Christian Heller.
%
% Permission is granted to copy, distribute and/or modify this document
% under the terms of the GNU Free Documentation License, Version 1.1 or
% any later version published by the Free Software Foundation; with no
% Invariant Sections, with no Front-Cover Texts and with no Back-Cover
% Texts. A copy of the license is included in the section entitled
% "GNU Free Documentation License".
%
% http://www.cybop.net
% - Cybernetics Oriented Programming -
%
% http://www.resmedicinae.org
% - Information in Medicine -
%
% Version: $Revision: 1.1 $ $Date: 2008-08-19 20:41:06 $ $Author: christian $
% Authors: Christian Heller <christian.heller@tuxtax.de>
%

\subsubsection{Data Transfer Object Reflection}
\label{data_transfer_object_reflection_heading}
\index{Data Transfer Object Pattern}
\index{DTO}
\index{Domain Model}
\index{Assembler}
\index{Translator Model}

The \emph{Data Transfer Object} (DTO) pattern proposes to bundle domain data
before sending/ receiving them among systems. An \emph{Assembler} packs/
unpacks needed domain data into/ from a flat data structure called DTO.

Comparing with the data mapper described before, the assembler's task of
translating between data models seems quite similar. Wouldn't it be possible,
hence, to use \emph{Translator} models (logic knowledge) similar to those
suggested for persistence, also for inter-system communication? Different types
of translator models could be provided for different communication protocols.
Again, communication mechanisms would be put into the \emph{System Control}
layer, and translator logic into application-related \emph{Knowledge} models.
