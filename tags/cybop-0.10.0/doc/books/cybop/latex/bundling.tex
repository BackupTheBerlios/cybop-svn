%
% $RCSfile: bundling.tex,v $
%
% Copyright (C) 2002-2008. Christian Heller.
%
% Permission is granted to copy, distribute and/or modify this document
% under the terms of the GNU Free Documentation License, Version 1.1 or
% any later version published by the Free Software Foundation; with no
% Invariant Sections, with no Front-Cover Texts and with no Back-Cover
% Texts. A copy of the license is included in the section entitled
% "GNU Free Documentation License".
%
% http://www.cybop.net
% - Cybernetics Oriented Programming -
%
% http://www.resmedicinae.org
% - Information in Medicine -
%
% Version: $Revision: 1.1 $ $Date: 2008-08-19 20:41:05 $ $Author: christian $
% Authors: Christian Heller <christian.heller@tuxtax.de>
%

\subsection{Bundling}
\label{bundling_heading}

OOP
\emph{Classes} as known from \emph{Object Oriented Programming} (OOP) bundle
attributes and methods which often causes additional inter-dependencies in a
system because classes do not only have to relate to other classes for accessing
their attributes but also for using the methods offered by them.

Bean/ Enterprise Java Bean (EJB)

- JMS Eigenschaften werden in EJBs eingebaut, so dass diese selbststaendig
andere EJBs aufrufen koennen.
- Verlagerung eines Teils der Application Logic into domain
--> bliebe nur noch 1 Schicht uebrig, die Domain (Gehirn)
--> Unguenstig, da mehrere verschiedene Sichten auf Domain gewuenscht und notwendig;
daher immer nur Domain-Extracts in Application dargestellt!
- How about adding a GUI to every EJB so that it can be displayed automatically?
--> useless: users want different guis and EJB data need to be displayed in
different contexts
==> the concept of merging more and more functionality into components a la EJB
is not arbitrarily continuable, nor always desired (different views on domain)
