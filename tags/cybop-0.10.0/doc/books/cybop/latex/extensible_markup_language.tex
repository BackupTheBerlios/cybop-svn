%
% $RCSfile: extensible_markup_language.tex,v $
%
% Copyright (C) 2002-2008. Christian Heller.
%
% Permission is granted to copy, distribute and/or modify this document
% under the terms of the GNU Free Documentation License, Version 1.1 or
% any later version published by the Free Software Foundation; with no
% Invariant Sections, with no Front-Cover Texts and with no Back-Cover
% Texts. A copy of the license is included in the section entitled
% "GNU Free Documentation License".
%
% http://www.cybop.net
% - Cybernetics Oriented Programming -
%
% http://www.resmedicinae.org
% - Information in Medicine -
%
% Version: $Revision: 1.1 $ $Date: 2008-08-19 20:41:06 $ $Author: christian $
% Authors: Christian Heller <christian.heller@tuxtax.de>
%

\subsubsection{Extensible Markup Language}
\label{extensible_markup_language_heading}
\index{Extensible Markup Language}
\index{XML}
\index{World Wide Web}
\index{WWW}
\index{World Wide Web Consortium}
\index{W3C}
\index{Standard Generalized Markup Language}
\index{SGML}
\index{Hypertext Markup Language}
\index{HTML}
\index{Document Publishing}
\index{Data Transfer}
\index{GUI Design}
\index{Workflow Composition}
\index{Database Storage}
\index{Domain Modelling}

A popular, very flexible, yet simple language playing an increasingly important
role in the exchange of a wide variety of data on the \emph{World Wide Web}
(WWW) and elsewhere is the \emph{Extensible Markup Language} (XML) \cite{xml},
defined by the \emph{World Wide Web Consortium} (W3C) \cite{w3c}. Being a text
format derived as simplified subset (dialect) of the
\emph{Standard Generalized Markup Language} (SGML) \cite{sgml}, it allows to
structure and store information hierarchically as \emph{Document} file. Norman
Walsh \cite{walsh} writes:

\begin{quote}
    A markup language is a mechanism to \emph{identify structures} in a document.
    The XML specification defines a standard way to \emph{add markup} to documents.
\end{quote}

And the XML Cover Pages \cite{sgmlmetamarkup} state:

\begin{quote}
    Both SGML and XML are \emph{meta} languages because they are used for
    defining \emph{markup} languages. A markup language defined using SGML or
    XML has a specific vocabulary (labels for elements and attributes) and a
    declared syntax (grammar defining the hierarchy and other features).
\end{quote}

Historically, markup languages became widely known through the
\emph{Hypertext Markup Language} (HTML) as language of the Web. To overcome its
limitations, XML was originally \emph{designed to meet the challenges of large-
scale electronic publishing} \cite{xml}. Today, XML is applied in many different
areas, for example:

\begin{itemize}
    \item[-] Document Publishing \cite{docbook}
    \item[-] Data Transfer \cite{soap}
    \item[-] GUI Design \cite{xul, uiml, ecml, xaml}
    \item[-] Workflow Composition \cite{oio, intershop}
    \item[-] Database Storage \cite{exist, dbxml}
    \item[-] Domain Modelling \cite{xmlschema, damloil}
\end{itemize}

Yet in the opinion of Robin Cover \cite{xmlsemantics}, the usability of XML for
domain modelling is limited. He writes:

\begin{quote}
    Just like its parent metalanguage (SGML), XML has no formal mechanism to
    support the declaration of semantic integrity constraints, and XML processors
    have no means of validating object semantics even if these are declared
    informally in an XML DTD. XML processors will have no inherent understanding
    of document object semantics because XML (meta-)markup languages have no
    predefined application-level processing semantics. XML thus formally governs
    syntax only -- not semantics.

    In fact, XML syntax is designed for representing an encoded serialization,
    and thus has a very limited range of expression for modeling complex object
    semantics, where \emph{Semantics} fundamentally means an intricate web of
    constrained relationships and properties. Otherwise stated: XML is a poor
    language for data modelling \ldots
\end{quote}

The XML-based language described in chapter \ref{cybernetics_oriented_language_heading}
proves the opposite. By applying a common knowledge modelling schema (chapter
\ref{knowledge_schema_heading}), it allows to model arbitrary meta information
(complex object semantics). Robin Cover continues \cite{xmlsemantics}:

\begin{quote}
    The notion of \emph{Attribute} might have been more useful except that XML
    supports only a flat data model for the value of an attribute in a
    name-value pair (essentially \emph{String}). This flat model cannot easily
    capture complex attribute notions such as would be predicated of abstracted
    real world objects, where attribute values are themselves typically
    represented by complex objects, either owned or referenced.
\end{quote}

This criticism of Cover is absolutely correct. It can be circumvented, though.
The language introduced in chapter \ref{cybernetics_oriented_language_heading}
permits one attribute to store a (file) path to an external compound knowledge
template and is thus capable of representing compound properties (complex
attribute notions). One problem remains, however: When serialising compound
knowledge models consisting of other compound models, the quotation mark as
attribute value delimiter is not sufficient, because the beginning and end of
an attribute value may get mixed up. Solving it, the XML standard needs to be
injured (chapter \ref{cybernetics_oriented_language_heading}).
