%
% $RCSfile: pattern_merger.tex,v $
%
% Copyright (C) 2002-2008. Christian Heller.
%
% Permission is granted to copy, distribute and/or modify this document
% under the terms of the GNU Free Documentation License, Version 1.1 or
% any later version published by the Free Software Foundation; with no
% Invariant Sections, with no Front-Cover Texts and with no Back-Cover
% Texts. A copy of the license is included in the section entitled
% "GNU Free Documentation License".
%
% http://www.cybop.net
% - Cybernetics Oriented Programming -
%
% http://www.resmedicinae.org
% - Information in Medicine -
%
% Version: $Revision: 1.1 $ $Date: 2008-08-19 20:41:08 $ $Author: christian $
% Authors: Christian Heller <christian.heller@tuxtax.de>
%

\subsection{Pattern Merger}
\label{pattern_merger_heading}
\index{CYBOI using Patterns}
\index{Model View Controller Pattern in CYBOI}
\index{MVC in CYBOI}
\index{Data Mapper Pattern in CYBOI}
\index{Data Transfer Object Pattern in CYBOI}
\index{DTO in CYBOI}
\index{Pipes and Filters Pattern in CYBOI}
\index{Microkernel Pattern in CYBOI}
\index{Broker Pattern in CYBOI}
\index{CYBOI as Peer to Peer System}
\index{CYBOI as GUI Renderer}
\index{Pattern-less Application Development in CYBOI}

A variety of software patterns (section \ref{pattern_heading}) can be found
when inspecting CYBOI's architecture. Most of them, especially those relying on
\emph{Object Oriented} (OO) principles, are used in an \emph{adapted} form, as
described following.

Firstly, there are those that can be summed up under the umbrella term
\emph{Translator Patterns}: \emph{Model View Controller} (MVC),
\emph{Data Mapper} and \emph{Data Transfer Object} (DTO). As section
\ref{translator_architecture_heading} tried to show, they all contain two state
models, one logic model and a controlling unit, which is why it is possible to
unify- and place them into one common architecture. CYBOI represents the unit
controlling all action, on a low system level. It stores state- as well as
logic models in one common knowledge tree, and uses the rules encoded in logic
models to translate state models into each other.

Secondly, there is the \emph{Pipes and Filters} pattern which CYBOI uses not
only to instantiate knowledge templates, but also for system communication.
An input (i/p) state (like a persistent, serialised knowledge template) runs
through a cascade of filters, namely \emph{Creator}, \emph{Communicator},
\emph{Converter} and \emph{Translator}, before it is processed (as transient
knowledge model) inside the system, to be finally forwarded in opposite
direction through the same filters, resulting in an output (o/p) state.

Thirdly, CYBOI acts as \emph{Microkernel} and \emph{Broker}, at the same time.
It calls special threads (internal servers) managing data input/ output (i/o),
and has the capability to communicate with remote systems (external servers),
for data transfer. The actual impulse for communication comes from a passive
knowledge model (adapter) that is actively processed by CYBOI. Since that
impulse is \emph{not} a direct method call, but either a \emph{send-} or
\emph{receive} operation with varying parameters, special \emph{Proxy} models
are not needed anymore. CYBOI may act as client and server, at the same time,
which enables the applications running within it to act as \emph{Peer to Peer}
(P2P) systems (section \ref{peer_node_heading}). It incorporates a signal
(event) loop (like the broker) and handles low-level system (socket)
communication (like the bridge).

The fact that future versions of CYBOI will be able to interpret CYBOL
knowledge templates containing \emph{Graphical User Interface} (GUI)
descriptions, makes it a \emph{GUI Renderer}. The task of a renderer is to
translate GUI models into hardware-understandable function calls and protocols.
%(section \ref{user_interface_modelling_heading}).
In the case of CYBOI, the
graphical environment supported first will be the \emph{X Window System} (X)
\emph{XFree86} \cite{xfree86} variant. The step towards rendering models given
in \emph{Hyper Text Markup Language} (HTML) format is not far then, so that
CYBOI may act not only as web server, but also as web browser. All that,
together with further additions, will make it virtually an \emph{All-Rounder}
system.

The usage of simplified forms of patterns like \emph{Composite} (inheritance
omitted), \emph{Whole-Part}, \emph{Wrapper} or \emph{Layers}, for knowledge
storage, was already mentioned in section \ref{hidden_patterns_heading}.

Unfavourable patterns as mentioned in section \ref{pattern_systematics_heading}
(those with \emph{global access} or \emph{bidirectional dependencies}) were
avoided.

Finally, the merged appearance of patterns in CYBOI (and CYBOL for that matter)
brings software development one step closer to \emph{pattern-less} application
programming. Application developers are freed from the burden of repeatedly
figuring out suitable patterns, and enabled to concentrate on modelling pure
domain knowledge, based on the concept of \emph{Hierarchy}, instead.
