%
% $RCSfile: primitives.tex,v $
%
% Copyright (C) 2002-2008. Christian Heller.
%
% Permission is granted to copy, distribute and/or modify this document
% under the terms of the GNU Free Documentation License, Version 1.1 or
% any later version published by the Free Software Foundation; with no
% Invariant Sections, with no Front-Cover Texts and with no Back-Cover
% Texts. A copy of the license is included in the section entitled
% "GNU Free Documentation License".
%
% http://www.cybop.net
% - Cybernetics Oriented Programming -
%
% http://www.resmedicinae.org
% - Information in Medicine -
%
% Version: $Revision: 1.1 $ $Date: 2008-08-19 20:41:08 $ $Author: christian $
% Authors: Christian Heller <christian.heller@tuxtax.de>
%

\subsection{Primitives}
\label{primitives_heading}
\index{Primitive}
\index{State Primitive}
\index{Binary Digit}
\index{Bit}
\index{Byte}
\index{Word}
\index{Double Word}
\index{File}
\index{Hard Disk Drive}
\index{HDD}
\index{Random Access Memory}
\index{RAM}
\index{Multipurpose Internet Mail Extension}
\index{MIME}
\index{Text MIME Type}
\index{Image MIME Type}
\index{Audio MIME Type}
\index{Video MIME Type}
\index{Application MIME Type}

All software is based on the final two states called \emph{one} \& \emph{zero},
or \emph{on} \& \emph{off}, or \emph{true} \& \emph{false}, or similarly. A
\emph{Binary Digit} (Bit) can take on the values \emph{zero} or \emph{one}; it
represents the final abstraction of any software model and can be easily mapped
to hardware. A second unit, the \emph{Byte}, consists of eight Bits. One
\emph{Word} is made up of two Bytes, one \emph{Double Word} of four Bytes and
so forth. Data in a \emph{File} on a \emph{Harddisk Drive} (HDD) partition or
another storage medium are saved in form of Bit sequences, just like data in
\emph{Random Access Memory} (RAM). It is up to a program to interpret these
data correctly, in the desired manner.

\begin{figure}[ht]
    \begin{center}
        \begin{footnotesize}
        \begin{tabular}{| p{70mm} | p{35mm} |}
            \hline
            \textbf{Primitivum} & \textbf{Size} [Byte]\\
            \hline
            Date, Time, Complex, Fraction, Term & many\\
            \hline
            Double, Float, Vector, String & 8, 12, 16 or more\\
            \hline
            Integer, Pointer, Word, Short, Byte, Character & 1, 2, 4, 7, 8 or more\\
            \hline
            Bit, Boolean & 0 or 1 Bit\\
            \hline
        \end{tabular}
        \end{footnotesize}
        \caption{State Primitives sorted after their Granularity}
        \label{primitives_table}
    \end{center}
\end{figure}

Many programming languages offer a number of basic types, also called
\emph{Primitives}, which are combinations of different numbers of Bits. Table
\ref{primitives_table} shows some of them, together with their possible memory
usage. Besides the primitive types that are included in a programming language,
there are other forms of storing data. Section \ref{language_heading} said that
not only a \emph{String}, but also an \emph{Image} or a \emph{Sound} can
represent a \emph{Quality}, that is a \emph{Term} with special meaning. The
format of such data sequences is often defined as
\emph{Multipurpose Internet Mail Extension} (MIME) type, for example:

\begin{itemize}
    \item[-] \emph{text:} sgml, xml, html, rtf, tex, txt
    \item[-] \emph{image:} jpeg, png, gif, bmp
    \item[-] \emph{audio:} ogg, mp3, wav
    \item[-] \emph{video:} mpeg, qt, avi
    \item[-] \emph{application:} kword, sxw
\end{itemize}
