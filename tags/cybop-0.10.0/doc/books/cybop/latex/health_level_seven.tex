%
% $RCSfile: health_level_seven.tex,v $
%
% Copyright (C) 2002-2008. Christian Heller.
%
% Permission is granted to copy, distribute and/or modify this document
% under the terms of the GNU Free Documentation License, Version 1.1 or
% any later version published by the Free Software Foundation; with no
% Invariant Sections, with no Front-Cover Texts and with no Back-Cover
% Texts. A copy of the license is included in the section entitled
% "GNU Free Documentation License".
%
% http://www.cybop.net
% - Cybernetics Oriented Programming -
%
% http://www.resmedicinae.org
% - Information in Medicine -
%
% Version: $Revision: 1.1 $ $Date: 2008-08-19 20:41:07 $ $Author: christian $
% Authors: Christian Heller <christian.heller@tuxtax.de>
%

\subsubsection{Health Level Seven}
\label{health_level_seven_heading}
\index{Health Level Seven}
\index{HL7}
\index{Reference Information Model}
\index{RIM}
\index{Clinical Document Architecture}
\index{CDA}
\index{XML}
\index{Common Message Element Type}
\index{CMET}
\index{Refined Message Information Model}
\index{RMIM}
\index{Hospital Information System}
\index{HIS}
\index{Practice Management System}
\index{PMS}

\emph{Health Level Seven} (HL7), as it describes itself \cite{hl7}, is:
\textit{a not-for-profit, ANSI-accredited standards developing organization
dedicated to providing a comprehensive framework and related standards for the
exchange, integration, sharing, and retrieval of electronic health information
that supports clinical practice and the management, delivery and evaluation of
health services.} The more than 2000 individuals representing over 500
corporate members, world-wide, do for a great part belong to healthcare
industry, implementing its interests. Accordingly, HL7's endeavors are
sponsored, in part, by that industry. Its name \emph{Level Seven}, after
\cite{rogers}, refers to the highest level of the ISO OSI communication model
(section \ref{systems_interconnection_heading}).

Besides the mentioned framework called \emph{Reference Information Model} (RIM),
the organisation worked out a number of specifications for the exchange of messages
and documents, newer formats being the \emph{Clinical Document Architecture} (CDA),
an XML-only specification, and the \emph{Common Message Element Type} (CMET)
\cite{marley}, a reusable component.

Thomas Beale criticised in \cite{openhealth}, that CDA did not have very strong
semantics for some of its detailed parts, since they were derived from XHTML,
and that it tended to mix presentation and representation concerns somewhat.
Furthermore, CDA had recently incorporated some RIM classes into its content
level, via the creation of a \emph{Refined Message Information Model} (RMIM)
which were a pity, because it reduced its genericness and made it dependend on
the RIM, which were essentially an analysis pattern of domain relationships,
not a model of recording.

Instead, as Beale writes in a later message to \cite{openhealth}, they (HL7)
might start thinking about generic solutions, which incorporate clinical
models, separated from their XML schemas, for a start. \ldots\ Single level XML
approaches didn't have much long term future in his opinion, because they
didn't properly separate clinical models from information representation, which
were required to allow compositional clinical models and specialisable clinical
models to be built independently from the software.

HL7's recommendations found partly application in a greater number of
\emph{Hospital Information Systems} (HIS), but rarely in smaller or
medium-sized \emph{Practice Management Systems} (PMS).
