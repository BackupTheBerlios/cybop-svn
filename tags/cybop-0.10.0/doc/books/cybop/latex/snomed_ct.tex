%
% $RCSfile: snomed_ct.tex,v $
%
% Copyright (C) 2002-2008. Christian Heller.
%
% Permission is granted to copy, distribute and/or modify this document
% under the terms of the GNU Free Documentation License, Version 1.1 or
% any later version published by the Free Software Foundation; with no
% Invariant Sections, with no Front-Cover Texts and with no Back-Cover
% Texts. A copy of the license is included in the section entitled
% "GNU Free Documentation License".
%
% http://www.cybop.net
% - Cybernetics Oriented Programming -
%
% http://www.resmedicinae.org
% - Information in Medicine -
%
% Version: $Revision: 1.1 $ $Date: 2008-08-19 20:41:08 $ $Author: christian $
% Authors: Christian Heller <christian.heller@tuxtax.de>
%

\subsubsection{SNOMED CT}
\label{snomed_ct_heading}
\index{Systematized Nomenclature of Medicine}
\index{SNOMED}
\index{SNOMED Clinical Terms}
\index{SNOMED CT}
\index{SNOMED Reference Terminology}
\index{SNOMED RT}
\index{Read Codes}
\index{READ}
\index{ICD-9-CM}
\index{ICD-10}
\index{ICD-03}
\index{OPCS-4}
\index{LOINC}
\index{SNOMED International}
\index{College of American Pathologists}
\index{CAP}

The \emph{Systematized Nomenclature of Medicine} (SNOMED) \emph{Clinical Terms}
(SNOMED CT) is: \textit{a dynamic, scientifically validated clinical health care
terminology and infrastructure that makes health care knowledge more usable and
accessible.} The SNOMED CT core terminology: \textit{contains over 364,400
health care concepts with unique meanings and formal logic-based definitions
organized into hierarchies. As of January 2005, the fully populated table with
unique descriptions for each concept contains more than 984,000 descriptions.
Approximately 1.45 million semantic relationships exist to enable reliability
and consistency of data retrieval.} \cite{snomed}

SNOMED CT was created by combining the content and structure of the SNOMED
\emph{Reference Terminology} (SNOMED RT) with the United Kingdom's (UK)
\emph{Read Codes} (READ) clinical terms. Meanwhile, mappings and integrations
for further standards exist, e.g. for several ICD versions (ICD-9-CM, ICD-10,
ICD-O3), OPCS-4 and LOINC.

Scheme: hybrid enumerative-compositional\\
Maintainer: \emph{SNOMED International} and \emph{College of American Pathologists} (CAP)

==

--- Klaus Veil <klaus@veil.net.au> wrote:

> Nandalal,
> �
> The concerns about closedness and excessive cost were indeed the main
> drivers that finally convinced CAP in the USA that an "open" international
> Standards Development Organisation was the only viable way forward. �CAP
> have committed to transfer the SNOMED IP to the new SDO.
> �
> Klaus
> 
> � _____ �
> 
> From: openhealth@yahoogroups.com [mailto:openhealth@yahoogroups.com] On
> Behalf Of Nandalal Gunaratne
> Sent: Monday, 5 February 2007 20:06
> To: openhealth@yahoogroups.com
> Subject: RE: [openhealth] OSHCA Conference Topics
> 
> 
> 
> Klaus,
> 
> Most of asia use ICD and other WHO standards. SNOMED
> is considered too expensive and too closed. I hope the
> new initiative would change that.
> 
> Nandalal
