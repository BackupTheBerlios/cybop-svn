%
% $RCSfile: xml_based_programming.tex,v $
%
% Copyright (C) 2002-2008. Christian Heller.
%
% Permission is granted to copy, distribute and/or modify this document
% under the terms of the GNU Free Documentation License, Version 1.1 or
% any later version published by the Free Software Foundation; with no
% Invariant Sections, with no Front-Cover Texts and with no Back-Cover
% Texts. A copy of the license is included in the section entitled
% "GNU Free Documentation License".
%
% http://www.cybop.net
% - Cybernetics Oriented Programming -
%
% http://www.resmedicinae.org
% - Information in Medicine -
%
% Version: $Revision: 1.1 $ $Date: 2008-08-19 20:41:09 $ $Author: christian $
% Authors: Christian Heller <christian.heller@tuxtax.de>
%

\section{XML Based Programming}
\label{xml_based_programming_heading}

%
% $RCSfile: user_interface_modelling.tex,v $
%
% Copyright (C) 2002-2008. Christian Heller.
%
% Permission is granted to copy, distribute and/or modify this document
% under the terms of the GNU Free Documentation License, Version 1.1 or
% any later version published by the Free Software Foundation; with no
% Invariant Sections, with no Front-Cover Texts and with no Back-Cover
% Texts. A copy of the license is included in the section entitled
% "GNU Free Documentation License".
%
% http://www.cybop.net
% - Cybernetics Oriented Programming -
%
% http://www.resmedicinae.org
% - Information in Medicine -
%
% Version: $Revision: 1.1 $ $Date: 2008-08-19 20:41:09 $ $Author: christian $
% Authors: Christian Heller <christian.heller@tuxtax.de>
%

\subsection{User Interface Modelling}
\label{user_interface_modelling_heading}

- GUIs bisher nie einheitlich beschrieben wie RDB mit ERM und DDL/SQL
- jetzt erst erreichen GUIs mit der UIML den Stand von DDL/SQL
- so wie RDBMS durch OODBMS ersetzt wurden und damit entfielen,
OODBMS mit OO Domain Layer verschmolzen ist, kann evtl. auch das GUI
mit der Domain verschmolzen werden! ==> NEIN! weil Domain Daten unterschiedlich
im GUI dargestellt werden koennen (Edit, Liste, Tabelle, Tree) und Anwender
verschiedene Wuensche und Anforderungen an das GUI haben, weswegen die Entwickler
die GUIs flexibel erstellen und handhaben muessen, was nicht gegeben waere,
wenn das GUI vom DomainModel fest vorgegeben waere

Common libraries used for data display design are the \emph{GIMP Toolkit} (GTK)
of the \emph{General (GNU) Image Manipulation Program} (GIMP) and the \emph{Qt Toolkit},
both written in C++, as well as the \emph{Abstract Window Toolkit} (AWT)/ \emph{Swing}
for Java.

mention and shortly describe \emph{GUI Renderer} (section pattern_merger refers to here)

%
% $RCSfile: bean.tex,v $
%
% Copyright (C) 2002-2008. Christian Heller.
%
% Permission is granted to copy, distribute and/or modify this document
% under the terms of the GNU Free Documentation License, Version 1.1 or
% any later version published by the Free Software Foundation; with no
% Invariant Sections, with no Front-Cover Texts and with no Back-Cover
% Texts. A copy of the license is included in the section entitled
% "GNU Free Documentation License".
%
% http://www.cybop.net
% - Cybernetics Oriented Programming -
%
% http://www.resmedicinae.org
% - Information in Medicine -
%
% Version: $Revision: 1.1 $ $Date: 2008-08-19 20:41:05 $ $Author: christian $
% Authors: Christian Heller <christian.heller@tuxtax.de>
%

\subsubsection{Bean}
\label{bean_heading}

- also Borland's Delphi Component Library (first of its kind)

%
% $RCSfile: xml_user_interface_language.tex,v $
%
% Copyright (C) 2002-2008. Christian Heller.
%
% Permission is granted to copy, distribute and/or modify this document
% under the terms of the GNU Free Documentation License, Version 1.1 or
% any later version published by the Free Software Foundation; with no
% Invariant Sections, with no Front-Cover Texts and with no Back-Cover
% Texts. A copy of the license is included in the section entitled
% "GNU Free Documentation License".
%
% http://www.cybop.net
% - Cybernetics Oriented Programming -
%
% http://www.resmedicinae.org
% - Information in Medicine -
%
% Version: $Revision: 1.1 $ $Date: 2008-08-19 20:41:09 $ $Author: christian $
% Authors: Christian Heller <christian.heller@tuxtax.de>
%

\subsubsection{XML User Interface Language}
\label{xml_user_interface_language_heading}

XML User Interface Language (XUL)

Mozilla + Opera

"Web Forms 2.0"?

%
% $RCSfile: user_interface_markup_language.tex,v $
%
% Copyright (C) 2002-2008. Christian Heller.
%
% Permission is granted to copy, distribute and/or modify this document
% under the terms of the GNU Free Documentation License, Version 1.1 or
% any later version published by the Free Software Foundation; with no
% Invariant Sections, with no Front-Cover Texts and with no Back-Cover
% Texts. A copy of the license is included in the section entitled
% "GNU Free Documentation License".
%
% http://www.cybop.net
% - Cybernetics Oriented Programming -
%
% http://www.resmedicinae.org
% - Information in Medicine -
%
% Version: $Revision: 1.1 $ $Date: 2008-08-19 20:41:09 $ $Author: christian $
% Authors: Christian Heller <christian.heller@tuxtax.de>
%

\subsubsection{User Interface Markup Language}
\label{user_interface_markup_language_heading}

User Interface Markup Language (UIML)


%
% $RCSfile: workflow.tex,v $
%
% Copyright (C) 2002-2008. Christian Heller.
%
% Permission is granted to copy, distribute and/or modify this document
% under the terms of the GNU Free Documentation License, Version 1.1 or
% any later version published by the Free Software Foundation; with no
% Invariant Sections, with no Front-Cover Texts and with no Back-Cover
% Texts. A copy of the license is included in the section entitled
% "GNU Free Documentation License".
%
% http://www.cybop.net
% - Cybernetics Oriented Programming -
%
% http://www.resmedicinae.org
% - Information in Medicine -
%
% Version: $Revision: 1.1 $ $Date: 2008-08-19 20:41:09 $ $Author: christian $
% Authors: Christian Heller <christian.heller@tuxtax.de>
%

\subsection{Workflow}
\label{workflow_heading}

\input{pipeline_and_pipelet}
%
% $RCSfile: open_infrastructure_for_outcomes.tex,v $
%
% Copyright (C) 2002-2008. Christian Heller.
%
% Permission is granted to copy, distribute and/or modify this document
% under the terms of the GNU Free Documentation License, Version 1.1 or
% any later version published by the Free Software Foundation; with no
% Invariant Sections, with no Front-Cover Texts and with no Back-Cover
% Texts. A copy of the license is included in the section entitled
% "GNU Free Documentation License".
%
% http://www.cybop.net
% - Cybernetics Oriented Programming -
%
% http://www.resmedicinae.org
% - Information in Medicine -
%
% Version: $Revision: 1.1 $ $Date: 2008-08-19 20:41:08 $ $Author: christian $
% Authors: Christian Heller <christian.heller@tuxtax.de>
%

\subsubsection{Open Infrastructure for Outcomes}
\label{open_infrastructure_for_outcomes_heading}


many UNIX commands which are connected by many pipes and such form a pipeline

http://openflow.sourceforge.net/index.html
"Welcome to the home page of the OpenFlow Project. OpenFlow aims to
build a workflow and document flow management system for the Italian
Public Administration."
They don't seem to have any code available yet and some of the dates
stem from 1993!

wftk: Open-source workflow toolkit http://www.vivtek.com/wftk/
this work was paid for via collabnet.  This is a C implementation.

Micro-Workflow: http://micro-workflow.com/Research/  Started off as an
academic project, the example on this page is medically oriented.  Open
source might be coming however as the author states: "My plan is to give
out the framework under the terms of the GNU General Public License.
Currently I am working on cleaning up the code. As soon as I'm through I
will make it available from these Web pages."

I can't really verify this, but at one time I read the the OMG jFlow
workflow service was being added to  the open source real-time ORB
project TAO here:
http://www.cs.wustl.edu/~schmidt/TAO-status.html

DSTC's Elvin project: http://elvin.dstc.edu.au/intro/overview.html
While described as \textit{a content-based routing service}, there are workflow
aspects to this software.  I would think that content-based routing
would be very interesting to medical software folks.  My plan is to
start experimenting with Elvin after our content managment system comes
on-line (the RFP mentioned above).

Bud P. Bruegger \cite{openoutcomesgeneral}
Thanks for the pointer to the overview paper on PROforma--definitely the
best of all I've seen.

At 01:24 PM 14-03-01 -0800, Andrew po-jung Ho wrote:
>In summary, PROforma looks promising and not overly complex. It could be a
>good standard for Openflow and OIO to consider adopting. Since I don't
>have much experience with workflow systems, I look forward to you, Paolo,
>and other list member's guidance.

In the following are some thoughts on PROforma and workflow models in
general.  They are based on a still incomplete understanding of the
situation, but maybe it can help bring the discussion further.

It seems to me that PROforma is highly specialized and far from a general
purpose workflow model.  Also, apart from workflow, it deals a lot with
decision support.  Something that general purpose workflow systems usually
have but that I didn't find in PROforma (did I miss it?) is a description
of how work flows withing a group of people.  A typical example in an
administrative organisation would be a document/application/etc. to flow
between people of different roles, each of whom add their contribution...

PROforma seems so specialized to clinical decision support that I wonder
whether it would be applicable as a workflow component in a LIMS.

So what workflow model/component to use???  A difficult question.  I looked
around quite a bit so far and haven't found a single thing that convinces
me all the way (obviously for my purposes).

On the standard side, I'm not convinced that the WfMC or OMG standards are
really the way to go  (look at the critique by the former IETF SWAP leader
that I sent you earlier--I don't think these things have really been
addressed by WfMG's new wf-XML that does not seem to break with the
problematic reference model).

I spent quite some time reading agent-technology workflow papers and non of
these people seems to even look at the WfMG standards.  It seems they use
petri-nets and similar as models and some use some other process
description standards (that probably are mostly relevant to manufacturing).

So unless there is a strong need for interoperation with a WfMG standard
complient external workflow system, I'm not sure wether it's worth the
effort of adding the effort and complexity that comes with the standard.

Other features that are not standardised seem much more important to
me.  For example, the possibility to negotiate as it is typical for
agent-based approaches. (As mentioned in section 8.2 of
http://www.acl.icnet.uk/PUBLICATIONS/ms364.pdf that you pointed
out).  Also, a peer-to-peer architecture that allows collaboration and
scalability way beyond the boundaries of a singe administrative
control.  Such features seems to be almost incompatible with the current
standards.

Going to actually available systems, my impression is that there are
roughly three categories of systems that call themselves "workflow" systems:

1. Task-Management systems such as those used in Ticket tracking and
help-desk systems.  Their basic entity is a task.  The notion of "flow"
comes in through the possibility of defining task dependencies.  But this
is very cumbersome and way too limited for many kinds of workflow needs.

Typically, these systems assign human resources to tasks (who is
responsible for execution, who created the task), but lack any notion of
other resources, such as for example documents that "flow" around in the
work process.  (To be fair, some deal with descriptions of the task coming
via e-mail etc.--but that's very specialized and not general at all).

2. Workflow systems that model an actual flow of tasks by connecting tasks
in some kind of a graph.  Most serious workflow systems and the WfMG model
fall into this category.

There is probably a major difference of how systems deal with resources
that are relevant to work processes.  All systems probably model human
resources.  Most systems (but not all) allow you that tasks reference their
input and output resources.  IMO, the mechanism of referencing resources
from task makes it difficult to understand the flow of resources
(documents, data) in the workflow.  I personally believe that models that
emphasise tasks at the cost of resources can be rather limited in some
kinds of workflow applications (such as LIMS) where resources (eg. samples
and specimen in a LIMS) are first class citizens.

3. There seems to be a few workflow systems that model workflow as complete
graphs and also treat resources as first class citizens.  This was my naive
idea (before I looked into workflow systems in more detail) of a workflow
component for a LIMS (see
http://www.sistema.it/labinfo/single.html#Relationships between Entities)
but also seems to be used by systems such as PIPER
(http://bioinformatics.org/piper/) and Khoros
(http://www.khoral.com/ideas/technology/cantata.pdf).

Piper actually looks VERY interesting but it seems too early to be used
(still alpha).  Interesting that Piper does not seem to bother about any
workflow standard.  I suppose that treating resources as first-class
citizens may be incompatible with the WfMG approach.  (Interesting for the
OIO folks is that the upper layers of Piper are written in Python).

Another product that may be able to manage workflows is the Narval software
agent (http://www.logilab.org/narval/).  <quote>Narval is the acronym of
"Network Assistant Reasoning with a Validating Agent Language". It is a
personal network assistant based on artificial intelligence and agent
technologies. It executes recipes (sequences of actions) to perform tasks.
It is easy to specify a new action using XML and to implement it using
Python. Recipes can be built and debugged using a graphical
interface.  </quote>

It has been used to automate some special workflows (eg the translation of
Linux Gazette to French
http://www.linuxgazette.com/issue59/chauvat.html).  It is written in Python
and seems to be quite user-friendly and easy to program  (as compared to
some Java agent systems).  I think it may be worth while to look into
Narval (and a different paradigm of thinking) to see how that would be
applicable to workflow in healthcare applications.  My gutt feeling is
quite positive...

Workflow
http://sourceforge.net/projects/tikiwiki/


- in Microsoft Windows "Longhorn": Win-FX XAML (YAML??)
http://www.ondotnet.com/pub/a/dotnet/2004/01/19/longhorn.html
http://www.codeproject.com/csharp/treemaps.asp
\cite{xaml}
http://www.xaml.net/

- ebXML (with OWL support??, by Sun Microsystems)
- W3C XForms

Extensible Programming for the 21st Century/ XML Based Programming \cite{wilson}:
- suggests to put an additional layer of abstraction (in XML) on top of current
programming languages
- programmers would be able to customise their "views" of software without
modifying the underlying model

"We believe that next-generation programming systems will store source code as XML,
rather than as flat text. Programmers will not see or edit XML tags; instead, their
editors will render these models to create human-friendly views, just as web browsers
render HTML. For example, a program might be stored on disk like this:"

<doc>Only replace below threshold</doc>
<cond>
  <test>
    <compare-expr operator="less">
      <field-expr field="age"><evaluate>record</evaluate></field-expr>
      <evaluate>threshold</evaluate>
    </compare-expr>
  </test>
  <body>
    <invoke-expr method="release"><evaluate>record</evaluate></invoke-expr>
  </body>
</cond>

- would be viewed and edited like this:

// Only replace below threshold
if (record.age < threshold) {
    record.release();
}

or:

if (record.age < threshold)    /* Only replace below threshold */
{
        record.release();
}

or even this (Scheme language):

;;; Only replace below threshold
(if (< (record 'age) threshold)
    (record 'release))
