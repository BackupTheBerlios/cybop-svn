%
% $RCSfile: episode_based.tex,v $
%
% Copyright (C) 2002-2008. Christian Heller.
%
% Permission is granted to copy, distribute and/or modify this document
% under the terms of the GNU Free Documentation License, Version 1.1 or
% any later version published by the Free Software Foundation; with no
% Invariant Sections, with no Front-Cover Texts and with no Back-Cover
% Texts. A copy of the license is included in the section entitled
% "GNU Free Documentation License".
%
% http://www.cybop.net
% - Cybernetics Oriented Programming -
%
% http://www.resmedicinae.org
% - Information in Medicine -
%
% Version: $Revision: 1.1 $ $Date: 2008-08-19 20:41:06 $ $Author: christian $
% Authors: Christian Heller <christian.heller@tuxtax.de>
%

\subsection{Episode Based}
\label{episode_based_heading}
\index{Episode Based EHR}
\index{Patient Centered Medical Record}
\index{Health Issue of an Episode Based EHR}
\index{Clinical Episode of an Episode Based EHR}
\index{Clinical Encounter of an Episode Based EHR}
\index{Clinical Item of an Episode Based EHR}
\index{Partial Contact of an Episode Based EHR}
\index{Problem Oriented Medical Record}
\index{POMR}

Historically, it took a long time until the concept of a modern EHR crystalised
out. An early form of a time-oriented medical record stems from Hippocrates
(5th century BC) who wanted to accurately reflect the course of a disease and
indicate its possible causes. In 1907, the \emph{Mayo Clinic} (formed by the
American surgeon William Mayo) adopted one separate file for each patient, to
be able to obtain a better overview of his complete disease history. This
innovation was the origin of the \emph{Patient Centered Medical Record} as
known today, as \cite{mihandbook} means.

The discussion on how to model an ideal EHR already lasts for decades and has
not finished. Recent proposals brought in some new perspectives and ideas. One
of them turns around the so-called \emph{Episode-based} EHR \cite{westerhof}.
In the centre of these considerations stands a structure that is described in a
more pragmatic way by Karsten Hilbert of GNUmed \cite{gnumed}. He sees a
complex EHR as hierarchical composition of the following items:

\begin{itemize}
    \item[-] Health Issue
    \item[-] Clinical Episode
    \item[-] Clinical Encounter
    \item[-] Clinical Item
\end{itemize}

The additional concept of a \emph{Partial Contact} as known from the Dutch
\emph{Episode Model} does not integrate into this hierarchy. But after Hilbert,
\emph{Partial Contacts} could be easily derived from existing EHR data by
aggregating all \emph{Clinical Items} that belong to the same
\emph{Clinical Encounter} and the same \emph{Clinical Episode}.

\emph{Clinical Items} are typically elements in the \emph{SOAP} format of
progress notes, as known from the \emph{Problem Oriented Medical Record} (POMR)
\cite{weed} that was introduced by Lawrence L. Weed in the 1960s. SOAP stands
for:

\begin{itemize}
    \item[-] \emph{Subjective:} Complaints as phrased by the patient
    \item[-] \emph{Objective:} Findings of physicians and nurses
    \item[-] \emph{Assessment:} Test results and conclusions, such as a diagnosis
    \item[-] \emph{Plan:} Medical plan, for example treatment or policy
\end{itemize}
