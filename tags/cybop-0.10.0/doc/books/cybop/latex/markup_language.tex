%
% $RCSfile: markup_language.tex,v $
%
% Copyright (C) 2002-2008. Christian Heller.
%
% Permission is granted to copy, distribute and/or modify this document
% under the terms of the GNU Free Documentation License, Version 1.1 or
% any later version published by the Free Software Foundation; with no
% Invariant Sections, with no Front-Cover Texts and with no Back-Cover
% Texts. A copy of the license is included in the section entitled
% "GNU Free Documentation License".
%
% http://www.cybop.net
% - Cybernetics Oriented Programming -
%
% http://www.resmedicinae.org
% - Information in Medicine -
%
% Version: $Revision: 1.1 $ $Date: 2008-08-19 20:41:07 $ $Author: christian $
% Authors: Christian Heller <christian.heller@tuxtax.de>
%

\subsection{Markup Language}
\label{markup_language_heading}
\index{Markup Language}
\index{World Wide Web}
\index{WWW}
\index{Style of a Document}
\index{Structure of a Document}
\index{Content of a Document}
\index{TeX}
\index{Lamport TeX}
\index{LaTeX}
\index{LaTeXe}
\index{Scribe}
\index{Standard Generalized Markup Language}
\index{SGML}
\index{Extensible Markup Language}
\index{XML}
\index{Hypertext Markup Language}
\index{HTML}
\index{DocBook}
\index{Text Encoding Initiative}
\index{TEI}

At latest with the distribution of the \emph{World Wide Web} (WWW),
\emph{Markup Languages} increasingly gained in popularity. A markup language
separates the presentation \emph{Style} of a document from its logical
\emph{Structure} and \emph{Content}. Well-known representatives of markup
languages, two famous of which being described in the following sections, are
\cite{wikipedia}:

\begin{itemize}
    \item[-] \emph{\TeX} / \emph{Lamport \TeX} (\LaTeX, \LaTeXe)
    \item[-] \emph{Scribe}
    \item[-] \emph{Standard Generalized Markup Language} (SGML)
    \item[-] \emph{Extensible Markup Language} (XML)
    \item[-] \emph{Hypertext Markup Language} (HTML)
    \item[-] \emph{DocBook}
    \item[-] \emph{Text Encoding Initiative} (TEI)
\end{itemize}

Recently, more and more projects appear that try to use markup languages not
just for document markup, but also for declarative programming.% More on this in
%section \ref{xml_based_programming_heading}.
Before coming to the actual markup languages, the paradigm of
\emph{Literate Programming} and its idea to use markup tokens for distinguishing
source code and documentation, is investigated.

%
% $RCSfile: literate_programming.tex,v $
%
% Copyright (C) 2002-2008. Christian Heller.
%
% Permission is granted to copy, distribute and/or modify this document
% under the terms of the GNU Free Documentation License, Version 1.1 or
% any later version published by the Free Software Foundation; with no
% Invariant Sections, with no Front-Cover Texts and with no Back-Cover
% Texts. A copy of the license is included in the section entitled
% "GNU Free Documentation License".
%
% http://www.cybop.net
% - Cybernetics Oriented Programming -
%
% http://www.resmedicinae.org
% - Information in Medicine -
%
% Version: $Revision: 1.1 $ $Date: 2008-08-19 20:41:07 $ $Author: christian $
% Authors: Christian Heller <christian.heller@tuxtax.de>
%

\subsubsection{Literate Programming}
\label{literate_programming_heading}
\index{Literate Programming}
\index{Token Character}
\index{Commercial At @}
\index{Re-ordering of Code}
\index{Typeset Code and Documentation}
\index{Cross Referencing}
\index{Source Code Documentation}
\index{JavaDoc}
\index{Doxygen}
\index{DOC++}

Ross Williams writes in \cite[section 1.1]{williams}:

\begin{quote}
    A traditional computer program consists of a text file containing program
    code. Scattered in amongst the program code are comments which describe the
    various parts of the code. In \emph{Literate Programming}, the emphasis is
    reversed. Instead of writing code containing documentation, the literate
    programmer writes documentation containing code.
\end{quote}

In other words, \emph{Literate Programming} pays more attention to proper source
code documentation than classical programming languages do. It mostly offers
special \emph{Token} characters like the \emph{Commercial At} character \emph{@}
for example, which serve as code delimiters. The delimited blocks are determined
by particular tools such as a preprocessor that filters out program code to be
processed further. All source information together (input document, commentaries,
program code) is used to generate typeset documentation files in one or more
formats.

Williams \cite{williams} means that the literate programming system provided
far more than: \textit{just a reversal of the priority of comments and code.}
In its full-blown form, a good literate programming facility could provide:

\begin{itemize}
    \item[-] \emph{Re-ordering of Code:} Some programming languages force the
        programmer to give the various program parts in a particular order.
    \item[-] \emph{Typeset Code and Documentation:} Because a literate programming
        utility sees all the code, it can use its knowledge of the programming
        language and the features of the typesetting language to typeset the
        program code as if it were appearing in a technical journal.
    \item[-] \emph{Cross referencing:} Because a literate tool sees all the code
        and documentation, it is able to generate extensive cross referencing
        information in the typeset documentation, which makes the printed program
        document more easy to navigate and partially compensates for the lack of
        an automatic searching facility when reading printed documentation.
\end{itemize}

It is true, the actual instructions and algorithms in between commentaries are
written in (or translated into) a system programming- or other kind of language.
But literate programming places its focus on source code \emph{Documentation}
for which it uses \emph{Markup} tokens, which is why it was classified under
\emph{Markup Language} in this work.

Although literate programming itself has not gained that much popularity, its
idea of using markup tokens to generate more expressive source code documentation
has. Several up-to-date programming environments make use of it. A well-known
example is the \emph{JavaDoc} tool \cite{javadoc}; other systems are
\emph{Doxygen} \cite{doxygen} or \emph{DOC++} \cite{docpp}.

%
% $RCSfile: tex_and_latex.tex,v $
%
% Copyright (C) 2002-2008. Christian Heller.
%
% Permission is granted to copy, distribute and/or modify this document
% under the terms of the GNU Free Documentation License, Version 1.1 or
% any later version published by the Free Software Foundation; with no
% Invariant Sections, with no Front-Cover Texts and with no Back-Cover
% Texts. A copy of the license is included in the section entitled
% "GNU Free Documentation License".
%
% http://www.cybop.net
% - Cybernetics Oriented Programming -
%
% http://www.resmedicinae.org
% - Information in Medicine -
%
% Version: $Revision: 1.1 $ $Date: 2008-08-19 20:41:09 $ $Author: christian $
% Authors: Christian Heller <christian.heller@tuxtax.de>
%

\subsubsection{TeX and LaTeX}
\label{tex_and_latex_heading}
\index{Tex}
\index{Lamport TeX}
\index{LaTeX}
\index{LaTeXe}
\index{Device Independent Format}
\index{DVI}
\index{PDFTeX}
\index{Portable Document Format}
\index{PDF}
\index{TeX User Group}
\index{TUG}
\index{Desktop Publishing}
\index{DTP}

The special-purpose \TeX\ \cite{tex} language is the centre-piece of a
typesetting system which, due to its well-formatted output of complex
mathematical formulas and generally high-quality typesetting, is especially
popular among academic circles of mathematicians, physicists and computer
scientists \cite{latextutorial}. The Wikipedia encyclopedia \cite{wikipedia}
writes:

\begin{quote}
    \TeX\ is a macro and token based language: many commands, including most
    user-defined ones, are expanded on the fly until only unexpandable tokens
    remain which get executed. Expansion itself is practically side-effect free.
    Tail recursion of macros takes no memory, and if-then-else constructs are
    available. \ldots

    The \TeX\ system has precise knowledge of the sizes of all characters and
    symbols, and using this information, it computes the optimal arrangement of
    letters per line and lines per page. It then produces a \emph{Device Independent}
    (DVI) file containing the final locations of all characters. The DVI file
    can be printed directly given an appropriate printer driver, or it can be
    converted to other formats.
\end{quote}

The \emph{PDFTeX} translator program is often used to bypass all DVI generation,
by creating \emph{Portable Document Format} (PDF) files directly.

Nowadays, \TeX\ is mostly used with a template extension called \emph{Lamport \TeX}
(\LaTeX, \LaTeXe) \cite{latex}. The Indian \emph{\TeX\ Users Group} (TUG) writes
\cite{latextutorial}: \textit{\LaTeX\ is a document preparation system which
adds a set of functions that make the \TeX\ language friendlier than using
the primitives provided by it.} It offers programmable \emph{Desktop Publishing}
(DTP) features and extensive facilities for automating most aspects of
typesetting. \cite{wikipedia}

The most important feature of \TeX\ for this work is its kind of relating meta-
with structural information. Two examples may help here:

\begin{scriptsize}
    \begin{verbatim}
    \documentclass[a4paper,12pt]{book}
    \includegraphics[scale=0.3]{path/file.pdf}
    \end{verbatim}
\end{scriptsize}

The first statement determines \emph{book} as document class for a document to
be written. It contains additional information such as paper- and font size, in
square brackets. The second statement refers to a graphics file to be included.
The additional information given in square brackets here is the scale factor.
With a different syntax, but in a comparable manner, the knowledge modelling
language introduced in chapter \ref{cybernetics_oriented_language_heading} does
relate structural- with meta information.

%
% $RCSfile: extensible_markup_language.tex,v $
%
% Copyright (c) 2002-2007. Christian Heller. All rights reserved.
%
% Permission is granted to copy, distribute and/or modify this document
% under the terms of the GNU Free Documentation License, Version 1.1 or
% any later version published by the Free Software Foundation; with no
% Invariant Sections, with no Front-Cover Texts and with no Back-Cover
% Texts. A copy of the license is included in the section entitled
% "GNU Free Documentation License".
%
% http://www.cybop.net
% - Cybernetics Oriented Programming -
%
% Version: $Revision: 1.1 $ $Date: 2007-07-17 20:02:36 $ $Author: christian $
% Authors: Christian Heller <christian.heller@tuxtax.de>
%

\section{Extensible Markup Language}
\label{extensible_markup_language_heading}
\index{Extensible Markup Language}

The \emph{Extensible Markup Language} (XML) is ...

