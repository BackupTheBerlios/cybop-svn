%
% $RCSfile: overall_placement.tex,v $
%
% Copyright (C) 2002-2008. Christian Heller.
%
% Permission is granted to copy, distribute and/or modify this document
% under the terms of the GNU Free Documentation License, Version 1.1 or
% any later version published by the Free Software Foundation; with no
% Invariant Sections, with no Front-Cover Texts and with no Back-Cover
% Texts. A copy of the license is included in the section entitled
% "GNU Free Documentation License".
%
% http://www.cybop.net
% - Cybernetics Oriented Programming -
%
% http://www.resmedicinae.org
% - Information in Medicine -
%
% Version: $Revision: 1.1 $ $Date: 2008-08-19 20:41:08 $ $Author: christian $
% Authors: Christian Heller <christian.heller@tuxtax.de>
%

\subsection{Overall Placement}
\label{overall_placement_heading}
\index{CYBOI as Knowledge-Hardware Interface}
\index{Java-CYBOP Analogies}
\index{CYBOP-Java Analogies}
\index{CYBOI as Virtual Machine}
\index{Computer Hardware}
\index{Operating System}
\index{OS}
\index{CYBOL}

Considering an overall computer system architecture, \emph{CYBOI} is situated
between the application knowledge existing in form of \emph{CYBOL} templates
and the \emph{Hardware} controlled by an \emph{Operating System} (OS), as was
shown in figure \ref{connection_figure}. CYBOI can thus also be called a
\emph{Knowledge-Hardware-Interface} (synonymous with \emph{Mind-Brain-Interface}).

\begin{table}[ht]
    \begin{center}
        \begin{footnotesize}
        \begin{tabular}{| p{35mm} | p{35mm} | p{35mm} |}
            \hline
            \textbf{Criterion} & \textbf{Java World} & \textbf{CYBOP World}\\
            \hline
            Theory & OOP in Java & CYBOP\\
            \hline
            Language & Java & CYBOL\\
            \hline
            Interpreter & Java VM & CYBOI\\
            \hline
        \end{tabular}
        \end{footnotesize}
        \caption{Analogies between the Java- and CYBOP World}
        \label{analogies_table}
    \end{center}
\end{table}

There are analogies to other systems run by language interpretation. Table
\ref{analogies_table} shows those between the \emph{Java-} and \emph{CYBOP}
world. Both are based on a programming theory, have a language and interpreter.
A theoretical model of a computer hardware- or -software system may be called
an \emph{Abstract Computer} or \emph{Abstract Machine} \cite{wikipedia}. If
being implemented as software simulation, or if containing an interpreter, it
is called a \emph{Virtual Machine} (VM). Kernighan and Pike write in their book
\emph{Practice of Programming} \cite{kernighan1999}:

\begin{quote}
� � Virtual machines are a wonderful, old idea, that latterly, through Java and
    the \emph{Java Virtual Machine} (JVM), came into fashion again. They are a
    simple possibility to gain portable and efficient program code, which can
    be written in a higher programming language.
\end{quote}

In that sense, CYBOI is certainly a VM. It provides low-level, platform-dependent
system functionality, close to the OS, together with a unified knowledge schema
(chapter \ref{knowledge_schema_heading}) which allows CYBOL applications to be
truly portable, well extensible and easier to program, because developers need
to concentrate on domain knowledge only. Since CYBOI interprets CYBOL sources
\emph{live} at system runtime, without the need for previous compilation (as in
Java), changes to CYBOL sources get into effect right away, without restarting
the system.

The use of an OS, however, has to be seen as temporary workaround. One future
aim is to remove all OS dependencies by stepwise integrating hardware device
driving functionality and other OS concepts into CYBOI (chapter
\ref{summary_and_outlook_heading}).
