%
% $RCSfile: knowledge_ontology.tex,v $
%
% Copyright (C) 2002-2008. Christian Heller.
%
% Permission is granted to copy, distribute and/or modify this document
% under the terms of the GNU Free Documentation License, Version 1.1 or
% any later version published by the Free Software Foundation; with no
% Invariant Sections, with no Front-Cover Texts and with no Back-Cover
% Texts. A copy of the license is included in the section entitled
% "GNU Free Documentation License".
%
% http://www.cybop.net
% - Cybernetics Oriented Programming -
%
% http://www.resmedicinae.org
% - Information in Medicine -
%
% Version: $Revision: 1.1 $ $Date: 2008-08-19 20:41:07 $ $Author: christian $
% Authors: Christian Heller <christian.heller@tuxtax.de>
%

\subsection{Knowledge Ontology}
\label{knowledge_ontology_heading}
\index{Knowledge Ontology}
\index{Composition}
\index{Ontology}
\index{Ontological Level}
\index{Granularity}

The previous sections tried to demonstrate the importance of \emph{Composition}
for knowledge modelling. One technique that was mentioned in this context are
\emph{Ontologies}. Section \ref{conceptual_network_heading} introduced some of
its numerous definitions. Section \ref{association_elimination_heading}
demonstrated how the principle of \emph{Hierarchy} may be applied to obtain an
\emph{Ontology}. The layers forming an ontology were called
\emph{Ontological Level}.

Basically, an ontology represents a systematic description of complex domain
contexts. This work uses its own adapted definition, and considers an ontology
to be \emph{a strict hierarchy of abstract models, organised in levels of
growing granularity, that are solely unidirectionally related}.

Terminologies as described in section \ref{terminology_heading} may be used to
specify the basic elements of an ontology. Every term may be represented by an
own abstract model (concept) containing a number of strings, one for each
terminology system. Further strings may stand for language translations, which
has importance for \emph{Internationalisation}.

The following examples may seem simple, but want to strengthen the hierarchical
thinking of the reader, under consideration of the granularity of models.

%
% $RCSfile: biological_systems.tex,v $
%
% Copyright (C) 2002-2008. Christian Heller.
%
% Permission is granted to copy, distribute and/or modify this document
% under the terms of the GNU Free Documentation License, Version 1.1 or
% any later version published by the Free Software Foundation; with no
% Invariant Sections, with no Front-Cover Texts and with no Back-Cover
% Texts. A copy of the license is included in the section entitled
% "GNU Free Documentation License".
%
% http://www.cybop.net
% - Cybernetics Oriented Programming -
%
% http://www.resmedicinae.org
% - Information in Medicine -
%
% Version: $Revision: 1.1 $ $Date: 2008-08-19 20:41:05 $ $Author: christian $
% Authors: Christian Heller <christian.heller@tuxtax.de>
%

\subsubsection{Biological Systems}
\label{biological_systems_heading}
\index{Biological System as Ontology}
\index{Ontological Layer}
\index{Parallel Layer}
\index{Stratum}

One example showing the hierarchical structuring of biological systems is
mentioned in \cite{sengbusch}. Its models are listed in decreasing granularity,
in table \ref{biological_table}.

\begin{table}[ht]
    \begin{center}
        \begin{footnotesize}
        \begin{tabular}{| p{105mm} |}
            \hline
            \textbf{Biological System}\\
            \hline
            Ecosystem\\
            \hline
            Biocoenosis (Living Community)\\
            \hline
            Multiple Cell Organism\\
            \hline
            Single Cell Organism (Protozoa)\\
            \hline
            Organelle (Mitochondrie, Chloroplast)\\
            \hline
            Supra Molecular Complex (Ribosome, Chromosome, Membrane)\\
            \hline
            Small Molecule\\
            \hline
        \end{tabular}
        \end{footnotesize}
        \caption{Hierarchical Structuring of Biological Systems}
        \label{biological_table}
    \end{center}
\end{table}

It is important not to mix ontological layers with parallel layers. In
\emph{Geology} or \emph{Biology}, the latter (also called \emph{Stratum}) may
be layers of material arranged one on top of another (such as a layer of tissue
or cells in an organism) \cite{wordnet}. However, these are not \emph{composed}
of each other. Ontological layers, on the other hand, have a different level of
granularity, each so that higher-level abstractions are composed of lower-level
abstractions.

%
% $RCSfile: logical_book.tex,v $
%
% Copyright (C) 2002-2008. Christian Heller.
%
% Permission is granted to copy, distribute and/or modify this document
% under the terms of the GNU Free Documentation License, Version 1.1 or
% any later version published by the Free Software Foundation; with no
% Invariant Sections, with no Front-Cover Texts and with no Back-Cover
% Texts. A copy of the license is included in the section entitled
% "GNU Free Documentation License".
%
% http://www.cybop.net
% - Cybernetics Oriented Programming -
%
% http://www.resmedicinae.org
% - Information in Medicine -
%
% Version: $Revision: 1.1 $ $Date: 2008-08-19 20:41:07 $ $Author: christian $
% Authors: Christian Heller <christian.heller@tuxtax.de>
%

\subsubsection{Logical Book}
\label{logical_book_heading}
\index{Logical Book as Ontology}
\index{Extension of Ontologies}
\index{Physical Book as Ontology}

The logical structure of a \emph{Book} shall serve as second example. A
\emph{Chapter} may consist of \emph{Paragraphs}. Yet it may become necessary to
first subdivide \emph{Chapter} into \emph{Sections} which then consist of
\emph{Paragraphs}, as shown in table \ref{book_table}.

All ontologies can get extended \emph{up-} or \emph{downwards}, by adding
further levels, at any later point in design time. But they can as well get
extended by inserting \emph{Intermediate Layers} between two already existing
ones. However, additional levels should only get introduced if there really is
a need for them.

\begin{table}[ht]
    \begin{center}
        \begin{footnotesize}
        \begin{tabular}{| p{105mm} |}
            \hline
            \textbf{Model Category}\\
            \hline
            Library\\
            \hline
            Book\\
            \hline
            Part\\
            \hline
            Chapter\\
            \hline
            Section\\
            \hline
            Paragraph\\
            \hline
            Sentence\\
            \hline
            Word\\
            \hline
            Character\\
            \hline
        \end{tabular}
        \end{footnotesize}
        \caption{Logical Book}
        \label{book_table}
    \end{center}
\end{table}

In contrast to the division of a \emph{logical} book, a \emph{physical} book
may be structured completely differently, for example into \emph{Binding},
\emph{Cover} and \emph{Pages}. Of course, the contents of an ontology heavily
depends on the intended area of application (knowledge domain) of the software
to be created.

%
% $RCSfile: interdisciplinary_science.tex,v $
%
% Copyright (C) 2002-2008. Christian Heller.
%
% Permission is granted to copy, distribute and/or modify this document
% under the terms of the GNU Free Documentation License, Version 1.1 or
% any later version published by the Free Software Foundation; with no
% Invariant Sections, with no Front-Cover Texts and with no Back-Cover
% Texts. A copy of the license is included in the section entitled
% "GNU Free Documentation License".
%
% http://www.cybop.net
% - Cybernetics Oriented Programming -
%
% http://www.resmedicinae.org
% - Information in Medicine -
%
% Version: $Revision: 1.1 $ $Date: 2008-08-19 20:41:07 $ $Author: christian $
% Authors: Christian Heller <christian.heller@tuxtax.de>
%

\subsubsection{Interdisciplinary Science}
\label{interdisciplinary_science_heading}
\index{Interdisciplinary Science}
\index{System of Sciences}
\index{Sciences as Ontology}
\index{Cybernetics}

A third, certainly very subjective example tries to sort a number of known
\emph{Sciences} into one common system (table \ref{sciences_table}).
\emph{Arts}, \emph{Linguistics}, \emph{Mathematics} and \emph{Informatics} have
an extra status: They deal with already abstracted knowledge (paintings, music,
language, numbers) and can be used as utility by any of the other sciences.

\begin{table}[ht]
    \begin{center}
        \begin{footnotesize}
        \begin{tabular}{| p{35mm} | p{70mm} |}
            \hline
            \textbf{Scientific Subject} & \textbf{Example Model}\\
            \hline
            Astronomy & Celestial Body (Big Bang, Cosmos)\\
            \hline
            Biology & Living Thing (Human, Animal, Plant, Virus)\\
            \hline
            Geography & Dead Thing (Air, Fire, Stone, Crystal)\\
            \hline
            Chemistry & Compounds (Water, DNA)\\
            \hline
            Physics & Particles (Elementary Particle, Atom, Matter, Energy)\\
            \hline
            Philosophy / Religion & Dialectic Dualism (Matter/Anti-Matter, +/-, 0/1)\\
            \hline
        \end{tabular}
        \end{footnotesize}
        \caption{System of Sciences}
        \label{sciences_table}
    \end{center}
\end{table}

The whole effort of finding new ways for representing knowledge, as done in
this work, is an \emph{inter-disciplinary} undertaking itself, touching various
fields of science. The world (nature) needs to be understood in its basics so
that humans are enabled to copy its concepts and put them into artificial
models -- exactly what \emph{Cybernetics} is all about.

%
% $RCSfile: car_model.tex,v $
%
% Copyright (C) 2002-2008. Christian Heller.
%
% Permission is granted to copy, distribute and/or modify this document
% under the terms of the GNU Free Documentation License, Version 1.1 or
% any later version published by the Free Software Foundation; with no
% Invariant Sections, with no Front-Cover Texts and with no Back-Cover
% Texts. A copy of the license is included in the section entitled
% "GNU Free Documentation License".
%
% http://www.cybop.net
% - Cybernetics Oriented Programming -
%
% http://www.resmedicinae.org
% - Information in Medicine -
%
% Version: $Revision: 1.1 $ $Date: 2008-08-19 20:41:05 $ $Author: christian $
% Authors: Christian Heller <christian.heller@tuxtax.de>
%

\subsubsection{Car Model}
\label{car_model_heading}
\index{Car Model as Ontology}
\index{Computer Aided Design}
\index{CAD}
\index{Computer Aided Manufacturing}
\index{CAM}
\index{Unidirectional Relation}
\index{Layers of an Abstract Model}
\index{Levels of an Abstract Model}

A \emph{Computer Aided Design} (CAD)/ \emph{Computer Aided Manufacturing} (CAM)
system of a car manufacturer will have a \emph{Car} model like the one shown in
table \ref{car_table}.

\begin{table}[ht]
    \begin{center}
        \begin{footnotesize}
        \begin{tabular}{| p{105mm} |}
            \hline
            \textbf{Model Category}\\
            \hline
            Car\\
            \hline
            Body, Chassis, Engine, Transmission\\
            \hline
            Door, Axle, Wheel, Cylinder\\
            \hline
            Window, Suspension, Plunger\\
            \hline
        \end{tabular}
        \end{footnotesize}
        \caption{Car Model}
        \label{car_table}
    \end{center}
\end{table}

The ontology contains multiple categories of models which are composed of each
other. An \emph{Engine} consists of a \emph{Cylinder} which consists of a
\emph{Plunger} and so on. That is why people speak of different \emph{Layers}
or \emph{Levels} of abstract models. An \emph{Axle} belongs to one level and a
\emph{Chassis} belongs to another, higher level. In a good ontology, the
relations between models are always \emph{unidirectional}, that is a chassis
can link to axles but not the other way.

%
% $RCSfile: macrocosm_and_microcosm.tex,v $
%
% Copyright (C) 2002-2008. Christian Heller.
%
% Permission is granted to copy, distribute and/or modify this document
% under the terms of the GNU Free Documentation License, Version 1.1 or
% any later version published by the Free Software Foundation; with no
% Invariant Sections, with no Front-Cover Texts and with no Back-Cover
% Texts. A copy of the license is included in the section entitled
% "GNU Free Documentation License".
%
% http://www.cybop.net
% - Cybernetics Oriented Programming -
%
% http://www.resmedicinae.org
% - Information in Medicine -
%
% Version: $Revision: 1.1 $ $Date: 2008-08-19 20:41:07 $ $Author: christian $
% Authors: Christian Heller <christian.heller@tuxtax.de>
%

\subsubsection{Macrocosm and Microcosm}
\label{macrocosm_and_microcosm_heading}
\index{Macrocosm as Part of an Ontology}
\index{Microcosm as Part of an Ontology}
\index{Astronomical Particles as Ontology}
\index{Universe}
\index{Top Level Model}
\index{Concept}
\index{Electronic Health Record}
\index{EHR}
\index{Electronic Insurance Record}
\index{EIR}

Table \ref{astronomical_table} lists \emph{Astronomical Particles}
\cite{fernandezdavid, arnett}. It ends with the \emph{Universe} and an
undefinable \emph{Macrocosm}.

\begin{table}[ht]
    \begin{center}
        \begin{footnotesize}
        \begin{tabular}{| p{40mm} | p{65mm} |}
            \hline
            \textbf{Category} & \textbf{Example Model}\\
            \hline
            Macrocosm & (Infinity)\\
            \hline
            Universe & Our Universe with its Laws of Nature\\
            \hline
            Heap of Galaxies & Local Group, Heap of Virgo\\
            \hline
            Galaxy & Milky Way (our), Andromeda, Magellan's Clouds\\
            \hline
            Planetary (Solar) System & Sol (Sun), 51 Pegasi\\
            \hline
            Star/ Planet & Beta Pictoris, Mercury, Venus, Earth, Mars\\
            \hline
        \end{tabular}
        \end{footnotesize}
        \caption{Astronomical Particles}
        \label{astronomical_table}
    \end{center}
\end{table}

When trying to abstract things (in software), there has to be some limit, a
kind of \emph{Top Level Model}. It represents the \emph{Concept} to be
described. For a medical information system, one such top level model will be
the \emph{Electronic Health Record} (EHR); for an insurance application, it
will be the \emph{Electronic Insurance Record} (EIR); and so on.

Models do not only have to be limited \emph{upwards}; the same holds true for
modelling towards \emph{Microcosm}. Table \ref{physical_table} organises
particles, as used by natural sciences, into several categories.

\begin{table}[ht]
    \begin{center}
        \begin{footnotesize}
        \begin{tabular}{| p{50mm} | p{55mm} |}
            \hline
            \textbf{Category} & \textbf{Example Model}\\
            \hline
            Physical Compound & Air, Water, Fire, Ground\\
            \hline
            Chemical Compound/ Molecule & H$_{2}$, O$_{2}$, O$_{3}$, H$_{2}$O\\
            \hline
            Crystal & C (Diamond)\\
            \hline
            Atom (Chemical Element) & H, He, O\\
            \hline
            Elementary Particle & Quark, Lepton (Electron, Neutrino)\\
            \hline
            Urelement & (Primary Particle)\\
            \hline
            Microcosm & (Infinity)\\
            \hline
        \end{tabular}
        \end{footnotesize}
        \caption{Physical Particles}
        \label{physical_table}
    \end{center}
\end{table}

Although the real world seems to be built like that (infinite, nobody knowing
what comes beneath the \emph{Quark} particles) -- in software modelling it
makes no sense (and is actually impossible) to neverendingly introduce lower
and lower levels, towards \emph{Microcosm}. On some point, the hierarchy has to
be stopped, to be able to abstract it in software. The later chapter
\ref{state_and_logic_heading} gives an overview of common knowledge primitives.

