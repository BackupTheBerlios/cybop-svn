%
% $RCSfile: agent_with_mental_state.tex,v $
%
% Copyright (C) 2002-2008. Christian Heller.
%
% Permission is granted to copy, distribute and/or modify this document
% under the terms of the GNU Free Documentation License, Version 1.1 or
% any later version published by the Free Software Foundation; with no
% Invariant Sections, with no Front-Cover Texts and with no Back-Cover
% Texts. A copy of the license is included in the section entitled
% "GNU Free Documentation License".
%
% http://www.cybop.net
% - Cybernetics Oriented Programming -
%
% http://www.resmedicinae.org
% - Information in Medicine -
%
% Version: $Revision: 1.1 $ $Date: 2008-08-19 20:41:05 $ $Author: christian $
% Authors: Christian Heller <christian.heller@tuxtax.de>
%

\subsection{Agent with Mental State}
\label{agent_with_mental_state_heading}
\index{Agent Oriented Programming}
\index{AGOP}
\index{Agent}
\index{Mental State of an Agent}
\index{Formal Knowledge Representation Language of AGOP}
\index{Agent Programming Language of AGOP}
\index{Conversion Method of AGOP}

One design paradigm that early recognised the advantages of splitting software
into low-level system control and high-level knowledge, is
\emph{Agent Oriented Programming} (AGOP) (section
\ref{agent_oriented_programming_heading}). \emph{Agents}, as active software
components (which in this work means: \emph{running in an own process}), have
a \emph{Mental State} representing their knowledge, which they are able to
interpret and manipulate. This approach was copied in CYBOP.

Of the three elements \emph{Formal Knowledge Representation Language},
\emph{Agent Programming Language} and \emph{Conversion Method}, which an AGOP
system, after section \ref{agent_oriented_programming_heading}, needs in order
to be complete, this work provides the first two in form of the
\emph{Cybernetics Oriented Language} (CYBOL) and the
\emph{Cybernetics Oriented Interpreter} (CYBOI), described in sections
\ref{cybernetics_oriented_language_heading} and
\ref{cybernetics_oriented_interpreter_heading}, respectively. Thereby, CYBOI
itself is not a language, but represents a ready system, written in the
\emph{C} programming language. A method for converting traditional applications
into agents is not provided, since methodologies are clearly \emph{not} a topic
of this work.

Of course, there are differences distinguishing
\emph{Cybernetics Oriented Programming} (CYBOP) from traditional AGOP systems.
CYBOP needs an own interpreter, because of its new knowledge representation
philosophy and -language, which are the topic of chapters
\ref{knowledge_schema_heading} and \ref{cybernetics_oriented_language_heading}.
One reason for the difficult handling and intransparency of many traditional
knowledge representation languages is that they mix two kinds of knowledge:
\emph{State-} and \emph{Logic} descriptions. More on how this is avoided in
CYBOP in chapter \ref{state_and_logic_heading}.

One may wonder why such a supposedly advantageous architecture is not used by
all of today's systems? One reason may be that AGOP is still a rather young
technology lacking the necessary popularity. Another reason may be the bad
reputation of AGOP systems (and just about everything that has to do with
knowledge representation) among average developers -- partly because of their
immaturity, but mostly because of their complicated knowledge models and
-handling. Looking at the often quite cryptic appearance of the corresponding
languages, one tends to understand the developers' dislike.
