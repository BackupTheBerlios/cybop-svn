%
% $RCSfile: date_and_rule.tex,v $
%
% Copyright (C) 2002-2008. Christian Heller.
%
% Permission is granted to copy, distribute and/or modify this document
% under the terms of the GNU Free Documentation License, Version 1.1 or
% any later version published by the Free Software Foundation; with no
% Invariant Sections, with no Front-Cover Texts and with no Back-Cover
% Texts. A copy of the license is included in the section entitled
% "GNU Free Documentation License".
%
% http://www.cybop.net
% - Cybernetics Oriented Programming -
%
% http://www.resmedicinae.org
% - Information in Medicine -
%
% Version: $Revision: 1.1 $ $Date: 2008-08-19 20:41:06 $ $Author: christian $
% Authors: Christian Heller <christian.heller@tuxtax.de>
%

\subsection{Date and Rule}
\label{date_and_rule_heading}
\index{Date and Rule}
\index{Expert System}
\index{Relational Database}
\index{Structured Query Language}
\index{SQL}
\index{Existential Conjunctive Logic}
\index{EC Logic}
\index{Modus Ponens, Inference Rule}
\index{Modus Tollens, Inference Rule}
\index{Forward Chaining}
\index{Backward Chaining}
\index{View, Implication}
\index{Trigger, Implication}
\index{Prolog}
\index{Microplanner}
\index{Backtracking}

Two kinds of systems that gained greater popularity are \emph{Expert Systems}
and \emph{Relational Databases}. After Sowa \cite{sowa}, both differed more in
quantity than in quality: \textit{Expert systems use repeated executions of
rules on relatively small amounts of data, while database systems execute short
chains of rules on large amounts of data.} Over time, their differences
decreased and today, the \emph{Structured Query Language} (SQL) for relational
databases supports the same logical functions as early expert systems.

Both, expert systems and relational databases, have common logical foundations
and store data in a subset of logic called \emph{Existential Conjunctive} (EC)
logic. EC is based on two logical operators: the \emph{Existential Quantifier}
$\exists$ and the \emph{Conjunction} $\wedge$; the \emph{Universal Quantifier}
$\forall$ and other operators ($\sim$, $\supset$, $\vee$) are never used.
Sowa \cite[p. 163]{sowa} writes: \textit{While variables in a query are governed
by existential quantifiers, those in a rule are governed by universal quantifiers.}

The two primary inference rules of the above-mentioned systems are called
\emph{Modus Ponens} (putting) and \emph{Modus Tollens} (taking away). Although
being simple, the power of these rules comes from their combination and
repeated execution. While repeated execution of modus ponens is called
\emph{Forward Chaining}, that of modus tollens is called \emph{Backward Chaining}.
In SQL, an implication used in backward chaining is called \emph{View}, and that
used in forward chaining is called \emph{Trigger} \cite{sowa}.

Besides \emph{Prolog} (section \ref{logical_programming_heading}) and \emph{SQL}
(section \ref{data_manipulation_language_heading}), the \emph{Microplanner}
language \cite[p. 157]{sowa} uses the so-called \emph{Backtracking} technique
to answer a query: If one of a sequence of aims cannot be satisfied, the language
tracks back to a previous aim and tries a different option. Although equivalent
queries in Prolog and SQL differ in their syntax, the semantics is the same.
\textit{Logic determines the structure of a query}, as Sowa \cite[p. 159]{sowa}
means.

To sum up, one can say that previous sections distinguished between \emph{Domain-}
and \emph{Application Models}. What was shown in this section, however, is that
many systems and their corresponding languages rely on a separation of \emph{Data}
(in state variables) and \emph{Rules} (logic). This will be of importance in
chapter \ref{state_and_logic_heading}.
