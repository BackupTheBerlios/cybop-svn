%
% $RCSfile: signal_handling.tex,v $
%
% Copyright (C) 2002-2008. Christian Heller.
%
% Permission is granted to copy, distribute and/or modify this document
% under the terms of the GNU Free Documentation License, Version 1.1 or
% any later version published by the Free Software Foundation; with no
% Invariant Sections, with no Front-Cover Texts and with no Back-Cover
% Texts. A copy of the license is included in the section entitled
% "GNU Free Documentation License".
%
% http://www.cybop.net
% - Cybernetics Oriented Programming -
%
% http://www.resmedicinae.org
% - Information in Medicine -
%
% Version: $Revision: 1.1 $ $Date: 2008-08-19 20:41:08 $ $Author: christian $
% Authors: Christian Heller <christian.heller@tuxtax.de>
%

\subsection{Signal Handling}
\label{signal_handling_heading}
\index{CYBOI Signal Handling}
\index{Central Processing Unit}
\index{CPU}

Depending on the signal model's kind of abstraction, two different signal
handling procedures may be called: \emph{handle\_compound} or
\emph{handle\_operation} (both in the \emph{handler} module). While the former
breaks down composed signals (algorithms) into basic operations, the latter
executes primitive signals (operations) directly, in form of low-level
instructions, which may go down to direct calls of the instruction set of the
\emph{Central Processing Unit} (CPU).

Actual knowledge model changes, in other words the application of well-defined
\emph{Logic-} to \emph{State} models, is done by primitive operations only.
