%
% $RCSfile: inflexible_architecture.tex,v $
%
% Copyright (C) 2002-2008. Christian Heller.
%
% Permission is granted to copy, distribute and/or modify this document
% under the terms of the GNU Free Documentation License, Version 1.1 or
% any later version published by the Free Software Foundation; with no
% Invariant Sections, with no Front-Cover Texts and with no Back-Cover
% Texts. A copy of the license is included in the section entitled
% "GNU Free Documentation License".
%
% http://www.cybop.net
% - Cybernetics Oriented Programming -
%
% http://www.resmedicinae.org
% - Information in Medicine -
%
% Version: $Revision: 1.1 $ $Date: 2008-08-19 20:41:07 $ $Author: christian $
% Authors: Christian Heller <christian.heller@tuxtax.de>
%

\paragraph{Inflexible Architecture}
\label{inflexible_architecture_heading}
\index{Inflexible Architecture}

First and foremost, the static coupling of classes leads to an inflexible
design. The names and number of attributes and methods as integral part of a
class cannot be changed dynamically later-on; only their values can. The class
structure represents a solution to a current problem. If it is static, then
future requirements cannot be considered. Adaptation issues and workarounds,
affecting stability and security, are thus to be expected.
