%
% $RCSfile: dissolution_of_process_concept.tex,v $
%
% Copyright (C) 2002-2008. Christian Heller.
%
% Permission is granted to copy, distribute and/or modify this document
% under the terms of the GNU Free Documentation License, Version 1.1 or
% any later version published by the Free Software Foundation; with no
% Invariant Sections, with no Front-Cover Texts and with no Back-Cover
% Texts. A copy of the license is included in the section entitled
% "GNU Free Documentation License".
%
% http://www.cybop.net
% - Cybernetics Oriented Programming -
%
% http://www.resmedicinae.org
% - Information in Medicine -
%
% Version: $Revision: 1.1 $ $Date: 2008-08-19 20:41:06 $ $Author: christian $
% Authors: Christian Heller <christian.heller@tuxtax.de>
%

\paragraph{The Dissolution of the Process Concept}
\label{dissolution_of_process_concept_heading}

Because all resources belonging to an application are situated under the
corresponding branch of the knowledge tree carrying the application's name, a
\emph{Process Table} (sometimes called \emph{Process Control Block}) as found
in classical \emph{Operating Systems} (OS) might not be necessary anymore.

After Tanenbaum \cite{tanenbaum2001}, an entry in the process table of an OS
contained information about the process' state, its program counter, stack
pointer, memory allocation, the status of its open files, its accounting and
scheduling information, and everything else about the process that must be
saved when the process is switched from \emph{running} to \emph{ready} or
\emph{blocked} state so that it can be restarted later as if it had never been
stopped.

see: \cite[p. 81]{tanenbaum2001}

- if there are no processes, there is no state information
- program counter: ?? (real time)
- a stack is not needed ??, since all states belong to the one knowledge tree
- CYBOP application system keeps all resources under one knowledge tree branch;
no \emph{wild} memory allocation --> no memory allocation information to keep
--> no context switches necessary when changing from one process to another
- open files: ?? (must not be accessed in parallel)
- accounting: ??
- scheduling: ??

- Operations are low-level and belong to CYBOI.
- The interruption of long-lasting executions of operations is a low-level issue
that must be solved within CYBOI, but \emph{not} by dividing application models
in RAM up into abstract processes.
- The investigation of concepts applicable for this interruption, however, are
an own topic that does not originally belong to this work.
