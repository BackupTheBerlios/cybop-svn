%
% $RCSfile: generics.tex,v $
%
% Copyright (C) 2002-2008. Christian Heller.
%
% Permission is granted to copy, distribute and/or modify this document
% under the terms of the GNU Free Documentation License, Version 1.1 or
% any later version published by the Free Software Foundation; with no
% Invariant Sections, with no Front-Cover Texts and with no Back-Cover
% Texts. A copy of the license is included in the section entitled
% "GNU Free Documentation License".
%
% http://www.cybop.net
% - Cybernetics Oriented Programming -
%
% http://www.resmedicinae.org
% - Information in Medicine -
%
% Version: $Revision: 1.1 $ $Date: 2008-08-19 20:41:06 $ $Author: christian $
% Authors: Christian Heller <christian.heller@tuxtax.de>
%

\subsection{Generics}
\label{generics_heading}
\index{Generics}
\index{Generic Programming}
\index{Function Template}
\index{Class Template}
\index{Standard Template Library}
\index{STL}
\index{Eiffel}
\index{Java}
\index{VB.NET}
\index{C\#}
\index{Reuse through Parameterisation}
\index{Dynamic Typing}

\emph{Generic Programming} received its name from the \emph{Generics} it uses.
Wikipedia \cite{wikipedia} writes: \textit{Generics is a technique that allows
one value to take different datatypes (so-called polymorphism) as long as
certain contracts such as subtypes and signature are kept.} \emph{Templates}
are one technique providing generics. They allow the writing of code without
considering the data type that code will eventually be used with. Two kinds of
templates exist \cite{wikipedia}:

\begin{itemize}
    \item[-] \emph{Function Template:} behaving like a function that can accept
        arguments of many different types
    \item[-] \emph{Class Template:} extending the same concept to classes;
        often used to make generic containers
\end{itemize}

Using templates of the C++ \emph{Standard Template Library} (STL) \cite{stl},
a list may be declared by writing \texttt{list<T>}, where \emph{T} represents
the type that may be substituted as needed. A linked list of integers, for
example, would be created with \texttt{list<int>}. After \cite{wikipedia},
there are three primary drawbacks to the use of templates:

\begin{enumerate}
    \item Less portable code due to the poor support for templates in compilers
    \item Difficult development of templates due to unhelpful error messages
        produced by compilers
    \item Bloated code due to the extra code (instantiated template) generated
        by compilers
\end{enumerate}

Meanwhile, many other OOP languages like \emph{Eiffel}, \emph{Java},
\emph{VB.NET} and \emph{C\#} provide generic facilities. Being used to improve
the customisability of code at compile time, they retain the efficiency of
statically configured code. However, in practice (own experience of the author)
it is often hard for programmers to understand and handle generic techniques.
Czarnecki \cite{czarnecki}, who summarises generic programming as \textit{Reuse
through Parameterisation}, criticises that it: \textit{limits code generation
to substituting generic type parameters with concrete types and welding together
pre-existing fragments of code in a fixed pattern.} \emph{Dynamic Typing}
(section \ref{typeless_programming_heading}) is one possibility to circumvent
the need for generic programming. The interpreter program introduced in chapter
\ref{cybernetics_oriented_interpreter_heading} uses dynamic typing; it
references all knowledge via neutral pointers whose meaning gets determined
only at runtime.
