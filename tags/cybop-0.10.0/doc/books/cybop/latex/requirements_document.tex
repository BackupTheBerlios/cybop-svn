%
% $RCSfile: requirements_document.tex,v $
%
% Copyright (C) 2002-2008. Christian Heller.
%
% Permission is granted to copy, distribute and/or modify this document
% under the terms of the GNU Free Documentation License, Version 1.1 or
% any later version published by the Free Software Foundation; with no
% Invariant Sections, with no Front-Cover Texts and with no Back-Cover
% Texts. A copy of the license is included in the section entitled
% "GNU Free Documentation License".
%
% http://www.cybop.net
% - Cybernetics Oriented Programming -
%
% http://www.resmedicinae.org
% - Information in Medicine -
%
% Version: $Revision: 1.1 $ $Date: 2008-08-19 20:41:08 $ $Author: christian $
% Authors: Christian Heller <christian.heller@tuxtax.de>
%

\subsection{Requirements Document}
\label{requirements_document_heading}
\index{Res Medicinae Requirements Document}
\index{DocBook DTD}
\index{The Linux Documentation Project}
\index{TLDP}

With the help of German medical doctors, a \emph{Requirements Document}
\cite{resmedicinae2001} was created and is meanwhile being updated and extended
since about five years. It basically describes an EHR and the information it
should include.

Since the document itself is just a hierarchical model consisting of parts, it
can well be represented in CYBOL. Unfortunately, a document processor that can
read and render CYBOL, in the style of \emph{LaTeX} \cite{latex}, has not been
written to date (although CYBOI might integrate this functionality one day). It
was therefore decided to write the requirements document in SGML/ XML, using
the \emph{DocBook} DTD \cite{docbook} and tools described in
\emph{The Linux Documentation Project} (TLDP) \cite{linuxdoc}.
