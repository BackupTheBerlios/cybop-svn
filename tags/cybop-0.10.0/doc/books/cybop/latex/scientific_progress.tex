%
% $RCSfile: scientific_progress.tex,v $
%
% Copyright (C) 2002-2008. Christian Heller.
%
% Permission is granted to copy, distribute and/or modify this document
% under the terms of the GNU Free Documentation License, Version 1.1 or
% any later version published by the Free Software Foundation; with no
% Invariant Sections, with no Front-Cover Texts and with no Back-Cover
% Texts. A copy of the license is included in the section entitled
% "GNU Free Documentation License".
%
% http://www.cybop.net
% - Cybernetics Oriented Programming -
%
% http://www.resmedicinae.org
% - Information in Medicine -
%
% Version: $Revision: 1.1 $ $Date: 2008-08-19 20:41:08 $ $Author: christian $
% Authors: Christian Heller <christian.heller@tuxtax.de>
%

\section*{Scientific Progress}
\label{scientific_progress_heading}
%\addcontentsline{toc}{section}{Scientific Progress}

An \emph{Abstraction} allows to capture the real world by representing it in
simplified models. Such models contain only the essential aspects of a special
domain. Any unimportant nuances, in the considered context, are neglected.
Correct abstract models is what makes science easy. Good science \emph{can} be
\emph{easy}. If it is not, then probably either:

\begin{itemize}
    \item[-] there is a \emph{mistake} in the model
    \item[-] it is not fully \emph{understood} by the scientist him/ herself
    \item[-] the explaining person wants to \emph{keep back} knowledge, making
        others look clueless
\end{itemize}

One of the biggest hindrances to scientific progress is too much or false
respect for existing solutions. No theory/ model/ concept is ever finished;
no document/ software/ product is ever fully completed. There is always room
for improvements. In the end, it is all just a person's subjective perception
and an arbitrary, abstract extract of the real world.

It is always worth reviewing and questionning everything in depth, again and
again. Standstill means regress. The best example showing how to work around
these criticisms is the \emph{Free and Open Source Software} (FOSS) movement
where all the time, existing solutions are rewritten, to be improved.
