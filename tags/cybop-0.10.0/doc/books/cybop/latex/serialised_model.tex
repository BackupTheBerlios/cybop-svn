%
% $RCSfile: serialised_model.tex,v $
%
% Copyright (C) 2002-2008. Christian Heller.
%
% Permission is granted to copy, distribute and/or modify this document
% under the terms of the GNU Free Documentation License, Version 1.1 or
% any later version published by the Free Software Foundation; with no
% Invariant Sections, with no Front-Cover Texts and with no Back-Cover
% Texts. A copy of the license is included in the section entitled
% "GNU Free Documentation License".
%
% http://www.cybop.net
% - Cybernetics Oriented Programming -
%
% http://www.resmedicinae.org
% - Information in Medicine -
%
% Version: $Revision: 1.1 $ $Date: 2008-08-19 20:41:08 $ $Author: christian $
% Authors: Christian Heller <christian.heller@tuxtax.de>
%

\subsubsection{Serialised Model}
\label{serialised_model_heading}
\index{CYBOL Serialised Model Example}

A possible (but not necessarily recommended) alternative to the linking of
external resources is to store such information (as binary code) \emph{inline}
in the CYBOL knowledge template. One case in which it is necessary to store all
information inline in the model is \emph{Serialisation}.

A CYBOL address management application that does not rely on the existence of a
\emph{Database Management System} (DBMS) probably has to store addresses in
form of serialised files, such as the one shown following. It contains two
parts representing dynamically extensible lists, one for \emph{phone\_numbers}
and another one for \emph{addresses}:

\begin{scriptsize}
    \begin{verbatim}
<model>
    <part name="honorific_prefix" channel="inline" abstraction="string" model="Dr."/>
    <part name="given_name" channel="inline" abstraction="string" model="Tux"/>
    <part name="family_name" channel="inline" abstraction="string" model="Penguin"/>
    <part name="phone_numbers" channel="inline" abstraction="cybol" model="(
        <part name="home" channel="inline" abstraction="string" model="123"/>
        <part name="work" channel="inline" abstraction="string" model="456"/>
        <part name="mobile" channel="inline" abstraction="string" model="789"/>
    )"/>
    <part name="addresses" channel="inline" abstraction="cybol" model="(
        ...
    )"/>
</model>
    \end{verbatim}
\end{scriptsize}

The serialisation of CYBOL models causes one problem: Due to the double
hierarchy (section \ref{double_hierarchy_heading}) to which belong
\emph{Whole-Part} relations (stored in XML attributes) and
\emph{Meta Information} (stored in XML tags), it is not possible to store CYBOL
models in an XML-conform manner. Instead of referencing external files
containing the corresponding CYBOL \emph{Part} models, a serialised
\emph{Whole} model has to contain these inline.

While XML tags were invented as pairs consisting of a \emph{begin} and an
\emph{end} tag, XML attribute values are enclosed by simple quotation marks.
Hence, the beginning markup of an attribute value does not look any different
than its ending markup. This is a true problem, because serialised whole-part
hierarchies of CYBOL models, with attribute values containing complete sub
models with their own attributes, would get completely mixed up in pure XML
notation.

It was therefore inevitable to break XML-conformity and introduce two
additional markup tokens \emph{"(} and \emph{)"}, indicating the beginning and
end of an XML attribute value. The tokens are extensions of the quotation marks
of standard XML attributes, with one left/ right parenthesis, respectively.
That way, the degree to which attributes are nested becomes countable and it is
always clear to which tag an attribute belongs.
