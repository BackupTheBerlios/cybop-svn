%
% $RCSfile: application_server.tex,v $
%
% Copyright (C) 2002-2008. Christian Heller.
%
% Permission is granted to copy, distribute and/or modify this document
% under the terms of the GNU Free Documentation License, Version 1.1 or
% any later version published by the Free Software Foundation; with no
% Invariant Sections, with no Front-Cover Texts and with no Back-Cover
% Texts. A copy of the license is included in the section entitled
% "GNU Free Documentation License".
%
% http://www.cybop.net
% - Cybernetics Oriented Programming -
%
% http://www.resmedicinae.org
% - Information in Medicine -
%
% Version: $Revision: 1.1 $ $Date: 2008-08-19 20:41:05 $ $Author: christian $
% Authors: Christian Heller <christian.heller@tuxtax.de>
%

\section{Application Server}
\label{application_server_heading}
\index{Application Server}
\index{Presentation Clients}
\index{Standalone Systems}
\index{Operating System}
\index{OS}
\index{Layers}
\index{Tier}
\index{1 Tier}
\index{n Tier}
\index{Server Process}
\index{Server}
\index{Client}

One well-known system, nowadays, is the \emph{Application Server}. The name
implies that this system is to \emph{serve} other systems, so-called
\emph{Presentation Clients} (section \ref{presentation_client_heading}). It may
be programmed in languages like \emph{Java}, \emph{Python}, \emph{Smalltalk},
\emph{C++}, \emph{C} or others more.

On the other hand, there are systems running all by themselves, without any
access to/ from another system -- so-called \emph{Standalone Systems}. In
reality, they hardly exist since most applications run in a surrounding
\emph{Operating System} (OS) and are thus not really \emph{alone}. An OS may be
called \emph{standalone} but mostly, even that consists of a number of sub
processes solving background tasks. That is why the name \emph{standalone} is
used when one wants to place emphasis on the system itself, neglecting its
communication with others.

Many kinds of application servers exist. Multiple services are offered by them,
for example storage or persistence handling but also application- and domain
specific functionality. A healthcare environment, as example, may contain several
servers, each fulfilling one task such as person identification, resource access
decision, image access and so on -- just like people in real life have abilities
and professions.

Systems of an IT environment are structured into so-called \emph{Layers},
another name for which is \emph{Tier}. The application server alone represents
a \emph{1 Tier} environment. The more systems of different type (presentation
client, application server, database server) are added to an environment, the
more tiers are added. For that reason, distributed client-server environments
are called \emph{n Tier}.

When people talk about a \emph{Server}, they very often mean a \emph{Computer}
on which a \emph{Server Process} is running. This is neither completely wrong nor
absolutely correct. A computer can run many different processes, only some of
which may be servers. Hence, the computer can act as \emph{Server} but also as
\emph{Client}, at the same time.
