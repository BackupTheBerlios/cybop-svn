%
% $RCSfile: deterministic_and_stochastic_behaviour.tex,v $
%
% Copyright (C) 2002-2008. Christian Heller.
%
% Permission is granted to copy, distribute and/or modify this document
% under the terms of the GNU Free Documentation License, Version 1.1 or
% any later version published by the Free Software Foundation; with no
% Invariant Sections, with no Front-Cover Texts and with no Back-Cover
% Texts. A copy of the license is included in the section entitled
% "GNU Free Documentation License".
%
% http://www.cybop.net
% - Cybernetics Oriented Programming -
%
% http://www.resmedicinae.org
% - Information in Medicine -
%
% Version: $Revision: 1.1 $ $Date: 2008-08-19 20:41:06 $ $Author: christian $
% Authors: Christian Heller <christian.heller@tuxtax.de>
%

\subsubsection{Deterministic- and Stochastic Behaviour}
\label{deterministic_and_stochastic_behaviour_heading}
\index{Deterministic Behaviour}
\index{Stochastic Behaviour}
\index{Probabilistic Behaviour}
\index{Fuzzy Logic}
\index{Artificial Neural Network}
\index{ANN}

Systems can be distinguished by their behaviour, which can be \emph{deterministic}
or \emph{stochastic} (probabilistic). While the elements of the first work in a
predictable way, probabilistic systems are not fully transparent and their
results are only \emph{likely}, but never \emph{certain}. Living systems are
entirely probabilistic, because firstly, not all of their elements are known
and secondly, they always consist of sub systems on different functional
levels. \cite{sengbusch}

Two areas dealing with the simulation of stochastic behaviour are \emph{Fuzzy Logic}
%(section \ref{fuzzy_logic_heading})
and \emph{Artificial Neural Networks} (ANN).
%(section \ref{artificial_neural_network_heading}).
Most software systems
though, need \emph{reliable} (deterministic) behaviour delivering
\emph{predictable} results. Deterministic systems are therefore in the focus of
the research done in this work.
