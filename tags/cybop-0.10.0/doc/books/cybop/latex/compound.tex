%
% $RCSfile: compound.tex,v $
%
% Copyright (C) 2002-2008. Christian Heller.
%
% Permission is granted to copy, distribute and/or modify this document
% under the terms of the GNU Free Documentation License, Version 1.1 or
% any later version published by the Free Software Foundation; with no
% Invariant Sections, with no Front-Cover Texts and with no Back-Cover
% Texts. A copy of the license is included in the section entitled
% "GNU Free Documentation License".
%
% http://www.cybop.net
% - Cybernetics Oriented Programming -
%
% http://www.resmedicinae.org
% - Information in Medicine -
%
% Version: $Revision: 1.1 $ $Date: 2008-08-19 20:41:06 $ $Author: christian $
% Authors: Christian Heller <christian.heller@tuxtax.de>
%

\subsubsection{Compound}
\label{compound_heading}
\index{Compound}
\index{Composition}
\index{Tree}
\index{Hierarchy}
\index{Artificial Intelligence}
\index{AI}
\index{Concept}
\index{Schema}
\index{Monades Theory}
\index{Parent Item}
\index{Child Item}
\index{Whole}
\index{Part}
\index{Container}
\index{Element}
\index{Genealogy}
\index{Unidirectional Relation}
\index{Hierarchical State}
\index{Hierarchical Procedure}
\index{Microcosm}
\index{Macrocosm}
\index{Particle}

\emph{Composition} is the third kind of abstraction that humans use to
understand their environment. It is an important instrument for the human mind
to associate information, that is to acquire, store and recall \emph{Knowledge}.
Every item can be recognised as a \emph{Compound} of smaller items and can
therefore also be called \emph{Tree} or \emph{Hierarchy}. The subject of
\emph{Artificial Intelligence} (AI) talks of \emph{Concept} or \emph{Schema}
\cite{sowa}.

The great philosopher and mathematician Gottfried Wilhelm Leibnitz (1646-1716)
made extensive use of the principle of hierarchy. His entire theory of
\emph{Monades} \cite{leibnitz} is based on it.

In software design, the terms \emph{Parent} and \emph{Child} are often used to
describe both, the items in a composition relation and the items in a
categorisation (inheritance) relationship. To avoid misunderstandings, this
work sticks to the terms \emph{Super} and \emph{Sub} for categorisation and to
the terms \emph{Whole} and \emph{Part} for composition. Yet other terms to
describe items of a composition would be \emph{Container} and \emph{Element}.

One obvious analogy comes from \emph{Genealogy} where parents have children who
become parents of their own children and so forth. Displaying these relations
between family members in a graph, leads to a tree which may represent both,
a category tree (for property inheritance) as well as a composition tree (for
children ownership).

Taking the example of a \emph{Human Being}, one could say that it is composed
of organs such as \emph{Eye}, \emph{Ear}, \emph{Heart}, \emph{Brain},
\emph{Arm} and further, also smaller parts. John F. Sowa \cite[p. 109]{sowa}
writes on this:

\begin{quote}
    From different viewpoints, the human body can be considered an aggregate of
    organs, an aggregate of cells, or an aggregate of molecules. Each viewpoint
    affects the terminology used to talk about the body, but not the body itself.
\end{quote}

That is, one and the same real world object may be represented in many
different ways. It is important to note the \emph{unidirectional} kind of
relations: A human being is composed of organs but an organ is never composed
of a human being! Another example is that of a \emph{Book}: physically, it may
be composed of a \emph{Paperback Cover} and \emph{Paper Pages}; logically,
however, it is usually separated into \emph{Part}, \emph{Chapter},
\emph{Section}, \emph{Paragraph}, \emph{Sentence}, \emph{Word} and
\emph{Character}. Obviously, knowledge representation always depends on what
one wants to express in which context. Philippe Ameline who works in the
\emph{Nautilus-Odyssee} project \cite{nautilus}, writes in \cite{openhealth}:

\begin{quote}
    Lets take a comparison with \emph{Geography}: you can build an ontology
    (which consists of compound structures) in order to describe natural
    objects (mountains, rivers, \ldots). But if you build artificial frontiers
    and call them \emph{Countries}, you cannot semantically include these
    concepts inside the geographical domain. -- That is the very reason why
    human beings, very frustrated, had to invent the political domain ;-)
    \ldots\ I think that there (are similar differences) in medicine (as in)
    the geographical domain.
\end{quote}

Not only \emph{States} may be represented as compound; \emph{Procedures} may be
hierarchical as well. The process \emph{Take Book from Library}, for example,
may have the following structure:

\begin{itemize}
    \item[-] Check Catalogue
    \begin{itemize}
        \item[-] Investigate suitable Books
        \item[-] Note Registration Number
    \end{itemize}
    \item[-] Obtain Book
    \begin{itemize}
        \item[-] Look for Shelf
        \item[-] Take off Book
    \end{itemize}
    \item[-] Borrow Book
\end{itemize}

Returning to human thinking, one realises that in the end, everything in
universe can be put into variable hierarchical models, that is consists of
smaller items and belongs to a bigger item. From the physical point of view,
nobody knows where this hierarchy really stops, towards \emph{Microcosm} as
well as towards \emph{Macrocosm}. There is no \emph{absolute}, \emph{basic}
item. A \emph{Particle} as concept exists only in the human mind, placed
somewhere between micro- and macrocosm, with hypothetic borders.
