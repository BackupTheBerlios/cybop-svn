%
% $RCSfile: archetype.tex,v $
%
% Copyright (C) 2002-2008. Christian Heller.
%
% Permission is granted to copy, distribute and/or modify this document
% under the terms of the GNU Free Documentation License, Version 1.1 or
% any later version published by the Free Software Foundation; with no
% Invariant Sections, with no Front-Cover Texts and with no Back-Cover
% Texts. A copy of the license is included in the section entitled
% "GNU Free Documentation License".
%
% http://www.cybop.net
% - Cybernetics Oriented Programming -
%
% http://www.resmedicinae.org
% - Information in Medicine -
%
% Version: $Revision: 1.1 $ $Date: 2008-08-19 20:41:05 $ $Author: christian $
% Authors: Christian Heller <christian.heller@tuxtax.de>
%

\subsection{Archetype}
\label{archetype_heading}
\index{Archetype}
\index{Electronic Health Record}
\index{EHR}
\index{Good European/ EHR}
\index{GEHR}
\index{Open EHR}
\index{SNOMED CT}
\index{Archetype Definition Language}
\index{ADL}
\index{Constraint Form of ADL}
\index{cADL}
\index{Data Definition Form of ADL}
\index{dADL}
\index{Template Form of ADL}
\index{tADL}
\index{First Order Predicate Logic}
\index{FOPL}

With the aim of providing the means to build usable, maintainable, extensible
\emph{Electronic Health Records} (EHR), the \emph{Archetype} as design concept
was introduced in the \emph{Design Principles} document of the
\emph{Good European/ EHR} (GEHR) project, which was later renamed into
\emph{Open EHR} \cite{openehr}. Their website states: \textit{An archetype is a
re-usable, formal model of a domain concept.} Archetypes adhere to ontological
principles; they can be composed of other archetypes or atomic elements. Their
use is not limited to EHR building, despite OpenEHR's focus on the medical
domain.

Comparing archetypes with terminologies, Beale \cite{openehrtechnical} writes
that a terminology like for example SNOMED Clinical Terms (SNOMED CT) had the
form of a semantic network, i.e. \ldots\ with an ontological flavour. However,
because rigorous design principles were not always applied, they tended to be
internally inconsistent and had a lot of pre-coordination in them, while what
was really needed was a generative/ compositional terminology. Further, SNOMED
could tell what the meanings of the parts of e.g. a complete blood count test
are, but it were not going to provide a model of an actual blood test. This is
where archetypes \ldots\ would come in; they were about information
\emph{in use}, not definitions of reality (as terminologies). \ldots\ So --
even if SNOMED was perfect, it wouldn't do everything. It were a knowledge
support part of the environment, and it could be used to name things and
perform inferencing (\textit{draw a conclusion/ deduction} \cite{websters}).

An \emph{Archetype Definition Language} (ADL) \cite{adl} was created for the
specification of archetypes. A corresponding ADL document has the following
structure:

\begin{scriptsize}
    \begin{verbatim}
    archetype_id = <"some.archetype.id">
    adl_version = <"2.0">
    is_controlled = <True>
    parent_archetype_id = <"some.other.archetype.id">
    concept = <[concept_code]>
    original_language = <"lang">
    translations = <
    ...
    >
    description = <
    ...
    >
    definition = <
    cADL structural section
    >
    invariant = <
    assertions
    >
    ontology = <
    ...
    >
    revision_history = <
    ...
    >
    \end{verbatim}
\end{scriptsize}

Many separate sections can be identified in this archetype structure, and
various syntaxes are used for them. Table \ref{adl_table} gives an overview of
the structural elements of an ADL archetype. In addition to the single
sections, it mentions two further syntaxes, for templates and constraints on
data instances.

\begin{table}[ht]
    \begin{center}
        \begin{footnotesize}
        \begin{tabular}{| p{30mm} | p{30mm} | p{45mm} |}
            \hline
            \textbf{Element} & \textbf{Syntax} & \textbf{Purpose}\\
            \hline
            archetype structure & \emph{Archetype Definition Language} (ADL) & glue syntax\\
            \hline
            definition section & \emph{Constraint Form of ADL} (cADL) &
                constraints definition\\
            \hline
            description, ontology and other sections & \emph{Data Definition Form of ADL} (dADL) &
                data definition\\
            \hline
            template & \emph{Template Form of ADL} (tADL) &
                formalism to compose archetypes into larger constraint structures, used in particular contexts at runtime\\
            \hline
            data instances & \emph{First Order Predicate Logic} (FOPL) &
                constraints on data which are instances of some information model (e.g. expressed in UML)\\
            \hline
        \end{tabular}
        \end{footnotesize}
        \caption{Structural Elements of an ADL-defined Archetype \cite{adl}}
        \label{adl_table}
    \end{center}
\end{table}

Chapter \ref{cybernetics_oriented_language_heading} will define a new language
that is based on just one syntax: the \emph{Extensible Markup Language} (XML)
(section \ref{extensible_markup_language_heading}), an easy-to-grasp pure text
format. Despite its limited vocabulary of just four tags and four attributes,
that language may encode a rich set of knowledge constructs, including meta
information and constraints.
