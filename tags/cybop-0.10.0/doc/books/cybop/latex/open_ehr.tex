%
% $RCSfile: open_ehr.tex,v $
%
% Copyright (C) 2002-2008. Christian Heller.
%
% Permission is granted to copy, distribute and/or modify this document
% under the terms of the GNU Free Documentation License, Version 1.1 or
% any later version published by the Free Software Foundation; with no
% Invariant Sections, with no Front-Cover Texts and with no Back-Cover
% Texts. A copy of the license is included in the section entitled
% "GNU Free Documentation License".
%
% http://www.cybop.net
% - Cybernetics Oriented Programming -
%
% http://www.resmedicinae.org
% - Information in Medicine -
%
% Version: $Revision: 1.1 $ $Date: 2008-08-19 20:41:08 $ $Author: christian $
% Authors: Christian Heller <christian.heller@tuxtax.de>
%

\subsubsection{Open EHR}
\label{open_ehr_heading}
\index{Open Electronic Health Record}
\index{openEHR}
\index{Good European/ Electronic Health Record}
\index{GEHR}
\index{Dual Model Approach}
\index{Archetype}

The \emph{Open Electronic Health Record} (openEHR) \cite{openehr} initiative,
previously called the \emph{Good European/ Electronic Health Record} (GEHR),
arose from a European standardisation effort but is now based in Australia.

Pursueing an idea named the \emph{Dual Model Approach} (section
\ref{dual_model_approach_heading}), which uses a \emph{Meta-Level Architecture}
as described by the \emph{Reflection} pattern (section \ref{reflection_heading}),
it wants to specify so-called \emph{Archetypes} -- formally specified knowledge
templates of requirements for representing and communicating EHR information.

The effort is based on the idea of no-cost open standards and free contribution.
