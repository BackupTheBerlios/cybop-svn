%
% $RCSfile: aspect_oriented_programming.tex,v $
%
% Copyright (C) 2002-2008. Christian Heller.
%
% Permission is granted to copy, distribute and/or modify this document
% under the terms of the GNU Free Documentation License, Version 1.1 or
% any later version published by the Free Software Foundation; with no
% Invariant Sections, with no Front-Cover Texts and with no Back-Cover
% Texts. A copy of the license is included in the section entitled
% "GNU Free Documentation License".
%
% http://www.cybop.net
% - Cybernetics Oriented Programming -
%
% http://www.resmedicinae.org
% - Information in Medicine -
%
% Version: $Revision: 1.1 $ $Date: 2008-08-19 20:41:05 $ $Author: christian $
% Authors: Christian Heller <christian.heller@tuxtax.de>
%

\subsection{Aspect Oriented Programming}
\label{aspect_oriented_programming_heading}
\index{Aspect Oriented Programming}
\index{AOP}
\index{Object Oriented Programming}
\index{OOP}
\index{Concern Interface}
\index{Aspect}
\index{Meta Object Protocol}
\index{MOP}
\index{Mixin Programming Concept}
\index{Crosscutting Concern}
\index{Common Concern}
\index{Aspect Marker Interface}
\index{Component Oriented Programming}
\index{COP}
\index{Development Aspect}
\index{Production Aspect}
\index{Join Point}
\index{Pointcut}
\index{Advice}
\index{Inter-Type Declaration}
\index{Aspect Oriented Software Development}
\index{AOSD}
\index{Aspect Weaver}
\index{Join Point Representation}
\index{Join Point Model}
\index{JPM}
\index{Singleton Pattern}
\index{Lifecycle Method}

Another alternative avoiding redundant code caused by the implementation of
concern interfaces (section \ref{separation_of_concerns_heading}) is the
so-called \emph{Aspect Oriented Programming} (AOP), which is an extension to
\emph{Object Oriented Programming} (OOP). \emph{Aspects} are a possibility to
separate and define concern areas addressed in program code. Wikipedia
\cite{wikipedia} writes that:

\begin{quote}
    Aspects emerged out of object-oriented programming and have functionality
    similar to using a meta-object protocol (section \ref{reflection_heading}).
    Aspects relate closely to programming concepts like subjects, mixins, and
    delegation.
\end{quote}

The Avalon documentation \cite{avalon} means that AOP were: \textit{the next
logical step after separation of concerns}. Many concerns could not be
centrally addressed using the standard mechanisms of programming. Using AOP, it
were possible to do so in a simple fashion.

The AspectJ documentation \cite{aspectj} writes that the motivation for AOP had
been the realisation that there are issues or concerns (like a security policy)
that cut across many of the natural units of modularity of an application and
are not well captured by traditional programming methodologies. For
\emph{Object Oriented Programming} (OOP) languages, the natural unit of
modularity were the \emph{Class}. But in these languages, some concerns were
not easily turned into classes because they'd \emph{cut across} classes. So
these weren't reusable, couldn't be refined or inherited, were spread
throughout the program in an undisciplined way and, in short, were difficult to
work with. AOP were a way of modularising \emph{Crosscutting Concerns} much
like OOP were a way of modularising \emph{Common Concerns}. The later chapter
\ref{statics_and_dynamics_heading} will come back to these two kinds of
concerns in short.

As shown in the previous sections, the \emph{Avalon} project \cite{avalon} uses
concern interfaces (sometimes called \emph{Aspect Marker Interfaces}) and
\emph{Component Oriented Programming} (COP) to define its concerns, what
frequently leads to redundant implementations. Other projects, for instance
\emph{AspectJ} \cite{aspectj} and \emph{AspectWerkz} \cite{aspectwerkz},
provide AOP facilities whose aim is the clean modularisation of crosscutting
concerns such as those in the following list, which AspectJ divides into
\emph{Development Aspects}:

\begin{itemize}
    \item[-] Tracing
    \item[-] Profiling and Logging
    \item[-] Pre- and Post-Conditions
    \item[-] Contract Enforcement
    \item[-] Configuration Management
\end{itemize}

and \emph{Production Aspects}:

\begin{itemize}
    \item[-] Change Monitoring
    \item[-] Context Passing
    \item[-] Providing Consistent Behaviour
\end{itemize}

The AspectJ project as an implementation of AOP for Java adds just one new
concept to that language -- the \emph{Join Point}, which is a well-defined
point in the program flow. A number of new constructs are introduced as well: A
\emph{Pointcut} picks out certain join points and values at those points; an
\emph{Advice} is a piece of code that is executed when a join point is reached.
Both do dynamically affect the program flow. \emph{Inter-Type Declarations}, on
the other hand, statically affect a program's class hierarchy, namely the
members of its classes and the relationships between classes. AspectJ's
\emph{Aspect}, finally, is the unit of modularity for crosscutting concerns. It
behaves somewhat like a Java class, summarising the constructs described
before, that is pointcuts, advices and inter-type declarations.

\emph{Aspect Oriented Software Development} (AOSD) is about developing programs
that rely on AOP principles and languages -- or language extensions,
respectively. Such programs get compiled slightly differently than usual. An
\emph{Aspect Weaver} generates a \emph{Join Point Representation} of the
program, by merging code and aspects. Only afterwards, the program is compiled
into an executable.

Although AOP seems to successfully address the problem of crosscutting concerns,
it also brings with yet another programming paradigm that application developers
have to get familiar with. The new concepts and constructs further complicate
software development. The source code gets further fractured because of AOP's
additional modules. It is harder to follow the program flow and one has less
control on the behaviour of classes, so that the usage of development tools
becomes inevitable. But besides this rather general criticism, there is more
specific ones: The \emph{Join Point Model} (JPM), for example, is reviewed in
\cite{huttenhuis} which points out some unsolved issues in AOP, while
investigating the following properties of join points:

\begin{itemize}
    \item[-] \emph{Granularity:} only some locations in program code are
        suitable as join points
    \item[-] \emph{Encapsulation:} there is no effective way to protect join
        points from aspect-imposed modification
    \item[-] \emph{Semantics:} low-level join point identification leads to
        tight coupling between aspects and join points
    \item[-] \emph{Jumping Aspects:} context-sensitive join points execute
        different aspect code
    \item[-] \emph{Sharing:} dependencies and order of execution are unclear
        when having several pieces of advice at the same join point
\end{itemize}

At least some of the concerns that AOP addresses could be implemented with
lifecycle-techniques as well. A logger, for example, could be created once at
system startup. But instead of accessing it across static methods, as suggested
by the \emph{Singleton} pattern (section \ref{singleton_heading}), or executing
an aspect-oriented \emph{Advice} when a join point is reached, the reference to
the logger instance could simply be forwarded from component to component,
using a special \emph{globalise} lifecycle method, so that each would be able
to access the logger. However, also this solution becomes tedious with a
growing number of objects to be forwarded. The new kind of programming
introduced in part \ref{contribution_heading} of this work therefore suggests
to put general functionality (concerns, aspects) into an interpreter program
acting close to hardware and providing the general functionality to application
systems executed by it.
