%
% $RCSfile: expressiveness.tex,v $
%
% Copyright (C) 2002-2008. Christian Heller.
%
% Permission is granted to copy, distribute and/or modify this document
% under the terms of the GNU Free Documentation License, Version 1.1 or
% any later version published by the Free Software Foundation; with no
% Invariant Sections, with no Front-Cover Texts and with no Back-Cover
% Texts. A copy of the license is included in the section entitled
% "GNU Free Documentation License".
%
% http://www.cybop.net
% - Cybernetics Oriented Programming -
%
% http://www.resmedicinae.org
% - Information in Medicine -
%
% Version: $Revision: 1.1 $ $Date: 2008-08-19 20:41:06 $ $Author: christian $
% Authors: Christian Heller <christian.heller@tuxtax.de>
%

\subsection{Expressiveness}
\label{expressiveness_heading}

- show how the issues criticised in \ref{extensible_markup_language_heading} are addressed
- for example, the semantics is not contained in tags/attributes, but \emph{inherent}
in the hierarchical meta structure of CYBOL

Q: Why doesn't CYBOL use the standard structure as intended by XML,
that is tags for parts and attributes for meta information (properties)?

A1: The parts belonging to one knowledge template may have an unpredictable
number of properties (meta information). If attributes were used to model
these, thousands, may be millions of attribute names of all knowledge templates
ever to be created would have to be known beforehand. But this would break
CYBOL's flexibility and would not be future-proof.

A2: The CYBOP knowledge schema relies on the fixed numbers of four different
tags and four different attributes. Only because these numbers are fixed, it
was possible at all to create a general schema covering all kinds of knowledge.

A3: Properties (meta information) are themselves models that have a name,
channel, abstraction (type) and primitive or compound value. If properties were
put into XML attributes, these information could not be given.

A4: Moreover, it is possible to keep meta information, namely constraints,
about properties, too. This wouldn't be possible with properties as attributes.

- merger of all classes into just one (knowledge schema) in CYBOI
OOP Klassen merge meta information and actual content (part models)
- big disadvantage since information is not expressive (semantic info is missing)
- therefore so many efforts are dealing with this problem (semantic web, ontologies etc.)

==

\cite{w3c}
An XSLT stylesheet specifies the presentation of a class of XML documents by
describing how an instance of the class is transformed into an XML document
that uses a formatting vocabulary, such as (X)HTML or XSL-FO.

XSL is a language for expressing style sheets. An XSL style sheet is, like
with CSS, a file that describes how to display an XML document of a given type.
XSL shares the functionality and is compatible with CSS2 (although it uses a
different syntax). It also adds:
- A transformation language for XML documents: XSLT. Originally intended to
perform complex styling operations, like the generation of tables of contents
and indexes, it is now used as a general purpose XML processing language.
XSLT is thus widely used for purposes other than XSL, like generating HTML web
pages from XML data.
- Advanced styling features, expressed by an XML document type which defines a
set of elements called Formatting Objects, and attributes (in part borrowed from
CSS2 properties and adding more complex ones.

XML for Ontology:
XML is widely predicted to improve the degree of interoperation between commerce
agents on the Internet. Yet XML does not address ontology and provides only a
syntactic representation of knowledge. For this reason, many Internet commerce
initiatives are developing taxonomies to support XML-based interoperation.
These developments mostly focus upon the identification of standard "tags",
and not the underlying ontology. Interoperation is therefore dependent upon
each trading partner agreeing to use particular tag sets and using these
consistently. It is not clear that this strategy will achieve the degree of
interoperation and flexibility sought by Ontology.Org.

>"x++ is the world's first object-oriented language that is entirely
>based on XML's syntactical structure. x++ conforms with the XML version
>1.0 specification..."
>
>  http://xplusplus.sourceforge.net/index.htm

http://avalon.apache.org/sandbox/merlin/meta/
- XML partly used for description/ specification/ configuration of components
- compare meta model to CYBOP!
- also: Intershop
- XML in general used by configuration/ property files of more and more applications

http://www.ontology.org/main/papers/faq.html
How does ontology relate to the eXtensible Markup Language (XML)?

\emph{Extensible Stylesheet Language} (XSL) is a family of recommendations for
defining XML document transformation and presentation. It consists of three parts:
\emph{XSL Transformations} (XSLT), \emph{XML Path Language} (XPath) and
\emph{XSL Formatting Objects} (XSL-FO).

%http://www.oasis-open.org/committees/tc\_home.php?wg_abbrev=cam
%see document /edu/grad/tmp/oasis\_cam\_technical\_brochure.pdf
%for short description (on first page) of XSD, XSLT, XPath, RDF, OWL

FORMATS: XML, DTD, SGML, DOCBOOK, RELAX NG, Schematron, XML Schema, XSLT
%\cite{} [http://xml.ascc.net/resource/schematron/schematron.html]
%The Schematron (description + code example)

Q: How to put required relations into CYBOL models (see \ref{feature_model_heading})?
Example: If the application has GUI capabilities, then use graphical dialogues
to report log messages!
A: It does not make sense to put this constraint into the GUI model.
It should be stored in the user configuration settings instead.
At system startup, a logic model would then normally read the user settings
and set a flag for the logger to use graphical dialogues instead of textual messages.
--> That means, as a recommendation to application developers, that not all
constraints should be put into the \emph{constraint} tag! Rather, the developer
should take some time to bethink, and sort all information where it actually belongs,
leaving the single knowledge models as independent from each other as possible.
