%
% $RCSfile: hidden_patterns.tex,v $
%
% Copyright (C) 2002-2008. Christian Heller.
%
% Permission is granted to copy, distribute and/or modify this document
% under the terms of the GNU Free Documentation License, Version 1.1 or
% any later version published by the Free Software Foundation; with no
% Invariant Sections, with no Front-Cover Texts and with no Back-Cover
% Texts. A copy of the license is included in the section entitled
% "GNU Free Documentation License".
%
% http://www.cybop.net
% - Cybernetics Oriented Programming -
%
% http://www.resmedicinae.org
% - Information in Medicine -
%
% Version: $Revision: 1.1 $ $Date: 2008-08-19 20:41:07 $ $Author: christian $
% Authors: Christian Heller <christian.heller@tuxtax.de>
%

\subsection{Hidden Patterns}
\label{hidden_patterns_heading}
\index{Patterns in CYBOL}
\index{Hidden Patterns in CYBOL}

There are a number of software patterns (section \ref{pattern_heading}) that
may not be obvious (hidden) at first sight, but have been considered in the
design of the CYBOL language.

Most obviously, CYBOL knowledge templates follow the \emph{Composite} pattern,
in a simplified form. All templates represent a compound consisting of part
templates, which leads to a tree-like structure. But this also means that related
patterns (see section \ref{pattern_systematics_heading}) like \emph{Whole-Part}
and \emph{Wrapper} are representable by CYBOL knowledge templates. A template
as whole wraps its parts.

Knowledge templates with similar granularity can be collected in one directory,
in other words one common ontological level. Templates with smaller granularity,
that is those that the more coarse-grained templates consist of, can be placed
in another common layer and so forth. What comes out of it is a system of levels
-- one variant of the \emph{Layers} pattern.
