%
% $RCSfile: odyssee.tex,v $
%
% Copyright (C) 2002-2008. Christian Heller.
%
% Permission is granted to copy, distribute and/or modify this document
% under the terms of the GNU Free Documentation License, Version 1.1 or
% any later version published by the Free Software Foundation; with no
% Invariant Sections, with no Front-Cover Texts and with no Back-Cover
% Texts. A copy of the license is included in the section entitled
% "GNU Free Documentation License".
%
% http://www.cybop.net
% - Cybernetics Oriented Programming -
%
% http://www.resmedicinae.org
% - Information in Medicine -
%
% Version: $Revision: 1.1 $ $Date: 2008-08-19 20:41:07 $ $Author: christian $
% Authors: Christian Heller <christian.heller@tuxtax.de>
%

\subsubsection{Odyssee}
\label{odyssee_heading}
\index{Odyssee}
\index{Logiciel Nautilus}

The \emph{Odyssee} open source project \cite{nautilus} contains a terminology
(\emph{Lexique}) of more than 35,000 (French) terms, each with a code, at the
core of its system. Additionally, it contains a \emph{Semantic Network} of
links between terms of the Lexique, to give sense. Links can be \emph{is a},
\emph{belongs to} or \emph{has unit}. Philippe Ameline writes
\cite{openehrtechnical}:

\begin{quotation}
    In Odyssee, we describe all that we can with trees. If we compare the
    Lexique with medical vocabulary, trees are sentences made of its words.
    Each node of a tree is an object with fields like the Lexique's code,
    complement (to store numbers or external codes), degree of evidence (from
    0=no to 100=certain). Trees can also contain free text sentences \ldots

    In Odyssee, each and every structured document is a tree; you just have to
    look at the Lexique term at its root to know what it is. The whole patient
    record can even be seen as a huge tree with (the) term \emph{Patient} as
    root. Trees can be shown \emph{as is} or, for report generation, be
    translated to natural langage sentences.
\end{quotation}

Scheme: compositional\\
Maintainer: Odyssee \emph{Non-Profit Organisation} (NPO), \emph{Logiciel Nautilus}
