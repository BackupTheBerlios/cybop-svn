%
% $RCSfile: free_publishing.tex,v $
%
% Copyright (C) 2002-2008. Christian Heller.
%
% Permission is granted to copy, distribute and/or modify this document
% under the terms of the GNU Free Documentation License, Version 1.1 or
% any later version published by the Free Software Foundation; with no
% Invariant Sections, with no Front-Cover Texts and with no Back-Cover
% Texts. A copy of the license is included in the section entitled
% "GNU Free Documentation License".
%
% http://www.cybop.net
% - Cybernetics Oriented Programming -
%
% http://www.resmedicinae.org
% - Information in Medicine -
%
% Version: $Revision: 1.1 $ $Date: 2008-08-19 20:41:06 $ $Author: christian $
% Authors: Christian Heller <christian.heller@tuxtax.de>
%

\section*{Free Publishing}
\label{free_publishing_heading}
%\addcontentsline{toc}{section}{Free Publishing}

Reputation in the scientific world strongly depends on the number of
publications in scientific journals, conference proceedings, magazines etc., of
which some have greater kudos, some less. A \emph{Philosophiae Doctor} (PhD)
student, for example, is expected to publish in some of the \emph{acknowledged}
journals, in order to be conferred a doctorate. The grant of project fundings
by local-, national- or \emph{European Union} (EU) governements and sponsorship
of a professor's department at university depend on it as well. Some unfair
practices and shortcomings of the current system of publication shall therefore
be mentioned here. There are at least four disadvantages of publishing in
scientific journals. An author:

\begin{itemize}
    \item is almost always forced to assign his copyright to the publisher;
    \item has very little chance of publishing completely new ideas, since
        evaluators (which are to guarantee a certain \emph{scientific level})
        sieve those which seem too crazy or are unknown to them and do not
        match state-of-the-art science, so that really new ideas can hardly
        become popular in this way;
    \item has to wait many months before being informed about article
        acceptance, sometimes further months to presentation at a conference
        and yet more months until a journal/ proceedings are finally available
        -- which, besides the unfine delay, is enough time for an evaluator to
        adapt the best ideas and publish them in a modified form before;
    \item and everyone else have to pay money for receiving journals (even for
        the one containing the author's own work), or become a member of
        certain scientific societies for some discount -- which means that the
        work is not freely accessible.
\end{itemize}

Further, there is something often labelled \emph{Citation Mafia}. Whether an
article gets published in a journal or not depends on it being accepted by a
number of reviewers (normally three). In order to avoid personal battles, the
article author never gets to know the evaluators' names or proficiency and has
to blindly rely on the \emph{good taste} of a conference's program committee.
However, evaluators, although tied to ethical standards, often seem to have
their list of \emph{friends} or seem to just prefer authors who have already
published elsewhere, leading to circles of scientists citing each other, quite
independent from the quality of their papers. Logically, also here, there are a
number of disadvantages:

\begin{itemize}
    \item Young scientists have a hard life and need a long time for getting
        their articles accepted, independent from how innovative they are.
    \item Mafioso scientists often warm up old stories or deliver well-formulated,
        but rubbish articles not earning the predicate \emph{scientific}.
\end{itemize}

Don't ask for proof -- I don't have it. But almost everybody in the scientific
business knows about these issues. Unfortunately, only few people \cite{jobb}
talk about- or try to change them. Obviously, many scientists prefer to either
play the same old game or are scared of personal disadvantages. However, it
feels like increasingly more researchers, in particular the new generation,
become aware that these drawbacks hinder scientific progress and new solutions
need to be found. Well, there is free online journals such as the
\emph{Journal of Free and Open Source Medical Computing} (JOSMC) \cite{josmc}
or the \emph{BioMed Central} (BMC) \cite{bmc} publisher, where research
articles are: \textit{free to access immediately, peer reviewed,
citation-tracked \ldots}

Although this document cannot deliver solutions to the above-mentioned
problems, it mentioned those to inform the reader and spur further discussion.
Supportive actions in this process would be that:

\begin{itemize}
    \item scientists acknowledge no-cost entry open source conferences like
        LinuxTag \& Co. \cite{linuxtag} as alternatives to traditional ones
    \item professors more readily accept citations of free knowledge sources
        such as Wikipedia \cite{wikipedia} in scientific works of their students
    \item students and scientists publish their works (code and documentation)
        under open source licenses
\end{itemize}
