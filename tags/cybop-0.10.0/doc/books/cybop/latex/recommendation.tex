%
% $RCSfile: recommendation.tex,v $
%
% Copyright (C) 2002-2008. Christian Heller.
%
% Permission is granted to copy, distribute and/or modify this document
% under the terms of the GNU Free Documentation License, Version 1.1 or
% any later version published by the Free Software Foundation; with no
% Invariant Sections, with no Front-Cover Texts and with no Back-Cover
% Texts. A copy of the license is included in the section entitled
% "GNU Free Documentation License".
%
% http://www.cybop.net
% - Cybernetics Oriented Programming -
%
% http://www.resmedicinae.org
% - Information in Medicine -
%
% Version: $Revision: 1.1 $ $Date: 2008-08-19 20:41:08 $ $Author: christian $
% Authors: Christian Heller <christian.heller@tuxtax.de>
%

\subsection{Recommendation}
\label{recommendation_heading}
\index{Recommendation for Pattern Usage}
\index{Itemisation Patterns}
\index{Objectification Patterns}
\index{1:1 Association Patterns}
\index{1:n Association Patterns}
\index{Recursion Patterns}
\index{Unidirectional Relation}
\index{Bidirectionalism Patterns}
\index{Endless Loop through Pattern}
\index{Unpredictable Behaviour through Pattern}
\index{Polymorphism Patterns}
\index{Fragile Base Class Problem}
\index{Grouping Patterns}
\index{Ontology}
\index{Global Access Patterns}

The first category \emph{Itemisation} (objectification) is the base of any
modelling activity and clearly necessary.

The next three categories \emph{1:1 Association}, \emph{1:n Association} and
\emph{Recursion} are special kinds of associations that rely exclusively on
\emph{unidirectional} relations and result in a clean architecture which is why
their usage is strongly recommended.

\emph{Bidirectionalism}, on the other hand, is an \emph{ill} variant of the three
aforementioned categories and should be avoided wherever possible. Patterns in
this category are one reason for endless loops and unpredictable behaviour since
it becomes very difficult to trace the effects that changes in one place of a
system have on others (section \ref{bidirectional_dependency_heading}).

\emph{Polymorphism} is a good thing. It relies on categorisation and due to
inheritance can avoid a tremendous amount of otherwise redundant source code.
However, it also makes understanding a system more difficult, since the whole
architecture must be understood before being able to manipulate code correctly.
Unwanted source code changes caused by inheritance dependencies are often
described with the term \emph{Fragile Base Class Problem} (section
\ref{fragile_base_class_heading}).

\emph{Grouping} models is essential to keep overview in a complex software
system. A very promising technology to support this are \emph{Ontologies}
\cite{hellerkunze}. A lot of thought-work has to go into them but if they are
well thought-out, they are clearly recommended.

The habit of \emph{globally accessing} models is banned since OOP (section
\ref{object_oriented_programming_heading}) became popular. However, it is not
banned completely. Patterns like \emph{Singleton} encapsulate and bundle global
access but they still permit it. They disregard any dependencies and relations
in a system, such are a security risk and reason for untraceable data changes.
This work sees the whole category of \emph{Global Access} as potentially
dangerous and \emph{cannot} recommend its patterns (section
\ref{global_access_heading}).

To sum this up: The different kinds of software patterns investigated in
section \ref{pattern_heading} showed various advantages, but also weaknesses
(bidirectionalism, global access, partly polymorphism), which became obvious
through the new pattern systematics introduced in the previous section. It now
turns out that the weaknesses show up in exactly those categories of patterns,
which do not follow the principles of human thinking. The resulting
recommendations of this section were considered in the design of the
\emph{Cybernetics Oriented Language} (CYBOL) and the
\emph{Cybernetics Oriented Interpreter} (CYBOI), described in the later
chapters \ref{cybernetics_oriented_language_heading} and
\ref{cybernetics_oriented_interpreter_heading}.
