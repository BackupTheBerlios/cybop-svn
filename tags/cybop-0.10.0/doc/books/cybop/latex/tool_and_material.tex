%
% $RCSfile: tool_and_material.tex,v $
%
% Copyright (C) 2002-2008. Christian Heller.
%
% Permission is granted to copy, distribute and/or modify this document
% under the terms of the GNU Free Documentation License, Version 1.1 or
% any later version published by the Free Software Foundation; with no
% Invariant Sections, with no Front-Cover Texts and with no Back-Cover
% Texts. A copy of the license is included in the section entitled
% "GNU Free Documentation License".
%
% http://www.cybop.net
% - Cybernetics Oriented Programming -
%
% http://www.resmedicinae.org
% - Information in Medicine -
%
% Version: $Revision: 1.1 $ $Date: 2008-08-19 20:41:09 $ $Author: christian $
% Authors: Christian Heller <christian.heller@tuxtax.de>
%

\subsection{Tool \& Material}
\label{tool_and_material_heading}
\index{Tools \& Materials Approach}
\index{Domain}
\index{Application}

In software engineering, the term \emph{Domain} stands for a special field of
business in which software systems are applied. Frequently, system development
methods distinguish between data belonging to the \emph{Domain} and
functionality defining the actual \emph{Application} working \emph{on} the
domain. The system family engineering mentioned before is one example.

This view is comparable to the well-known \emph{Tools \& Materials} approach
\cite{tandm} which is based on the distinction of \emph{active} applications
(tools) working on \emph{passive} domain data (material). \textit{Materials can
never be accessed directly, but only by using appropriate tools}, as
\cite{tandm} writes. This simple idea is an important pre-condition for the
separate treatment of \emph{System} and \emph{Knowledge}, as explained in
chapter \ref{statics_and_dynamics_heading} of this work.
