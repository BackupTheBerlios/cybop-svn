%
% $RCSfile: call_for_developers.tex,v $
%
% Copyright (C) 2002-2008. Christian Heller.
%
% Permission is granted to copy, distribute and/or modify this document
% under the terms of the GNU Free Documentation License, Version 1.1 or
% any later version published by the Free Software Foundation; with no
% Invariant Sections, with no Front-Cover Texts and with no Back-Cover
% Texts. A copy of the license is included in the section entitled
% "GNU Free Documentation License".
%
% http://www.cybop.net
% - Cybernetics Oriented Programming -
%
% http://www.resmedicinae.org
% - Information in Medicine -
%
% Version: $Revision: 1.1 $ $Date: 2008-08-19 20:41:05 $ $Author: christian $
% Authors: Christian Heller <christian.heller@tuxtax.de>
%

\section{Call for Developers}
\label{call_for_developers_heading}

\emph{CYBOP}'s concepts want to ease application programming. From now on,
developers can focus on pure domain knowledge, which they encode in form of
CYBOL (XML) models. For this to become possible, a lot of standard functionality
had (and has) to be integrated into the CYBOI interpreter. It does contain (or
will so in future) firstly, all kinds of communication mechanisms and secondly,
more and more hardware control functionality.

For our CYBOP \emph{Free/ Open Source Software} (FOSS) project, we therefore
steadily look for developers with interest in one of the following topics:

\begin{itemize}
    \item[-] \emph{Graphical User Interface} (GUI) Design: Experience with
        toolkits like \emph{Qt}, \emph{GTK}, \emph{wxWindows}, \emph{Tcl/Tk} or
        the like may be helpful, but does CYBOI itself not use these. It
        integrates low-level graphics routines which currently base on the
        \emph{Xlibs} libraries for UNIX' \emph{X Window System} (XFree86). The
        corresponding functionality is still missing for other platforms like
        MS \emph{Windows} or Apple Macintosh \emph{OS X}.
    \item[-] \emph{Textual User Interface} (TUI) Design: Experience with
        libraries such as \emph{ncurses} or \emph{slang} may be helpful. For
        CYBOI, however, low-level \emph{Console}/ \emph{Terminal} programming
        is necessary.
    \item[-] Web User Interface (WUI) Design: In CYBOI, pure CYBOL models get
        translated into pure HTML models. There is no mix-up of HTML with other
        code, as known from \emph{JSP} or \emph{PHP}. Although knowledge of the
        latter and related technologies like \emph{JavaScript} may be helpful,
        these are not used in CYBOI.
    \item[-] Socket Communication: Sockets are essential for system
        communication. They differ slightly between platforms and not all kinds
        have been implemented in CYBOI yet. Some experience with UNIX file-,
        Win- and TCP sockets is needed.
    \item[-] Database Communication: Traditional mechanisms like \emph{ODBC} or
        \emph{JDBC} ease and standardise the communication with database
        systems. It still needs to be figured out whether to use these or
        better to write our own low-level SQL statements in CYBOI.
    \item[-] Data Conversion: Different kinds of communication require
        different data transfer models. The same counts for persistently
        storing data in various file formats. Therefore, a huge number of
        parsers/ serialisers and CYBOL encoders/ decoders will be needed.
        Experience with import/ export filters and file format conversion is
        very welcome.
    \item[-] OS Concepts: CYBOI is the active core system managing passive
        knowledge models and running directly on an \emph{Operating System}
        (OS) or \emph{Hardware}. It already provides signalling and
        prioritising itself and further OS concepts are planned to be
        integrated, wherever useful. Help in this would be appreciated.
\end{itemize}

If you think you might like to work on one of these topics, or just want to try
out and learn by doing, or have further questions, just check out our website
\emph{http://www.cybop.net} or contact one of the mailing lists mentioned there!
