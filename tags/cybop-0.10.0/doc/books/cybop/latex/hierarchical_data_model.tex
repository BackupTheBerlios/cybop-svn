%
% $RCSfile: hierarchical_data_model.tex,v $
%
% Copyright (C) 2002-2008. Christian Heller.
%
% Permission is granted to copy, distribute and/or modify this document
% under the terms of the GNU Free Documentation License, Version 1.1 or
% any later version published by the Free Software Foundation; with no
% Invariant Sections, with no Front-Cover Texts and with no Back-Cover
% Texts. A copy of the license is included in the section entitled
% "GNU Free Documentation License".
%
% http://www.cybop.net
% - Cybernetics Oriented Programming -
%
% http://www.resmedicinae.org
% - Information in Medicine -
%
% Version: $Revision: 1.1 $ $Date: 2008-08-19 20:41:07 $ $Author: christian $
% Authors: Christian Heller <christian.heller@tuxtax.de>
%

\subsubsection{Hierarchical Data Model}
\label{hierarchical_data_model_heading}

One of the first models to structure domain data was a simple \emph{Hierarchy},
being used in \emph{Hierarchical Databases} such as ... (VSAM),
in the 1960s and early 1970s. Many of them are still running now,
for instance in insurance companies who have not yet migrated their
systems to modern technologies.

Knowledge Engineering Systems make use of hierarchical data, today.

- Domain Engineering in general, vertical and horizontal separation
- see \cite{inpulse} paper

The following techniques of domain modelling are neither sorted historically,
nor after the SEP phase they are used in (analysis, design, implementation).
The order of their appearance is determined by similarities they have. New
techniques are added stepwise, in a didactic manner, one building on the
other.

The feature modelling, for example, is a domain analysis- and not an
implementation technique but mentioned here because it is based on a special
technique of abstraction worth considering. Moreover it is not exclusively used
for analysis, but also it represents the beginning of design in a software
engineering process, as mentioned in section \ref{abstraction_gaps_heading}.

- add Feature Model to domain modelling, because it is a hierarchical model
