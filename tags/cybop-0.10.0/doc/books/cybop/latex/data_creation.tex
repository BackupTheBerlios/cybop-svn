%
% $RCSfile: data_creation.tex,v $
%
% Copyright (C) 2002-2008. Christian Heller.
%
% Permission is granted to copy, distribute and/or modify this document
% under the terms of the GNU Free Documentation License, Version 1.1 or
% any later version published by the Free Software Foundation; with no
% Invariant Sections, with no Front-Cover Texts and with no Back-Cover
% Texts. A copy of the license is included in the section entitled
% "GNU Free Documentation License".
%
% http://www.cybop.net
% - Cybernetics Oriented Programming -
%
% http://www.resmedicinae.org
% - Information in Medicine -
%
% Version: $Revision: 1.1 $ $Date: 2008-08-19 20:41:06 $ $Author: christian $
% Authors: Christian Heller <christian.heller@tuxtax.de>
%

\subsection{Data Creation}
\label{data_creation_heading}
\index{CYBOI Data Creation}
\index{CYBOI Data Encapsulation}
\index{Compound Structure as Multi Dimensional Container}
\index{Knowledge Memory consisting of Compound- and Primitive Models}

The functionality of CYBOI as system is built around the manipulation of states
in memory. The question how states, representing data, are stored is therefore
of great importance. Besides containers like the \emph{Internals Memory},
\emph{Knowledge Memory} and \emph{Signal Memory}, belonging to its
infrastructure, CYBOI uses special structures encapsulating primitive data. Not
only types like \emph{Character}, \emph{Integer}, \emph{Float} or \emph{String}
are wrapped this way, also \emph{Operations} are. Any compositions of these are
stored as \emph{Compound}.

Encapsulated primitives have the advantage of being forwardable as reference
(memory address pointer), instead of as copy. This ensures that redundant data
are avoided and states of manipulated primitives are not lost. \emph{Logic}
operations are stored in form of a string indicating their name. Necessary
references to input/ output (i/o) \emph{State} models need to be provided as
meta information, by the compound model surrounding the operation.

The \emph{Compound}, as most important CYBOI structure capable of representing
state- as well as logic knowledge, and capable of emulating a map, collection,
list and tree, deserves closer inspection. Essentially, it is a container able
to recursively reference instances of its own, thus spanning up a tree of parent
nodes (\emph{Whole} models) that may have child nodes (\emph{Part} models). In
fulfillment of the requirements of the knowledge schema introduced in section
\ref{knowledge_representation_heading}, a compound consists of a combination of
many arrays containing a part's:

\begin{itemize}
    \item[-] \emph{Name:} serving as unique \emph{Identifier} (ID)
    \item[-] \emph{Model:} the actual contents (may be a part node)
    \item[-] \emph{Abstraction:} the kind of abstraction (type) of the model
    \item[-] \emph{Details:} further meta information (properties, constraints)
\end{itemize}

One may call the \emph{Compound} a \emph{multi-dimensional} container but it is
probably easier described as large table with many columns, whereby the values
of one row describe exactly one part model, and thus belong together.

The previously mentioned \emph{Knowledge Memory} may be seen as huge tree
consisting of compound- as well as primitive models. Its root is always a
\emph{Compound} -- it, so to say, concentrates all knowledge in just one point,
the single concepts being branches of it.

The essential procedures for managing data in memory are \emph{create} and
\emph{destroy} (\emph{creator} modules). Three additional kinds of procedures
are provided for compound- and other container-like structures: \emph{set},
\emph{get} and \emph{remove} (\emph{accessor} modules). It should be noted at
this point that CYBOL applications have no direct access to these procedures,
so that \emph{wild} memory allocation is not possible. A knowledge model can
only be created using the corresponding CYBOL operation \emph{create\_part}.
