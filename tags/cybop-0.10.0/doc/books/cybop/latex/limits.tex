%
% $RCSfile: limits.tex,v $
%
% Copyright (C) 2002-2008. Christian Heller.
%
% Permission is granted to copy, distribute and/or modify this document
% under the terms of the GNU Free Documentation License, Version 1.1 or
% any later version published by the Free Software Foundation; with no
% Invariant Sections, with no Front-Cover Texts and with no Back-Cover
% Texts. A copy of the license is included in the section entitled
% "GNU Free Documentation License".
%
% http://www.cybop.net
% - Cybernetics Oriented Programming -
%
% http://www.resmedicinae.org
% - Information in Medicine -
%
% Version: $Revision: 1.1 $ $Date: 2008-08-19 20:41:07 $ $Author: christian $
% Authors: Christian Heller <christian.heller@tuxtax.de>
%

\section{Limits}
\label{limits_heading}
\index{CYBOP Limits}

Naturally, there are \emph{Limits} to CYBOP. For instance, it:

\begin{itemize}
    \item[-] does not claim to be \emph{the} approach for all kinds of
        programming problems, although it thinks to contribute suitable
        concepts for at least standard business application development.
        However, its usability for hardware-close systems with Real Time (RT)
        requirements, or for control engineering is questionnable and yet to be
        investigated;
    \item[-] depends on the existence of a system with knowledge-processing
        capabilities, which current \emph{Operating Systems} (OS) are not. The
        CYBOI delivered with it is quite mature, but still lacks functionality
        like different \emph{User Interfaces} (UI), various \emph{import/ export}
        (i/e) filters/ translators, better error handling, prioritising and
        further OS features. Only functionality already implemented in CYBOI can
        also be used in CYBOL. But because CYBOI is free software, continuously
        developed in an open project, new features shall be implementable shortly;
    \item[-] has no type-checking features like classical compilers. This is
        the cost of flexibility. The knowledge schema is the only type
        structure provided by CYBOI. All domain- and application knowledge is
        hold externally, in CYBOL knowledge templates, and interpreted only at
        runtime;
    \item[-] will have performance problems when using UI models, especially
        graphical ones, because these are sent in complete to the graphics
        adapter card, whenever a minor change is made. Techniques have to be
        found, that update only clips of a UI model, in graphics memory. The
        difficulty herein is that CYBOL application knowledge has \emph{no}
        direct access to system-level functionality;
    \item[-] does not eliminate all abstraction gaps in a SEP. Requirements
        described informally by an analysis document have to be mapped to CYBOL
        knowledge templates, which then represent the application to be created.
        Although analysts and experts may create CYBOL models right from the
        project start, there will probably never be a true replacement for the
        written requirements analysis document, as one form of abstraction.
        However, if not the informal descriptions of its models, the document
        itself may be created in CYBOL, since it represents knowledge.
\end{itemize}

%\input{persistency}
%Ein Haupt-Denkproblem habe ich immer noch mit persistent gemachten
%Laufzeit-Modellen. Z.B. koennte man in einer Personalverwaltung die
%personenbezogenen Daten in einer CYBOL Datei speichern und die
%Adressdaten in einer anderen CYBOL Datei, aehnlich, wie man Datensaetze
%in verschiedenen Tabellen einer relationalen DB ablegt. Jede CYBOL Datei
%wuerde als Namen eine eindeutige ID bekommen und im Datensatz einer
%Person wuerde als Adressfeld lediglich die ID der Adresse als Verweis
%stehen. Doch wie sage ich einem System, welche Daten es zusammen, und
%welche in getrennten Dateien speichern soll?
