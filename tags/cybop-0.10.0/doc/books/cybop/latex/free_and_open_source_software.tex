%
% $RCSfile: free_and_open_source_software.tex,v $
%
% Copyright (C) 2002-2008. Christian Heller.
%
% Permission is granted to copy, distribute and/or modify this document
% under the terms of the GNU Free Documentation License, Version 1.1 or
% any later version published by the Free Software Foundation; with no
% Invariant Sections, with no Front-Cover Texts and with no Back-Cover
% Texts. A copy of the license is included in the section entitled
% "GNU Free Documentation License".
%
% http://www.cybop.net
% - Cybernetics Oriented Programming -
%
% http://www.resmedicinae.org
% - Information in Medicine -
%
% Version: $Revision: 1.1 $ $Date: 2008-08-19 20:41:06 $ $Author: christian $
% Authors: Christian Heller <christian.heller@tuxtax.de>
%

\subsection{Free and Open Source Software}
\label{free_and_open_source_software_heading}
\index{Free and Open Source Software}
\index{FOSS}
\index{Free/ Libre Open Source Software}
\index{FLOSS}
\index{General Public License}
\index{GPL}
\index{Free Documentation License}
\index{FDL}
\index{Open Source Software}
\index{OSS}

Just like CYBOP (including CYBOL and CYBOI) \cite{cybop}, \emph{Res Medicinae}
\cite{resmedicinae} is developed within a \emph{Free/ Libre Open Source Software}
(FLOSS) project. Its source code, resources and documentation are placed under
GNU's \emph{General Public License} (GPL) (section
\ref{gnu_general_public_license_heading}) and \emph{Free Documentation License}
(FDL) (section \ref{gnu_free_documentation_license_heading}), respectively.
That means they can be freely redistributed and modified under the terms of
these licences. Although distributed in the hope that they will be useful, the
program and its resources come \emph{without any warranty}, without even the
implied warranty of \emph{merchantability or fitness for a particular purpose}.
See \cite{gnulicences} for details.

More information on \emph{Open Source Software} (OSS) in general can be found
at \cite{opensource}. There are plenty of resources for further background
reading, a German one being the \emph{Open Source Jahrbuch 2004}
\cite{opensourcejahrbuch2004}. To what concerns FLOSS in the medical arena,
many other projects exist. Comprehensive lists of these can be found at
\cite{euspirit, linuxmednews, medhowto}.
