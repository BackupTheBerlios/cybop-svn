%
% $RCSfile: migration_to_cybol.tex,v $
%
% Copyright (C) 2002-2008. Christian Heller.
%
% Permission is granted to copy, distribute and/or modify this document
% under the terms of the GNU Free Documentation License, Version 1.1 or
% any later version published by the Free Software Foundation; with no
% Invariant Sections, with no Front-Cover Texts and with no Back-Cover
% Texts. A copy of the license is included in the section entitled
% "GNU Free Documentation License".
%
% http://www.cybop.net
% - Cybernetics Oriented Programming -
%
% http://www.resmedicinae.org
% - Information in Medicine -
%
% Version: $Revision: 1.1 $ $Date: 2008-08-19 20:41:07 $ $Author: christian $
% Authors: Christian Heller <christian.heller@tuxtax.de>
%

\section{Migration to CYBOL}
\label{migration_to_cybol_heading}

Developers who have developed an interest in the \emph{CYBOP} concepts may
eventually want to switch their systems to the \emph{CYBOL} programming language.
This section tries to provide these developers with the necessary steps
(something like a \emph{Hands-on Guide}) that allow a (more or less) careful
\emph{Migration}. \emph{Careful} means that after each step, one should have a
running system again, before taking the next step.

The guidelines are especially suited for developers with experience in object
oriented techniques, ideally programming in Java. Some knowledge in component
oriented programming (such as lifecycle methods) will be helpful in understanding
and implementing the following recommendations.

\paragraph{Caution!}
This migration guide is ad hoc and has not been tested. Use at your own risk!

\begin{itemize}
    \item[-] Use solely one single \emph{Exception} class. Eliminate any other
        special exception classes.
    \item[-] Make sure that instances are only created by special \emph{create}
        methods, one for every attribute. Eliminate all wild calls of the \emph{new}
        operator that are spread over the code.
    \item[-] Create four methods for each attribute (here called \emph{sample}):
        \emph{createSample}, \emph{destroySample}, \emph{setSample} and
        \emph{getSample}.
    \item[-] Collect all \emph{create} methods that are called during system
        startup and put them into a method \emph{initialise} that is called at
        startup.
    \item[-] Introduce a method \emph{finalise} as counterpart to \emph{initialise}
        and call the \emph{destroy} method of every attribute there. This
        \emph{finalise} method should be called at system shutdown.
    \item[-] Insert a new \emph{categorise} method which is called before
        \emph{initialise}, at system startup. Move all configuration code there.
    \item[-] Insert a method \emph{decategorise} as counterpart to \emph{categorise}.
        It should be called at system shutdown, after the \emph{finalise} method.
        Save (write) any configuration settings there.
    \item[-] Introduce your own top-most framework class \emph{Item} that all
        other classes inherit from. This is somewhat difficult. Several classes
        in a Java application need to inherit from standard JDK classes. For now,
        only let those classes inherit from \emph{Item} which currently have
        no (that is the \emph{java.lang.Object}) or an own class as parent.
    \item[-] Add the lifecycle methods \emph{categorise}, \emph{decategorise},
        \emph{initialise} and \emph{finalise} to the \emph{Item} class.
        All subclasses which use any of these methods should also call the
        superclass' implementation of the used method.
    \item[-] Build a general \emph{create} method capable of creating instances
        from a class name that was handed over as string parameter. This method
        should determine a class with \texttt{Class c = Class.forName(classnameAsString)}.
        The instance will then be created using \texttt{c.newInstance()}.
    \item[-] Try to fix certain classes (ideally only one called \emph{Controller})
        whose task it will be to catch events. Let this class implement all event
        interfaces used by the system and also implement the necessary event
        handling methods, enforced by the interfaces.
    \item[-] Add a method \emph{handle} to the \emph{Controller} class which is
        called by all other handling methods. It receives an event as parameter
        and contains simple \emph{if-then} comparisons to filter out the single
        events and call a method which finally processes the event.
    \item[-] Move as much method (\emph{not} lifecycle method) functionality as
        possible into the \emph{Controller} class.
    \item[-] Structure all other (domain) model classes \emph{hierarchically}.
        Make sure only \emph{unidirectional} dependencies exist among them.
    \item[-] Let the \emph{Item} class encapsulate two additional attributes of
        type \emph{java.lang.Object} and \emph{javax.swing.MutableTreeNode}.
        Add the corresponding \emph{set} and \emph{get} methods.
    \item[-] Implement or use an \emph{XML Handler} for reading and writing
        CYBOL/ XML files.
    \item[-] Move hierarchical domain model classes into \emph{CYBOL} code.
        The java code is responsible for reading, altering and writing
        \emph{CYBOL} files.
    \item[-] Now structure methods hierarchically as well and move them into
        \emph{CYBOL} code.
    \item[-] It is now probably easier to use the \emph{CYBOI} interpreter
        (written in C) than to further maintain your own interpreter (written
        in Java).
    \item[-] That's it. Corrections, hints and improvements are very welcome!
\end{itemize}
