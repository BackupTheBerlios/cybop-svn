%
% $RCSfile: inversion_of_control.tex,v $
%
% Copyright (C) 2002-2008. Christian Heller.
%
% Permission is granted to copy, distribute and/or modify this document
% under the terms of the GNU Free Documentation License, Version 1.1 or
% any later version published by the Free Software Foundation; with no
% Invariant Sections, with no Front-Cover Texts and with no Back-Cover
% Texts. A copy of the license is included in the section entitled
% "GNU Free Documentation License".
%
% http://www.cybop.net
% - Cybernetics Oriented Programming -
%
% http://www.resmedicinae.org
% - Information in Medicine -
%
% Version: $Revision: 1.1 $ $Date: 2008-08-19 20:41:07 $ $Author: christian $
% Authors: Christian Heller <christian.heller@tuxtax.de>
%

\subsection{Inversion of Control}
\label{inversion_of_control_heading}
\index{Inversion of Control Pattern}
\index{IoC}
\index{Component}
\index{Application Programming Interface}
\index{API}
\index{Framework}
\index{Observer Pattern}
\index{Chain of Responsibility Pattern}
\index{Bidirectional Dependency}
\index{Strong Coupling}
\index{Dependency Injection}
\index{Unidirectional Dependency}
\index{Security by Design}
\index{Container}
\index{Component Isolation}
\index{Component Lifecycle}

The \emph{Avalon} documentation \cite{avalon} defines a \emph{Component} as
\emph{a passive entity that performs a specific role}. According to this
definition, a passive entity has to employ a \emph{passive}
\emph{Application Programming Interface} (API) which, after \cite{avalon}, were
one that is acted upon, versus one that acts itself.

As could be seen in section \ref{framework_heading}, it is often desirable to
let a framework play the role of the main program coordinating and sequencing
events and application activity. \emph{Observers} (section \ref{observer_heading})
are applied to realise a callback mechanism; a \emph{Chain of Responsibility}
(section \ref{chain_of_responsibility_heading}) is often set up among objects
so that they can react to certain messages in a delegation hierarchy. Both
patterns rely on bidirectional dependencies whose negative effects such as
\emph{Strong Coupling} were mentioned before (section
\ref{bidirectional_dependency_heading}).

The \emph{Inversion of Control} (IoC) pattern \cite{avalon}, in a recent
article of Martin Fowler also called \emph{Dependency Injection}
\cite{fowlerioc}, shows a way out. It is mentioned here (and was not in the
pattern section \ref{pattern_heading}) because of its importance to COP. The
pattern refers to a major semantic detail: a parent object controlling child
objects (components) through their passive API, but \emph{not} vice-versa,
which results in solely \emph{unidirectional} dependencies. As a side-effect,
this principle enforces \emph{Security by Design} in that the flow of control
(object access) is always directed \emph{top-to-bottom}. A parent instantiates
its child components, hands over important parameters (which were configured by
the parent), and then calls the component's methods. Components are not
autonomous. They have no state apart from the parent and also have no way of
obtaining a reference to implementations of parent parameters without the
parent giving them the implementation they need.

Parent objects that have the ability to host child objects are often called
\emph{Container}. A container can provide a passive API itself which allows yet
other containers to control that container. A hierarchical container example
can be found in the \emph{Pico Container} project \cite{picocontainer}. The
container manages things like loading of configuration files, resolution of
dependencies, component handling and isolation, and lifecycle support.
