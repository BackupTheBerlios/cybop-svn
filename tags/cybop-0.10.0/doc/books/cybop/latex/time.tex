%
% $RCSfile: time.tex,v $
%
% Copyright (C) 2002-2008. Christian Heller.
%
% Permission is granted to copy, distribute and/or modify this document
% under the terms of the GNU Free Documentation License, Version 1.1 or
% any later version published by the Free Software Foundation; with no
% Invariant Sections, with no Front-Cover Texts and with no Back-Cover
% Texts. A copy of the license is included in the section entitled
% "GNU Free Documentation License".
%
% http://www.cybop.net
% - Cybernetics Oriented Programming -
%
% http://www.resmedicinae.org
% - Information in Medicine -
%
% Version: $Revision: 1.1 $ $Date: 2008-08-19 20:41:09 $ $Author: christian $
% Authors: Christian Heller <christian.heller@tuxtax.de>
%

\subsubsection{Time}
\label{time_heading}
\index{Time}
\index{Process}
\index{Sub Process}
\index{Order of Sub Processes}
\index{Algorithm}
\index{Occurence of Sub Processes}
\index{Duration of Part Processes}

A third kind of conceptual interaction that humans use to place themselves and
the environment into their very own model of the universe is \emph{Time}. Section
\ref{compound_heading} showed on the example of \emph{Take Book from Library}
that any \emph{Process} can be split into \emph{Sub Processes} and thus
represents a structure with \emph{Hierarchical Character}.

In most cases, the \emph{Order} in which sub processes are executed, is very
important. Without it, no meaningful \emph{Algorithm} could ever be created.
A process thus needs to know about the \emph{Occurrence} of its sub processes
and this sequence information is usually stored in units of time.

Moreover, the \emph{Whole} process sets a time frame that all \emph{Part}
processes, in sum, cannot exceed. Their \emph{Duration} is limited. Again,
process and sub processes have some kind of conceptual relation; in this case
over time.
