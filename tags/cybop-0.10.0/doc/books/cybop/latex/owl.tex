%
% $RCSfile: owl.tex,v $
%
% Copyright (C) 2002-2008. Christian Heller.
%
% Permission is granted to copy, distribute and/or modify this document
% under the terms of the GNU Free Documentation License, Version 1.1 or
% any later version published by the Free Software Foundation; with no
% Invariant Sections, with no Front-Cover Texts and with no Back-Cover
% Texts. A copy of the license is included in the section entitled
% "GNU Free Documentation License".
%
% http://www.cybop.net
% - Cybernetics Oriented Programming -
%
% http://www.resmedicinae.org
% - Information in Medicine -
%
% Version: $Revision: 1.1 $ $Date: 2008-08-19 20:41:08 $ $Author: christian $
% Authors: Christian Heller <christian.heller@tuxtax.de>
%

\subsection{OWL}
\label{owl_heading}
\index{Web Ontology Language}
\index{OWL}
\index{CYBOL-OWL Comparison}

Since the \emph{Web Ontology Language} (OWL) described in section
\ref{web_ontology_language_heading} is a vocabulary extension to RDF, the
points explained in the context of RDF before do count for OWL as well. Further
considerations are done using the following OWL code example representing an
\emph{incomplete} extract of a description of \emph{Wine} (potable liquid)
\cite{owlguide}:

\begin{scriptsize}
    \begin{verbatim}
<rdf:RDF ...
    <owl:Class rdf:ID="Wine">
        <rdfs:subClassOf rdf:resource="&food;PotableLiquid"/>
        <rdfs:label xml:lang="en">wine</rdfs:label>
        <rdfs:label xml:lang="fr">vin</rdfs:label>
        ...
    </owl:Class>
    ...
    <owl:Class rdf:ID="WineColour">
        <rdfs:subClassOf rdf:resource="#WineDescriptor"/>
        <owl:oneOf rdf:parseType="Collection">
            <owl:Thing rdf:about="#White"/>
            <owl:Thing rdf:about="#Rose"/>
            <owl:Thing rdf:about="#Red"/>
        </owl:oneOf>
    </owl:Class>
    ...
    <owl:Class rdf:ID="Vintage">
        <rdfs:subClassOf>
            <owl:Restriction>
                <owl:onProperty rdf:resource="#vintageOf"/>
                <owl:minCardinality rdf:datatype="&xsd;nonNegativeInteger">
                    1
                </owl:minCardinality>
            </owl:Restriction>
        </rdfs:subClassOf>
    </owl:Class>
</rdf:RDF>
    \end{verbatim}
\end{scriptsize}

A class \emph{Wine} inheriting from a super class \emph{PotableLiquid} is
defined first. Labels in several languages are given. Further, two classes
\emph{WineColour} and \emph{Vintage} are defined. Properties referring to the
\emph{WineColour} class may take on one out of a collection of three colour
values. The \emph{Vintage} class has the restriction that at least one property
\emph{vintageOf} must exist. If the \emph{Wine} class or one of its sub classes
is connected to the \emph{WineColour} or \emph{Vintage} class (what is not
shown here to keep the code example simple), it has to consider their possible
restrictions.

An approximated CYBOL pendant to the OWL example above is shown following:

\begin{scriptsize}
    \begin{verbatim}
<model>
    <part name="wine" channel="http" abstraction="cybol" model="example">
        <property name="super" channel="file" abstraction="cybol" model="food/potable.cybol"/>
        <property name="label" channel="file" abstraction="cybol" model="terminology/wine.cybol"/>
        <property name="colour" channel="inline" abstraction="string" model="">
            <constraint name="choice" channel="inline" abstraction="string" model="red,white"/>
        </property>
        <property name="vintage" channel="inline" abstraction="integer" model="">
            <constraint name="requirement" channel="inline" abstraction="boolean" model="true"/>
        </property>
    </part>
</model>
    \end{verbatim}
\end{scriptsize}

\emph{Wine} as one part of a greater model (such as a catalogue) is described.
Its properties are equivalent to those in the OWL example above. The values of
the \emph{colour} property are constrained to a choice of two colours, and the
\emph{vintage} property is required to exist, i.e. it is not deletable.
