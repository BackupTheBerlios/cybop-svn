%
% $RCSfile: hardware_description_language.tex,v $
%
% Copyright (C) 2002-2008. Christian Heller.
%
% Permission is granted to copy, distribute and/or modify this document
% under the terms of the GNU Free Documentation License, Version 1.1 or
% any later version published by the Free Software Foundation; with no
% Invariant Sections, with no Front-Cover Texts and with no Back-Cover
% Texts. A copy of the license is included in the section entitled
% "GNU Free Documentation License".
%
% http://www.cybop.net
% - Cybernetics Oriented Programming -
%
% http://www.resmedicinae.org
% - Information in Medicine -
%
% Version: $Revision: 1.1 $ $Date: 2008-08-19 20:41:07 $ $Author: christian $
% Authors: Christian Heller <christian.heller@tuxtax.de>
%

\subsection{Hardware Description Language}
\label{hardware_description_language_heading}
\index{Hardware Description Language}
\index{HDL}
\index{Electronic Circuit}
\index{Application Specific Integrated Circuit}
\index{ASIC}
\index{Field Programmable Gate Array}
\index{FPGA}
\index{Very High Speed Integrated Circuit}
\index{VHSIC}
\index{VHDL}
\index{Verilog HDL}
\index{SystemC}
\index{Electronic Data Interchange Format}
\index{EDIF}
\index{Joint Electron Device Engineering Council}
\index{JEDEC}
\index{Programmable Logic Device}
\index{PLD}

\emph{Hardware Description Language} (HDL) is an umbrella term for any computer
language formally describing \emph{Electronic Circuits}, that is their design and
operation, as well as tests to verify their operation by means of \emph{Simulation}
\cite{wikipedia}. HDLs used for the design of digital circuits like
\emph{Application Specific Integrated Circuits} (ASIC) or
\emph{Field Programmable Gate Arrays} (FPGA) include:

\begin{itemize}
    \item[-] \emph{Very High Speed Integrated Circuit} (VHSIC) HDL (VHDL) \cite[standard 1164]{ieee}
    \item[-] \emph{Verilog HDL} \cite[standard 1364-2001]{ieee}
    \item[-] \emph{SystemC} \cite{doulos}
\end{itemize}

Although being similar, HDLs are not programming languages. \textit{HDL's syntax
and semantics include explicit notations for expressing time and concurrency which
are the primary attributes of hardware}, as \cite{wikipedia} writes and adds:

\begin{quote}
    An HDL compiler often works in several stages, first producing a logic
    description file in a proprietary format, then converting that to a logic
    description file in the industry-standard
    \emph{Electronic Data Interchange Format} (EDIF), then converting that to a
    \emph{Joint Electron Device Engineering Council} (JEDEC) format file. The
    JEDEC file contains instructions to a \emph{Programmable Logic Device} (PLD)
    programmer for building logic. On the other hand, a software (programming
    language) compiler generates instructions to a microprocessor for moving data.
\end{quote}

The following sections of this chapter will be about programming-, not hardware
description concepts.
