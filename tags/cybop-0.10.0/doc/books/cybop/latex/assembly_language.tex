%
% $RCSfile: assembly_language.tex,v $
%
% Copyright (C) 2002-2008. Christian Heller.
%
% Permission is granted to copy, distribute and/or modify this document
% under the terms of the GNU Free Documentation License, Version 1.1 or
% any later version published by the Free Software Foundation; with no
% Invariant Sections, with no Front-Cover Texts and with no Back-Cover
% Texts. A copy of the license is included in the section entitled
% "GNU Free Documentation License".
%
% http://www.cybop.net
% - Cybernetics Oriented Programming -
%
% http://www.resmedicinae.org
% - Information in Medicine -
%
% Version: $Revision: 1.1 $ $Date: 2008-08-19 20:41:05 $ $Author: christian $
% Authors: Christian Heller <christian.heller@tuxtax.de>
%

\subsection{Assembly Language}
\label{assembly_language_heading}
\index{Assembly Language}
\index{Numeric Languages}
\index{Program Keywords, Symbols, Abbreviations}
\index{System Programmer}
\index{Application Programmer}
\index{Interpreter}
\index{Higher Level Languages}
\index{Lower Level Instructions}
\index{Translation}
\index{Assembler}
\index{Compiler}
\index{Byte Code}

Languages of the layers described to here are numeric. That is, programs
written in them consist of long numerical series adapted to what a machine
expects. Starting with the level of \emph{Assembly Language}, programs contain
special \emph{Keywords}, symbols and abbreviations which are meaningful to
humans. While programs of the former levels are written by
\emph{System Programmers}, it is \emph{Application Programmers} who use
assembly- and higher-level languages to write a program.

Instructions of lower levels are always interpreted. The corresponding program
is called \emph{Interpreter}. It is running on the level below the one the
instructions stem from. An interpreter executes an instruction directly,
without generating a translated program. Higher-level languages, on the other
hand, get translated into lower-level instructions before being executed. Such
translator programs are called \emph{Assembler} or \emph{Compiler}. New forms
of programs (like those written in Java) also use a combination of both, being
first compiled into a special byte code and then interpreted at runtime.
