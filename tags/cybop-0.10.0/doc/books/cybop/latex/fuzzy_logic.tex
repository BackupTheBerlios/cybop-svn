%
% $RCSfile: fuzzy_logic.tex,v $
%
% Copyright (C) 2002-2008. Christian Heller.
%
% Permission is granted to copy, distribute and/or modify this document
% under the terms of the GNU Free Documentation License, Version 1.1 or
% any later version published by the Free Software Foundation; with no
% Invariant Sections, with no Front-Cover Texts and with no Back-Cover
% Texts. A copy of the license is included in the section entitled
% "GNU Free Documentation License".
%
% http://www.cybop.net
% - Cybernetics Oriented Programming -
%
% http://www.resmedicinae.org
% - Information in Medicine -
%
% Version: $Revision: 1.1 $ $Date: 2008-08-19 20:41:06 $ $Author: christian $
% Authors: Christian Heller <christian.heller@tuxtax.de>
%

\subsubsection{Fuzzy Logic}
\label{fuzzy_logic_heading}

\emph{Fuzzy Logic} a technique for reasoning under uncertainty, has been widely
used in industrial control systems. \cite{wikipedia}

Fuzzy logic is considered a superset of conventional (Boolean) logic that has
been extended to handle the concept of partial truth --
truth values between \emph{completely true} and \emph{completely false} -- by
mapping values according to a special mathematical function \cite{kantrowitz}.
