%
% $RCSfile: terminology.tex,v $
%
% Copyright (C) 2002-2008. Christian Heller.
%
% Permission is granted to copy, distribute and/or modify this document
% under the terms of the GNU Free Documentation License, Version 1.1 or
% any later version published by the Free Software Foundation; with no
% Invariant Sections, with no Front-Cover Texts and with no Back-Cover
% Texts. A copy of the license is included in the section entitled
% "GNU Free Documentation License".
%
% http://www.cybop.net
% - Cybernetics Oriented Programming -
%
% http://www.resmedicinae.org
% - Information in Medicine -
%
% Version: $Revision: 1.1 $ $Date: 2008-08-19 20:41:09 $ $Author: christian $
% Authors: Christian Heller <christian.heller@tuxtax.de>
%

\subsection{Terminology}
\label{terminology_heading}
\index{Terminology}
\index{Lexicon}
\index{Vocabulary}
\index{Nomenclature}
\index{Hierarchy}
\index{Semantic Link}
\index{Directed Acyclic Graph}
\index{DAG}

While a \emph{Lexicon} is a list of pure words, a \emph{Terminology} (sometimes
called \emph{Vocabulary}) can also contain phrases. Because it is a fixed list
of lots of terms, a terminology should exclude any link to a separate list of
concepts. When a terminology contains additional instructions describing how to
interpret each term, or dictating when to choose one over another
(prioritisation), it may be called a \emph{Nomenclature}. The knowledge schema
proposed in this work (chapter \ref{knowledge_schema_heading}) shall be capable
of storing codes of various terminology systems.

Lexicon and terminology stand for a \emph{Set} of words or terms, respectively.
To bring some structure into such a set, terms or concepts need to be ordered,
that is organised through a system of links, into a \emph{Hierarchy}, which
Rogers \cite{rogers} defines as a:

\begin{quote}
    \ldots\ tree-like structure, where things at the top of the tree are in some
    way more general or abstract than the things lower down. The nature of each
    link between each level in the tree may be explicit or only implied, and
    more than one flavour of semantic link can be used to build the tree (in
    which case it may be called a \emph{Mixed Hierarchy}).
\end{quote}

Kinds of hierarchies, as means of organisation, are:

\begin{itemize}
    \item[-] \emph{Subsumption Hierarchy} (Classification, Taxonomy): only
        \emph{is-a} relationships exist between parent-child pairs in the tree
    \item[-] \emph{Uniaxial Hierarchy:} each concept only ever has one parent,
        though it can have more than one child
    \item[-] \emph{Multiaxial Hierachy:} each concept can have more than one
        parent as well as more than one child
    \item[-] \emph{Exhaustive Multiaxial Hierarchy:} all concepts have all the
        parents as well as all the children they should have
\end{itemize}

As organisation \emph{Rules} count:

\begin{itemize}
    \item[-] \emph{Formalism}: an explicitly expressed set of rules, like the
        specification for how to tell what should (not) be a parent of a concept
    \item[-] \emph{Concept System} (Model): a system of \emph{Symbols} that
        stand in for concepts and/ or the links between them, and which may or
        may not be intended to be processed with reference to some formalism
    \item[-] \emph{Partonomy} (Mereology): a system of concepts and links
        intended to represent whole-part relationships specifically
\end{itemize}

On a yet higher abstract level, a \emph{Data Structure} may hold organisations
of concepts. Various types of data structures are:

\begin{itemize}
    \item[-] \emph{Network}: a mesh-like structure that connects terms or concepts
        using links; a hierarchy can be thought of as simple case of a network
    \item[-] \emph{Graph}: a network
    \item[-] \emph{Directed Graph}: a network in which each link has a \emph{Direction}
    \item[-] \emph{Directed Acyclic Graph} (DAG): a directed graph free of loops
\end{itemize}

A knowledge template expressed in the language that will be defined in chapter
\ref{cybernetics_oriented_language_heading} describes an uniaxial hierarchy,
that is its sub concepts have just one parent node. Its structure follows the
partonomy (mereology) organisation rules and represents a DAG.
