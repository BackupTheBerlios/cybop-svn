%
% $RCSfile: domain_specific_language.tex,v $
%
% Copyright (C) 2002-2008. Christian Heller.
%
% Permission is granted to copy, distribute and/or modify this document
% under the terms of the GNU Free Documentation License, Version 1.1 or
% any later version published by the Free Software Foundation; with no
% Invariant Sections, with no Front-Cover Texts and with no Back-Cover
% Texts. A copy of the license is included in the section entitled
% "GNU Free Documentation License".
%
% http://www.cybop.net
% - Cybernetics Oriented Programming -
%
% http://www.resmedicinae.org
% - Information in Medicine -
%
% Version: $Revision: 1.1 $ $Date: 2008-08-19 20:41:06 $ $Author: christian $
% Authors: Christian Heller <christian.heller@tuxtax.de>
%

\subsection{Domain Specific Language}
\label{domain_specific_language_heading}
\index{Domain Specific Language}
\index{DSL}
\index{General Purpose Language}
\index{GPL}
\index{Little Language}
\index{Application Language}
\index{Macro}
\index{Very High Level Language}
\index{GraphViz DOT}
\index{Mathematica}
\index{Yet Another Compiler Compiler}
\index{YACC}
\index{Universal Interactive Executive}
\index{UNIX}
\index{UNIX Shell Script}
\index{Lisp}
\index{Smalltalk}
\index{In-Language DSL}
\index{C++}
\index{Ruby}

While a \emph{General Purpose Language} (GPL), no matter if in form of a
scripting- or compiled programming language, can be used for performing a
variety of different tasks, a (usually declarative) \emph{Domain Specific Language}
(DSL), though less comprehensive, is more expressive in a special domain
context \cite{deursen}. After \cite{wikipedia}, DSLs may: \textit{enhance the
productivity, reliability, maintainability, portability and reusability of
software.} In Czarnecki's words \cite{czarnecki}, DSLs: \textit{increase the
abstraction level for a particular problem domain} and, being highly
intentional: \textit{allow users to work closely with domain concepts.}

Several synonyms are used to label a DSL, for example: \emph{Little Language},
\emph{Application Language}, \emph{Macro} or \emph{Very High Level Language}
\cite{wikipedia}. To the numerous representatives belong simple spreadsheet
\emph{Macros} as well as graph definition languages like \emph{GraphViz}'s
\emph{DOT} \cite{graphviz}, languages for numerical and symbolic computation as
used in \emph{Mathematica} \cite{mathematica}, or parser generator languages
like \emph{Yet Another Compiler Compiler} (YACC), found on
\emph{Universal Interactive Executive} (UNIX) systems. Even UNIX
\emph{Shell Scripts} can be considered a DSL, with emphasis on data
organisation. Further DSLs exist, yet are the boundaries between the concepts
of domain specific- and other languages quite blurry \cite{wikipedia}.

Martin Fowler \cite{fowlerdsl} mentions that the \emph{Lisp} \cite{commonlisp}
and \emph{Smalltalk} \cite{smalltalk} communities, rather than defining a new
language, frequently morph the GPL into a DSL, in a \emph{bottom-up} manner.
Such \emph{In-Language} DSLs, as he calls them, use constructs of the
programming language itself. Wondering why, programming in Smalltalk, he never
really felt the need to use a separate language, while, programming in C++/
Java/ C\#, quite often he did, he concludes \cite{fowlerdsl} that: \textit{the
more suitable languages (are) minimalist ones with a single basic idea that's
deeper and simpler than traditional languages (function application for lisp,
objects and messages for smalltalk)}, and finds that it is the
\textit{friendliness towards in-language DSLs} rather than \textit{static
versus dynamic typing} that let many software developers \textit{enjoy
programming in Smalltalk or Ruby so much more than in Java or C\#}.

Besides their limited usability outside the special domain they were created
for, to the problems that the usage of DSLs brings with belong after
\cite{menzies}:

\begin{itemize}
    \item[-] High cost of designing, implementing, and maintaining a DSL
    \item[-] Difficult finding of the proper scope
    \item[-] Difficult balancing between domain-specificity and GPL constructs
    \item[-] Potential loss of efficiency when compared with hand-coded software
\end{itemize}

The language introduced in chapter \ref{cybernetics_oriented_language_heading}
is simple and just because of that flexible enough to be applicable for
modelling the knowledge of arbitrary domains. It might have the potential to
replace some of the existing DSLs, the investigation of what is out of the
scope of this work, though.
