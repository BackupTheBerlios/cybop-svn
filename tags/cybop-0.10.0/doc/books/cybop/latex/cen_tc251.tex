%
% $RCSfile: cen_tc251.tex,v $
%
% Copyright (C) 2002-2008. Christian Heller.
%
% Permission is granted to copy, distribute and/or modify this document
% under the terms of the GNU Free Documentation License, Version 1.1 or
% any later version published by the Free Software Foundation; with no
% Invariant Sections, with no Front-Cover Texts and with no Back-Cover
% Texts. A copy of the license is included in the section entitled
% "GNU Free Documentation License".
%
% http://www.cybop.net
% - Cybernetics Oriented Programming -
%
% http://www.resmedicinae.org
% - Information in Medicine -
%
% Version: $Revision: 1.1 $ $Date: 2008-08-19 20:41:05 $ $Author: christian $
% Authors: Christian Heller <christian.heller@tuxtax.de>
%

\subsubsection{CEN/TC251}
\label{cen_tc251_heading}
\index{CEN/TC251}
\index{European Committee for Standardization}
\index{CEN}
\index{Standards Development Organisation}
\index{SDO}
\index{Electronic Health Record}
\index{EHR}
\index{Technical Committee 251}
\index{TC251}
\index{ENV 12265}
\index{ENV 13606}
\index{General Purpose Information Component}
\index{GPIC}
\index{EHR Communications Task Force}
\index{EHRcom Task Force}
\index{OpenEHR Archetype}
\index{HL7 CDA}

The \emph{European Committee for Standardization} (CEN) as association of
national \emph{Standards Development Organisations} (SDO) is working on
technical specifications for an \emph{Electronic Health Record} (EHR). The
\emph{Technical Committee} (TC) dealing with that task and medical informatics
in general carries the name \emph{TC251} \cite{centc251}.

While its pre-standard \emph{ENV 12265}, defined in 1995, focused on the EHR
\emph{Architecture}, the successor pre-standard \emph{ENV 13606}, published in
1999, placed more emphasis on \emph{Communication}. Among other things, it
defined a set of reusable components (compositions) called
\emph{General Purpose Information Component} (GPIC) \cite{marley}.
The \emph{ENV 13606} pre-standard consisted of four parts:

\begin{enumerate}
    \item Extended Architecture
    \item Domain Termlist
    \item Distribution Rules
    \item Messages for the Exchange of Information
\end{enumerate}

A third effort, the \emph{EHR Communications} (EHRcom) task force, set up in
December 2001, is to refine \emph{ENV 13606} and to propose a revision that
could be adopted by CEN as formal standard (EN), during 2004. Its current
activities, as described in \cite{ehrcom}, happen in five areas:

\begin{enumerate}
    \item \emph{Reference Model:} generic information model for EHR
        communication
    \item \emph{Archetype Interchange Specification:} generic language for EHR
        representation and communication
    \item \emph{Reference Archetypes and Term Lists:} range of templates
        reflecting clinical requirements and settings
    \item \emph{Security Features:} concepts to enable interaction with
        security components
    \item \emph{Exchange Models:} set of models that can form the basis of
        message-based or service-based communication
\end{enumerate}

\emph{EHRcom} places special focus on the \emph{Harmonisation} of different
standardisation efforts \cite{kalra2002} like CEN's \emph{GPIC}s, openEHR's
\emph{Archetypes} or HL7's \emph{CDA}, described in a later section. But this
is not an easy task. Bert Verhees, for example, reports \cite{openehrtechnical}
about name clashes between the reserved words of many programming languages and
HL7 data types (Set, Array) which made it impossible to use the standard as is
and necessitated a renaming of those types (into something like \emph{HL7Set}
and \emph{HL7Array}). In his opinion, a standard should be platform independent
(operating system- and programming language wise). Thomas Beale of openEHR
writes to this topic \cite{openehrtechnical}:

\begin{quote}
    \ldots\ almost all the issues \ldots\ are actually due to the HL7 data
    types, which CEN unfortunately decided to adopt/ adapt a long time ago. Tom
    Marley and others have struggled to find a version of them which a) remains
    faithful to the idea of HL7 but b) fixes some problems, like strange
    inheritance. Personally, \ldots\ I don't find the HL7 data types a good
    design at all \ldots\ and have made available the reasons in various
    standards discussions, along with many others who have pointed out the same
    problems. The result of this recently has been \ldots\ a new ISO work item
    called \emph{Data Types for Clinical Informatics} \ldots\ which will
    recognise three layers:

    \begin{enumerate}
        \item Inbuilt types (like in ISO 11404)
        \item General purpose clinical types (specified from requirements)
        \item Bindings to particular model systems (such as HL7)
    \end{enumerate}
\end{quote}
