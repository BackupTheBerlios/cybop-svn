%
% $RCSfile: simplified_c.tex,v $
%
% Copyright (C) 2002-2008. Christian Heller.
%
% Permission is granted to copy, distribute and/or modify this document
% under the terms of the GNU Free Documentation License, Version 1.1 or
% any later version published by the Free Software Foundation; with no
% Invariant Sections, with no Front-Cover Texts and with no Back-Cover
% Texts. A copy of the license is included in the section entitled
% "GNU Free Documentation License".
%
% http://www.cybop.net
% - Cybernetics Oriented Programming -
%
% http://www.resmedicinae.org
% - Information in Medicine -
%
% Version: $Revision: 1.1 $ $Date: 2008-08-19 20:41:08 $ $Author: christian $
% Authors: Christian Heller <christian.heller@tuxtax.de>
%

\subsection{Simplified C}
\label{simplified_c_heading}
\index{C Programming Language Simplification}

An early version of CYBOI was implemented in the \emph{Java} programming
language. Since, over time, its functionality was reduced to pure system
control, by moving application-specific features to CYBOL, there was no longer
a need for an \emph{Object Oriented Programming} (OOP) language, which Java is.
The OOP overhead caused by concepts like \emph{Inheritance} inevitably results
in lower performance, as compared to \emph{Structured and Procedural Programming}
(SPP) languages. Low-level system programming, close to hardware, focuses on
fast data processing. OOP concepts would only disturb here. A later (the
current) version of CYBOI was therefore rewritten in the slimmer \emph{C}
programming language.

It would, of course, be possible to implement CYBOI in other languages, too.
Candidates could be \emph{C++} or the increasingly popular \emph{Python}. Both
are OOP languages having similar dis-/advantages like \emph{Java}. However,
different CYBOI implementations are possible and as long as the CYBOL format
gets interpreted correctly, different languages, frameworks and libraries can
be used.

But also C (as other SPP languages) contains a number of unnecessary,
redundant constructs (section \ref{structured_and_procedural_programming_heading})
for one and the same concept (like three kinds of looping, for example) that
deserve the name \emph{Syntactic Sugar for the Programmer}. Some source code
simplifications have therefore been issued as implementation guideline, and
applied to CYBOI:

\begin{itemize}
    \item[-] Use only \emph{procedures}, not \emph{functions}! (A return value is
        just another parameter. There is no argument not to hand it over as such.)
    \item[-] Use only \emph{call by reference}, not \emph{call by value}!
        (Handing over parameters as copy creates redundant data. It is better
        to use references instead.)
    \item[-] Use only \emph{if-else} conditions, not \emph{case} statements!
        (Branching via simple conditions covers all necessary use cases.)
    \item[-] Use only \emph{while} \emph{endless} loops, not \emph{do-while-}
        or \emph{for} loops! (Merging the loop concept with a \emph{break}
        condition is not a good idea. The condition can be put into the loop's
        body, in order to realise \emph{pre-} or \emph{post-testing}.)
\end{itemize}
