%
% $RCSfile: interface_and_implementation.tex,v $
%
% Copyright (C) 2002-2008. Christian Heller.
%
% Permission is granted to copy, distribute and/or modify this document
% under the terms of the GNU Free Documentation License, Version 1.1 or
% any later version published by the Free Software Foundation; with no
% Invariant Sections, with no Front-Cover Texts and with no Back-Cover
% Texts. A copy of the license is included in the section entitled
% "GNU Free Documentation License".
%
% http://www.cybop.net
% - Cybernetics Oriented Programming -
%
% http://www.resmedicinae.org
% - Information in Medicine -
%
% Version: $Revision: 1.1 $ $Date: 2008-08-19 20:41:07 $ $Author: christian $
% Authors: Christian Heller <christian.heller@tuxtax.de>
%

\subsection{Interface and Implementation}
\label{interface_and_implementation_heading}
\index{Separation of Interface and Implementation}
\index{Java}
\index{Interface}
\index{Interface Pattern}
\index{Class}
\index{Decoupling of Components}
\index{Java Development Kit}
\index{JDK}

The \emph{Separation of Interface and Implementation} is a core concept of many
programming languages. \emph{Java}, for example, distinguishes \emph{Interface}
and \emph{Class} (section \ref{classification_heading}). Mark Grand's book
\emph{Patterns in Java} \cite{grand} refers to that separation simply as
\emph{Interface} pattern, one of whose uses, after him, were to
\emph{encapsulate components}, that is to:

\begin{itemize}
    \item[-] Force decoupling of different components
    \item[-] Ensure easy changes of interface implementations
    \item[-] Enable users to read interface documentations without having the
        implementation details clutter up their perception
    \item[-] Increase the reuse possibility in larger applications
\end{itemize}

The Java source code example below shows how the \emph{sayHello} method whose
header is declared in the \emph{HelloWorld} interface can be used as
\emph{Application Programming Interface} (API) without knowing anything about
the underlying implementation. Depending on the instance, the method may be
processed on a local or a remote system.

\begin{scriptsize}
    \begin{verbatim}
    package helloworld;
    public interface HelloWorld {
        void sayHello(String greeting);
    }

    package helloworld.impl.default;
    public class DefaultHelloWorld implements HelloWorld {
        void sayHello(String greeting) {
            System.out.println("HelloWorld Greeting: " + greeting);
        }
    }

    package helloworld.impl.remote;
    public class RemoteHelloWorld implements HelloWorld {
        private RemoteMessager remoteMessager;
        public RemoteHelloWorld(RemoteMessager rm) {
            remoteMessager = rm;
        }
        void sayHello(String greeting) {
            rm.sendMessage("HelloWorld Greeting: " + greeting);
        }
    }
    \end{verbatim}
\end{scriptsize}

Further details and recommendations for using interfaces, on examples specific
to the \emph{Java Development Kit} (JDK) \cite{java}, are given in \cite{avalon}.
