%
% $RCSfile: persistent_and_transient.tex,v $
%
% Copyright (C) 2002-2008. Christian Heller.
%
% Permission is granted to copy, distribute and/or modify this document
% under the terms of the GNU Free Documentation License, Version 1.1 or
% any later version published by the Free Software Foundation; with no
% Invariant Sections, with no Front-Cover Texts and with no Back-Cover
% Texts. A copy of the license is included in the section entitled
% "GNU Free Documentation License".
%
% http://www.cybop.net
% - Cybernetics Oriented Programming -
%
% http://www.resmedicinae.org
% - Information in Medicine -
%
% Version: $Revision: 1.1 $ $Date: 2008-08-19 20:41:08 $ $Author: christian $
% Authors: Christian Heller <christian.heller@tuxtax.de>
%

\subsection{Persistent and Transient}
\label{persistent_and_transient_heading}
\index{Integrated Circuit}
\index{IC}
\index{Read Only Memory}
\index{ROM}
\index{Basic Input/ Output System}
\index{BIOS}
\index{Personal Digital Assistant}
\index{PDA}
\index{Electrically Erasable Programmable ROM}
\index{EEPROM}
\index{Flash ROM}
\index{Hard Disk Drive}
\index{HDD}
\index{Random Access Memory}
\index{RAM}
\index{Persistent Data}
\index{Transient Data}
\index{Volatile Data}
\index{Central Processing Unit}
\index{CPU}

In the science of \emph{Informatics}, there are a few \emph{Integrated Circuits}
(IC) -- so-called \emph{Read Only Memories} (ROM) -- containing unchangeable
programs. The \emph{Basic Input/ Output System} (BIOS) of a computer is one
example for such programs, software of \emph{Personal Digital Assistants} (PDA)
or \emph{Mobile Phones} are others. Often, the BIOS is stored in
\emph{Electrically Erasable Programmable ROM} (EEPROM) or its later form called
\emph{Flash ROM}. It thereby gets writable. When writing it, the complete old
BIOS gets overwritten. Most software programs, however, reside on a
\emph{Hard Disk Drive} (HDD) as permanent storage medium.

In order to use them, such \emph{persistent} programs often need to be loaded
into an IC called \emph{Random Access Memory} (RAM) which, other than a ROM,
can be both read from and written to. Most RAMs are \emph{volatile} which means
that the data they store -- to which also belong programs -- are
\emph{transient}, that is get lost in case the computer is powered down. The
ability to manipulate data in memory is a pre-requisite to usefully work with a
computer.

One surely could, with some effort, rearchitect computers so to let the
\emph{Central Processing Unit} (CPU) communicate directly with persistent
memory (like a HDD), instead of RAM. One reason for not doing this is the
\emph{Performance} of computer systems -- RAM can be accessed much faster than a
HDD. Another reason is the independence from differing HDD designs.

What this section by its remarks tries to show is that \emph{static} software
becomes \emph{dynamic}, that is changeable over time, when being loaded into a
RAM whose data (represented by its state) can be processed (manipulated)
through a CPU. Although not being so new, this statement has importance for
some considerations later in this chapter.

While some kinds of software (like standard- or business applications) mainly
\emph{contain} static knowledge of a special domain, other kinds do mainly
\emph{use} that knowledge to dynamically control hardware. The point is that
traditionally, both kinds of software are mixed up. A business application has
to care about memory allocation, graphical input and output (even when using a
framework for that), communication mechanisms and more, although these are not
in its original interest. On the other hand, an operating system often contains
configuration knowledge about its structure or available devices that does not
actually belong into it. In order to achieve a clearer structure with less
dependencies and more flexibility, it is necessary to treat both kinds of
software differently.
