%
% $RCSfile: standardised_templates.tex,v $
%
% Copyright (C) 2002-2008. Christian Heller.
%
% Permission is granted to copy, distribute and/or modify this document
% under the terms of the GNU Free Documentation License, Version 1.1 or
% any later version published by the Free Software Foundation; with no
% Invariant Sections, with no Front-Cover Texts and with no Back-Cover
% Texts. A copy of the license is included in the section entitled
% "GNU Free Documentation License".
%
% http://www.cybop.net
% - Cybernetics Oriented Programming -
%
% http://www.resmedicinae.org
% - Information in Medicine -
%
% Version: $Revision: 1.1 $ $Date: 2008-08-19 20:41:09 $ $Author: christian $
% Authors: Christian Heller <christian.heller@tuxtax.de>
%

\paragraph{Standardised Knowledge Templates for Various Domains}
\label{standardised_templates_heading}

Much effort has to go into the \emph{Standardisation} of CYBOL templates for
various domains such as \emph{Transport}, \emph{Telecommunication} or
\emph{Healthcare}. Actually, translator templates eliminate the need for a
unification; every state- or logic model template can be translated into any
other. Nevertheless, it seems very useful to provide internationally
standardised model templates: They can serve as focal point, while developing a
system. The more compliant different models are, the less translating is needed
for data interchange between applications. For the greater part, this
standardisation process has to be carried by domain experts; less by software
specialists.

A closely related topic in this respect are \emph{Terminologies}. Section
\ref{implication_heading} proposed to use CYBOL for terminology modelling,
because it allowed to add meta information as well as to integrate constraints
into its compositional structure. That way, problems like \emph{Nonsense}
combinations (section \ref{schemes_heading}) or \emph{Post-hoc Classification}
(unforeseeable addition of new, unknown concepts that may prevent a meaningful
data analysis) might be avoided. But this surely has to be figured out in
future research works.

Besides for domain models like an \emph{Electronic Health Record} (EHR)
\cite{openehr} for medicine, knowledge templates could also be defined for
\emph{User Interfaces} (UI), may they be textual, graphical or for the
\emph{World Wide Web} (WWW). Another kind of standardised template could be one
for the creation of \emph{Requirements Analysis Documents}. All templates would
be exchangeable and reusable.

One point deserving special attention, is the handling of \emph{Constraints} in
CYBOL templates. Certainly, not all possible variants of constraints have been
thought-out in this work. More details are needed.
