%
% $RCSfile: user_interface_modelling.tex,v $
%
% Copyright (C) 2002-2008. Christian Heller.
%
% Permission is granted to copy, distribute and/or modify this document
% under the terms of the GNU Free Documentation License, Version 1.1 or
% any later version published by the Free Software Foundation; with no
% Invariant Sections, with no Front-Cover Texts and with no Back-Cover
% Texts. A copy of the license is included in the section entitled
% "GNU Free Documentation License".
%
% http://www.cybop.net
% - Cybernetics Oriented Programming -
%
% http://www.resmedicinae.org
% - Information in Medicine -
%
% Version: $Revision: 1.1 $ $Date: 2008-08-19 20:41:09 $ $Author: christian $
% Authors: Christian Heller <christian.heller@tuxtax.de>
%

\subsection{User Interface Modelling}
\label{user_interface_modelling_heading}

- GUIs bisher nie einheitlich beschrieben wie RDB mit ERM und DDL/SQL
- jetzt erst erreichen GUIs mit der UIML den Stand von DDL/SQL
- so wie RDBMS durch OODBMS ersetzt wurden und damit entfielen,
OODBMS mit OO Domain Layer verschmolzen ist, kann evtl. auch das GUI
mit der Domain verschmolzen werden! ==> NEIN! weil Domain Daten unterschiedlich
im GUI dargestellt werden koennen (Edit, Liste, Tabelle, Tree) und Anwender
verschiedene Wuensche und Anforderungen an das GUI haben, weswegen die Entwickler
die GUIs flexibel erstellen und handhaben muessen, was nicht gegeben waere,
wenn das GUI vom DomainModel fest vorgegeben waere

Common libraries used for data display design are the \emph{GIMP Toolkit} (GTK)
of the \emph{General (GNU) Image Manipulation Program} (GIMP) and the \emph{Qt Toolkit},
both written in C++, as well as the \emph{Abstract Window Toolkit} (AWT)/ \emph{Swing}
for Java.

mention and shortly describe \emph{GUI Renderer} (section pattern_merger refers to here)

%
% $RCSfile: bean.tex,v $
%
% Copyright (C) 2002-2008. Christian Heller.
%
% Permission is granted to copy, distribute and/or modify this document
% under the terms of the GNU Free Documentation License, Version 1.1 or
% any later version published by the Free Software Foundation; with no
% Invariant Sections, with no Front-Cover Texts and with no Back-Cover
% Texts. A copy of the license is included in the section entitled
% "GNU Free Documentation License".
%
% http://www.cybop.net
% - Cybernetics Oriented Programming -
%
% http://www.resmedicinae.org
% - Information in Medicine -
%
% Version: $Revision: 1.1 $ $Date: 2008-08-19 20:41:05 $ $Author: christian $
% Authors: Christian Heller <christian.heller@tuxtax.de>
%

\subsubsection{Bean}
\label{bean_heading}

- also Borland's Delphi Component Library (first of its kind)

%
% $RCSfile: xml_user_interface_language.tex,v $
%
% Copyright (C) 2002-2008. Christian Heller.
%
% Permission is granted to copy, distribute and/or modify this document
% under the terms of the GNU Free Documentation License, Version 1.1 or
% any later version published by the Free Software Foundation; with no
% Invariant Sections, with no Front-Cover Texts and with no Back-Cover
% Texts. A copy of the license is included in the section entitled
% "GNU Free Documentation License".
%
% http://www.cybop.net
% - Cybernetics Oriented Programming -
%
% http://www.resmedicinae.org
% - Information in Medicine -
%
% Version: $Revision: 1.1 $ $Date: 2008-08-19 20:41:09 $ $Author: christian $
% Authors: Christian Heller <christian.heller@tuxtax.de>
%

\subsubsection{XML User Interface Language}
\label{xml_user_interface_language_heading}

XML User Interface Language (XUL)

Mozilla + Opera

"Web Forms 2.0"?

%
% $RCSfile: user_interface_markup_language.tex,v $
%
% Copyright (C) 2002-2008. Christian Heller.
%
% Permission is granted to copy, distribute and/or modify this document
% under the terms of the GNU Free Documentation License, Version 1.1 or
% any later version published by the Free Software Foundation; with no
% Invariant Sections, with no Front-Cover Texts and with no Back-Cover
% Texts. A copy of the license is included in the section entitled
% "GNU Free Documentation License".
%
% http://www.cybop.net
% - Cybernetics Oriented Programming -
%
% http://www.resmedicinae.org
% - Information in Medicine -
%
% Version: $Revision: 1.1 $ $Date: 2008-08-19 20:41:09 $ $Author: christian $
% Authors: Christian Heller <christian.heller@tuxtax.de>
%

\subsubsection{User Interface Markup Language}
\label{user_interface_markup_language_heading}

User Interface Markup Language (UIML)

