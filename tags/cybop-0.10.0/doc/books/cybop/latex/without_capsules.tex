%
% $RCSfile: without_capsules.tex,v $
%
% Copyright (C) 2002-2008. Christian Heller.
%
% Permission is granted to copy, distribute and/or modify this document
% under the terms of the GNU Free Documentation License, Version 1.1 or
% any later version published by the Free Software Foundation; with no
% Invariant Sections, with no Front-Cover Texts and with no Back-Cover
% Texts. A copy of the license is included in the section entitled
% "GNU Free Documentation License".
%
% http://www.cybop.net
% - Cybernetics Oriented Programming -
%
% http://www.resmedicinae.org
% - Information in Medicine -
%
% Version: $Revision: 1.1 $ $Date: 2008-08-19 20:41:09 $ $Author: christian $
% Authors: Christian Heller <christian.heller@tuxtax.de>
%

\subsection{Without Capsules?}
\label{without_capsules_heading}
\index{Without Capsules}
\index{Side Effect}
\index{Knowledge Tree}
\index{Encapsulation}
\index{Well-Defined Knowledge Paths}
\index{Call by Reference}
\index{Call by Value}

Once again it has to be said that all this becomes possible only because all
domain/ application knowledge is stored together in one single tree structure
which is hold and managed by the \emph{Cybernetics Oriented Interpreter}
(CYBOI) (chapter \ref{cybernetics_oriented_interpreter_heading}). What was
traditionally criticised as \emph{Side Effect}, is now a \emph{wanted} effect.
Low-level system procedures within CYBOI forward just one pointer -- the root
of the knowledge tree, which they all may access and manipulate. Data values do
\emph{not} get copied among procedures; they exist just once in the knowledge
tree and may be used by any procedure. Of course, this also means that any
application has access to the knowledge of any other application. Ways ensuring
sufficient security have to be found here (section \ref{future_works_heading}).

Besides the \emph{Encapsulation} of data through \emph{Procedures}, there are
other forms of encapsulating data, such as the \emph{Class} (section
\ref{object_oriented_programming_heading}). One of its purposes was to preserve
transient data in memory, another to restrict access to certain data. In CYBOP,
both tasks are taken care of by CYBOI. It holds the singular knowledge tree and
manages access to it, through well-defined low-level procedures.

There are a number of advantages to this style of programming: An application
developer has no chance of accessing memory areas directly, which prevents
memory leaks and wrong pointers. Because all knowledge can be accessed through
well-defined paths into the knowledge tree only, arbitrary security mechanisms
may be applied and switched as needed, at runtime. Since all algorithms (logic
knowledge) work with references to data in the knowledge tree
(\emph{Call by Reference}), no more data values need to be copied locally
(\emph{Call by Value}), which ensures efficient memory usage. Errors are not to
be expected, because nonexisting knowledge references are simply ignored by
CYBOI.
