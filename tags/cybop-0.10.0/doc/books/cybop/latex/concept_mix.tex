%
% $RCSfile: concept_mix.tex,v $
%
% Copyright (C) 2002-2008. Christian Heller.
%
% Permission is granted to copy, distribute and/or modify this document
% under the terms of the GNU Free Documentation License, Version 1.1 or
% any later version published by the Free Software Foundation; with no
% Invariant Sections, with no Front-Cover Texts and with no Back-Cover
% Texts. A copy of the license is included in the section entitled
% "GNU Free Documentation License".
%
% http://www.cybop.net
% - Cybernetics Oriented Programming -
%
% http://www.resmedicinae.org
% - Information in Medicine -
%
% Version: $Revision: 1.1 $ $Date: 2008-08-19 20:41:06 $ $Author: christian $
% Authors: Christian Heller <christian.heller@tuxtax.de>
%

\paragraph{Concept Mix}
\label{concept_mix_heading}
\index{Concept Mix}

Further, specialised domain concepts identified during requirements analysis
(such as a \emph{Patient} being a kind of \emph{Person}) are often mixed up
with more general concepts as found during design (for example the application
of a proper \emph{Role} architecture instead of simple inheritance for the
person-patient relation). The lack of a proper separation between pure domain
knowledge (like a patient receiving a medication) and system control software
(like logging facilities or persistence mechanisms) was already explained in
detail in the previous chapter \ref{statics_and_dynamics_heading}. It frequently
leads to strong coupling between system layers and complicates software design.
