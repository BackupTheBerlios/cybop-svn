%
% $RCSfile: knowledge_query_and_manipulation_language.tex,v $
%
% Copyright (C) 2002-2008. Christian Heller.
%
% Permission is granted to copy, distribute and/or modify this document
% under the terms of the GNU Free Documentation License, Version 1.1 or
% any later version published by the Free Software Foundation; with no
% Invariant Sections, with no Front-Cover Texts and with no Back-Cover
% Texts. A copy of the license is included in the section entitled
% "GNU Free Documentation License".
%
% http://www.cybop.net
% - Cybernetics Oriented Programming -
%
% http://www.resmedicinae.org
% - Information in Medicine -
%
% Version: $Revision: 1.1 $ $Date: 2008-08-19 20:41:07 $ $Author: christian $
% Authors: Christian Heller <christian.heller@tuxtax.de>
%

\subsubsection{Knowledge Query and Manipulation Language}
\label{knowledge_query_and_manipulation_language_heading}
\index{Knowledge Query and Manipulation Language}
\index{KQML}
\index{Common Lisp}
\index{CL}

The \emph{Knowledge Query and Manipulation Language} (KQML) \cite{kqml} is a:
\textit{language and associated protocol by which intelligent software agents
can communicate to share information and knowledge}, as Tim Finin et al.
\cite{finin} write. Its syntax were based on a balanced parenthesis list,
because initial implementations had been done in Common Lisp (CL)
\cite{commonlisp}. After Finin et al., the initial element of the list were the
\emph{Performative} and the remaining elements were the performative's
\emph{Arguments} as keyword/ value pairs. The Free Wikipedia Encyclopedia
\cite{wikipedia} explains:

\begin{quote}
    The KQML message format and protocol can be used to interact with an
    intelligent system, either by an application program, or by another
    intelligent system. KQML's \emph{Performatives} are operations that agents
    perform on each other's \emph{Knowledge} and \emph{Goal} stores.
    Higher-level interactions such as \emph{Contract Nets} and
    \emph{Negotiation} are built using these. KQML's
    \emph{Communication Facilitators} coordinate the interactions of other
    agents to support \emph{Knowledge Sharing}.
\end{quote}

An example message representing a query about the price of a share of IBM stock
might be encoded as \cite{finin}:

\begin{scriptsize}
    \begin{verbatim}
    (ask-one
    :content (PRICE IBM ?price)
    :receiver stock-server
    :language LPROLOG
    :ontology NYSE-TICKS)
    \end{verbatim}
\end{scriptsize}

System communication and its elements like \emph{Sender}, \emph{Receiver},
\emph{Language} or \emph{Message Content} will be further investigated in
chapter \ref{state_and_logic_heading}. The new language introduced in chapter
\ref{cybernetics_oriented_language_heading} defines communication operations
(logic) accompanied by properties (meta information), much the same way
performatives have arguments. Also, that new language may not only be used to
encode knowledge for communication, but to represent knowledge of arbitrary
domains. By combining pre-defined, primitive operations, it may be used to
create more complex (higher-level) algorithms.
