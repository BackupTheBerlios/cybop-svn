%
% $RCSfile: constraint.tex,v $
%
% Copyright (C) 2002-2008. Christian Heller.
%
% Permission is granted to copy, distribute and/or modify this document
% under the terms of the GNU Free Documentation License, Version 1.1 or
% any later version published by the Free Software Foundation; with no
% Invariant Sections, with no Front-Cover Texts and with no Back-Cover
% Texts. A copy of the license is included in the section entitled
% "GNU Free Documentation License".
%
% http://www.cybop.net
% - Cybernetics Oriented Programming -
%
% http://www.resmedicinae.org
% - Information in Medicine -
%
% Version: $Revision: 1.1 $ $Date: 2008-08-19 20:41:06 $ $Author: christian $
% Authors: Christian Heller <christian.heller@tuxtax.de>
%

\subsubsection{Constraint}
\label{constraint_heading}
\index{Constraint}
\index{Table with constrained Number of Parts}
\index{Temperature with Minimum and Maximum}

The previous sections have discussed three kinds of conceptual interaction:
\emph{Space}, \emph{Mass} and \emph{Time}. They are used by a model (concept)
to position parts within its area of validity.

Yet this meta knowledge is not enough. Frequently, parts have to be
\emph{constrained} to maintain the validity of the whole model. The concept of
a \emph{Table}, for example, may consist of a \emph{Top} and one to four
\emph{Legs}. The additional meta information herein is the constraint of the
number of legs to at least \emph{one} and at most \emph{four}.

Another example regards the area of valid values that parts can take on. The
\emph{Temperature} of an alive human body lies somewhere between a
\emph{Minimum} of +30$^{\circ}$C and a \emph{Maximum} of +40$^{\circ}$C
(broad-minded estimation). A corresponding model has to remember these extrema
in order to be able to limit numbers to the correct temperature range.
