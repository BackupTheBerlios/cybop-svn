%
% $RCSfile: knowledge_engineering.tex,v $
%
% Copyright (C) 2002-2008. Christian Heller.
%
% Permission is granted to copy, distribute and/or modify this document
% under the terms of the GNU Free Documentation License, Version 1.1 or
% any later version published by the Free Software Foundation; with no
% Invariant Sections, with no Front-Cover Texts and with no Back-Cover
% Texts. A copy of the license is included in the section entitled
% "GNU Free Documentation License".
%
% http://www.cybop.net
% - Cybernetics Oriented Programming -
%
% http://www.resmedicinae.org
% - Information in Medicine -
%
% Version: $Revision: 1.1 $ $Date: 2008-08-19 20:41:07 $ $Author: christian $
% Authors: Christian Heller <christian.heller@tuxtax.de>
%

\section{Knowledge Engineering}
\label{knowledge_engineering_heading}
\index{Knowledge Engineering}
\index{KE}
\index{Data (Definition)}
\index{Information (Definition)}
\index{Knowledge (Definition)}
\index{Explicit Knowledge}
\index{Implicit Knowledge}
\index{Knowledge Representation}
\index{Date and Rule}

In order to correctly perform information input, memorising, processing and
output, systems need to know about the structure of the data they are
processing. While pure \emph{Data} abstract the real world in form of (mostly
machine-readable) characters (quality) and numbers (quantity) (more on this in
chapter \ref{knowledge_schema_heading}), their interpretation in a semantic
context may yield \emph{Information}. What coincides across the many different
definitions \cite{wikipedia} of the term \emph{Information}, are two
statements: it has to be recognisable and: it has to contain something new.
Organised information available in form of specific structures with
inter-relations is often called \emph{Knowledge}.

Quite often, knowledge gets shared into two kinds: \emph{explicit} (codified)
and \emph{implicit} (tacit). While the former, after \cite{nonaka}, referred to
knowledge that is transmittable in formal, systematic language, the latter had
a personal quality, which made it hard to formalise and communicate. This work
is about explicit knowledge; it wants to provide concepts and schemata for
expressing it as completely as possible (part \ref{contribution_heading}).

Over the years, many different \emph{Knowledge Representation} models have been
proposed. In software engineering, they represent different kinds of user
interfaces (textual, graphical, web), workflows, persistent- or transient data.
The biggest importance, however, models experience when representing a complete
\emph{Domain} (section \ref{domain_model_heading}), that is the special
business field an application system was developed for. After John F. Sowa
\cite{sowa}, \emph{Knowledge Engineering} (KE) could be defined as: \textit{the
branch of engineering that analyses knowledge about some subject and transforms
it to a computable form for some purpose.}

It has to be mentioned that the borders between \emph{Domain Engineering} (DE)
(section \ref{domain_engineering_heading}) and \emph{Knowledge Engineering} are
quite blurry. The \emph{Feature Model} (section \ref{feature_model_heading}),
although described as part of DE, can of course also be seen as one form of
knowledge representation, just like many other models. Both, DE as well as KE
specify formal languages and investigate how different models can be translated
into each other. This work, however, makes a split between DE and KE, because:

\begin{itemize}
    \item[-] their topics are usually treated as different fields of science
    \item[-] KE's focus is almost exclusively on representing (domain)
        knowledge in models
    \item[-] DE also considers their relation to the applications working on
        them
    \item[-] DE distinguishes models belonging to different phases in a
        \emph{Software Engineering Process} (SEP)
    \item[-] DE describes concrete implementation techniques
\end{itemize}

The following sections try to give a brief overview of the rather wide field of
knowledge representation and -engineering, raising only a few topics. Their
main intention is to show the existence of two different kinds of knowledge:
\emph{Date and Rule}, both of which can be structured according to an ontology,
what will be dealt with in the later chapters \ref{knowledge_schema_heading}
and \ref{state_and_logic_heading}.

%\input{undeterministic_knowledge}
%
% $RCSfile: representation_principles.tex,v $
%
% Copyright (C) 2002-2008. Christian Heller.
%
% Permission is granted to copy, distribute and/or modify this document
% under the terms of the GNU Free Documentation License, Version 1.1 or
% any later version published by the Free Software Foundation; with no
% Invariant Sections, with no Front-Cover Texts and with no Back-Cover
% Texts. A copy of the license is included in the section entitled
% "GNU Free Documentation License".
%
% http://www.cybop.net
% - Cybernetics Oriented Programming -
%
% http://www.resmedicinae.org
% - Information in Medicine -
%
% Version: $Revision: 1.1 $ $Date: 2008-08-19 20:41:08 $ $Author: christian $
% Authors: Christian Heller <christian.heller@tuxtax.de>
%

\subsection{Representation Principles}
\label{representation_principles_heading}
\index{Knowledge Representation Principles}
\index{Knowledge Engineering, Basic Principles}

Randall Davis, Howard Schrobe and Peter Szolovits (1993), cited by Sowa in
\cite[p. 134]{sowa}, summarise their review of the state-of-the-art in
knowledge engineering in form of five \emph{Basic Principles}. To them, a
knowledge representation is a:

\begin{enumerate}
    \item \emph{Surrogate:} symbols and the links between them, which form a
        model that simulates a system, serve as surrogates of physical items
        $\rightarrow$ chapter \ref{statics_and_dynamics_heading} will distinguish
        between virtual models (symbols) and real world (physical) items
    \item \emph{Set of ontological commitments:} an ontology determines the
        categories of things that may exist in an application domain
        $\rightarrow$ chapter \ref{knowledge_schema_heading} will introduce a
        knowledge schema permitting to apply an ontological structure to data
    \item \emph{Fragmentary theory of intelligent reasoning:} a description of
        the behaviour and interactions of things in a domain supports reasoning
        about them
        $\rightarrow$ chapter \ref{state_and_logic_heading} will separate state-
        and logic (behavioural) knowledge
    \item \emph{Medium of human expression:} a language facilitates
        communication between knowledge engineers and domain experts
        $\rightarrow$ chapter \ref{cybernetics_oriented_language_heading} will
        define a language for knowledge specification
    \item \emph{Medium for efficient computation:} encoded knowledge ensures
        efficient processing on computing equipment
        $\rightarrow$ chapter \ref{cybernetics_oriented_interpreter_heading} will
        describe a low-level interpreter that processes high-level knowledge
\end{enumerate}

%
% $RCSfile: date_and_rule.tex,v $
%
% Copyright (C) 2002-2008. Christian Heller.
%
% Permission is granted to copy, distribute and/or modify this document
% under the terms of the GNU Free Documentation License, Version 1.1 or
% any later version published by the Free Software Foundation; with no
% Invariant Sections, with no Front-Cover Texts and with no Back-Cover
% Texts. A copy of the license is included in the section entitled
% "GNU Free Documentation License".
%
% http://www.cybop.net
% - Cybernetics Oriented Programming -
%
% http://www.resmedicinae.org
% - Information in Medicine -
%
% Version: $Revision: 1.1 $ $Date: 2008-08-19 20:41:06 $ $Author: christian $
% Authors: Christian Heller <christian.heller@tuxtax.de>
%

\subsection{Date and Rule}
\label{date_and_rule_heading}
\index{Date and Rule}
\index{Expert System}
\index{Relational Database}
\index{Structured Query Language}
\index{SQL}
\index{Existential Conjunctive Logic}
\index{EC Logic}
\index{Modus Ponens, Inference Rule}
\index{Modus Tollens, Inference Rule}
\index{Forward Chaining}
\index{Backward Chaining}
\index{View, Implication}
\index{Trigger, Implication}
\index{Prolog}
\index{Microplanner}
\index{Backtracking}

Two kinds of systems that gained greater popularity are \emph{Expert Systems}
and \emph{Relational Databases}. After Sowa \cite{sowa}, both differed more in
quantity than in quality: \textit{Expert systems use repeated executions of
rules on relatively small amounts of data, while database systems execute short
chains of rules on large amounts of data.} Over time, their differences
decreased and today, the \emph{Structured Query Language} (SQL) for relational
databases supports the same logical functions as early expert systems.

Both, expert systems and relational databases, have common logical foundations
and store data in a subset of logic called \emph{Existential Conjunctive} (EC)
logic. EC is based on two logical operators: the \emph{Existential Quantifier}
$\exists$ and the \emph{Conjunction} $\wedge$; the \emph{Universal Quantifier}
$\forall$ and other operators ($\sim$, $\supset$, $\vee$) are never used.
Sowa \cite[p. 163]{sowa} writes: \textit{While variables in a query are governed
by existential quantifiers, those in a rule are governed by universal quantifiers.}

The two primary inference rules of the above-mentioned systems are called
\emph{Modus Ponens} (putting) and \emph{Modus Tollens} (taking away). Although
being simple, the power of these rules comes from their combination and
repeated execution. While repeated execution of modus ponens is called
\emph{Forward Chaining}, that of modus tollens is called \emph{Backward Chaining}.
In SQL, an implication used in backward chaining is called \emph{View}, and that
used in forward chaining is called \emph{Trigger} \cite{sowa}.

Besides \emph{Prolog} (section \ref{logical_programming_heading}) and \emph{SQL}
(section \ref{data_manipulation_language_heading}), the \emph{Microplanner}
language \cite[p. 157]{sowa} uses the so-called \emph{Backtracking} technique
to answer a query: If one of a sequence of aims cannot be satisfied, the language
tracks back to a previous aim and tries a different option. Although equivalent
queries in Prolog and SQL differ in their syntax, the semantics is the same.
\textit{Logic determines the structure of a query}, as Sowa \cite[p. 159]{sowa}
means.

To sum up, one can say that previous sections distinguished between \emph{Domain-}
and \emph{Application Models}. What was shown in this section, however, is that
many systems and their corresponding languages rely on a separation of \emph{Data}
(in state variables) and \emph{Rules} (logic). This will be of importance in
chapter \ref{state_and_logic_heading}.

%%
% $RCSfile: data_model.tex,v $
%
% Copyright (C) 2002-2008. Christian Heller.
%
% Permission is granted to copy, distribute and/or modify this document
% under the terms of the GNU Free Documentation License, Version 1.1 or
% any later version published by the Free Software Foundation; with no
% Invariant Sections, with no Front-Cover Texts and with no Back-Cover
% Texts. A copy of the license is included in the section entitled
% "GNU Free Documentation License".
%
% http://www.cybop.net
% - Cybernetics Oriented Programming -
%
% http://www.resmedicinae.org
% - Information in Medicine -
%
% Version: $Revision: 1.1 $ $Date: 2008-08-19 20:41:06 $ $Author: christian $
% Authors: Christian Heller <christian.heller@tuxtax.de>
%

\subsection{Data Model}
\label{data_model_heading}

A number of historical and more current domain modelling concepts are described
in brief in the following sections, some of them known from the field of database
technology.

- kind of relationships between the data and rules (knowledge)

%
% $RCSfile: hierarchical_data_model.tex,v $
%
% Copyright (C) 2002-2008. Christian Heller.
%
% Permission is granted to copy, distribute and/or modify this document
% under the terms of the GNU Free Documentation License, Version 1.1 or
% any later version published by the Free Software Foundation; with no
% Invariant Sections, with no Front-Cover Texts and with no Back-Cover
% Texts. A copy of the license is included in the section entitled
% "GNU Free Documentation License".
%
% http://www.cybop.net
% - Cybernetics Oriented Programming -
%
% http://www.resmedicinae.org
% - Information in Medicine -
%
% Version: $Revision: 1.1 $ $Date: 2008-08-19 20:41:07 $ $Author: christian $
% Authors: Christian Heller <christian.heller@tuxtax.de>
%

\subsubsection{Hierarchical Data Model}
\label{hierarchical_data_model_heading}

One of the first models to structure domain data was a simple \emph{Hierarchy},
being used in \emph{Hierarchical Databases} such as ... (VSAM),
in the 1960s and early 1970s. Many of them are still running now,
for instance in insurance companies who have not yet migrated their
systems to modern technologies.

Knowledge Engineering Systems make use of hierarchical data, today.

- Domain Engineering in general, vertical and horizontal separation
- see \cite{inpulse} paper

The following techniques of domain modelling are neither sorted historically,
nor after the SEP phase they are used in (analysis, design, implementation).
The order of their appearance is determined by similarities they have. New
techniques are added stepwise, in a didactic manner, one building on the
other.

The feature modelling, for example, is a domain analysis- and not an
implementation technique but mentioned here because it is based on a special
technique of abstraction worth considering. Moreover it is not exclusively used
for analysis, but also it represents the beginning of design in a software
engineering process, as mentioned in section \ref{abstraction_gaps_heading}.

- add Feature Model to domain modelling, because it is a hierarchical model

\input{network_data_model}
\input{entity_relationship_model}
%
% $RCSfile: object_oriented_model.tex,v $
%
% Copyright (C) 2002-2008. Christian Heller.
%
% Permission is granted to copy, distribute and/or modify this document
% under the terms of the GNU Free Documentation License, Version 1.1 or
% any later version published by the Free Software Foundation; with no
% Invariant Sections, with no Front-Cover Texts and with no Back-Cover
% Texts. A copy of the license is included in the section entitled
% "GNU Free Documentation License".
%
% http://www.cybop.net
% - Cybernetics Oriented Programming -
%
% http://www.resmedicinae.org
% - Information in Medicine -
%
% Version: $Revision: 1.1 $ $Date: 2008-08-19 20:41:07 $ $Author: christian $
% Authors: Christian Heller <christian.heller@tuxtax.de>
%

\subsubsection{Object Oriented Model}
\label{object_oriented_model_heading}

?? frame \cite{sowa}, as ancestor of OO and others

The introduction of object oriented programming (section
\ref{object_oriented_programming_heading}) made another software design
philosophy popular:
Every entity was now treated as \emph{Object} being a runtime-instance of a
\emph{Class} that was capable of inheriting \emph{Attributes} and \emph{Methods}
from a parent class. A \emph{Relation} was now called \emph{Association}.
Multiplicity was called \emph{Cardinality}.

The primary new ideas of \emph{Inheritance} between classes and classes owning not
only attributes but also \emph{Methods} could not be modeled well in entity relationship
models (section \ref{entity_relationship_model}) so that \emph{Data Mapper} layers
(section \ref{data_mapper_heading}) became necessary. In general, object oriented
models look not much different from entity relationship models.


%%
% $RCSfile: knowledge_map.tex,v $
%
% Copyright (C) 2002-2008. Christian Heller.
%
% Permission is granted to copy, distribute and/or modify this document
% under the terms of the GNU Free Documentation License, Version 1.1 or
% any later version published by the Free Software Foundation; with no
% Invariant Sections, with no Front-Cover Texts and with no Back-Cover
% Texts. A copy of the license is included in the section entitled
% "GNU Free Documentation License".
%
% http://www.cybop.net
% - Cybernetics Oriented Programming -
%
% http://www.resmedicinae.org
% - Information in Medicine -
%
% Version: $Revision: 1.1 $ $Date: 2008-08-19 20:41:07 $ $Author: christian $
% Authors: Christian Heller <christian.heller@tuxtax.de>
%

\subsection{Knowledge Map}
\label{knowledge_map_heading}

A \emph{Topic Map} is ... a standard paradigm for the interchange of knowledge structures.
http://www.topicmaps.org/
http://topicmap.com/topicmap/tools.html

A \emph{Concept Map} is a diagram meant to represent ideas, each idea being
represented by a shape (rectangle, ellipse, image, ...). The diagram becomes
a concept map when relations link these ideas and \emph{Hypertext Links} are
added to these ideas which allows navigation to either a detailed description,
for example a web page providing details on this idea, or another concept map.
\cite{} [http://www.thinkgraph.com/english/index.htm]

A \emph{Mind Map} is a powerful graphic technique which provides a universal
key to unlock the potential of the brain. It harnesses the full range of cortical
skills -- word, image, number, logic, rhythm, colour and spatial awareness --
in a single, uniquely powerful manner. In so doing, it gives you the freedom
to roam the infinite expanses of your brain. The Mind Map can be applied to
every aspect of life where improved learning and clearer thinking will enhance
human performance.
\cite{} [http://www.mind-map.com/EN/mindmaps/definition.html]

A \emph{Mind Map} is a picture that represents semantic connections between
portions of learned material.
\cite[http://en.wikipedia.org/wiki/Mind_mapping]{wikipedia}

https://sourceforge.net/projects/freemind/

%
% $RCSfile: agent_communication_language.tex,v $
%
% Copyright (C) 2002-2008. Christian Heller.
%
% Permission is granted to copy, distribute and/or modify this document
% under the terms of the GNU Free Documentation License, Version 1.1 or
% any later version published by the Free Software Foundation; with no
% Invariant Sections, with no Front-Cover Texts and with no Back-Cover
% Texts. A copy of the license is included in the section entitled
% "GNU Free Documentation License".
%
% http://www.cybop.net
% - Cybernetics Oriented Programming -
%
% http://www.resmedicinae.org
% - Information in Medicine -
%
% Version: $Revision: 1.1 $ $Date: 2008-08-19 20:41:05 $ $Author: christian $
% Authors: Christian Heller <christian.heller@tuxtax.de>
%

\subsection{Agent Communication Language}
\label{agent_communication_language_heading}
\index{Agent Communication Language}
\index{ACL}
\index{Artificial Intelligence}
\index{AI}
\index{Agent Oriented Programming}
\index{AGOP}

A whole palette of languages was suggested within the scientific field of
\emph{Artificial Intelligence} (AI). \emph{Agent Oriented Programming} (AGOP)
(section \ref{agent_oriented_programming_heading}), for example, uses
representation formats like the ones described following, for the knowledge
bases and communication of its agent systems. That is why such formats are
often labeled \emph{Agent Communication Language} (ACL).

%
% $RCSfile: knowledge_interchange_format.tex,v $
%
% Copyright (C) 2002-2008. Christian Heller.
%
% Permission is granted to copy, distribute and/or modify this document
% under the terms of the GNU Free Documentation License, Version 1.1 or
% any later version published by the Free Software Foundation; with no
% Invariant Sections, with no Front-Cover Texts and with no Back-Cover
% Texts. A copy of the license is included in the section entitled
% "GNU Free Documentation License".
%
% http://www.cybop.net
% - Cybernetics Oriented Programming -
%
% http://www.resmedicinae.org
% - Information in Medicine -
%
% Version: $Revision: 1.1 $ $Date: 2008-08-19 20:41:07 $ $Author: christian $
% Authors: Christian Heller <christian.heller@tuxtax.de>
%

\subsubsection{Knowledge Interchange Format}
\label{knowledge_interchange_format_heading}
\index{Knowledge Interchange Format}
\index{KIF}
\index{PostScript}
\index{PS}

The \emph{Knowledge Interchange Format} (KIF), as described in \cite{kif}, is:

\begin{itemize}
    \item[-] a language designed for use in the interchange of knowledge among
        disparate computer systems
    \item[-] not intended as a primary language for interaction with human users
    \item[-] not intended as an internal representation for knowledge within
        computer systems
    \item[-] in its purpose, roughly analogous to \emph{PostScript} (PS)
        (section \ref{page_description_language_heading})
    \item[-] not as efficient as a specialised representation for knowledge,
        but more general and programmer-readable
\end{itemize}

The idea behind KIF is that \cite{kif}: \textit{a computer system reads a
knowledge base in KIF, (and) converts the data into its own internal form
(pointer structures, arrays, etc.). All computation is done using these
internal forms. When the computer system needs to communicate with another
computer system, it maps its internal data structures into KIF.} KIF's design
is characterised by three features:

\begin{enumerate}
    \item \emph{Declarative Semantics:} independent from specific interpreters,
        as opposed to e.g. \emph{Prolog}
    \item \emph{Logically Comprehensive:} may express arbitrary logical
        sentences, as opposed to \emph{SQL} or \emph{Prolog}
    \item \emph{Meta Knowledge:} permits the introduction of new knowledge
        representation constructs, without changing the language
\end{enumerate}

The following syntax example \cite{kif} shows a logical term involving the
\emph{if} operator. If the object constant \emph{a} denotes a number, then the
term denotes the absolute value of that number:

\begin{scriptsize}
    \begin{verbatim}
    (if (> a 0) a (- a))
    \end{verbatim}
\end{scriptsize}

The language introduced in chapter \ref{cybernetics_oriented_language_heading}
may not only serve as interchange format between systems, but also for the
definition of user interfaces, workflows and domain models, altogether. It
treats state- and logic models as separate, composable concepts (chapter
\ref{state_and_logic_heading}), which KIF does not. Further, it provides the
means to express meta knowledge.

%
% $RCSfile: knowledge_query_and_manipulation_language.tex,v $
%
% Copyright (C) 2002-2008. Christian Heller.
%
% Permission is granted to copy, distribute and/or modify this document
% under the terms of the GNU Free Documentation License, Version 1.1 or
% any later version published by the Free Software Foundation; with no
% Invariant Sections, with no Front-Cover Texts and with no Back-Cover
% Texts. A copy of the license is included in the section entitled
% "GNU Free Documentation License".
%
% http://www.cybop.net
% - Cybernetics Oriented Programming -
%
% http://www.resmedicinae.org
% - Information in Medicine -
%
% Version: $Revision: 1.1 $ $Date: 2008-08-19 20:41:07 $ $Author: christian $
% Authors: Christian Heller <christian.heller@tuxtax.de>
%

\subsubsection{Knowledge Query and Manipulation Language}
\label{knowledge_query_and_manipulation_language_heading}
\index{Knowledge Query and Manipulation Language}
\index{KQML}
\index{Common Lisp}
\index{CL}

The \emph{Knowledge Query and Manipulation Language} (KQML) \cite{kqml} is a:
\textit{language and associated protocol by which intelligent software agents
can communicate to share information and knowledge}, as Tim Finin et al.
\cite{finin} write. Its syntax were based on a balanced parenthesis list,
because initial implementations had been done in Common Lisp (CL)
\cite{commonlisp}. After Finin et al., the initial element of the list were the
\emph{Performative} and the remaining elements were the performative's
\emph{Arguments} as keyword/ value pairs. The Free Wikipedia Encyclopedia
\cite{wikipedia} explains:

\begin{quote}
    The KQML message format and protocol can be used to interact with an
    intelligent system, either by an application program, or by another
    intelligent system. KQML's \emph{Performatives} are operations that agents
    perform on each other's \emph{Knowledge} and \emph{Goal} stores.
    Higher-level interactions such as \emph{Contract Nets} and
    \emph{Negotiation} are built using these. KQML's
    \emph{Communication Facilitators} coordinate the interactions of other
    agents to support \emph{Knowledge Sharing}.
\end{quote}

An example message representing a query about the price of a share of IBM stock
might be encoded as \cite{finin}:

\begin{scriptsize}
    \begin{verbatim}
    (ask-one
    :content (PRICE IBM ?price)
    :receiver stock-server
    :language LPROLOG
    :ontology NYSE-TICKS)
    \end{verbatim}
\end{scriptsize}

System communication and its elements like \emph{Sender}, \emph{Receiver},
\emph{Language} or \emph{Message Content} will be further investigated in
chapter \ref{state_and_logic_heading}. The new language introduced in chapter
\ref{cybernetics_oriented_language_heading} defines communication operations
(logic) accompanied by properties (meta information), much the same way
performatives have arguments. Also, that new language may not only be used to
encode knowledge for communication, but to represent knowledge of arbitrary
domains. By combining pre-defined, primitive operations, it may be used to
create more complex (higher-level) algorithms.

%
% $RCSfile: darpa_agent_markup_language.tex,v $
%
% Copyright (C) 2002-2008. Christian Heller.
%
% Permission is granted to copy, distribute and/or modify this document
% under the terms of the GNU Free Documentation License, Version 1.1 or
% any later version published by the Free Software Foundation; with no
% Invariant Sections, with no Front-Cover Texts and with no Back-Cover
% Texts. A copy of the license is included in the section entitled
% "GNU Free Documentation License".
%
% http://www.cybop.net
% - Cybernetics Oriented Programming -
%
% http://www.resmedicinae.org
% - Information in Medicine -
%
% Version: $Revision: 1.1 $ $Date: 2008-08-19 20:41:06 $ $Author: christian $
% Authors: Christian Heller <christian.heller@tuxtax.de>
%

\subsubsection{DARPA Agent Markup Language / Ontology Inference Layer}
\label{darpa_agent_markup_language_heading}
\index{DARPA Agent Markup Language}
\index{DAML}
\index{Ontology Inference Layer}
\index{OIL}
\index{DAML+OIL}
\index{Semantic Web}
\index{Extensible Markup Language}
\index{XML}
\index{Resource Description Framework}
\index{RDF}
\index{OWL}

The DAML+OIL language resulted from a combination of the DAML and OIL languages.
The \emph{DARPA Agent Markup Language} (DAML) \cite{damloil} was created in a
project run by the \emph{Defense Advanced Research Projects Agency} (DARPA) of the
\emph{United States of America} (USA); the \emph{Ontology Inference Layer} (OIL)
was created within the \emph{Information Science Technologies} (IST) program of
the \emph{European Union} (EU) \cite{rdfowlrelease}. Both projects aimed at
developing a language and tools to facilitate the concept of the
\emph{Semantic Web} (section \ref{semantic_web_heading}).

At the beginning of the project stood the realisation that: \textit{The use of
ontologies (section \ref{conceptual_network_heading}) provides a very powerful
way to describe objects and their relationships to other objects.} The DAML+OIL
language, being developed as an extension to the \emph{Extensible Markup Language}
(XML) (section \ref{extensible_markup_language_heading}) and the
\emph{Resource Description Framework} (RDF) (section \ref{semantic_web_heading}),
therefore provided a \cite{rdf} rich set of constructs with which to create
ontologies and to markup information so that it becomes machine-readable and
understandable. Much of the work in DAML and OIL has now been incorporated into
OWL (section \ref{web_ontology_language_heading}).

Chapter \ref{cybernetics_oriented_language_heading} will introduce a language
that is based on XML, too.

%\input{general_ontological_language}
%\emph{General Ontological Language} (GOL) \cite{degen}
%Dokument dazu liegt lokal in /tmp:
%explanation of KIF etc.; Sowa is referenced; see end of paper for relation to informatics

%
% $RCSfile: semantic_web.tex,v $
%
% Copyright (C) 2002-2008. Christian Heller.
%
% Permission is granted to copy, distribute and/or modify this document
% under the terms of the GNU Free Documentation License, Version 1.1 or
% any later version published by the Free Software Foundation; with no
% Invariant Sections, with no Front-Cover Texts and with no Back-Cover
% Texts. A copy of the license is included in the section entitled
% "GNU Free Documentation License".
%
% http://www.cybop.net
% - Cybernetics Oriented Programming -
%
% http://www.resmedicinae.org
% - Information in Medicine -
%
% Version: $Revision: 1.1 $ $Date: 2008-08-19 20:41:08 $ $Author: christian $
% Authors: Christian Heller <christian.heller@tuxtax.de>
%

\subsection{Semantic Web}
\label{semantic_web_heading}
\index{Semantic Web}
\index{Extensible Markup Language}
\index{XML}
\index{XML Schema}
\index{XML Data}
\index{Document Content Description}
\index{DCD}
\index{Schema for Object Oriented XML}
\index{SOX}
\index{XML Metadata Interchange}
\index{XMI}
\index{Resource Description Framework}
\index{RDF}
\index{Web Ontology Language}
\index{OWL}

As mentioned in section \ref{extensible_markup_language_heading}, the
\emph{Extensible Markup Language} (XML) \cite{rdfowlrelease} provides a:
\textit{set of rules for creating vocabularies that can bring structure to both
documents and data on the Web} and it: \textit{gives clear rules for syntax.}
XML Schemas \cite{xmlschema} served as: \textit{a method for composing XML
vocabularies.} Yet although XML were a powerful, flexible surface syntax for
structured documents, it imposed no \emph{Semantic Constraints} on the
\emph{Meaning} of these documents. Having investigated the usefulness of XML
for a \emph{meaningful} sharing of information units at the semantic level,
Robin Cover writes \cite{xmlsemantics}:

\begin{quote}
    \ldots\ the use of XML for \emph{Data Interchange} may already outweigh its
    use for \emph{Document Display}. For messaging and other transaction data,
    specifications approaching the level of formal semantics (e.g. KIF or KQML)
    are desirable, governing not just common (atomic) data types in business
    objects, but complex objects used by computer agents in large-scale business
    transactions. XML vocabularies supporting these applications will need to be
    defined in terms of precise object semantics.
\end{quote}

He lists a number of efforts dealing with the support for generic XML semantics,
that is \emph{Semantic Transparency} of XML in a broader sense, to provide
unambiguous semantic specification:

\begin{itemize}
    \item[-] \emph{XML Data} \cite{xmldata}
    \item[-] \emph{Document Content Description} (DCD) for XML \cite{dcd}
    \item[-] \emph{Schema for Object Oriented XML} (SOX) \cite{sox}
    \item[-] \emph{XML Metadata Interchange} (XMI) \cite{mda}
    \item[-] \emph{Resource Description Framework} (RDF) \cite{rdf}
    \item[-] \emph{Web Ontology Language} (OWL) \cite{owl}
\end{itemize}

The RDF and OWL as well-known efforts are mentioned in the following two
subsections. Both are often comprised under the umbrella term
\emph{Semantic Web}. Much of what is written about the semantic web sounds as
if it was a replacement technology for the Web as known today. Yet Eric Miller,
leader of W3C's semantic web activity, means \cite{rdfowlrelease}:

\begin{quote}
    In reality, it's more Web Evolution than Revolution. The Semantic Web is
    made through incremental changes, by bringing machine-readable descriptions
    to the data and documents already on the Web. XML, RDF and OWL enable the
    Web to be a global infrastructure for sharing both, documents and data which
    make searching and reusing information easier and more reliable as well.
\end{quote}

%
% $RCSfile: resource_description_framework.tex,v $
%
% Copyright (C) 2002-2008. Christian Heller.
%
% Permission is granted to copy, distribute and/or modify this document
% under the terms of the GNU Free Documentation License, Version 1.1 or
% any later version published by the Free Software Foundation; with no
% Invariant Sections, with no Front-Cover Texts and with no Back-Cover
% Texts. A copy of the license is included in the section entitled
% "GNU Free Documentation License".
%
% http://www.cybop.net
% - Cybernetics Oriented Programming -
%
% http://www.resmedicinae.org
% - Information in Medicine -
%
% Version: $Revision: 1.1 $ $Date: 2008-08-19 20:41:08 $ $Author: christian $
% Authors: Christian Heller <christian.heller@tuxtax.de>
%

\subsubsection{Resource Description Framework}
\label{resource_description_framework_heading}
\index{Resource Description Framework}
\index{RDF}
\index{Extensible Markup Language}
\index{XML}
\index{RDF Schema}
\index{XML Schema}
\index{OWL}

The \emph{Resource Description Framework} (RDF) \cite{rdf} as part of the
\emph{Semantic Web} provides a standard way for simple descriptions to be made.
It is: \textit{a simple data model for referring to objects (resources) and how
they are related. An RDF-based model can be represented in XML syntax.}
\cite{wikipedia}

RDF wants to achieve for \emph{Semantics} what XML has achieved for
\emph{Syntax} -- to provide a clear set of rules for creating descriptive
information. Both follow a special schema, \emph{RDF Schema} \cite{rdf} and
\emph{XML Schema} \cite{xmlschema}, respectively, which defines the structure
and vocabulary that may be used in the corresponding documents.

Many applications that use XML as syntax for data interchange, may apply the
RDF specifications to better support the exchange of actual knowledge on the
web. The RDF data framework is used \cite{rdfowlrelease} in: asset management,
enterprise integration and the sharing and reuse of data on the web. Example
applications combining information from multiple sources on the web
\cite{rdfowlrelease} include: library catalogs, world-wide directories, news-
and content aggregation, collections of music or photos.

In the words of Brian McBride \cite{rdfowlrelease}, chair of the RDF core
working group, his group had: \textit{turned the RDF specifications into both a
practical and mathematically precise foundation on which OWL and the rest of
the semantic web can be built.}

Chapter \ref{cybernetics_oriented_language_heading} will come back to RDF once
more, and compare it with the new language then introduced.

%
% $RCSfile: web_ontology_language.tex,v $
%
% Copyright (C) 2002-2008. Christian Heller.
%
% Permission is granted to copy, distribute and/or modify this document
% under the terms of the GNU Free Documentation License, Version 1.1 or
% any later version published by the Free Software Foundation; with no
% Invariant Sections, with no Front-Cover Texts and with no Back-Cover
% Texts. A copy of the license is included in the section entitled
% "GNU Free Documentation License".
%
% http://www.cybop.net
% - Cybernetics Oriented Programming -
%
% http://www.resmedicinae.org
% - Information in Medicine -
%
% Version: $Revision: 1.1 $ $Date: 2008-08-19 20:41:09 $ $Author: christian $
% Authors: Christian Heller <christian.heller@tuxtax.de>
%

\subsubsection{Web Ontology Language}
\label{web_ontology_language_heading}
\index{Web Ontology Language}
\index{OWL}
\index{Uniform Resource Indicator}
\index{URI}
\index{Resource Description Framework}
\index{RDF}
\index{DARPA Agent Markup Language}
\index{Ontology Inference Layer}
\index{DAML+OIL}
\index{Ontology}

The \emph{Web Ontology Language} (OWL) is \cite{owl}: \textit{a semantic markup
language for publishing and sharing ontologies on the world wide web \ldots\
which delivers richer integration and interoperability of data among descriptive
communities.} It uses \emph{Uniform Resource Indicators} (URI) for naming and
is an extension of the \emph{Resource Description Framework} (RDF), adding more
vocabulary for describing properties and classes, for example relations between
classes, cardinality, richer typing of properties, or enumerated classes. OWL
was originally derived from the \emph{DARPA Agent Markup Language} +
\emph{Ontology Inference Layer} (DAML+OIL) web ontology language (section
\ref{agent_communication_language_heading}).

In the understanding of OWL, an ontology is a subject- or domain specific
vocabulary which defines the terms used to describe and represent an area of
knowledge \cite{rdfowlrelease}. However, there are other definitions of the term
\emph{Ontology} which are given in section \ref{conceptual_network_heading}.
OWL aims to add to ontologies capabilities like \cite{rdfowlrelease}:

\begin{itemize}
    \item[-] Ability to be distributed across many systems
    \item[-] Scalability to web needs
    \item[-] Compatibility with web standards for accessibility and internationalisation
    \item[-] Openness and extensibility
\end{itemize}

It introduces keywords for the use of \emph{Classification}, \emph{Subclassing}
with \emph{Inheritance} and further abstraction principles. RDF is neutral
enough to permit such extensions. Also the language introduced in chapter
\ref{cybernetics_oriented_language_heading} may be extended with meta
properties, such as one for inheritance.


%\input{temporary_section_summary}
