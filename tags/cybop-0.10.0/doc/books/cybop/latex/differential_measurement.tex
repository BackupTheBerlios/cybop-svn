%
% $RCSfile: differential_measurement.tex,v $
%
% Copyright (C) 2002-2008. Christian Heller.
%
% Permission is granted to copy, distribute and/or modify this document
% under the terms of the GNU Free Documentation License, Version 1.1 or
% any later version published by the Free Software Foundation; with no
% Invariant Sections, with no Front-Cover Texts and with no Back-Cover
% Texts. A copy of the license is included in the section entitled
% "GNU Free Documentation License".
%
% http://www.cybop.net
% - Cybernetics Oriented Programming -
%
% http://www.resmedicinae.org
% - Information in Medicine -
%
% Version: $Revision: 1.1 $ $Date: 2008-08-19 20:41:06 $ $Author: christian $
% Authors: Christian Heller <christian.heller@tuxtax.de>
%

\subsection{Differential Measurement}
\label{differential_measurement_heading}

The \emph{Second} as unit of time is currently defined by a number of oscillations
of a carbon atom. The \emph{Meter} is defined as the way that light completes in
one second. The \emph{Gram} is defined by ...
As can be seen, current definitions can all be lead back to oscillations of an
atom. An oscillation is the change of something between two states. In the end,
all physical units are based on two states -- just like abstractions in informatics.

Similarly, each concept's expansion in space can only be estimated. The
\emph{Meter} is an arbitrary definition just like all units. It is currently
defined as ?? (wavelength of light), that is it depends on the second which
itself is an arbitrary concept.
The boundaries are unclear, again.
Density makes borders between items more or less clear to us.
So much to a short physical/ philosophical excursion -- as Einstein said,
everything is relative.

And even the unit \emph{Second} is an arbitrary concept, currently defined as
oscillations of an atom.
Categories help to recognise the world in a similar way, to make large numbers
of items capturable and comparable. \emph{Comprehension} is based on
\emph{Comparison}. The human brain compares and describes new concepts with
already known ones. \emph{Measurement} is based on \emph{Comparison}, too.
Yet none of these descriptions is absolute; all are based on \emph{Comparison}.

- Fowler: Range pattern (Time etc.)

Unit Time definition: \cite{platt}, p. 4
Velocity: \cite{platt}, p. 27

==> What is important is not the unit/values as such (they are estimations anyway,
abstracted as multiples of a unit such second or meter), it is the \emph{Relations}
between them that make our model of the world, for example: v = s / t
s und t as difference (infinitesimal), cannot be given absolute;
but v as a relation/logic/law between s and t is valid
- world is made countable by using types (quality) and numbers (quantity): statics
- translation of one model state into another: logic
- composition of logic: dynamics (algorithm, formula)
--> EHR: it is not so important which values an EHR contains, rather how they are
related as concept, for example: Blood Pressure consisting of systolic and diastolic

- statics Example: velocity = s and t; s and t are fundamental abstractions
represented with basic SI units [m] and [s]

- dynamics Example: multiply_fraction = multiply_numerators and multiply_denominators;
multiplication is lead back to simple addition inside CPU

==> That is, mankind's whole world model is based on comparison, in the case of
physics, the comparison with 7 basic SI units

- 5 out of 7 SI Units can be omitted; only Space (m) and Time (s) are necessary
to describe the Universe
- all other descriptions can be derived from m and s
