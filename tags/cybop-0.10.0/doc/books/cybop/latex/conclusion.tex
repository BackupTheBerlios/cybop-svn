%
% $RCSfile: conclusion.tex,v $
%
% Copyright (C) 2002-2008. Christian Heller.
%
% Permission is granted to copy, distribute and/or modify this document
% under the terms of the GNU Free Documentation License, Version 1.1 or
% any later version published by the Free Software Foundation; with no
% Invariant Sections, with no Front-Cover Texts and with no Back-Cover
% Texts. A copy of the license is included in the section entitled
% "GNU Free Documentation License".
%
% http://www.cybop.net
% - Cybernetics Oriented Programming -
%
% http://www.resmedicinae.org
% - Information in Medicine -
%
% Version: $Revision: 1.1 $ $Date: 2008-08-19 20:41:06 $ $Author: christian $
% Authors: Christian Heller <christian.heller@tuxtax.de>
%

\subsubsection{Conclusion}
\label{conclusion_heading}
\index{OOP Innovations}
\index{Object Oriented Programming}
\index{OOP}
\index{Structured and Procedural Programming}
\index{SPP}

As could be seen in the previous sections, OOP contributed many new concepts
to software design, thus trying to improve SPP. Most importantly, SPP data
structures (struct, record) got extended towards the \emph{Class} which does
not only hold data (attributes), but also operations (methods). This brought
with the concept of \emph{Encapsulation}, which permits only special methods of
an \emph{Object} (class instance) to access the data (properties) of that same
object. The next innovation was \emph{Inheritance}, which allows a class to
reuse the attributes and methods of its super class(es). Finally, inheritance
was used to introduce the concept of \emph{Polymorphism}, which lets objects
react differently, depending on the class they were instantiated with.

All of these concepts were true innovations as compared with traditional SPP
techniques. However, they have their own drawbacks: growth of the number of
dependencies within a system (links between classes), caused by the bundling of
attributes and methods; fragile base class problem; falsified container
contents with container inheritance. This work will not just revise these
concepts, but turn them upside down. Data (attributes) and operations/
algorithms (methods) are not bundled any longer; the resolution of inheritance
relationships at runtime gets eliminated and with it polymorphism; container
inheritance is not necessary any longer, since only one global container
structure (knowledge container) is used in a system. More on that in part
\ref{contribution_heading} of this work.
