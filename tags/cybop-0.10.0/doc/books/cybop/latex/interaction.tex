%
% $RCSfile: interaction.tex,v $
%
% Copyright (C) 2002-2008. Christian Heller.
%
% Permission is granted to copy, distribute and/or modify this document
% under the terms of the GNU Free Documentation License, Version 1.1 or
% any later version published by the Free Software Foundation; with no
% Invariant Sections, with no Front-Cover Texts and with no Back-Cover
% Texts. A copy of the license is included in the section entitled
% "GNU Free Documentation License".
%
% http://www.cybop.net
% - Cybernetics Oriented Programming -
%
% http://www.resmedicinae.org
% - Information in Medicine -
%
% Version: $Revision: 1.1 $ $Date: 2008-08-19 20:41:07 $ $Author: christian $
% Authors: Christian Heller <christian.heller@tuxtax.de>
%

\subsection{Interaction}
\label{interaction_heading}
\index{Interaction}

As was explained before, most abstract models of the human mind have a composed
nature, that is consist of smaller parts. If being compounds, they hold certain
information about their parts, namely their name, model and abstraction. More
on this in section \ref{knowledge_representation_heading}. But isn't there
other detailed knowledge a compound must have about its parts? What about the
order or position, the size or colour of parts within their compound?

%
% $RCSfile: meta_information.tex,v $
%
% Copyright (C) 2002-2008. Christian Heller.
%
% Permission is granted to copy, distribute and/or modify this document
% under the terms of the GNU Free Documentation License, Version 1.1 or
% any later version published by the Free Software Foundation; with no
% Invariant Sections, with no Front-Cover Texts and with no Back-Cover
% Texts. A copy of the license is included in the section entitled
% "GNU Free Documentation License".
%
% http://www.cybop.net
% - Cybernetics Oriented Programming -
%
% http://www.resmedicinae.org
% - Information in Medicine -
%
% Version: $Revision: 1.1 $ $Date: 2008-08-19 20:41:07 $ $Author: christian $
% Authors: Christian Heller <christian.heller@tuxtax.de>
%

\subsubsection{Meta Information}
\label{meta_information_heading}
\index{Meta Information}
\index{Visual Impressions of the Human Mind}
\index{Movement}
\index{Shape}
\index{Depth}
\index{Colour}
\index{Physical Dimension}
\index{Mass}
\index{Interaction}
\index{Relation}
\index{Conceptual Interaction}

To find an answer, the science of \emph{Psychology} needs to be called in. It
distinguishes between various aspects of a (visual) impression of the human
mind, as there are \emph{Movement}, \emph{Shape}, \emph{Depth} or \emph{Colour}
\cite{stoerig}. Looking closer at these, one quickly realises that they contain
representations of the classical physical dimensions that humans use to
describe the world:

\begin{itemize}
    \item[-] \emph{Movement} stands for changing the state of something over
        \emph{Time}
    \item[-] \emph{Shape} is how items would appear in a two-dimensional world,
        as known from \emph{Geometry}
    \item[-] \emph{Depth} (which is possible to recognise thanks to the human's
        ability for stereo vision) adds a third dimension to shapes, so that
        these become three-dimensional and form a \emph{Space}
    \item[-] \emph{Colour}, not being considered a dimension, tells about how
        items reflect \emph{Light}
\end{itemize}

Another physical value often used to abstract and describe the world is
\emph{Mass}. Again, it is not considered to be a dimension. If, according to
modern physics, not all of the impressions listed above are dimensions, what
else is common to them? -- All are used to express a special \emph{Interaction}.
(Einstein \cite{einstein} would probably prefer the term \emph{Relation}, to
better point out the relative nature of at least the space and time, in which a
whole and its parts interact.) To avoid conflicts with other sciences, this
document sticks to the term \emph{Conceptual Interaction}.

The following paragraphs will describe some conceptual interactions in more
detail and give examples for their understanding.

%
% $RCSfile: space.tex,v $
%
% Copyright (C) 2002-2008. Christian Heller.
%
% Permission is granted to copy, distribute and/or modify this document
% under the terms of the GNU Free Documentation License, Version 1.1 or
% any later version published by the Free Software Foundation; with no
% Invariant Sections, with no Front-Cover Texts and with no Back-Cover
% Texts. A copy of the license is included in the section entitled
% "GNU Free Documentation License".
%
% http://www.cybop.net
% - Cybernetics Oriented Programming -
%
% http://www.resmedicinae.org
% - Information in Medicine -
%
% Version: $Revision: 1.1 $ $Date: 2008-08-19 20:41:08 $ $Author: christian $
% Authors: Christian Heller <christian.heller@tuxtax.de>
%

\subsubsection{Space}
\label{space_heading}
\index{Space}
\index{Atom with Electrons}
\index{Part}
\index{Position}
\index{Graphical Frame consisting of Components}
\index{Human Body having Organs}

To the common concept of an \emph{Atom} belong a \emph{Core} and \emph{Electrons}.
The atom provides the \emph{Space} that the core and the electrons can fill with
their extension. For core and electrons, the atom represents the small universe
they live in. Moreover, the atom \emph{knows} about the \emph{Position} (more
correct \emph{Trajectory}) of each electron. Thus, one can say that the atom as
a \emph{Whole} interacts with its \emph{Parts} by means of space. Electrons, on
the other hand, know nothing about their own position within the atom; they do
not know about the existence of the atom at all.

A different example would be the \emph{Graphical Frame} of a software application.
It has an expansion that cannot be crossed by its children. Children may be a
\emph{Menu Bar}, \emph{Tool Bar} and \emph{Status Bar}. In order to be positioned
correctly, the frame has to know about their coordinates or orientation. Again,
this can be seen as an interaction over space.

A third and last example that was already stressed in previous sections would be
the \emph{Human Body} consisting of organs like \emph{Heart}, \emph{Brain} and
\emph{Arm}. Each organ has its special position within the body concept.
However, it is always useful to keep in mind that models (concepts) are an
abstraction, an \emph{Illusion}. Taking the example of the human body, how is
it constituted? Does belong to it the:

\begin{itemize}
    \item[-] Air in its lungths
    \item[-] Sweat leaving its skin
    \item[-] Radiation crossing it
    \item[-] Food being eaten
\end{itemize}

The human body, in reality, is not stable; it changes permanently, in all
dimensions. Human thinking only makes it stable by characterising it with
arbitrary properties, picked out of millions. It actually exists (in the same
state) for just an infinitesimal instant in time. The same counts for any other
real world items. Some elementary or yet smaller particles have a lifetime of
only a fraction of a second. But even within this minimal lifetime, they
probably take on millions of different states.

%
% $RCSfile: mass.tex,v $
%
% Copyright (C) 2002-2008. Christian Heller.
%
% Permission is granted to copy, distribute and/or modify this document
% under the terms of the GNU Free Documentation License, Version 1.1 or
% any later version published by the Free Software Foundation; with no
% Invariant Sections, with no Front-Cover Texts and with no Back-Cover
% Texts. A copy of the license is included in the section entitled
% "GNU Free Documentation License".
%
% http://www.cybop.net
% - Cybernetics Oriented Programming -
%
% http://www.resmedicinae.org
% - Information in Medicine -
%
% Version: $Revision: 1.1 $ $Date: 2008-08-19 20:41:07 $ $Author: christian $
% Authors: Christian Heller <christian.heller@tuxtax.de>
%

\subsubsection{Mass}
\label{mass_heading}
\index{Mass}
\index{Force}
\index{Solar System consisting of Planets}
\index{Artificial Neural Network}
\index{ANN}
\index{Weight}

A \emph{Solar System}, as concept, has very much in common with the atom. It
has a star, the \emph{Sun}, as its core and it has \emph{Planets} orbiting
around that star. Besides the conceptual interaction over space that also
exists here, there is another relation worth paying attention to: \emph{Mass}.
It has great influence on the \emph{Gravitational Force}.

Conceptually, the solar system can be treated as a closed field of \emph{Mass},
the sun representing the centre, the planets additions. The solar system as a
\emph{Whole} knows about the masses of its \emph{Parts}, what can be considered
a conceptual interaction.

Another example, taken from informatics, are \emph{Artificial Neural Networks}
(ANN) consisting of \emph{Neurons} and \emph{weighted} connections between them.
%(section \ref{artificial_neural_network_heading}).
Ideally, the ANN knows
about its neurons -- or, depending on its design, the layers that contain them
-- and the corresponding connections. This structural information needs to be
complemented by \emph{Weight} (\emph{Mass}) information indicating the strength
(importance) of an association between neurons.

%
% $RCSfile: time.tex,v $
%
% Copyright (C) 2002-2008. Christian Heller.
%
% Permission is granted to copy, distribute and/or modify this document
% under the terms of the GNU Free Documentation License, Version 1.1 or
% any later version published by the Free Software Foundation; with no
% Invariant Sections, with no Front-Cover Texts and with no Back-Cover
% Texts. A copy of the license is included in the section entitled
% "GNU Free Documentation License".
%
% http://www.cybop.net
% - Cybernetics Oriented Programming -
%
% http://www.resmedicinae.org
% - Information in Medicine -
%
% Version: $Revision: 1.1 $ $Date: 2008-08-19 20:41:09 $ $Author: christian $
% Authors: Christian Heller <christian.heller@tuxtax.de>
%

\subsubsection{Time}
\label{time_heading}
\index{Time}
\index{Process}
\index{Sub Process}
\index{Order of Sub Processes}
\index{Algorithm}
\index{Occurence of Sub Processes}
\index{Duration of Part Processes}

A third kind of conceptual interaction that humans use to place themselves and
the environment into their very own model of the universe is \emph{Time}. Section
\ref{compound_heading} showed on the example of \emph{Take Book from Library}
that any \emph{Process} can be split into \emph{Sub Processes} and thus
represents a structure with \emph{Hierarchical Character}.

In most cases, the \emph{Order} in which sub processes are executed, is very
important. Without it, no meaningful \emph{Algorithm} could ever be created.
A process thus needs to know about the \emph{Occurrence} of its sub processes
and this sequence information is usually stored in units of time.

Moreover, the \emph{Whole} process sets a time frame that all \emph{Part}
processes, in sum, cannot exceed. Their \emph{Duration} is limited. Again,
process and sub processes have some kind of conceptual relation; in this case
over time.

%
% $RCSfile: constraint.tex,v $
%
% Copyright (C) 2002-2008. Christian Heller.
%
% Permission is granted to copy, distribute and/or modify this document
% under the terms of the GNU Free Documentation License, Version 1.1 or
% any later version published by the Free Software Foundation; with no
% Invariant Sections, with no Front-Cover Texts and with no Back-Cover
% Texts. A copy of the license is included in the section entitled
% "GNU Free Documentation License".
%
% http://www.cybop.net
% - Cybernetics Oriented Programming -
%
% http://www.resmedicinae.org
% - Information in Medicine -
%
% Version: $Revision: 1.1 $ $Date: 2008-08-19 20:41:06 $ $Author: christian $
% Authors: Christian Heller <christian.heller@tuxtax.de>
%

\subsubsection{Constraint}
\label{constraint_heading}
\index{Constraint}
\index{Table with constrained Number of Parts}
\index{Temperature with Minimum and Maximum}

The previous sections have discussed three kinds of conceptual interaction:
\emph{Space}, \emph{Mass} and \emph{Time}. They are used by a model (concept)
to position parts within its area of validity.

Yet this meta knowledge is not enough. Frequently, parts have to be
\emph{constrained} to maintain the validity of the whole model. The concept of
a \emph{Table}, for example, may consist of a \emph{Top} and one to four
\emph{Legs}. The additional meta information herein is the constraint of the
number of legs to at least \emph{one} and at most \emph{four}.

Another example regards the area of valid values that parts can take on. The
\emph{Temperature} of an alive human body lies somewhere between a
\emph{Minimum} of +30$^{\circ}$C and a \emph{Maximum} of +40$^{\circ}$C
(broad-minded estimation). A corresponding model has to remember these extrema
in order to be able to limit numbers to the correct temperature range.

