%
% $RCSfile: hello_world.tex,v $
%
% Copyright (C) 2002-2008. Christian Heller.
%
% Permission is granted to copy, distribute and/or modify this document
% under the terms of the GNU Free Documentation License, Version 1.1 or
% any later version published by the Free Software Foundation; with no
% Invariant Sections, with no Front-Cover Texts and with no Back-Cover
% Texts. A copy of the license is included in the section entitled
% "GNU Free Documentation License".
%
% http://www.cybop.net
% - Cybernetics Oriented Programming -
%
% http://www.resmedicinae.org
% - Information in Medicine -
%
% Version: $Revision: 1.1 $ $Date: 2008-08-19 20:41:07 $ $Author: christian $
% Authors: Christian Heller <christian.heller@tuxtax.de>
%

\subsubsection{Hello World}
\label{hello_world_heading}
\index{CYBOL 'Hello, World!' Example}

A practice popularised by Brian Kernighan and Dennis Ritchie
\cite{kernighan1988} is to write as first program one that prints the words
\emph{Hello, World!}, when teaching/ learning a new programming language. Two
possible CYBOL versions of that minimal program are given following. The first
consists of only two operations: \emph{send} and \emph{exit}. The string
message to be displayed on screen is handed over as \emph{property} to the
\emph{send} operation, before the \emph{exit} operation shuts down the system:

\begin{scriptsize}
    \begin{verbatim}
<model>
    <part name="send_model_to_output" channel="inline" abstraction="operation" model="send">
        <property name="language" channel="inline" abstraction="string" model="tui"/>
        <property name="receiver" channel="inline" abstraction="string" model="user"/>
        <property name="message" channel="inline" abstraction="string" model="Hello, World!"/>
    </part>
    <part name="exit_application" channel="inline" abstraction="operation" model="exit"/>
</model>
    \end{verbatim}
\end{scriptsize}

The second example template is slightly more complex. It starts with creating a
domain model that consists of just one \emph{greeting} string. That string is
then sent as message to the human user via a \emph{Textual User Interface}
(TUI), just as in the first example. The difference is that now, the greeting
is not handed over as hard-coded \emph{string} value, but is read from the
runtime knowledge model, which is indicated by its \emph{abstraction} value:

\begin{scriptsize}
    \begin{verbatim}
<model>
    <part name="create_greeting" channel="inline" abstraction="operation" model="create_part">
        <property name="name" channel="inline" abstraction="string" model="greeting"/>
        <property name="channel" channel="inline" abstraction="string" model="inline"/>
        <property name="abstraction" channel="inline" abstraction="string" model="string"/>
        <property name="model" channel="inline" abstraction="string" model="Hello, World!"/>
    </part>
    <part name="send_model_to_output" channel="inline" abstraction="operation" model="send">
        <property name="language" channel="inline" abstraction="string" model="tui"/>
        <property name="receiver" channel="inline" abstraction="string" model="user"/>
        <property name="message" channel="inline" abstraction="knowledge" model="greeting"/>
    </part>
    <part name="destroy_greeting" channel="inline" abstraction="operation" model="destroy_part">
        <property name="name" channel="inline" abstraction="knowledge" model="greeting"/>
    </part>
    <part name="exit_application" channel="inline" abstraction="operation" model="exit"/>
</model>
    \end{verbatim}
\end{scriptsize}

The appearance of a \emph{create\_part}/ \emph{destroy\_part} pair in the second
example already suggests how an application lifecycle with \emph{startup-},
\emph{runtime-} and \emph{shutdown} phase could look like in CYBOL.
