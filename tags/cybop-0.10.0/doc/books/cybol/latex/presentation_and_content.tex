%
% $RCSfile: presentation_and_content.tex,v $
%
% Copyright (c) 2002-2007. Christian Heller. All rights reserved.
%
% Permission is granted to copy, distribute and/or modify this document
% under the terms of the GNU Free Documentation License, Version 1.1 or
% any later version published by the Free Software Foundation; with no
% Invariant Sections, with no Front-Cover Texts and with no Back-Cover
% Texts. A copy of the license is included in the section entitled
% "GNU Free Documentation License".
%
% http://www.cybop.net
% - Cybernetics Oriented Programming -
%
% Version: $Revision: 1.1 $ $Date: 2007-08-01 13:59:00 $ $Author: christian $
% Authors: Christian Heller <christian.heller@tuxtax.de>
%

\subsection{Presentation and Content}
\label{presentation_and_content_heading}
\index{Presentation and Content Example}
\index{Mathematical Markup Language}
\index{MathML}

The \emph{Mathematical Markup Language} (MathML) \cite{mathml} provides means
for representing mathematical expressions, that is \emph{Content} as well as
\emph{Presentation} of data. Both are discrete models, comparable to the
\emph{Domain} and \emph{User Interface} (UI) of a software application, which
can be translated into each other.

CYBOL uses just four tags (section \ref{semantics_heading}) but can represent
mathematical expressions as well. What MathML calls \emph{Content}, becomes a
\emph{Logic} knowledge template in CYBOL. The mathematical content of the
formula $(a + b)^{2}$ would be modelled as follows:

\begin{scriptsize}
    \begin{verbatim}
<model>
    <part name="addition" channel="inline" abstraction="operation" model="add">
        <property name="abstraction" channel="inline" abstraction="character" model="integer"/>
        <property name="summand_1" channel="inline" abstraction="knowledge" model=".domain.a"/>
        <property name="summand_2" channel="inline" abstraction="knowledge" model=".domain.b"/>
        <property name="sum" channel="inline" abstraction="knowledge" model=".domain.c"/>
    </part>
    <part name="exponentiation" channel="inline" abstraction="operation" model="power">
        <property name="base" channel="inline" abstraction="knowledge" model=".domain.c"/>
        <property name="power" channel="inline" abstraction="integer" model="2"/>
        <property name="result" channel="inline" abstraction="knowledge" model=".domain.r"/>
    </part>
</model>
    \end{verbatim}
\end{scriptsize}

And the formula's \emph{Presentation} would be defined by the following two
CYBOL \emph{State} knowledge templates, of which the second one represents the
\emph{Base} that is referenced by the first one:

\begin{scriptsize}
    \begin{verbatim}
<model>
    <part name="base" channel="file" abstraction="compound" model="domain/base.cybol">
        <property name="fence" channel="inline" abstraction="boolean" model="true"/>
    </part>
    <part name="power" channel="inline" abstraction="integer" model="2">
        <property name="superscript" channel="inline" abstraction="boolean" model="true"/>
    </part>
</model>

<model>
    <part name="summand_$1" channel="inline" abstraction="character" model="a"/>
    <part name="operator" channel="inline" abstraction="character" model="+"/>
    <part name="summand_$2" channel="inline" abstraction="character" model="b"/>
</model>
    \end{verbatim}
\end{scriptsize}
