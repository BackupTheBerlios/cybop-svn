%
% $RCSfile: operation_call.tex,v $
%
% Copyright (c) 2002-2007. Christian Heller. All rights reserved.
%
% Permission is granted to copy, distribute and/or modify this document
% under the terms of the GNU Free Documentation License, Version 1.1 or
% any later version published by the Free Software Foundation; with no
% Invariant Sections, with no Front-Cover Texts and with no Back-Cover
% Texts. A copy of the license is included in the section entitled
% "GNU Free Documentation License".
%
% http://www.cybop.net
% - Cybernetics Oriented Programming -
%
% Version: $Revision: 1.1 $ $Date: 2007-08-01 13:59:00 $ $Author: christian $
% Authors: Christian Heller <christian.heller@tuxtax.de>
%

\subsection{Operation Call}
\label{operation_call_heading}
\index{Operation Call Example}

As stated previously, logic models may access and manipulate state models. The
simplest form of a logic model is an operation with associated input/ output
(i/o) state models. The following CYBOL knowledge template calls an \emph{add}
operation, handing over i/o parameters as \emph{properties} of the
corresponding \emph{part}:

\begin{scriptsize}
    \begin{verbatim}
<model>
    <part name="addition" channel="inline" abstraction="operation" model="add">
        <property name="abstraction" channel="inline" abstraction="character" model="integer"/>
        <property name="summand_1" channel="inline" abstraction="integer" model="1"/>
        <property name="summand_2" channel="inline" abstraction="knowledge" model=".app.summand"/>
        <property name="sum" channel="inline" abstraction="knowledge" model=".app.result"/>
    </part>
</model>
    \end{verbatim}
\end{scriptsize}

The example nicely shows how state models can be given in various formats. The
\emph{summand\_1} is given as constant value, defined directly in the knowledge
template. Its type of abstraction is \emph{integer}. The \emph{summand\_2-} and
\emph{sum} parameters, on the other hand, are given as dot-separated references
to the runtime tree of knowledge models. Their type of abstraction is therefore
\emph{knowledge}.
