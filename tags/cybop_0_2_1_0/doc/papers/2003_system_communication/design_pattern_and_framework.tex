%
% $RCSfile: design_pattern_and_framework.tex,v $
%
% Copyright (c) 2001-2004. Christian Heller. All rights reserved.
%
% No copying, altering, distribution or any other actions concerning this
% document, except after explicit permission by the author!
% At some later point in time, this document is planned to be put under
% the GNU FDL license. For now, _everything_ is _restricted_ by the author.
%
% http://www.cybop.net
% - Cybernetics Oriented Programming -
%
% http://www.resmedicinae.org
% - Information in Medicine -
%
% @author Christian Heller <christian.heller@tuxtax.de>
%

\subsection{Design Pattern and Framework}
\label{design_pattern_and_framework_heading}

One well-known way to create a layered system with clear architecture and only
few interdependencies is the use of \emph{Software Patterns} (often divided
into \emph{Design-}, \emph{Architecture-} and other patterns). They help
recognizing recurring structures for application on similar problems.
Another, closely related technique are \emph{Frameworks}. They can help prevent
code duplication and development efforts.
Both concepts -- frameworks and patterns -- depend on each other and provide
higher flexibility for software components \cite{pree}.\\
This paper is not about frameworks but its solutions are extracted and used in
one called \emph{Cybernetics Oriented Programming} (CYBOP) \cite{cybop}.
Its main concept is based on the hierarchical structure of the universe
\cite{hellerbohl}. This very simple idea can perfectly be mapped on software
systems.

