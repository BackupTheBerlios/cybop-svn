%
% $RCSfile: object_id.tex,v $
%
% Copyright (c) 2001-2004. Christian Heller. All rights reserved.
%
% No copying, altering, distribution or any other actions concerning this
% document, except after explicit permission by the author!
% At some later point in time, this document is planned to be put under
% the GNU FDL license. For now, _everything_ is _restricted_ by the author.
%
% http://www.cybop.net
% - Cybernetics Oriented Programming -
%
% http://www.resmedicinae.org
% - Information in Medicine -
%
% @author Christian Heller <christian.heller@tuxtax.de>
%

\subsubsection{Object ID (OID)}
\label{object_id_heading}

Most database systems provide an own algorithm to generate primary keys for the
tables. But the applications that use our communication architecture shall also
be able to work if a database server is not reachable, e.g. due to a network
failure. Thats why the keys are generated locally, by each application.
Based on the assumption that every host in a network has a network card, it thereby
has a unique internet address. This number is concatinated with an exact time stamp
(nanoseconds). That is why the OID is unique in the global network and unique in
time.\\
The proposed approach uses the OID as file name for local storage and the same
OID as primary key in the main table of the database. Therewith, both models can
be mapped to each other. Of course, it is necessary to avoid overwriting of new
data in the database. If, for example, a network connection is cut and a little
later, one wants to get data from the local files and write them up in the restored
central database, it has to be made sure that nobody else has modified the data
during the offline-time. That is why there is another technique to ensure this
-- the time stamp.

