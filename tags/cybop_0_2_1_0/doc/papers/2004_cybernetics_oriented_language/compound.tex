%
% $RCSfile: compound.tex,v $
%
% Copyright (c) 2001-2004. Christian Heller. All rights reserved.
%
% No copying, altering, distribution or any other actions concerning this
% document, except after explicit permission by the author!
% At some later point in time, this document is planned to be put under
% the GNU FDL license. For now, _everything_ is _restricted_ by the author.
%
% http://www.cybop.net
% - Cybernetics Oriented Programming -
%
% http://www.resmedicinae.org
% - Information in Medicine -
%
% @author Christian Heller <christian.heller@tuxtax.de>
%

\subsection{Compound}
\label{compound_heading}

\emph{Composition} is the third kind of abstraction that humans use to understand
their environment. It is an important instrument for the human mind to associate
information, that is to acquire, store and recall \emph{Knowledge}. Every item is
recognized as a \emph{Compound} of smaller items and can therefore also be called
\emph{Tree} or \emph{Hierarchy}. The subject of \emph{Artificial Intelligence} (AI)/
\emph{Knowledge Engineering} talks of \emph{Concept} or \emph{Schema}.

In software design, the terms \emph{Parent} and \emph{Child} are often used to
describe both, the items in a composition as well as the items in a categorization
relationship (section \ref{category_heading}).
To avoid misunderstandings, this document sticks to the terms \emph{Super} and
\emph{Sub} for categorization and to the terms \emph{Whole} and \emph{Part}
\cite{sowa} for composition. Yet other terms to describe items of a composition
would be \emph{Container} and \emph{Element}.

To stick with the example of a \emph{Human Being}, one could say that it is
composed of \emph{Organs} such as \emph{Eye}, \emph{Ear}, \emph{Heart},
\emph{Brain}, \emph{Arm} and further, also smaller parts. Other examples are the
concept of an \emph{Atom} consisting of a \emph{Core} and \emph{Electrons} or
that of a physical \emph{Book} composed of a \emph{Paperback Cover} and
\emph{Paper Pages}.
However, knowledge representation always depends on what one wants to express in
which context. The \emph{Book}, for example, can be represented in many other
ways. Logically, it is usually separated into \emph{Part}, \emph{Chapter},
\emph{Section}, \emph{Paragraph}, \emph{Sentence}, \emph{Word} and \emph{Character}.

It is important to note the \emph{unidirectional} kind of relations: A human
being is composed of organs but an organ is never composed of a human being!

Not only \emph{static} items represent a compound; \emph{dynamic} items are
hierarchical as well. The process \emph{Take Book from Library}, for example,
may have the following structure:

\begin{itemize}
    \item{Check Catalogue}
    \begin{itemize}
        \item{Investigate suitable Books}
        \item{Note Registration Number}
    \end{itemize}
    \item{Organize Book}
    \begin{itemize}
        \item{Look for Shelf}
        \item{Take off Book}
    \end{itemize}
    \item{Borrow Book}
\end{itemize}

Returning to human thinking, one realizes that in the end, everything in universe
can be put into variable hierarchical models, that is consists of smaller items
and belongs to a bigger item. From the physical point of view, nobody knows where
this hierarchy really stops, towards \emph{Microcosm} as well as towards
\emph{Macrocosm}. There is no \emph{absolute}, \emph{basic} item.
A \emph{Particle} as concept exists only in the human mind, placed somewhere
between micro- and macrocosm, with hypothetic borders.
