%
% $RCSfile: category.tex,v $
%
% Copyright (c) 2001-2004. Christian Heller. All rights reserved.
%
% No copying, altering, distribution or any other actions concerning this
% document, except after explicit permission by the author!
% At some later point in time, this document is planned to be put under
% the GNU FDL license. For now, _everything_ is _restricted_ by the author.
%
% http://www.cybop.net
% - Cybernetics Oriented Programming -
%
% http://www.resmedicinae.org
% - Information in Medicine -
%
% @author Christian Heller <christian.heller@tuxtax.de>
%

\subsection{Category}
\label{category_heading}

Offline thinking (in terms of second order) enables humans not only to
discriminate items but also to \emph{categorize} them into superior groups.
Since it is impossible to exactly model the real world in complete, compromises
have to be made: People do not model every single item in their minds but rather
group them into \emph{Types} (\emph{Classes}) of common characteristics.

This kind of classification stems from the earliest days of ancient science.
\emph{Plato}'s pupil \emph{Aristotle} (being the teacher of \emph{Alexander the
Great}) was the first philosopher who logically captured and organized the world.
It was him who sorted items into clear groups which he called \emph{Categories}.
And it was him who first distinguished between \emph{enlivened} and
\emph{unenlivened} nature; who parted living forms into \emph{Plants},
\emph{Animals} and \emph{Humans}. The science of biology calls this classification
a \emph{Systematics}.

\emph{Categorization} (classification) can be seen from two sides, depending on
what direction of that relationship one wants to emphasize. Taking Aristotle's
examples, \emph{Living Thing} would be a \emph{Generalization} of \emph{Plants},
\emph{Animals} and \emph{Humans}. \emph{Human Being} would be a \emph{Specialization}
of \emph{Living Thing}.

Software developers call categorization an \emph{is-a} relationship and talk of
\emph{Super} and \emph{Sub} categories (sometimes also \emph{Parent} and
\emph{Child} categories). Section \ref{property_bundling_heading} mentioned that
object oriented programming uses categorization to let a sub class inherit
attributes and methods from its super class.
