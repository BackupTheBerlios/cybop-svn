%
% $RCSfile: semantics.tex,v $
%
% Copyright (c) 2001-2004. Christian Heller. All rights reserved.
%
% No copying, altering, distribution or any other actions concerning this
% document, except after explicit permission by the author!
% At some later point in time, this document is planned to be put under
% the GNU FDL license. For now, _everything_ is _restricted_ by the author.
%
% http://www.cybop.net
% - Cybernetics Oriented Programming -
%
% http://www.resmedicinae.org
% - Information in Medicine -
%
% @author Christian Heller <christian.heller@tuxtax.de>
%

\subsection{Semantics}
\label{semantics_heading}

CYBOL files can be used to model either \emph{static} or \emph{dynamic} aspects.
In both cases, the \emph{same} syntax (document structure) with \emph{identical}
vocabulary (tags and attributes) is applied. It is the attribute \emph{Values}
that make a difference in meaning. An \emph{Attribute} keeps meta information
about the contents of a \emph{Tag}. In CYBOL, the tag of main interest is
\emph{part}. Its attributes contain information about its:

\begin{itemize}
    \item{Name (to identify different parts)}
    \item{Model (compound or primitive)}
    \item{Position (in space, time or mass)}
    \item{Constraint (minima, maxima and further limitations)}
\end{itemize}

What is missing is a means to keep such meta information about an attribute,
too. How should an interpreter know if it deals with a \emph{compound} or a
\emph{primitive} model, with a position in \emph{Space} or in \emph{Time}?
It is therefore necessary to \emph{bundle} attributes in \emph{Pairs of Two},
one attribute containing the actual value and the second attribute containing
abstraction information about how the first attribute gets interpreted correctly.
The only exception is the name which gets always interpreted as string of characters.
The resulting attributes of the \emph{part} tag are:

\begin{itemize}
    \item{name}
    \item{part\_abstraction}
    \item{part\_model}
    \item{position\_abstraction}
    \item{position\_model}
    \item{constraint\_abstraction}
    \item{constraint\_model}
\end{itemize}

A list of defined, primitive abstraction values for CYBOL can be found in \cite{cybop}.
