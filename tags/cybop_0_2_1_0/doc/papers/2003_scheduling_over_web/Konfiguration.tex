\section{Konfiguration}

  \subsection{Inhalt}
    In diesem Abschnitt kann nur eine Beispielkonfiguration
    f�r die Webapplication beschrieben werden. Dies
    beschr�nkt sich auf ein bestimmtes Betriebssystem und
    auf bestimmte Tools. Bei anderen Betriebssystemen, bei
    anderen Tools und anderen Versionen von Betriebssystemen und 
    Tools kann die Konfiguration anders
    aussehen. Im Prinzip sollte aber der Ablauf �hnlich
    sein.

    Abgrenzung:
    An dieser Stelle wird nicht die Installation und
    Konfiguration des Betriebssystems beschrieben.
    Weiterhin wird hier auch nicht die Installation der
    verwendeten Tools behandelt. Dies w�rde den Rahmen
    des Kapitels sprengen. Dies ist die Aufgabe des
    Administrators.

  \subsection{Toolauswahl}

    F�r die Webapplication werden folgende Teile gebraucht.

    \begin{itemize}
      \item Betriebssystem
      \item Java-Version
      \item Webserver
      \item Servlet- und JSP-Engine (Applicationserver)
      \item Datenbank,
    \end{itemize}
    wobei bei der Beispielkonfiguration der Webserver und der 
    Applicationserver
    von Tomcat abgedeckt werden. Im normalen Betrieb werden Webserver
    und Applicationserver getrennt und nur die Servlet- und
    JSP-Anfragen werden zum Applicationserver weitergeleitet.

    Die Beispielkonfiguration wurde unter folgendem
    Betriebssystem und folgenden Tools realisiert:
    \begin{table}[h]
\centering
\caption{Tabelle zur Toolauswahl}
\begin{tabular}{|l|l|l|l|}

  \hline
  \textbf{Aufgabe} & \textbf{Name} & \textbf{Version} & \textbf{Beschreibung} \\ \hline\hline
  Betriebssystem    & Linux Debian            & 3.0          & Codename Woody        \\ \hline
  Java-Version (Standard) & Java 2 SDK       & 1.4.1        &                       \\ \hline
  Java-Version (Enterprise) & Java 2 SDK  EE & 1.3.1        &                       \\ \hline
  Web- und Applicationserver & Tomcat        & 4.0.6        & Servlet 2.3, JSP 1.2  \\ \hline
  Datenbank        & mySQL                   & 3.23         &                       \\ \hline
\end{tabular}
\end{table}


 
  \subsection{Voraussetzung}
    F�r die Konfiguration und den Betrieb der Webanwendung wird folgendes
    vorausgesetzt:

    \begin{itemize}
      \item installiertes Java
      \item Applicationserver (Tomcat), l�uft auf Port 8080
      \item Datenbank (mySQL), l�uft
      \item Webanwendung ist auf dem Server unter folgendem Verzeichnis verf�gbar:
            \begin{quote}
              /home/resmedicinae
            \end{quote}
            Dazu m�ssen von SourceForge vom Projekt "`Resmedicinae"' die Module
            \begin{itemize}
              \item lib
              \item etc
              \item share
              \item src
              \item bin
            \end{itemize}
            heruntergeladen werden.
    \end{itemize}

  \subsection{Konfiguration Tomcat}
    Die allgemeine Serverkonfiguration von Tomcat erfolgt in folgender Datei:
    \begin{quote}
      \$\{TOMCAT\_HOME\}/conf/server.xml   
    \end{quote}

    Die Datei ist eine XML-Datei und folgt also der XML-Syntax. 
    Die �nderung ist an folgender Stelle im XML-Syntaxbaum 
    vorzunehmen:
\begin{verbatim}
<Server ...>
  <Service name="Tomcat-Standalone" ...>
    <Engine name="Standalone" ...>
      <Host name="localhost" ...>
   
      </Host>
    </Engine>
  </Service>
</Server>
\end{verbatim}
    
    Folgende �nderung ist in dem Baum einzutragen:
\begin{verbatim}
<Context path="/resdata" 
         docBase="/home/resmedicinae/share/application/healthcare/resdata">
  <Parameter name="resdata.xml" 
             value="/home/resmedicinae/etc/resdata.xml"> 
  </Parameter>
</Context>
\end{verbatim}

    wobei die Werte folgende Bedeutungen haben: 
    
    \begin{table}[hptb]
\centering
\caption{Tabelle zur Beschreibung der Werte f�r die server.xml}
\begin{tabular}{|l|p{10cm}|}

  \hline
  \textbf{Schl�ssel}  & \textbf{Bedeutung} \\ \hline\hline
  path                & Mappingpfad zum Stammverzeichnis     \\ \hline
  docbase             & Stammverzeichnis der Webapplication   \\ \hline
  name                & Name des Parameters. Dieser ist f�r die 
                        Webapplication "`resdata"' immer "`resdata.xml"' \\ \hline
  value               & Wert des Parameters - f�r den Parameter "`resdata.xml"'
                        der Ort der Konfigurationsdatei        \\ \hline
\end{tabular}
\end{table}
  
    
    Weiterhin sind die Variablen JAVA\_HOME  und der CLASSPATH f�r TOMCAT zu setzen.
    Dies geschieht am besten in folgender Datei: 
    \begin{quote}
      \$\{TOMCAT\_HOME\}/bin/catalina.sh
    \end{quote}
     
    \begin{itemize}
      \item JAVA\_HOME \newline
            Wird gleich am Anfang des Skriptes gesetzt: 
            z.B. export JAVA\_HOME=/usr/local/java
      \item CLASSPATH \newline
            Wird nach der Sektion "\# Add on extra jar files to CLASSPATH\"
            eingef�gt. Ist alles standardm��ig eingerichtet worden, so mu� hier 
            folgende Zeile erg�nzt werden:
            \begin{quote}
              CLASSPATH="\$CLASSPATH": \newline
              /home/resmedicinae/lib/protomatter-1.1.7.jar: \newline
              /home/resmedicinae/lib/mm.mysql-2.0.14-bin.jar: \newline
              /home/resmedicinae/lib/mailapi.jar
            \end{quote}
    \end{itemize}
     

  \subsection{Konfiguration mySQL}
  
    F�r die Konfiguration von mySQL sind keine speziellen 
    Einstellungen n�tig. Das einzige, was gemacht werden mu�,
    ist die Erstellung der Datenbank. Dies geschieht mit den 
    mitgelieferten Scripten.
    \begin{quote}
      \$\{RESMEDICINAE\_HOME\}/bin/application/healthcare/resdata/build\_database.sh
    \end{quote}
    Ist mySQL im Standardverzeichnis installiert, sollte das Skript sofort 
    funktionieren. Ansonsten mu� in der 
    \begin{quote}
      \$\{RESMEDICINAE\_HOME\}/bin/set\_home.sh 
    \end{quote}
    der Parameter MYSQL\_HOME gesetzt werden.
    
  \subsection{Konfiguration der Webapplication}
    F�r die Anwendung werden alle ver�nderlichen Gr��en
    in einer Konfigurationsdatei abgelegt. Diese 
    Konfigurationsdatei ist im XML-Schema abgespeichert.
    
    An dieser Stelle werden die Eintr�ge beschrieben:

    \begin{table}[hbtp!]
  \caption{Eintr�ge der Konfigurationsdatei Sektion "`logging"'}
  \centering

  \begin{tabular} {|p{3.6cm}|p{10cm}|} 

    \hline
    
    \textbf{Schl�ssel}  & \textbf{Beschreibung} \\ \hline\hline

    log\_level          & Loglevel der Application (1..15)      \\ \hline
    log\_file           & Datei der Logeintr�ge                 \\ \hline
  \end{tabular}
\end{table}


\begin{table}[hbtp!]
  \caption{Eintr�ge der Konfigurationsdatei Sektion "`database"'}
  \centering
  \begin{tabular} {|p{3.6cm}|p{10cm}|} 

    \hline
    
    \textbf{Schl�ssel}  & \textbf{Beschreibung} \\ \hline\hline

    driver              & 
    Treiber der JDBC-Verbindung
    (org.gjt.mm.mysql.Driver)             \\ \hline
    url                 & 
    Verwendungs-URL f�r die JDBC-Verbing 
    (jdbc:mysql://localhost/resdata)                             \\ \hline
    user                & 
    Verwendete Benutzer der Datenbank   
    (resdata)                                                   \\ \hline
    pwd                 & 
    Verwendetes Pa�wort der Datenbank  
    (resdata)                                                   \\ \hline
  \end{tabular}
\end{table}

    
\begin{table}[hbtp!]
  \caption{Eintr�ge der Konfigurationsdatei Sektion "`mail"'}
  \centering
  \begin{tabular} {|p{3.6cm}|p{10cm}|} 

    \hline
    
    \textbf{Schl�ssel}  & \textbf{Beschreibung} \\ \hline\hline

    protocol            & 
    Protokoll der eMail-Verschickung  
    (smtp)                                                       \\ \hline
    host                & 
    Host f�r die eMail-Verschickung   
    (mail.gmx.de)                                               \\ \hline
    user                & 
    Login f�r das eMail-Konto   
    (xxx)                                                       \\ \hline
    pwd                 & 
    Pa�wort f�r das eMail-Konto   
    (???)                                                       \\ \hline
    fromadresse         & 
    Absenderadresse f�r die eMail-Verschickung  
    (xxx@yyy.net)                                               \\ \hline
    subject\_desired    & 
    Betreff f�r eMail-Verschickung Terminwunsch   
    (Web Terminwunsch)                                           \\ \hline
    subject\_confirm    & 
    Betreff f�r eMail-Verschickung Terminbest�tigung 
    (Terminbest�tigung)                                           \\ \hline
    subject\_cancel     & 
    Betreff f�r eMail-Verschickung Terminabsage  
    (Terminabsage)                                               \\ \hline
  \end{tabular}
\end{table}

    
\begin{table}[hbtp!]
  \caption{Eintr�ge der Konfigurationsdatei Sektion "`path"'}
  \centering
  \begin{tabular} {|p{3.6cm}|p{10cm}|} 

    \hline
    
    \textbf{Schl�ssel}  & \textbf{Beschreibung} \\ \hline\hline

    servlet\_path       & 
    Pfad f�r das Servlet \newline
    (http://localhost:8080/resdata/servlet/
    org.resmedicinae.application.healthcare.resdata.controller)\\ \hline
    jsp\_path           & 
    Pfad f�r JSP-Dateien  
    (/jsp/)                                                     \\ \hline
    css\_path           & 
    Pfad f�r CSS-Dateien \newline  
    (http://servername:8080/resdata/css/)                         \\ \hline
    image\_path         & 
    Pfad f�r Bild-Dateien \newline  
    (http://servername:8080/resdata/images/)                     \\ \hline
  \end{tabular}
\end{table}


\begin{table}[hbtp!]
  \caption{Eintr�ge der Konfigurationsdatei Sektion "`navigation"'}
  \centering
  \begin{tabular} {|p{3.6cm}|p{10cm}|} 

    \hline
    
    \textbf{Schl�ssel}  & \textbf{Beschreibung} \\ \hline\hline

    resmedicinae\_home  & 
    URL der Home-Seite f�r Resmedicinae  \newline
    (http://resmedicinae.sourceforge.net/index.html)             \\ \hline
    resdata\_home       & 
    URL der Home-Seite f�r ResData \newline 
    (http://servername:8080/resdata/jsp/ResDataStart.jsp)         \\ \hline
  \end{tabular}
\end{table}


\begin{table}[hbtp!]
  \caption{Eintr�ge der Konfigurationsdatei Sektion "`parameter"'}
  \centering
  \begin{tabular} {|p{3.6cm}|p{10cm}|} 

    \hline
    
    \textbf{Schl�ssel}  & \textbf{Beschreibung} \\ \hline\hline

    control\_param      & 
    Instanzbezeichnung f�r den Controller 
    (myController)                                               \\ \hline
    loginmodel\_param   & 
    Instanzbezeichnung f�r das Modell Login  
    (myModelLogin)                                              \\ \hline
    doctormodel\_param  & 
    Instanzbezeichnung f�r das Modell Doctor 
    (myModelDoctor)                                             \\ \hline
    datemodel\_param    & 
    Instanzbezeichnung f�r das Modell Date  
    (myModeldate)                                               \\ \hline
    mailmodel\_param    & 
    Instanzbezeichnung f�r das Modell Mail  
    (myModelMail)                                               \\ \hline
  \end{tabular}
\end{table}

\begin{table}[hbtp!]
  \caption{Eintr�ge der Konfigurationsdatei Sektion "`color"'}
  \centering
  \begin{tabular} {|p{3.6cm}|p{10cm}|} 

    \hline
    
    \textbf{Schl�ssel}  & \textbf{Beschreibung} \\ \hline\hline

    status\_no          & 
    Farbe f�r freie Termine  
    (blue)                                                       \\ \hline
    status\_request     & 
    Farbe f�r angeforderte Termine
    (red)                                                       \\ \hline
    status\_confirmed   & 
    Farbe f�r best�tigte Termine   
    (yellow)                                                     \\ \hline    
  \end{tabular}
\end{table}


 \newpage 
