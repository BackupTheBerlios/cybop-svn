\section{Servlets}
  \subsection{�berblick}
    Eine normale Anfrage �ber das Web folgt immer dem gleichen Schema. Ein 
    Benutzer stellt eine Anforderung (request) an einen Server. 
    Dieser Server antwort (response) auf diese Anforderung und schickt das 
    gew�nschte Ergebnis an den Benutzer zur�ck. Der Browser 
    stellt dann diese Antwort dar. An dieser Stelle wird zwischen zwei
    verschiedenen Verarbeitungsformen unterschieden. 
    
    \begin{itemize}
      \item Clientseitige Verarbeitung \newline
            Die Antwort des Servers wird erst auf dem Client 
            verarbeitet. Ein Beispiel f�r diese Techniken w�ren 
            Javascripts und Applets.
      \item Serverseitige Verarbeitung \newline
            Die Anfrage des Benutzer wird auf dem Server verarbeitet.
            Der Benutzer bekommt die generierte Antwort zur�ck
            und mu� diese nur noch darstellen. Ein Beispiel f�r diese Technik 
            w�re PHP, ASP und Servlets.
    \end{itemize}

		Ein Servlet ist eine in Java geschriebene Softwarekomponente in Java geschrieben. 
		Die Firma Sun entwickelte diese Anwendungsschnittstelle (API), um eine
		einfache M�glichkeit der serverseitigen Programmausf�hrung
		bereitzustellen. �ber diese Api ist es m�glich, auf Objekte der 
		Webkommunikation (request, response) zuzugreifen. Zu Ausf�hrung
		eines Servlets wird eine Servlet-Engine gebraucht. 
		
		Servlets unterst�tzen Session und Cookies. Sessions dienen dazu,
		Sitzungen eines Benutzers zu verfolgen und eventuell Informationen
		zu der Sitzung zu speichern. Cookies sind kleine Informationshappen, 
		die auf der Clientseite gespeichert werden k�nnen.

	\subsection{Arbeitsweise}
		Wird eine Anfrage an das Servlet gestellt, so wird diese 
		durch die Servlet-Engine (falls noch nicht geladen) mit
		init(ServletConfig) initialisiert.
		Nach der Initialisierung k�nnen mehrere Anfragen an das Servlet 
		gestellt werden, ohne dass es dabei neu initialisiert wird.
		Das bedeutet, dass ein Servlet mehrere Anfragen und damit mehrere 
		Benutzer bedienen kann, obwohl es nur einmal im Speicher geladen
		wird. 
		
		Bekommt ein Servlet eine Anfrage, so geschieht das mit
		service(ServletRequest, ServletResponse). Das Servlet liest 
		aus dem Request-Object
		die Parameter und schreibt in das Response-Object die
		Ausgabe an den Client. Dazwischen kann es jede Anwendungslogik 
		ausf�hren, die mit Java realisierbar ist. Davon bekommt der 
		Benutzer (Client) nichts mit, weil zum Client nur das
		Ergebnis (z. B. HTML) geschickt wird. 
		
		Wird das Servlet nicht mehr gebraucht, so wird es mit destroy()
		freigegeben.
		
		