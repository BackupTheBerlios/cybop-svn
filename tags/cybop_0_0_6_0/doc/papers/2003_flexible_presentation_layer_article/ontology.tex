\section{Ontologies}

An ontology is a catalogue of types that are depending on each other in
hierarchical order. It is a formal specification concerning a particular
objective. CYBOP consists of three such ontologies:
\begin{itemize}
    \item{Basic Ontology}
    \item{Model Ontology}
    \item{System Ontology}
\end{itemize}

Figure \ref{Model Ontology} shows the model ontology. The layer super types
are {\tt Record}, {\tt Unit}, {\tt Heading} and {\tt Description}.
These classes are super types of all classes in a particular ontological level.\\
The right side shows a concrete implementation of the model ontology --
the \emph{Electronic Health Record} \cite{openehr}. This data structure
contains all information concerning a particular patient. The figure shows
{\tt Problem} types in level {\tt Unit}. These consist of episodes containing
instances of {\tt PartialContact}. In level {\tt Heading}, the structural
elements of a partial contact can be found -- {\tt Subjective}, {\tt Objective},
{\tt Assessment} and {\tt Plan}. Therapeutical issues are placed in level
{\tt Description} -- such as {\tt Medication} with particular dose.\\
\includepicture{8}{eps/ModelOntologyUML.eps}{Model Ontology}{Model Ontology}{Model Ontology}{Model Ontology}
As shown, the concept of ontology can be used to organize data structures in
a hierarchical order by defining logical layers with super types.
